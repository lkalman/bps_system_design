\hypertarget{a-budapest-school-modell}{%
\section{A Budapest School Modell}\label{a-budapest-school-modell}}

A Budapest School Modellben azt mutatjuk be, amit az elmúlt években a
tanulásról megtanultunk. Annak az útnak a tapasztalatait összegezzük,
amelynek során az álomból az első csoportjaink létrejöttek, majd újabbak és
újabbak alakultak. Ahogy nőtt a Budapest School, úgy kezdtük egyre
jobban megérteni, mit is csinálunk és miként kell ezt a tanárok, a
családok, a gyerekek, valamint a társadalom és a jogalkotó elvárásaihoz
igazítanunk.

%% A Budapest School Modellben azt mutatjuk be, amit az elmúlt években a
%% tanulásról megtanultunk. Annak az útnak a tapasztalatait összegezzük,
%% amely során az álomból az első csoportjaink létrejöttek, majd újabbak és
%% újabbak alakultak. Ahogy nőtt a Budapest School, úgy kezdtük egyre
%% jobban megérteni, mit is csinálunk és miként kell ezt a tanárok, a
%% családok, a gyerekek, valamint a társadalom és a jogalkotó elvárásaihoz
%% igazítanunk.

%% Az elmúlt években azon dolgoztunk, hogy magántanulók közösségéből
%% államilag elfogadott iskolává válhasson az, amit mi Budapest Schoolnak
%% hívunk. Arra a kérdésre kerestük a választ, hogy létrehozhatunk-e egy
%% személyre szabott, önvezérelt tanulási modellt úgy, hogy az mindenben
%% megfeleljen a törvényi előírásoknak. Ezért írtunk először egy
%% \emph{kerettantervet}, ami arra a kérdésre válaszol, hogy mit tanulnak a
%% gyerekek. Amikor ezzel elkészültünk, leírtuk azt is, hogyan tanulnak, és
%% mire a kérdés minden részlete kibomlott előttünk, elkészült a
%% \emph{pedagógiai programunk} is. Menet közben ráébredtünk, hogy a mit és
%% a hogyan kérdése közötti határok jóval elmosódottabbak, mint azt elsőre
%% gondolnánk. Egy Budapest School Modell alapján működő iskolába vagy
%% osztályba járó gyerekek ugyanis éppannyira dönthetnek a tartalomról,
%% mint amennyire formálói lehetnek a mindennapok működésének is.

%% Így jött létre a Budapest School Modell, ami azt írja le, hogy a mai
%% tudásunk szerint hogyan szervezünk tanulási környezeteket. Ugyanazt a
%% modellt használhatjuk írni, olvasni, számolni tanuló kisiskolások
%% mindennapjainak szervezéséhez (alsó tagozat), egy mély szakmai
%% képességre fókuszáló középiskola megszervezéséhez (technikum) vagy egy
%% általános\break
%% gimnázium működtetéséhez.

%% A BPS modell a tanulás szervezését, azaz a pedagógiáját a motiváció
%% {\autocite{Pink2011}}, a fejlődési szemlélet {\autocite{Dweck2006}} és
%% az elmélyült gyakorlás {\autocite{Ericsson2016}} pszichológiai kutatási
%% eredményei és a modern tanulásszervezési paradigmák, mint az önvezérelt
%% {\autocite{Mitra2012}} és a személyre szabott {\autocite{Khan2012}}
%% tanulás alapján határozza meg.

%% Az iskola a gyerekek számára egy önvezérelt és személyre szabott
%% tanulási környezetet biztosít. Az igényekre agilisan reagál, és közben
%% stabil, biztonságos és kiszámítható rendszerként biztosítja az iskola és
%% más iskolák közötti átjárhatóságot, a továbbtanulás lehetőségét, a
%% jogszabályok betartását és az iskola elszámoltathatóságát.

%% \hypertarget{bps-modell-tobb-iskolaban}{%
%% \subsection{BPS Modell több iskolában}\label{bps-modell-tobb-iskolaban}}

%% A BPS modell egy iskola működési rendszerét írja le. Rendszer, mert
%% definiálja az iskola szereplőit, azok interakcióját és
%% együttműködésüket. Mindeközben amit ebben a dokumentumban leírunk, csak
%% egy modell, mert minden iskolában kicsit másként működnek majd a dolgok,
%% hiszen emberek a modellt a saját céljaik szerint alakíthatják.

%% BPS modell alapján több különböző iskola tud működni. A BPS modell
%% ugyanúgy érvényes tud lenni egy általános iskolában, egy gimnáziumban,
%% egy technikumban, vagy akárcsak egy iskola kísérleti osztályában.
% :LK

Így jött létre a Budapest School Modell, ami azt írja le, hogy a mai
tudásunk szerint hogyan szervezünk tanulási környezeteket. Ugyanazt a
modellt használhatjuk írni, olvasni, számolni tanuló kisiskolások
mindennapjainak szervezéséhez (alsó tagozat), egy mély szakmai
képességre fókuszáló középiskola megszervezéséhez (technikum) vagy egy
általános\break
gimnázium működtetéséhez.

A BPS modell a tanulás szervezését, azaz a pedagógiáját a motiváció
{\autocite{Pink2011}}, a fejlődési szemlélet {\autocite{Dweck2006}} és
az elmélyült gyakorlás {\autocite{Ericsson2016}} pszichológiai kutatási
eredményei és a modern tanulásszervezési paradigmák, mint az önvezérelt
{\autocite{Mitra2012}} és a személyre szabott {\autocite{Khan2012}}
tanulás alapján határozza meg.

Az iskola a gyerekek számára egy önvezérelt és személyre szabott
tanulási környezetet biztosít. Az igényekre agilisan reagál, és közben
stabil, biztonságos és kiszámítható rendszerként biztosítja az iskola és
más iskolák közötti átjárhatóságot, a továbbtanulás lehetőségét, a
jogszabályok betartását és az iskola elszámoltathatóságát.

A BPS modell egy iskola működési rendszerét írja le. Rendszer, mert
definiálja az iskola szereplőit, azok interakcióját és
együttműködésüket. Mindeközben amit ebben a dokumentumban leírunk, csak
egy modell, mert minden iskolában kicsit másként működnek majd a dolgok,
hiszen emberek a modellt a saját céljaik szerint alakíthatják.

BPS modell alapján több különböző iskola tud működni. A BPS modell
ugyanúgy érvényes tud lenni egy általános iskolában, egy gimnáziumban,
egy technikumban, vagy akárcsak egy iskola kísérleti osztályában.
