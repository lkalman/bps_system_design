\hypertarget{a-szemelyre-szabhato-kozossegi-iskola}{%
\section{A személyre szabható közösségi
iskola}\label{a-szemelyre-szabhato-kozossegi-iskola}}

Az iskola több helyszínen működik. A gyerekek az iskolában kisebb
közösségekbe tartoznak, a
tanulóközösségekbe (\ref{a-budapest-school-tanulokozossegei}.~fejezet, \pageref{a-budapest-school-tanulokozossegei}.~oldal).
Ezek a közösségek legfeljebb hat évfolyamot átölelő, 6-60 fős tanulási
közösségekként működnek. A tanulóközösségek se jogi, se szervezeti
szempontból nem önálló iskolák, nem összevont osztályok. A
tanulóközösség nem más, mint az iskola egyik szervezeti egysége, ami
önálló identitással, sajátos szubkultúrával, szabályokkal rendelkezhet.

A kisközösségükön belül a gyerekek többfajta csoportbontásban tanulnak,
és idősebb korban (10-12 év felett) különböző osztályok, közösségek
tagjai is összeállhatnak egy-egy foglalkozásra, kurzusra, de
identitásban, ,,egymáshoztartozás-érzésben'' a tanulóközösség az
elsődleges közösségük. A közösség tagjai együtt és egymástól tanulnak.

Az egyes tanulóközösségeket
tanulásszervező tanárok (tanulásszervezők, \ref{tanulasszervezo}.~fejezet, \pageref{tanulasszervezo}.~oldal)
csapata vezeti. Minden gyereknek van egy kitüntetett tanára, a
mentora (\ref{mentor}.~fejezet, \pageref{mentor}.~oldal), aki egyéni
figyelmével a fejlődésben segíti. Minden gyerek a mentortanára
segítségével és a szülők aktív részvételével trimeszterenként
meghatározza a
saját tanulási céljait (\ref{sajat-tanulasi-celok}.~fejezet, \pageref{sajat-tanulasi-celok}.~oldal).

A tanulásszervezők
modulokat (\ref{tanulasi-tanitasi-egysegek-a-modulok}.~fejezet, \pageref{tanulasi-tanitasi-egysegek-a-modulok}.~oldal)
hirdetnek ezen célokból és a tantárgyak tanulási eredmények alapján. A
modulok reflektálnak a mai világ alapvető kérdéseire, integrálják a
tudományterületeket és művészeti ágakat, azaz a tantárgyakat, és egyenlő
lehetőséget adnak a tudásszerzésre, az önálló gondolkodásra és az
alkotásra a gyerekek mindennapjaiban.

A modulok végeztével a gyerekek eredményei bekerülnek saját
portfóliójukba (\ref{portfolio}.~fejezet, \pageref{portfolio}.~oldal),
ami tartalmazhat önálló vagy csoportos alkotásokat, tudáspróbákat,
vizsgafeladatokat, egymás felé történő visszajelzéseket, a fejlődést jól
mérő dokumentációkat vagy bármit, amire a gyerek és tanárai büszkék, vagy
amit fontosnak tartanak. Erre a portfólióra épül a Budapest School
visszajelző és értékelő rendszere (\ref{visszajelzes-ertekeles}.~fejezet, \pageref{visszajelzes-ertekeles}.~oldal).

A gyerekek mindennapjait meghatározó modulok több műveltségi területet,
többféle kompetenciát, több tantárgy anyagát is lefedhetik, és egy
tantárgy anyagát több modul is érintheti. Ezért is mondhatjuk, hogy a
BPS-iskolákban a tantárgyközi tevékenységek vannak előtérben. Az iskola
szándéka, hogy a gyerekek folyamatosan fejlődjenek a világ tudományos
megismerésében (STEM), a saját és mások kulturális közegéhez való
kapcsolódásban (KULT), valamint a testi-lelki egyensúlyuk fenntartásában
(Harmónia), vagyis a
kiemelt tantárgyközi fejlesztési területeken (\ref{kiemelt-fejlesztesi-teruletek}.~fejezet, \pageref{kiemelt-fejlesztesi-teruletek}.~oldal).

Az iskola szerint az a tanárok döntése, hogy a gyerekek kémiaórán
kísérleteznek-e, vagy kísérletezésórán foglalkoznak kémiával. A iskola\break
annyit határoz meg, hogy a 7--10. évfolyamszinten kémia tantárgyhoz
kapcsolódóan 73 különböző tanulási eredményt kell elérni,
kísérletezéssel\break
kapcsolatban pedig 21 különböző tanulási eredményt több
különböző tantárgyból (ezekből csak 8 kapcsolódik a kémia tantárgyhoz).

Tehát a tantárgyak a tanulás tartalmi elemeinek forrásai és keretei: a
tanulandó dolgok halmazaként működnek. Az, hogy milyen csoportosításban
történik a tanulás, a szaktanárokra van bízva. A gyerekek lehet, hogy
csak félévente, az elszámolás időszakában találkoznak a tantárgyak
taxonómiájával. Ebben az időszakban veti össze minden gyerek és mentor,
hogy amit tanultak, alkottak, és amiben fejlődtek, az hogyan viszonyul a
társadalom és a törvények elvárásaihoz, a Nemzeti Alaptantervhez.

Az iskola a NAT-tantárgyak témaköreit, tartalmát és követelményeit
\emph{tanulási eredmények} halmazaként adja meg. A gyerekek feladata az
iskolában, hogy tanulási eredményeket érjenek el, és így sajátítsák el a
tantárgyak által szabott követelményeket. Tanulási eredményeket modulok
elvégzésével (is) el lehet érni, tehát a modulok elsődleges feladata,
hogy a tanulási eredményekhez vezető utat mutassák.

Az iskolában egyszerre jelennek meg a NAT tantárgyi elvárásai, a
közoktatást szabályozó törvények szándékai, a gyerekek saját céljai és a
mai világra való integrált reflexió.

\hypertarget{a-tanulas-rendszerszemleletu-megkozelitese}{%
\subsection{A tanulás rendszerszemléletű
megközelítése}\label{a-tanulas-rendszerszemleletu-megkozelitese}}

Az oktatás tartalmának előzetes szabályozása helyett a Budapest School a
tanulás módjára helyezi a hangsúlyt. Az iskola alapelve, hogy integratív
módon folyamatosan keresse és fejlessze a pedagógiai, pszichológiai és
szervezetfejlesztési módszereket, amelyek korszerű módon tudják segíteni
a tanulás tanulását, az egyéni és csoportos fejlődést, a konfliktusok
feloldását.

A tanulás tartalmát tekintve a Budapest School a NAT tartalmára
támaszkodik. A Budapest School Modell pedig a tanulás rendszerét, annak
folyamatát szabályozza. E dokumentum alapján a NAT tanulási eredményein
történő végighaladás mellett a Budapest School nagy hangsúlyt fektet a
gyerekek saját tanulási céljaira és a célállítás módjára.
