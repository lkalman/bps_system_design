\hypertarget{a-tanev-ritmusa}{%
\section{A tanév ritmusa}\label{a-tanev-ritmusa}}

A tanév több ciklus ismétlődésével írható le. A személyre szabott, saját
célok által irányított tanulást a trimeszterek ciklusa határozza meg, az
iskolarendszerben megszokott értékeléseket és bizonyítványokat a féléves
ciklus határozza meg. Miért vezeti be az iskola a trimesztereket? Hogy
tanévenként kétszer szánjunk minőségi időt arra, hogy a gyerekek
ránézzenek a tanulási útjaikra, ha kell, tervezzenek újra, és kapjanak új lendületet.


\hypertarget{trimeszterek}{%
\subsection{Trimeszterek}\label{trimeszterek}}

A tanév tanulási útját három szakaszra bontja az iskola, ezeket három
mérföldkő zárja le. Ezeket az időszakokat hívjuk \emph{trimesztereknek}.
Egy trimeszteren belül a tanulási célok tervezése után következik a
tanulás, és a ciklust a visszajelzés és értékelés zárja. Amint egy
ciklus véget ér, elkezdődik egy új.

Ez a felosztás követi az üzleti világ negyedéves tervezését, néhány
egyetem trimeszterekre bontását, de leginkább az évszakokat. Minden
periódus után értékeljük az elmúlt három hónapot, ünnepeljük az
eredményeket, és megtervezzük a következő időszakot.

Az egyes trimeszterek átlagosan 12 hétig tartanak úgy, hogy a tanítással
eltöltött napokat és a tanítási szüneteket mindig az állami oktatásirányítás
által kiadott tanév rendjéhez igazodva határozzuk meg. A trimeszterek
első hete mindig a tervezéssel, utolsó hete mindig az értékeléssel
telik. Trimeszterenként átlagosan további egy hét a közösség építésével,
önálló tanulással zajlik.

A trimesztereken belül az egyes tanulóközösségek között lehetnek néhány
hetes eltérések, melyek a közösség sajátosságait követik.

A ciklusok állandósága adja a tanulás irányításához szükséges kereteket.
Ezek megtartásáért az egyes tanulóközösségek tanulásszervezői felelnek,
működésüket a fenntartó monitorozza. A tanév ritmusát, a
trimeszteralapú tervezés és a félévalapú iskolarendszer-szintű
visszajelzés illeszkedését a következő táblázat mutatja.

\newcolumntype{L}{>{\begin{minipage}[t]{.2\textwidth}\strut\raggedright\hangindent=1em}l<{\strut\end{minipage}}}
\newcolumntype{M}{>{\begin{minipage}[t]{.6\textwidth}\strut\raggedright\hangindent=1em}l<{\strut\end{minipage}}}
\begin{longtable}[]{LM}
\renewcommand{\arraystretch}{1.5}
% \toprule
% \hline
\textbf{Időszak} & \textbf{Tevékenység}\tabularnewline
% \midrule
\hline
\endhead
szeptember & {közösségépítés\\
saját célok meghatározása\\
modulok kialakítása és meghirdetése}\tabularnewline
\hline
október & tanulás, alkotás\tabularnewline
\hline
november & tanulás, alkotás\tabularnewline
\hline
december & {portfólió frissítése\\
reflexiók\\
visszajelzések\\
célok felülvizsgálata\\
modulok változtatása igény esetén}\tabularnewline
\hline
január & {tanulás, alkotás\\
féléves értékelés kiadása}\tabularnewline
\hline
február & tanulás, alkotás\tabularnewline
\hline
március & {portfólió frissítése\\
reflexiók\\
visszajelzések\\
célok felülvizsgálata\\
modulok változtatása igény esetén}\tabularnewline
\hline
április & tanulás, alkotás\tabularnewline
\hline
május & tanulás, alkotás\tabularnewline
\hline
fél június & évzárás, értékelés, bizonyítványok\tabularnewline
% \bottomrule
\end{longtable}

%% \newcolumntype{L}{|>{\begin{minipage}[t]{.3\textwidth}\strut\raggedright}l<{\strut\end{minipage}}}
%% \newcolumntype{M}{|>{\begin{minipage}[t]{.4\textwidth}\strut\raggedright}l<{\strut\end{minipage}}|}
%% \begin{longtable}[]{LM}
%% % \toprule
%% \hline
%% \textbf{Időszak} & \textbf{Tevékenység}\tabularnewline
%% % \midrule
%% \hline
%% \endhead
%% szeptember & {közösségépítés\\
%% saját célok meghatározása\\
%% modulok kialakítása és meghirdetése}\tabularnewline
%% \hline
%% október & tanulás, alkotás\tabularnewline
%% \hline
%% november & tanulás, alkotás\tabularnewline
%% \hline
%% december & {portfólió frissítése\\
%% reflexiók\\
%% visszajelzések\\
%% célok felülvizsgálata\\
%% modulok változtatása igény esetén}\tabularnewline
%% \hline
%% január & {tanulás, alkotás\\
%% féléves értékelés kiadása}\tabularnewline
%% \hline
%% február & tanulás, alkotás\tabularnewline
%% \hline
%% március & {portfólió frissítése\\
%% reflexiók\\
%% visszajelzések\\
%% célok felülvizsgálata\\
%% modulok változtatása igény esetén}\tabularnewline
%% \hline
%% április & tanulás, alkotás\tabularnewline
%% \hline
%% május & tanulás, alkotás\tabularnewline
%% \hline
%% fél június & évzárás, értékelés, bizonyítványok\tabularnewline
%% \bottomrule
%% \end{longtable}

%% \newcolumntype{L}{>{\begin{minipage}[t]{.3\textwidth}\strut\raggedright}l<{\strut\end{minipage}}}
%% \newcolumntype{M}{>{\begin{minipage}[t]{.4\textwidth}\strut\raggedright}l<{\strut\end{minipage}}}
%% \begin{longtable}[]{LM}
%% % \toprule
%% \textbf{Időszak} & \textbf{Tevékenység}\tabularnewline
%% \midrule
%% \midrule
%% \endhead
%% szeptember & {közösségépítés\\
%% saját célok meghatározása\\
%% modulok kialakítása és meghirdetése}\tabularnewline
%% \hline
%% október & tanulás, alkotás\tabularnewline
%% \hline
%% november & tanulás, alkotás\tabularnewline
%% \hline
%% december & {portfólió frissítése\\
%% reflexiók\\
%% visszajelzések\\
%% célok felülvizsgálata\\
%% modulok változtatása igény esetén}\tabularnewline
%% \hline
%% január & {tanulás, alkotás\\
%% féléves értékelés kiadása}\tabularnewline
%% \hline
%% február & tanulás, alkotás\tabularnewline
%% \hline
%% március & {portfólió frissítése\\
%% reflexiók\\
%% visszajelzések\\
%% célok felülvizsgálata\\
%% modulok változtatása igény esetén}\tabularnewline
%% \hline
%% április & tanulás, alkotás\tabularnewline
%% \hline
%% május & tanulás, alkotás\tabularnewline
%% \hline
%% fél június & évzárás, értékelés, bizonyítványok\tabularnewline
%% \bottomrule
%% \end{longtable}

%% \begin{longtable}[]{@{}ll@{}}
%% \toprule
%% Időszak & Tevékenység\tabularnewline
%% \midrule
%% \endhead
%% szeptember & közösségépítés\tabularnewline
%% & saját célok meghatározása\tabularnewline
%% & modulok kialakítása és meghirdetése\tabularnewline
%% október & tanulás, alkotás\tabularnewline
%% november & tanulás, alkotás\tabularnewline
%% december & portfólió frissítése\tabularnewline
%% & reflexiók\tabularnewline
%% & visszajelzések\tabularnewline
%% & célok felülvizsgálata\tabularnewline
%% & modulok változtatása igény esetén\tabularnewline
%% január & tanulás, alkotás\tabularnewline
%% & féléves értékelés kiadása\tabularnewline
%% február & tanulás, alkotás\tabularnewline
%% március & portfólió frissítése\tabularnewline
%% & reflexiók\tabularnewline
%% & visszajelzések\tabularnewline
%% & célok felülvizsgálata\tabularnewline
%% & modulok változtatása igény esetén\tabularnewline
%% április & tanulás, alkotás\tabularnewline
%% május & tanulás, alkotás\tabularnewline
%% fél június & évzárás, értékelés, bizonyítványok\tabularnewline
%% \bottomrule
%% \end{longtable}

\hypertarget{feleves-es-evvegi-ertekeles}{%
\subsection{Féléves és évvégi
értékelés}\label{feleves-es-evvegi-ertekeles}}

A külső rendszereknek és a törvényeknek való megfelelés miatt az iskola
(az ötödik évfolyamtól) két félévre bontva határozza meg a
tananyagtartalmakat, és ezzel összhangban az iskola automatikusan félévenkénti értékelést
állít ki a portfólió alapján, ami nem más, mint a
portfólióba bekerült értékelések összegyűjtése.

A félév vége január vége (amit egy miniszteri rendelet évente határoz
meg), ami a második trimeszterbe esik, ezért a féléves értékelést az
első trimeszter végén rögzített állapot szerint adja ki az iskola. Az
5--12. évfolyamon lévő gyerekeknek januárban módjuk és lehetőségük van a
portfóliójuk frissítésére, ha a féléves értékelés és az osztályzatokra
váltás eredménye számukra iskolaváltás, továbbtanulás miatt vagy egyéb okból
fontos. Ilyenkor januárban alkalmuk van javítani, ha 
továbbtanulás miatt szükségük van osztályzatokra. Részletesebben az
évfolyamokról, osztályzatokról, bizonyítványokról szóló fejezet (\ref{evfolyam-osztalyzatok-bizonyitvany}.~fejezet, \pageref{evfolyam-osztalyzatok-bizonyitvany}.~oldal)
határozza meg a folyamatot.

Év végén, június hónap fele marad az évvégi zárásokra és igény szerint az
osztályzatokra váltásra való felkészülésre, ami a portfólió frissítését,
bővítését, kiegészítését jelenti.
