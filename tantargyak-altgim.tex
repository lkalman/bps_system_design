\hypertarget{tantargyak-tartalma-a-tanulasi-eredmenyek}{%
\section{Tantárgyak tartalma, a tanulási
eredmények}\label{tantargyak-tartalma-a-tanulasi-eredmenyek}}

\hypertarget{allampolgari-ismeretek}{%
\subsection{Állampolgári ismeretek}\label{allampolgari-ismeretek}}

\newsavebox{\evfbox}
\savebox{\evfbox}{\bfseries\ 12. évf.\ }
\newdimen{\evflength}
\settowidth{\evflength}{\usebox{\evfbox}}
\newdimen{\columnlength}
\setlength{\columnlength}{\textwidth}
\addtolength{\columnlength}{-\evflength}

\begin{longtable}[]{p{\evflength}@{\strut}>{\begin{minipage}{\columnlength}\strut}l<{\strut\end{minipage}}}
  \bfseries 8. évf. & \bfseries Állampolgári ismeretek\endhead
  \hline
&
  Megérti a családnak mint a társadalom alapvető intézményének szerepét,
  és értelmezi jellemzőit.
\tabularnewline
\hline
&
  Lokálpatriotizmusa megerősödik, személyiségébe beépülnek a\break nemzeti
  közösséghez tartozás, a hazaszeretet emocionális össze-\break tevői.
\tabularnewline
\hline
&
  Ismeri a demokratikus jogállam működésének alapvető sajátosságait,
  alapvető kötelezettségeit.
\tabularnewline
\hline
&
  Jártasságot szerez mindennapi ügyeinek intézésében.
\tabularnewline
\hline
&
  Saját pénzügyeiben tudatos döntéseket hoz.
\tabularnewline
\hline
&
  Értelmezi a családnak mint a társadalom alapvető intézményének szerepét
  és jellemzőit.
\tabularnewline
\hline
&
  Értelmezi a családi kohézió alapelemeit, jellemzőit: együttműködés,
  szeretetközösség, kölcsönösség, tisztelet.
\tabularnewline
\hline
&
  Felismeri a családi szocializációnak az ember életútját befolyásoló
  jelentőségét.
\tabularnewline
\hline
&
  Ismeri a magyar állam alapvető intézményeinek feladatkörét és
  működését.
\tabularnewline
\hline
&
  Értelmezi a törvényalkotás folyamatát.
\tabularnewline
\hline
&
  Ismeri a saját településének, lakóhelyének alapvető jellemzőit,
  értelmezi a településen működő intézmények és szervezetek szerepét és
  működését.
\tabularnewline
\hline
&
  A lakóhelyével kapcsolatos javaslatokat fogalmaz meg, tervet készít a
  település fejlesztésének lehetőségeiről.
\tabularnewline
\hline
&
  Felismeri a jogok és kötelességek közötti egyensúly kialakításának és
  fenntartásának fontosságát, megismeri a haza iránti kötelezettségeit,
  feladatait.
\tabularnewline
\hline
&
  Ismeri településének, lakóhelyének kulturális, néprajzi értékeit, a
  település történetének alapvető eseményeit és fordulópontjait.
\tabularnewline
\hline
&
  Megfogalmazza a nemzeti identitás jelentőségét az egyén és a közösség
  szempontjából is.
\tabularnewline
\hline
&
  Felismeri a nemzetek, nemzetállamok fontosságát a globális vi-\break
  lágban.
\tabularnewline
\hline
&
  Megismeri és értelmezi a honvédelem jelentőségét, feladatait és
  szerepét.
\tabularnewline
\hline
&
  Azonosítja a mindennapi ügyintézés alapintézményeit, az alapvető
  ellátó rendszerek funkcióját és működési sajátosságait.
\tabularnewline
\hline
&
  Azonosítja az igazságszolgáltatás intézményeit és működésük
  jellemzőit, megismeri az alapvető ellátórendszereket és funkcióikat.
\tabularnewline
\hline
&
  Megismeri és értelmezi a diákmunka alapvető jogi feltételeit,
  kereteit.
\tabularnewline
\hline
&
  Információkat gyűjt és értelmez a foglalkoztatási helyzetről, a
  szakmaszerkezet változásairól.
\tabularnewline
\hline
&
  Ismeri a családi háztartás összetevőit, értelmezi a család
  gazdálkodását meghatározó és befolyásoló tényezőket.
\tabularnewline
\hline
&
  Felismeri a családi háztartás gazdasági-pénzügyi fenntarthatóságának
  és a környezettudatos életvitel kialakításának társadalmi
  jelentőségét.
\tabularnewline
\hline
&
  Értelmezi az állam gazdasági szerepvállalásának területeit.
\tabularnewline
\hline
&
  Felismeri a közteherviselés gazdasági, társadalmi és erkölcsi
  jelentőségét, a társadalmi felelősségvállalás fontosságát.
\tabularnewline
\hline
&
  Fogyasztási szokásaiban érvényesíti a tudatosság szempontjait is.
\tabularnewline
\hline
&
  Felismeri a véleménynyilvánítás, az érvelés, a párbeszéd és a vita
  társadalmi hasznosságát.
\tabularnewline
\hline
&
  Képes arra, hogy feladatai egy részét a társas tanulás keretében
  végezze el.
\tabularnewline
\hline
&
  Önállóan vagy társaival együttműködve javaslatokat fogalmaz meg,
  tervet, tervezetet készít.
\tabularnewline
\hline
&
  Önállóan vagy segítséggel használja az infokommunikációs esz-\break közöket.
\tabularnewline
\hline
\end{longtable}

% \hypertarget{evfolyamon}{%
% \subsubsection{8. évfolyamon}\label{evfolyamon}}

% \hypertarget{evfolyamon-1}{%
% \subsubsection{12. évfolyamon}\label{evfolyamon-1}}

\begin{longtable}[]{p{\evflength}@{\strut}>{\begin{minipage}{\columnlength}\strut}l<{\strut\end{minipage}}}
  \bfseries 12. évf. & \bfseries Állampolgári ismeretek\endhead
  \hline
&
  Megérti a család szerepét, alapvető feladatait az egyén és a nemzet
  szempontjából egyaránt.
\tabularnewline
\hline
&
  Értékeli a nemzeti identitás jelentőségét az egyén és a közösség
  szempontjából is.
\tabularnewline
\hline
&
  Ismeri a választások alapelveit és a törvényhozás folyamatát.
\tabularnewline
\hline
&
  Megismeri a demokratikus jogállam működésének alapvető sa-\break játosságait.
\tabularnewline
\hline
&
  Érti és vallja a haza védelmének, a nemzetért történő tenni akarás
  fontosságát.
\tabularnewline
\hline
&
  A mindennapi életének megszervezésében alkalmazza a jogi és
  gazdasági-pénzügyi ismereteit.
\tabularnewline
\hline
&
  Saját pénzügyeiben tudatos döntéseket hoz.
\tabularnewline
\hline
&
  Felismeri az életpálya-tervezés és a munkavállalás egyéni és
  társadalmi jelentőségét.
\tabularnewline
\hline
&
  Ismeri a munka világát érintő alapvető jogi szabályozást, a
  munkaerőpiac jellemzőit, tájékozódik a foglalkoztatás és a
  szakmaszerkezet változásairól.
\tabularnewline
\hline
&
  Értelmezi a családnak mint a társadalom alapvető intézményének szerepét
  és jellemzőit.
\tabularnewline
\hline
&
  Társaival megbeszéli a párválasztás, a családtervezés fontos
  szakaszait, szempontjait és a gyermekvállalás demográfiai
  jelentőségét: tájékozódás, minták, orientáló példák, átgondolt
  tervezés, felelősség.
\tabularnewline
\hline
&
  Felismeri, hogy a családtagok milyen szerepet töltenek be a
  szocializáció folyamatában.
\tabularnewline
\hline
&
  Értelmezi a családi szocializációnak az ember életútját befolyásoló
  jelentőségét.
\tabularnewline
\hline
&
  Felismeri az alapvető emberi jogok egyetemes és társadalmi
  je-\break lentőségét.
\tabularnewline
\hline
&
  Bemutatja Magyarország alaptörvényének legfontosabb részeit:
  alapvetés; az állam; szabadság és felelősség.
\tabularnewline
\hline
&
  Érti a társadalmi normák és az egyéni cselekedetek, akaratok, célok
  egyeztetésének, összehangolásának követelményét. Elméleti és
  tapasztalati úton ismereteket szerez a társadalmi
  felelősségvállalásról, a segítségre szorulók támogatásának
  lehetőségeiről.
\tabularnewline
\hline
&
  Megérti a honvédelem szerepét az ország biztonságának fenntartásában,
  megismeri a haza védelmének legfontosabb feladatcsoportjait és
  területeit, az egyén kötelezettségeit.
\tabularnewline
\hline
&
  Felismeri és értelmezi az igazságosság, az esélyegyenlőség
  biztosításának jelentőségét és követelményeit.
\tabularnewline
\hline
&
  Értelmezi a választójog feltételeit és a választások alapelveit.
\tabularnewline
\hline
&
  Értelmezi a törvényalkotás folyamatát.
\tabularnewline
\hline
&
  Megérti a nemzeti érzület sajátosságait és a hazafiság fontosságát,
  lehetséges megnyilvánulási formáit.
\tabularnewline
\hline
&
  Véleményt alkot a nemzetállamok és a globalizáció összefüg-\break géseiről.
\tabularnewline
\hline
&
  Felismeri a világ magyarsága mint nemzeti közösség összetartozásának
  jelentőségét.
\tabularnewline
\hline
&
  Érti és felismeri a honvédelem mint nemzeti ügy jelentőségét.
\tabularnewline
\hline
&
  Felismeri és értékeli a helyi, regionális és országos közgyűjteményeknek
  a nemzeti kulturális örökség megőrzésében betöltött szerepét.
\tabularnewline
\hline
&
  Azonosítja a mindennapi ügyintézés alapintézményeit.
\tabularnewline
\hline
&
  Életkori sajátosságainak megfelelően jártasságot szerez a jog
  területének mindennapi életben való alkalmazásában.
\tabularnewline
\hline
&
  Tájékozott a munkavállalással kapcsolatos szabályokban.
\tabularnewline
\hline
&
  Megtervezi egy fiktív család költségvetését.
\tabularnewline
\hline
&
  Saját pénzügyi döntéseit körültekintően, megalapozottan hozza meg.
\tabularnewline
\hline
&
  Megismeri a megalapozott, körültekintő hitelfelvétel szempontjait,
  illetve feltételeit.
\tabularnewline
\hline
&
  Azonosítja az állam gazdasági szerepvállalásának elemeit.
\tabularnewline
\hline
&
  Felismeri és megérti a közteherviselés nemzetgazdasági, társadalmi és
  morális jelentőségét.
\tabularnewline
\hline
&
  Életvitelébe beépülnek a tudatos fogyasztás elemei, életmódjában
  figyelmet fordít a környezeti terhelés csökkentésére, érvényesíti a
  fogyasztóvédelmi szempontokat.
\tabularnewline
\hline
&
  Értelmezi a vállalkozás indítását befolyásoló tényezőket.
\tabularnewline
\hline
&
  Felismeri a véleménynyilvánítás, az érvelés, a párbeszéd és a vita
  társadalmi hasznosságát.
\tabularnewline
\hline
&
  Képes arra, hogy feladatait akár önálló, akár társas tanulás révén
  végezze el, célorientáltan képes az együttműködésre.
\tabularnewline
\hline
&
  Önállóan vagy társaival együttműködve javaslatokat fogalmaz meg.
\tabularnewline
\hline
&
  Tiszteletben tartja a másik ember értékvilágát, gondolatait és
  véleményét, ha szükséges, kritikusan viszonyul emberi cselekedetekhez,
  magatartásformákhoz.
\tabularnewline
\hline
&
  Megismerkedik a tudatos médiafogyasztói magatartással és a közösségi
  média használatával.
\tabularnewline
\hline
&
  A tanulási tevékenységek szakaszaiban használja az infokommunikációs
  eszközöket, lehetőségeket, tisztában van azok szerepével, innovációs
  potenciáljával és veszélyeivel is.
\tabularnewline
\hline
\end{longtable}

\hypertarget{biologia}{%
\subsection{Biológia}\label{biologia}}

% \hypertarget{evfolyamon-2}{%
% \subsubsection{9-10. évfolyamon}\label{evfolyamon-2}}

\savebox{\evfbox}{\bfseries\ 9--10. évf.\ }
\settowidth{\evflength}{\usebox{\evfbox}}
\setlength{\columnlength}{\textwidth}
\addtolength{\columnlength}{-\evflength}
\begin{longtable}[]{p{\evflength}@{\strut}>{\begin{minipage}{\columnlength}\strut}l<{\strut\end{minipage}}}
  \bfseries 9--10. évf. & \bfseries Biológia\endhead
  \hline
&
  A mindennapi élettel összefüggő problémák megoldásában alkalmazza a
  természettudományos gondolkodás műveleteit, rendelkezik a biológiai
  problémák vizsgálatához szükséges gyakorlati készségekkel.
\tabularnewline
\hline
&
  Az élő rendszerek belső működésének és a környezettel való
  kapcsolataiknak az
  elemzésében alkalmazza a rendszerszintű gondolkodás műveleteit.
\tabularnewline
\hline
&
  Életközösségek vizsgálata alapján értelmezi a környezet és az
  élőlények felépítése és működése közötti összefüggést, érti az
  ökológiai egyensúly jelentőségét, érvel a biológiai sokféleség
  megőrzése mellett.
\tabularnewline
\hline
&
  Az emberi test és pszichikum felépítéséről és működéséről\break szerzett
  ismereteit alkalmazza önismeretének fejlesztésében, egészséges életvitelének
  kialakításában.
\tabularnewline
\hline
&
  Felismeri a helyi és a globális környezeti problémák összefüggését,
  érvel a Föld és a Kárpát-medence természeti értékeinek védelme
  mellett, döntéseket hoz és cselekszik a fenntarthatóság érdekében.
\tabularnewline
\hline
&
  Ismeri a biológiai kutatások alapvető céljait, legfontosabb
  területeit, értékeli az élet megértésében, az élővilág megismerésében
  és megóvásában játszott szerepét.
\tabularnewline
\hline
&
  Példákkal igazolja a biológiai ismereteknek a világképünk és a
  technológia fejlődésében betöltött szerepét, gazdasági és társadalmi
  jelentőségét.
\tabularnewline
\hline
&
  Az élő rendszerek vizsgálata során felismeri az analógiákat,
  korrelációkat, alkalmazza a statisztikus és a rendszerszintű
  gondolkodás műveleteit, kritikusan és kreatívan mérlegeli a
  lehetőségeket, bizonyítékokra alapozva érvel, több szempontot is
  figyelembe vesz.
\tabularnewline
\hline
&
  A vizsgált biológiai jelenségek magyarázatára feltevést fogalmaz
  meg, ennek bizonyítására vagy cáfolatára kísérletet tervez és
  kivitelez, azonosítja és beállítja a kísérleti változókat,
  megfigyeléseket és méréseket végez.
\tabularnewline
\hline
&
  Biológiai vonatkozású adatokat elemez, megfelelő formába rendez,
  ábrázol, ezek alapján előrejelzéseket, következtetéseket fogalmaz meg,
  a már ábrázolt adatokat értelmezi.
\tabularnewline
\hline
&
  Egyénileg és másokkal együttműködve célszerűen és biztonságosan
  alkalmaz biológiai vizsgálati módszereket, ismeri a fénymikroszkóp
  működésének alapelvét, képes azt használni.
\tabularnewline
\hline
&
  Érti a biológia molekuláris szintű vizsgálati módszereinek elméleti
  alapjait és felhasználási lehetőségeit, értékeli a biológiai
  kutatásokból származó nagy mennyiségű adat feldolgozásának
  jelentőségét.
\tabularnewline
\hline
&
  A biológiai jelenségek vizsgálata során digitális szöveget, képet,
  videót keres, értelmez és felhasznál, vizsgálja azok megbízhatóságát,
  jogszerű és etikus felhasználhatóságát.
\tabularnewline
\hline
&
  Biológiai vizsgálatok során elvégzi az adatrögzítés és -rendezés
  műveleteit, ennek alapján tényekkel alátámasztott következtetéseket
  von le.
\tabularnewline
\hline
&
  Ismeri a tudományos közlések lényegi jellemzőit.
\tabularnewline
\hline
&
  Tájékozódik a biotechnológia és a bioetika kérdéseiben, ezekről folyó
  vitákban tudományosan megalapozott érveket alkot.
\tabularnewline
\hline
&
  A valós és virtuális tanulási közösségekben, másokkal együttműködve
  megtervez és kivitelez biológiai vizsgálatokat, pro-\break jekteket.
\tabularnewline
\hline
&
  Tudja a biológiai problémákat és magyarázatokat a megfelelő szinttel
  összefüggésben értelmezni.
\tabularnewline
\hline
&
  Tényekkel bizonyítja az élőlények elemi összetételének hasonlóságát, a
  biogén elemek, a víz, az ATP és a makromolekulák élő szervezetekben
  betöltött alapvető szerepét.
\tabularnewline
\hline
&
  Megérti, miért és hogyan mehetnek végbe viszonylag alacsony
  hőmérsékleten, nagy sebességgel kémiai reakciók a sejtekben, vizsgálja
  az enzimműködést befolyásoló tényezőket.
\tabularnewline
\hline
&
  Értékeli és példákkal igazolja a különféle szintű biológiai
  szabályozás szerepét az élő rendszerek normál működési állapotának
  fenntartásában.
\tabularnewline
\hline
&
  Magyarázza, hogy a sejt az élő szervezetek szerkezeti és működési
  egysége.
\tabularnewline
\hline
&
  Ábrák, animációk alapján értelmezi, és biológiai tényekkel
  alátámasztja, hogy a vírusok az élő és élettelen határán állnak.
\tabularnewline
\hline
&
  A felépítés és működés összehasonlítása alapján bemutatja a sejtes
  szerveződés kétféle formájának közös jellemzőit és alapvető
  különbségeit, értékeli ezek jelentőségét.
\tabularnewline
\hline
&
  Tényekkel igazolja a baktériumok anyagcseréjének sokfélesége, a gyors
  szaporodásuk és alkalmazkodóképességük közötti összefüggést.
\tabularnewline
\hline
&
  Felismeri az összetett sejttípus mikroszkóppal megfigyelhető
  sejtalkotóit, magyarázza a sejt anyagcsere-fo\-lya\-ma\-ta\-i\-nak
  lé-\break nyegét.
\tabularnewline
\hline
&
  Ismeri az örökítőanyag többszintű szerveződését, képek, animációk
  alapján értelmezi a sejtekben zajló biológiai információ tárolásának,
  átírásának és kifejeződésének folyamatait.
\tabularnewline
\hline
&
  Tudja, hogy a sejtekben és a sejtek között bonyolult jelforgalmi
  hálózatok működnek, amelyek befolyásolják a génműködést, és felelősek
  lehetnek a normál és a kóros működésért is.
\tabularnewline
\hline
&
  Összehasonlítja a sejtosztódás típusait, megfogalmazza ezek biológiai
  szerepét, megérti, hogy a soksejtű szervezetek a megtermékenyített
  petesejt és utódsejtjei meghatározott számú osztódásával és
  differenciálódásával alakulnak ki.
\tabularnewline
\hline
&
  Felismeri az összefüggést a rák kialakulása és a sejtciklus zavarai
  között, megérti, hogy mit tesz a sejt és a szervezet a daganatok
  kialakulásának megelőzéséért.
\tabularnewline
\hline
&
  Ismeri az őssejt fogalmát, különféle típusait és azok jellemzőit,
  különbséget tesz őssejt és daganatsejt között.
\tabularnewline
\hline
&
  Fénymikroszkópban, ábrán vagy fotón felismeri és jellemzi a főbb
  állati és növényi szövettípusokat, elemzi, hogy milyen funkciók
  hatékony elvégzésére specializálódtak.
\tabularnewline
\hline
&
  Ismeri és példákkal bizonyítja az élőlények szén- és
  energiaforrásainak különféle lehetőségeit, az anyagcseretípusok
  közötti különbséget.
\tabularnewline
\hline
&
  Vázlatrajzok, folyamatábrák és animációk alapján azonosítja a
  fotoszintézis és a sejtlégzés fő szakaszainak sejten belüli helyét és
  struktúráit, a fontosabb anyagokat és az energiaátalakítás jellemzőit.
\tabularnewline
\hline
&
  A sejtszintű anyagcserefolyamatok alapján magyarázza a növények és
  az állatok közötti ökológiai szintű kapcsolatot, a termelő és fogyasztó
  szervezetek közötti anyagforgalmat.
\tabularnewline
\hline
&
  A földi élet keletkezését biológiai kísérletek és elméletek alapján
  magyarázza.
\tabularnewline
\hline
&
  Érti és tényekkel igazolja az ősbaktériumok különleges élőhelyeken
  való életképességét.
\tabularnewline
\hline
&
  Biológiai és csillagászati tények alapján mérlegeli a Földön kívüli
  élet valószínűsíthető feltételeit és lehetőségeit.
\tabularnewline
\hline
&
  Ismeri az örökítőanyag bázissorrendjének vagy bázisainak
  megváltozásához vezető folyamatokat, konkrét esetekben azonosítja ezek
  következményeit.
\tabularnewline
\hline
&
  A géntechnológia céljának és módszertani alapjainak ismeretében,
  kritikai szemlélettel elemzi a genetikai módosítások előnyeit és
  kockázatait.
\tabularnewline
\hline
&
  Érti az örökítőanyagban tárolt információ és a kifejeződő
  tulajdonságok közötti összefüggést, megkülönbözteti a genotípust és a
  fenotípust.
\tabularnewline
\hline
&
  Megérti a genetikai információ nemzedékek közötti átadásának
  törvényszerűségeit, ezeket konkrét esetek elemzésében alkalmazza.
\tabularnewline
\hline
&
  Felismeri a kapcsolatot az életmód és a gének kifejeződése között,
  érti, hogy a sejt és az egész szervezet jellemzőinek kialakításában és
  fenntartásában kiemelt szerepe van a környezet általi
  génaktivitás-változásoknak.
\tabularnewline
\hline
&
  Megérti a természetes változatosság szerveződését, az evolúciós
  változások eredetét és elterjedését magyarázó elemi folyamatokat,
  felismer és magyaráz mikro- és makroszintű evolúciós jelenségeket.
\tabularnewline
\hline
&
  Példákkal igazolja, hogy a szelekció a különböző szerveződési
  szinteken értelmezhető tulajdonságokon keresztül egyidejűleg hat.
\tabularnewline
\hline
&
  Példákkal mutatja be az élővilág főbb csoportjainak evolúciós
  újításait, magyarázza, hogy ezek hogyan segítették elő az adott
  élőlénycsoport elterjedését.
\tabularnewline
\hline
&
  Érti és elfogadja, hogy a mai emberek egy fajhoz tartoznak, és az
  evolúció során kialakult nagyrasszok értékükben nem különböznek, a
  biológiai és kulturális örökségük az emberiség közös kincse.
\tabularnewline
\hline
&
  Morfológiai, molekuláris biológiai adatok alapján egyszerű
  származástani kapcsolatokat elemez, törzsfát készít.
\tabularnewline
\hline
&
  Ismeri az evolúció befolyásolásának lehetséges módjait (például
  mesterséges szelekció, fajtanemesítés, géntechnológia), értékeli ezek
  előnyeit és esetleges hátrányait.
\tabularnewline
\hline
&
  Megérti a környezeti állapot és az ember egészsége közötti
  összefüggéseket, azonosítja az ember egészségét veszélyeztető
  tényezőket, felismeri a megelőzés lehetőségeit, érvényesíti az
  elővigyázatosság elvét.
\tabularnewline
\hline
&
  Elemzi az ember mozgásképességének biokémiai, szövettani és
  biomechanikai alapjait, ezeket összefüggésbe hozza a mindennapi élet,
  a sport és a munka mozgásformáival, értékeli a rendszeres testmozgás
  szerepét egészségének megőrzésében.
\tabularnewline
\hline
&
  Az emberi test kültakarójának, váz- és izomrendszerének elemzése
  alapján magyarázza az ember testképének, testalkatának és
  mozgásképességének biológiai alapjait.
\tabularnewline
\hline
&
  A táplálkozás, a légzés, a keringés és a kiválasztás
  szervrendszerének elemzése alapján magyarázza az emberi szervezet
  anyag- és energiaforgalmi működésének biológiai alapjait.
\tabularnewline
\hline
&
  Az ideg-, hormon- és immunrendszer elemzése alapján magyarázza az
  emberi szervezet információs rendszerének biológiai alapjait.
\tabularnewline
\hline
&
  Felsorolja az emberi egyedfejlődés főbb szakaszait, magyarázza hogyan
  és miért változik a szervezetünk az életkor előrehaladásával, értékeli
  a fejlődési szakaszok egészségvédelmi szempontjait, önmagát is
  elhelyezve ebben a rendszerben.
\tabularnewline
\hline
&
  Ismeri a férfi és a női nemi szervek felépítését és működését, a
  másodlagos nemi jellegeket és azok kialakulási folyamatát, ismereteit
  összekapcsolja a szaporító szervrendszer egészségtanával.
\tabularnewline
\hline
&
  Biológiai ismereteit is figyelembe véve értékeli az emberi
  szexualitásnak a 
  párkapcsolattal és a tudatos családtervezéssel összefüggő
  jelentőségét.
\tabularnewline
\hline
&
  Megérti a fogamzásgátlók hatékonyságáról szóló információkat, a
  személyre szabott, orvosilag ellenőrzött fogamzásgátlás fontosságát.
\tabularnewline
\hline
&
  Ismeri a fogamzás feltételeit, a terhesség jeleit, bemutatja a magzat
  fejlődésének szakaszait, értékeli a terhesség alatti egészséges
  életmód jelentőségét.
\tabularnewline
\hline
&
  A biológiai működések alapján magyarázza a stressz fogalmát, felismeri
  a tartós stressz egészségre gyakorolt káros hatásait, igyekszik azt
  elkerülni, csökkenteni.
\tabularnewline
\hline
&
  Ismeri a gondolkodási folyamatokat és az érzelmi és motivációs
  működéseket meghatározó tényezőket, értékeli az érzelmi és az értelmi
  fejlődés kapcsolatát.
\tabularnewline
\hline
&
  Ismeri a mentális egészség jellemzőit, megérti annak feltételeit, ezek
  alapján megtervezi az egészségmegőrző magatartásához szükséges
  életviteli elemeket.
\tabularnewline
\hline
&
  Megérti az idegsejtek közötti jelátviteli folyamatokat, és azokat kapcsolatba
  hozza a tanulás és emlékezés folyamataival, a drogok
  hatásmechanizmusával.
\tabularnewline
\hline
&
  Az agy felépítése és funkciója alapján magyarázza az információk
  feldolgozásával, a tanulással összefüggő folyamatokat, értékeli a
  tanulási képesség jelentőségét az egyén és a közösség szempontjából.
\tabularnewline
\hline
&
  Biológiai folyamatok alapján magyarázza a függőség kialakulását,
  felismeri a függőségekhez vezető tényezőket, ezek kockázatait és
  következményeit.
\tabularnewline
\hline
&
  Ismeri az orvosi diagnosztika, a szűrővizsgálatok és védőoltások
  célját, lényegét, értékeli ezek szerepét a betegségek megelőzésében és
  a gyógyulásban.
\tabularnewline
\hline
&
  Megkülönbözteti a házi- és a szakorvosi ellátás funkcióit, ismeri az
  orvoshoz fordulás módját, tisztában van a kórházi ellátás indokaival,
  jellemzőivel.
\tabularnewline
\hline
&
  Ismeri a leggyakoribb fertőző betegségek kiváltó okait, ismeri a
  fertőzések elkerülésének lehetőségeit és a járványok elleni védekezés
  módjait.
\tabularnewline
\hline
&
  Ismeri a leggyakoribb népbetegségek (pl. szívinfarktus, stroke,
  cukorbetegség, allergia, asztma) kockázati tényezőit, felismeri ezek
  kezdeti tüneteit.
\tabularnewline
\hline
&
  Képes a bekövetkezett balesetet, rosszullétet felismerni.
\tabularnewline
\hline
&
  Képes a sérült vagy beteg személy ellátását a rendelkezésre álló
  eszközökkel (vagy eszköz nélkül) megkezdeni, segítséget (szükség
  esetén mentőt) hívni.
\tabularnewline
\hline
&
  Szükség esetén alkalmazza a felnőtt alapszintű újraélesztés műveleteit
  (CPR), képes a félautomata defibrillátort alkalmazni.
\tabularnewline
\hline
&
  Példákkal mutatja be a fontosabb hazai szárazföldi és vizes
  életközösségek típusait, azok jellemzőit és előfordulásait.
\tabularnewline
\hline
&
  Megfigyelések, leírások és videók alapján azonosítja a populációk
  közötti kölcsönhatások típusait, az ezzel összefüggő etológiai
  jellemzőket, bemutatja ezek jellegét, jelentőségét.
\tabularnewline
\hline
&
  Érti az ökológiai mutatókkal, bioindikációs vizsgálatokkal megvalósuló
  környezeti állapotelemzések céljait, adott esetben alkalmazza azok
  módszereit.
\tabularnewline
\hline
&
  Ismeri a levegő-, a víz- és a talajszennyezés forrásait, a szennyező
  anyagok típusait és példáit, konkrét esetek alapján elemzi az
  életközösségekre gyakorolt hatásukat.
\tabularnewline
\hline
&
  Felismeri és példákkal igazolja az állatok viselkedésének a
  környezethez való alkalmazkodásban játszott szerepét.
\tabularnewline
\hline
&
  Felismeri a természetes élőhelyeket veszélyeztető tényezőket, kifejti
  álláspontját az élőhelyvédelem szükségességéről, egyéni és társadalmi
  megvalósításának lehetőségeiről.
\tabularnewline
\hline
&
  Érti a biológiai sokféleség fogalmát, ismer a meghatározásra alkalmas
  módszereket, értékeli a bioszféra stabilitásának megőrzésében játszott
  szerepét.
\tabularnewline
\hline
&
  Érti az ökológiai egyensúly fogalmát, értékeli a jelentőségét,
  példákkal igazolja az egyensúly felborulásának lehetséges
  következményeit.
\tabularnewline
\hline
&
  Érti az ökológiai rendszerek működése és a biológiai sokféleség
  közötti kapcsolatot, konkrét életközösségek vizsgálata alapján
  táplálkozási piramist, hálózatot elemez.
\tabularnewline
\hline
&
  Konkrét példák alapján vizsgálja a bioszférában végbemenő
  folyamatokat, elemzi ezek idő- és térbeli viszonyait, azonosítja az
  emberi tevékenységgel való összefüggésüket.
\tabularnewline
\hline
&
  A kutatások adatai és előrejelzései alapján értelmezi a globális
  éghajlatváltozás élővilágra gyakorolt helyi és bioszféra szintű
  következményeit.
\tabularnewline
\hline
&
  Példák alapján elemzi a levegő-, a víz- és a talajszennyeződés, az
  ipari és természeti katasztrófák okait és ezek következményeit, az
  emberi tevékenységnek az élőhelyek változásához vezető hatását, ennek
  alapján magyarázza egyes fajok veszélyeztetettségét.
\tabularnewline
\hline
&
  Érti és elfogadja, hogy a jövőbeli folyamatokat a jelen cselekvései
  alakítják, tudja, hogy a folyamatok tervezése, előrejelzése
  számítógépes modellek alapján lehetséges.
\tabularnewline
\hline
&
  Értékeli a környezet- és természetvédelem fontosságát, megérti a
  nemzetközi összefogások és a hazai törekvések jelentőségét,
  döntéshozatalai során saját személyes érdekein túl a természeti
  értékeket és az egészségmegőrzési szempontokat is mérlegeli.
\tabularnewline
\hline
&
  Történeti adatok és jelenkori esettanulmányok alapján értékeli a
  mezőgazdaság, az erdő- és vadgazdaság, valamint a halászat természetes
  életközösségekre gyakorolt hatását, példák alapján bemutatja az
  ökológiai szempontú, fenntartható gazdálkodás technológiai
  lehetőségeit.
\tabularnewline
\hline
&
  Megérti a biotechnológiai eljárások és a bionika eredményeinek
  alkalmazási lehetőségeit, értékeli az információs technológiák
  alkalmazásának orvosi, biológiai jelentőségét.
\tabularnewline
\hline
&
  Érvel a Föld mint élő bolygó egyedisége mellett, tényekre alapozottan
  és kritikusan értékeli a természeti okokból és az emberi hatásokra
  bekövetkező változásokat.
\tabularnewline
\hline
&
  Ismeri a Kárpát-medence élővilágának sajátosságait, megőrzendő
  értékeit, ezeket összekapcsolja a hazai nemzeti parkok
  tevékenységével.
\tabularnewline
\hline
\end{longtable}

\hypertarget{digitalis-kultura}{%
\subsection{Digitális kultúra}\label{digitalis-kultura}}

% \hypertarget{evfolyamon-3}{%
% \subsubsection{4. évfolyamon}\label{evfolyamon-3}}

\savebox{\evfbox}{\bfseries\ 4. évf.\ }
\settowidth{\evflength}{\usebox{\evfbox}}
\setlength{\columnlength}{\textwidth}
\addtolength{\columnlength}{-\evflength}

\begin{longtable}[]{p{\evflength}@{\strut}>{\begin{minipage}{\columnlength}\strut}l<{\strut\end{minipage}}}
  \bfseries 4. évf. & \bfseries Digitális kultúra\endhead
  \hline
&
  Elmélyülten dolgozik digitális környezetben, önellenőrzést végez.
\tabularnewline
\hline
&
  Megvizsgálja és értékeli az általa vagy társai által alkalmazott,
  létrehozott, megvalósított eljárásokat.
\tabularnewline
\hline
&
  Társaival együttműködve online és offline környezetben egyaránt megold
  különböző feladatokat, ötleteit, véleményét megfogalmazza, részt vesz
  a közös álláspont kialakításában.
\tabularnewline
\hline
&
  Kiválasztja az általa ismert informatikai eszközök és alkalmazások
  közül azokat, melyek az adott probléma megoldásához szükségesek.
\tabularnewline
\hline
&
  Eredményétől függően módosítja a problémamegoldás folyamatában az
  adott, egyszerű tevékenységsorokat.
\tabularnewline
\hline
&
  A rendelkezésére álló eszközökkel, forrásokból meggyőződik a talált
  vagy kapott információk helyességéről.
\tabularnewline
\hline
&
  Közvetlen otthoni vagy iskolai környezetéből megnevez néhány
  informatikai eszközt, felsorolja fontosabb jellemzőit.
\tabularnewline
\hline
&
  Megfogalmazza, néhány példával alátámasztja, hogyan könnyíti meg a
  felhasználó munkáját az adott eszköz alkalmazása.
\tabularnewline
\hline
&
  Egyszerű feladatokat old meg informatikai eszközökkel. esetenként
  tanítói segítséggel összetett funkciókat is alkalmaz.
\tabularnewline
\hline
&
  Önállóan vagy tanítói segítséggel választ más tantárgyak tanulásának
  támogatásához applikációkat, digitális tananyagot, oktatójátékot,
  képességfejlesztő digitális alkalmazást.
\tabularnewline
\hline
&
  Kezdetben tanítói segítséggel, majd önállóan használ néhány
  életkorának megfelelő alkalmazást, elsősorban információgyűjtés,
  gyakorlás, egyéni érdeklődésének kielégítése céljából.
\tabularnewline
\hline
&
  A feladathoz, problémához digitális eszközt, illetve alkalmazást,
  applikációt, felhasználói felületet választ; felsorol néhány érvet
  választásával kapcsolatosan.
\tabularnewline
\hline
&
  Adott szempontok alapján megfigyel néhány grafikai alkalmazással
  készített produktumot; személyes véleményét megfogalmazza.
\tabularnewline
\hline
&
  Grafikai alkalmazással egyszerű, közvetlenül hasznosuló rajzot,
  grafikát, dokumentumot hoz létre.
\tabularnewline
\hline
&
  Egy rajzos dokumentumot adott szempontok alapján értékel, módosít.
\tabularnewline
\hline
&
  Állításokat fogalmaz meg grafikonokról, infografikákról,
  táblázatokról; a kapott információkat felhasználja napi tevékenysége
  során.
\tabularnewline
\hline
&
  Információkat keres, a talált adatokat felhasználja digitális
  produktumok létrehozására.
\tabularnewline
\hline
&
  Értelmezi a problémát, a megoldási lehetőségeket eljátssza,
  megfogalmazza, egyszerű eszközök segítségével megvalósítja.
\tabularnewline
\hline
&
  Információt keres az interneten más tantárgyak tanulása során, és
  felhasználja azt.
\tabularnewline
\hline
&
  Egyszerű prezentációt, ábrát, egyéb segédletet készít.
\tabularnewline
\hline
&
  Felismer, eljátszik, végrehajt néhány hétköznapi tevékenysége során
  tapasztalt, elemi lépésekből álló, adott sorrendben végrehajtandó
  cselekvést.
\tabularnewline
\hline
&
  Egy adott, a mindennapi életből vett algoritmust elemi lépésekre bont,
  értelmezi a lépések sorrendjét, megfogalmazza az algoritmus várható
  kimenetelét.
\tabularnewline
\hline
&
  Feladat, probléma megoldásához többféle algoritmust próbál ki.
\tabularnewline
\hline
&
  A valódi vagy szimulált programozható eszköz mozgását értékeli, hiba
  esetén módosítja a kódsorozatot a kívánt eredmény eléréséig.
  tapasztalatait megfogalmazza, megvitatja társaival.
\tabularnewline
\hline
&
  Adott feltételeknek megfelelő kódsorozatot tervez és hajtat végre,
  történeteket, meserészleteket jelenít meg padlórobottal vagy más
  eszközzel.
\tabularnewline
\hline
&
  Alkalmaz néhány megadott algoritmust tevékenység, játék során, és
  néhány egyszerű esetben módosítja azokat.
\tabularnewline
\hline
&
  Információkat keres az interneten, egyszerű eljárásokkal meggyőződik
  néhány az interneten talált információ igazságértékéről.
\tabularnewline
\hline
&
  Kiválasztja a számára releváns információt, felismeri a hamis
  információt.
\tabularnewline
\hline
&
  Tisztában van a személyes adat fogalmával, törekszik megőrzésére,
  ismer néhány példát az e-világ veszélyeivel kapcsolatban.
\tabularnewline
\hline
&
  Ismeri és használja a kapcsolattartás formáit és a kommunikáció
  lehetőségeit a digitális környezetben.
\tabularnewline
\hline
&
  Ismeri a mobileszközök alkalmazásának előnyeit, korlátait, etikai
  vonatkozásait.
\tabularnewline
\hline
&
  Közvetlen tapasztalatokat szerez a digitális eszközök használatával
  kapcsolatban.
\tabularnewline
\hline
&
  Képes feladat, probléma megoldásához megfelelő applikáció, digitális
  tananyag, oktatójáték, képességfejlesztő digitális alkalmazás
  kiválasztására.
\tabularnewline
\hline
&
  Ismer néhány kisiskolások részére készített portált,
  információforrást, digitálistananyag-lelőhelyet.
\tabularnewline
\hline
\end{longtable}

% \hypertarget{evfolyamon-4}{%
% \subsubsection{5--8. évfolyamon}\label{evfolyamon-4}}

\savebox{\evfbox}{\bfseries\ 5--8. évf.\ }
\settowidth{\evflength}{\usebox{\evfbox}}
\setlength{\columnlength}{\textwidth}
\addtolength{\columnlength}{-\evflength}

\begin{longtable}[]{p{\evflength}@{\strut}>{\begin{minipage}{\columnlength}\strut}l<{\strut\end{minipage}}}
  \bfseries 5--8. évf. & \bfseries Digitális kultúra\endhead
  \hline
&
  Önállóan használja a digitális eszközöket, az online kommunikáció
  eszközeit, tisztában van az ezzel járó veszélyekkel.
\tabularnewline
\hline
&
  Elsajátítja a digitális írástudás eszközeit, azokkal feladatokat old
  meg.
\tabularnewline
\hline
&
  Megismeri a felmerülő problémák megoldásának módjait, beleértve az
  adott feladat megoldásához szükséges algoritmus értelmezését,
  alkotását és számítógépes program készítését és kódolását a
  blokkprogramozás eszközeivel.
\tabularnewline
\hline
&
  Digitális tudáselemek felhasználásával, társaival együttműködve
  különböző problémákat old meg.
\tabularnewline
\hline
&
  Megismeri a digitális társadalom elvárásait, lehetőségeit és
  ve-\break szélyeit.
\tabularnewline
\hline
&
  Célszerűen választ a feladat megoldásához használható informatikai
  eszközök közül.
\tabularnewline
\hline
&
  Az informatikai eszközöket önállóan használja, a tipikus felhasználói
  hibákat elkerüli, és elhárítja az egyszerűbb felhasználói szintű
  hibákat.
\tabularnewline
\hline
&
  Értelmezi az informatikai eszközöket működtető szoftverek
  hibajelzéseit, és azokról beszámol.
\tabularnewline
\hline
&
  Önállóan használja az operációs rendszer felhasználói felületét.
\tabularnewline
\hline
&
  Önállóan kezeli az operációs rendszer mappáit, fájljait és a
  felhőszolgáltatásokat.
\tabularnewline
\hline
&
  Használja a digitális hálózatok alapszolgáltatásait.
\tabularnewline
\hline
&
  Tapasztalatokkal rendelkezik a digitális jelek minőségével,
  kódolásával, tömörítésével, továbbításával kapcsolatos problémák
  kezeléséről.
\tabularnewline
\hline
&
  Egy adott feladat kapcsán önállóan hoz létre szöveges vagy multimédiás
  dokumentumokat.
\tabularnewline
\hline
&
  Ismeri és tudatosan alkalmazza a szöveges és multimédiás dokumentum
  készítése során a szöveg formázására, tipográfiájára vonatkozó
  alapelveket.
\tabularnewline
\hline
&
  A tartalomnak megfelelően alakítja ki a szöveges vagy a multimédiás
  dokumentum szerkezetét, illeszti be, helyezi el és formázza meg a
  szükséges objektumokat.
\tabularnewline
\hline
&
  Ismeri és kritikusan használja a nyelvi eszközöket (például
  he-\break lyesírás-ellenőrzés, elválasztás).
\tabularnewline
\hline
&
  A szöveges dokumentumokat többféle elrendezésben jeleníti meg papíron,
  tisztában van a nyomtatás környezetre gyakorolt hatásaival.
\tabularnewline
\hline
&
  Ismeri a prezentációkészítés alapszabályait, és azokat alkal-\break mazza.
\tabularnewline
\hline
&
  Etikus módon használja fel az információforrásokat, tisztában van a
  hivatkozás szabályaival.
\tabularnewline
\hline
&
  Digitális eszközökkel önállóan rögzít és tárol képet, hangot és
  videót.
\tabularnewline
\hline
&
  Digitális képeken képkorrekciót hajt végre.
\tabularnewline
\hline
&
  Ismeri egy bittérképes rajzolóprogram használatát, azzal ábrát készít.
\tabularnewline
\hline
&
  Bemutatókészítő vagy szövegszerkesztő programban rajzeszközökkel
  ábrát készít.
\tabularnewline
\hline
&
  Érti, hogyan történik az egyszerű algoritmusok végrehajtása a
  digitális eszközökön.
\tabularnewline
\hline
&
  Megkülönbözteti, kezeli és használja az elemi adatokat.
\tabularnewline
\hline
&
  Értelmezi az algoritmus végrehajtásához szükséges adatok és az
  eredmények kapcsolatát.
\tabularnewline
\hline
&
  Egyszerű algoritmusokat elemez és készít.
\tabularnewline
\hline
&
  Ismeri a kódolás eszközeit.
\tabularnewline
\hline
&
  Adatokat kezel a programozás eszközeivel.
\tabularnewline
\hline
&
  Ismeri és használja a programozási környezet alapvető eszközeit.
\tabularnewline
\hline
&
  Ismeri és használja a blokkprogramozás alapvető építőelemeit.
\tabularnewline
\hline
&
  A probléma megoldásához vezérlési szerkezetet (szekvencia, elágazás és
  ciklus) alkalmaz a tanult blokkprogramozási nyelven.
\tabularnewline
\hline
&
  Az adatokat táblázatos formába rendezi és formázza.
\tabularnewline
\hline
&
  Cellahivatkozásokat, matematikai tudásának megfelelő képleteket,
  egyszerű statisztikai függvényeket használ táblázatkezelő programban.
\tabularnewline
\hline
&
  Az adatok szemléltetéséhez diagramot készít.
\tabularnewline
\hline
&
  Problémákat old meg táblázatkezelő program segítségével.
\tabularnewline
\hline
&
  Tapasztalatokkal rendelkezik hétköznapi jelenségek számítógépes
  szimulációjáról.
\tabularnewline
\hline
&
  Vizsgálni tudja a szabályozó eszközök hatásait a tantárgyi
  alkalmazásokban.
\tabularnewline
\hline
&
  Ismeri az információkeresés technikáját, stratégiáját és több keresési
  szempont egyidejű érvényesítésének lehetőségét.
\tabularnewline
\hline
&
  Önállóan keres információt, a találatokat hatékonyan szűri.
\tabularnewline
\hline
&
  Az internetes adatbáziskezelő rendszerek keresési űrlapját helyesen
  tölti ki.
\tabularnewline
\hline
&
  Ismeri, használja az elektronikus kommunikáció lehetőségeit, a családi
  és az iskolai környezetének elektronikus szolgáltatásait.
\tabularnewline
\hline
&
  Ismeri és betartja az elektronikus kommunikációs szabályokat.
\tabularnewline
\hline
&
  Mozgásokat vezérel szimulált vagy valós környezetben.
\tabularnewline
\hline
&
  Adatokat gyűjt szenzorok segítségével.
\tabularnewline
\hline
&
  Tapasztalatokkal rendelkezik az eseményvezérlésről.
\tabularnewline
\hline
&
  Ismeri a térinformatika és a 3d megjelenítés lehetőségeit.
\tabularnewline
\hline
&
  Tapasztalatokkal rendelkezik az iskolai oktatáshoz kapcsolódó
  mobileszközökre fejlesztett alkalmazások használatában.
\tabularnewline
\hline
&
  Tisztában van a hálózatokat és a személyes információkat érintő
  fenyegetésekkel, alkalmazza az adatok védelmét biztosító
  le-\break hetőségeket.
\tabularnewline
\hline
&
  Védekezik az internetes zaklatás különböző formái ellen, szükség
  esetén segítséget kér.
\tabularnewline
\hline
&
  Ismeri a digitális környezet, az e-világ etikai problémáit.
\tabularnewline
\hline
&
  Ismeri az információs technológia fejlődésének gazdasági, környezeti,
  kulturális hatásait.
\tabularnewline
\hline
&
  Ismeri az információs társadalom múltját, jelenét és várható jövőjét.
\tabularnewline
\hline
&
  Online gyakorolja az állampolgári jogokat és kötelességeket.
\tabularnewline
\hline
\end{longtable}

% \hypertarget{evfolyamon-5}{%
% \subsubsection{9-12. évfolyamon}\label{evfolyamon-5}}

\savebox{\evfbox}{\bfseries\ 9--12. évf.\ }
\settowidth{\evflength}{\usebox{\evfbox}}
\setlength{\columnlength}{\textwidth}
\addtolength{\columnlength}{-\evflength}

\begin{longtable}[]{p{\evflength}@{\strut}>{\begin{minipage}{\columnlength}\strut}l<{\strut\end{minipage}}}
  \bfseries 9--12. évf. & \bfseries Digitális kultúra\endhead
  \hline
&
  Ismeri az informatikai eszközök és a működtető szoftvereik célszerű
  választásának alapelveit, használja a digitális hálózatok
  alapszolgáltatásait, az online kommunikáció eszközeit, tisztában van
  az ezzel járó veszélyekkel, ezzel összefüggésben ismeri a
  segítségnyújtási, segítségkérési lehetőségeket.
\tabularnewline
\hline
&
  Gyakorlatot szerez dokumentumok létrehozását segítő eszközök
  használatában.
\tabularnewline
\hline
&
  Megismeri az adatkezelés alapfogalmait, képes a nagyobb adatmennyiség
  tárolását, hatékony feldolgozását biztosító eszközök és módszerek
  alapszintű használatára, érti a működésüket.
\tabularnewline
\hline
&
  Megismeri az algoritmikus probléma megoldásához szükséges módszereket
  és eszközöket, megoldásukhoz egy magas szintű formális programozási
  nyelv fejlesztői környezetét önállóan használja.
\tabularnewline
\hline
&
  Hatékonyan keres információt; az IKT-tudáselemek felhasználásával
  társaival együttműködve problémákat old meg.
\tabularnewline
\hline
&
  Ismeri az e-világ elvárásait, lehetőségeit és veszélyeit.
\tabularnewline
\hline
&
  Ismeri és tudja használni a célszerűen választott informatikai
  eszközöket és a működtető szoftvereit, ismeri a felhasználási
  lehetőségeket.
\tabularnewline
\hline
&
  Ismeri a digitális eszközök és a számítógépek fő egységeit, ezek
  fejlődésének főbb állomásait, tendenciáit.
\tabularnewline
\hline
&
  Tudatosan alakítja informatikai környezetét, ismeri az ergonomikus
  informatikai környezet jellemzőit, figyelembe veszi a digitális
  eszközök egészségkárosító hatásait, óvja maga és környezete
  egészségét.
\tabularnewline
\hline
&
  Önállóan használja az informatikai eszközöket, elkerüli a tipikus
  felhasználói hibákat, elhárítja az egyszerűbb felhasználói hibákat.
\tabularnewline
\hline
&
  Céljainak megfelelően használja a mobileszközök és a számítógépek
  operációs rendszereit.
\tabularnewline
\hline
&
  Igénybe veszi az operációs rendszer és a számítógépes hálózat
  alapszolgáltatásait.
\tabularnewline
\hline
&
  Követi a technológiai változásokat a digitális információforrások
  használatával.
\tabularnewline
\hline
&
  Használja az operációs rendszer segédprogramjait, és elvégzi a
  munkakörnyezet beállításait.
\tabularnewline
\hline
&
  Tisztában van a digitális kártevők elleni védekezés lehetősé-\break geivel.
\tabularnewline
\hline
&
  Használja az állományok tömörítését és a tömörített állományok
  kibontását.
\tabularnewline
\hline
&
  Ismeri egy adott feladat megoldásához szükséges digitális eszközök és
  szoftverek kiválasztásának szempontjait.
\tabularnewline
\hline
&
  Speciális dokumentumokat hoz létre, alakít át és formáz meg.
\tabularnewline
\hline
&
  Tapasztalatokkal rendelkezik a formanyomtatványok, a sablonok, az
  előre definiált stílusok használatáról.
\tabularnewline
\hline
&
  Gyakorlatot szerez a fotó-, hang-, videó-, multimédia-szer\-kesz\-tő, a
  bemutatókészítő eszközök használatában.
\tabularnewline
\hline
&
  Alkalmazza az információkeresés során gyűjtött multimédiás
  alapelemeket új dokumentumok készítéséhez.
\tabularnewline
\hline
&
  Dokumentumokat szerkeszt és helyez el tartalomkezelő rend-\break szerben.
\tabularnewline
\hline
&
  Ismeri a HTML formátumú dokumentumok szerkezeti elemeit, érti a CSS
  használatának alapelveit; több lapból álló webhelyet készít.
\tabularnewline
\hline
&
  Létrehozza az adott probléma megoldásához szükséges rasztergrafikus
  ábrákat.
\tabularnewline
\hline
&
  Létrehoz vektorgrafikus ábrákat.
\tabularnewline
\hline
&
  Digitálisan rögzít képet, hangot és videót, azokat manipulálja.
\tabularnewline
\hline
&
  Tisztában van a raszter- és a vektorgrafikus ábrák tárolási és
  szerkesztési módszereivel
\tabularnewline
\hline
&
  Érti az egyszerű problémák megoldásához szükséges tevékenységek
  lépéseit és kapcsolatukat.
\tabularnewline
\hline
&
  Ismeri a következő elemi adattípusok közötti különbségeket: egész,
  valós szám, karakter, szöveg, logikai.
\tabularnewline
\hline
&
  Ismeri az elemi és összetett adattípusok közötti különbségeket.
\tabularnewline
\hline
&
  Érti egy algoritmus-leíró eszköz alapvető építőelemeit, érti a
  típusalgoritmusok felhasználásának lehetőségeit.
\tabularnewline
\hline
&
  Példákban, feladatok megoldásában használja egy formális programozási
  nyelv fejlesztői környezetének alapszolgálta-\break tásait.
\tabularnewline
\hline
&
  Szekvencia, elágazás és ciklus segítségével algoritmust hoz létre, és
  azt egy magas szintű formális programozási nyelven kó-\break dolja.
\tabularnewline
\hline
&
  A feladat megoldásának helyességét teszteli.
\tabularnewline
\hline
&
  Adatokat táblázatba rendez.
\tabularnewline
\hline
&
  Táblázatkezelővel adatelemzést és számításokat végez.
\tabularnewline
\hline
&
  A problémamegoldás során függvényeket célszerűen használ.
\tabularnewline
\hline
&
  Nagy adathalmazokat tud kezelni.
\tabularnewline
\hline
&
  Az adatokat diagramon szemlélteti.
\tabularnewline
\hline
&
  Ismeri az adatbáziskezelés alapfogalmait.
\tabularnewline
\hline
&
  Az adatbázisban interaktív módon keres, rendez és szűr.
\tabularnewline
\hline
&
  A feladatmegoldás során az adatbázisba adatokat visz be, módosít és
  töröl, űrlapokat használ, jelentéseket nyomtat.
\tabularnewline
\hline
&
  Strukturáltan tárolt nagy adathalmazokat kezel, azokból egyedi és
  összesített adatokat nyer ki.
\tabularnewline
\hline
&
  Tapasztalatokkal rendelkezik hétköznapi jelenségek számítógépes
  szimulációjáról.
\tabularnewline
\hline
&
  Hétköznapi, oktatáshoz készült szimulációs programokat használ.
\tabularnewline
\hline
&
  Tapasztalatokat szerez a kezdőértékek változtatásának hatásairól a
  szimulációs programokban.
\tabularnewline
\hline
&
  Ismeri és alkalmazza az információkeresési stratégiákat és
  technikákat, a találati listát a problémának megfelelően szűri,
  ellenőrzi annak hitelességét.
\tabularnewline
\hline
&
  Etikus módon használja fel az információforrásokat, tisztában van a
  hivatkozás szabályaival.
\tabularnewline
\hline
&
  Használja a két- vagy többrésztvevős kommunikációs lehetőségeket és
  alkalmazásokat.
\tabularnewline
\hline
&
  Ismeri és alkalmazza a fogyatékkal élők közötti kommunikáció eszközeit
  és formáit.
\tabularnewline
\hline
&
  Az online kommunikáció során alkalmazza a kialakult viselkedési
  kultúrát és szokásokat, a szerepelvárásokat.
\tabularnewline
\hline
&
  Ismeri és használja a mobiltechnológiát, kezeli a mobileszközök
  operációs rendszereit, és mobilalkalmazásokat használ.
\tabularnewline
\hline
&
  Céljainak megfelelő alkalmazást választ, az alkalmazás funkcióira,
  kezelőfelületére vonatkozó igényeit megfogalmazza.
\tabularnewline
\hline
&
  Az applikációkat önállóan telepíti.
\tabularnewline
\hline
&
  Az iskolai oktatáshoz kapcsolódó mobileszközökre fejlesztett
  alkalmazások használata során együttműködik társaival.
\tabularnewline
\hline
&
  Tisztában van az e-világ ---~e-szolgáltatások, e-ügyintézés,
  e-kereskedelem, e-állampolgárság, it-gazdaság, környezet, kultúra,
  információvédelem~--- biztonsági és jogi kérdéseivel.
\tabularnewline
\hline
&
  Tisztában van a digitális személyazonosság és az információhitelesség
  fogalmával.
\tabularnewline
\hline
&
  A gyakorlatban alkalmazza az adatok védelmét biztosító lehetőségeket.
\tabularnewline
\hline
\end{longtable}

\hypertarget{drama-es-szinhaz}{%
\subsection{Dráma és színház}\label{drama-es-szinhaz}}

% \hypertarget{evfolyamon-6}{%
% \subsubsection{7-8. évfolyamon}\label{evfolyamon-6}}

\savebox{\evfbox}{\bfseries\ 7--8. évf.\ }
\settowidth{\evflength}{\usebox{\evfbox}}
\setlength{\columnlength}{\textwidth}
\addtolength{\columnlength}{-\evflength}

\begin{longtable}[]{p{\evflength}@{\strut}>{\begin{minipage}{\columnlength}\strut}l<{\strut\end{minipage}}}
  \bfseries 7--8. évf. & \bfseries Dráma és színház\endhead
  \hline
&
  Ismeri és alkalmazza a különböző verbális és nonverbális kommunikációs
  eszközöket.
\tabularnewline
\hline
&
  Ismeri és alkalmazza a drámai és színházi kifejezés formáit.
\tabularnewline
\hline
&
  Aktívan részt vesz többféle dramatikus tevékenységben tanári
  irányítással, önállóan, illetve társakkal való együttműködésben.
\tabularnewline
\hline
&
  Saját gondolatot, témát, üzenetet fogalmaz meg a témához általa
  alkalmasnak ítélt dramatikus közlésformában.
\tabularnewline
\hline
&
  Megfogalmazza egy színházi előadás kapcsán élményeit, gon-\break dolatait.
\tabularnewline
\hline
&
  Felfedezi a tér, az idő, a tempó, a ritmus sajátosságait és
  össze-\break függéseit.
\tabularnewline
\hline
&
  Megfigyeli, azonosítja és értelmezi a tárgyi világ jelenségeit.
\tabularnewline
\hline
&
  Felidézi a látott, hallott, érzékelt verbális, vokális, vizuális,
  kinetikus hatásokat.
\tabularnewline
\hline
&
  Kitalál és alkalmaz elképzelt verbális, vokális, vizuális, kinetikus
  hatásokat.
\tabularnewline
\hline
&
  Tudatosan irányítja és összpontosítja figyelmét a környezete
  je-\break lenségeire.
\tabularnewline
\hline
&
  Koncentrált figyelemmel végzi a játékszabályok adta keretek között
  tevékenységeit.
\tabularnewline
\hline
&
  Megfigyeli, azonosítja és értelmezi a környezetéből érkező hatásokra
  adott saját válaszait.
\tabularnewline
\hline
&
  Értelmezi önmagát a csoport részeként, illetve a csoportos tevékenység
  alkotó közreműködőjeként.
\tabularnewline
\hline
&
  Fejleszti az együttműködésre és a konszenzus kialakítására irányuló
  gyakorlatát.
\tabularnewline
\hline
&
  Adekvát módon alkalmazza a verbális és nonverbális kifejezés
  eszközeit.
\tabularnewline
\hline
&
  Az alkotótevékenység során használja a megismert kifejezési
  formákat.
\tabularnewline
\hline
&
  Felfedezi a tárgyi világ kínálta eszközöket, ezek művészi formáit (pl.
  a bábot és a maszkot).
\tabularnewline
\hline
&
  Használja a tér sajátosságaiban rejlő lehetőségeket.
\tabularnewline
\hline
&
  Felfedezi a szerepbe lépésben és az együttjátszásban rejlő
  le-\break hetőségeket.
\tabularnewline
\hline
&
  Felismeri és alapszinten alkalmazza a kapcsolat létrehozásának és
  fenntartásának technikáit.
\tabularnewline
\hline
&
  Felfedezi a kommunikációs jelek jelentéshordozó és jelentésteremtő
  erejét.
\tabularnewline
\hline
&
  Felfedezi a feszültség élményét és szerepét a dramatikus
  tevé-\break kenységekben.
\tabularnewline
\hline
&
  Felfedezi a színházi kommunikáció erejét.
\tabularnewline
\hline
&
  Alkalmazza a tanult dramatikus technikákat a helyzetek
  megje-\break lenítésében.
\tabularnewline
\hline
&
  Felismeri a helyzetek feldolgozása során a szerkesztésben rejlő
  lehetőségeket.
\tabularnewline
\hline
&
  Megkülönbözteti és alapszinten alkalmazza a dramaturgiai
  alap-\break fogalmakat.
\tabularnewline
\hline
&
  Értelmezi a megélt, a látott-hallott-olvasott, a kitalált történeteket
  a különböző dramatikus tevékenységek révén.
\tabularnewline
\hline
&
  Felismeri a színházi élmény fontosságát.
\tabularnewline
\hline
&
  A színházi előadást a dramatikus tevékenységek kiindulópontjául is
  használja.
\tabularnewline
\hline
&
  Felismeri és azonosítja a dramatikus szituációk jellemzőit (szereplők,
  viszonyrendszer, cél, szándék, akarat, konfliktus, feloldás).
\tabularnewline
\hline
&
  Felismeri és megvizsgálja a problémahelyzeteket és azok lehetséges
  megoldási alternatíváit.
\tabularnewline
\hline
&
  Felismeri és azonosítja a dráma és a színház formanyelvi sajátosságait
  a látott előadásokban.
\tabularnewline
\hline
\end{longtable}

% \hypertarget{evfolyamon-7}{%
% \subsubsection{9-10. évfolyamon}\label{evfolyamon-7}}

\savebox{\evfbox}{\bfseries\ 9--10. évf.\ }
\settowidth{\evflength}{\usebox{\evfbox}}
\setlength{\columnlength}{\textwidth}
\addtolength{\columnlength}{-\evflength}

\begin{longtable}[]{p{\evflength}@{\strut}>{\begin{minipage}{\columnlength}\strut}l<{\strut\end{minipage}}}
  \bfseries 9--10. évf. & \bfseries Dráma és színház\endhead
  \hline
&
  Az élmény megélésén keresztül jusson el a megértésig.
\tabularnewline
\hline
&
  Alkalmat kapjon az önmagára és a világra vonatkozó kérdések
  megfogalmazására és a válaszok keresésére.
\tabularnewline
\hline
&
  Különböző élethelyzeteket védett környezetben, biztonságos keretek
  között vizsgáljon.
\tabularnewline
\hline
&
  A drámán és színjátékon keresztül tanulja meg az önkifejezést.
\tabularnewline
\hline
&
  Részt vegyen közösségépítésben és közösségi alkotásban.
\tabularnewline
\hline
&
  Verbális és nonverbális kommunikációs készségei fejlődjenek.
\tabularnewline
\hline
&
  Empátiás készségeit erősítse a dramatikus tevékenységekben való
  együttműködéssel.
\tabularnewline
\hline
&
  Komplex látásmódot alakítson ki a dráma és a színház társadalmi,
  történelmi és kulturális szerepének megértésével.
\tabularnewline
\hline
&
  Szabályjátékok, népi játékok
\tabularnewline
\hline
&
  Dramatikus játékok (szöveggel, hanggal, bábbal, zenével, mozgással,
  tánccal)
\tabularnewline
\hline
&
  Rögtönzés
\tabularnewline
\hline
&
  Saját történetek feldolgozása
\tabularnewline
\hline
&
  Műalkotások feldolgozása
\tabularnewline
\hline
&
  Dramaturgiai alapfogalmak
\tabularnewline
\hline
&
  A színház kifejezőeszközei (szöveg, hang, báb, zene, mozgás, tánc)
\tabularnewline
\hline
&
  Színházi műfajok, stílusok
\tabularnewline
\hline
&
  Színházi előadás megtekintése
\tabularnewline
\hline
&
  Szabályjátékok
\tabularnewline
\hline
&
  Dramatikus játékok (szöveggel, hanggal, bábbal, zenével, mozgással,
  tánccal)
\tabularnewline
\hline
&
  Rögtönzés
\tabularnewline
\hline
&
  Saját történetek feldolgozása
\tabularnewline
\hline
&
  Műalkotások feldolgozása
\tabularnewline
\hline
&
  Dramaturgiai ismeretek
\tabularnewline
\hline
&
  A színház kifejezőeszközei (szöveg, hang, báb, zene, mozgás, tánc)
\tabularnewline
\hline
&
  Dráma- és színháztörténet
\tabularnewline
\hline
&
  Dráma- és színházelmélet
\tabularnewline
\hline
&
  Kortárs dráma és színház
\tabularnewline
\hline
&
  Színjátékos tevékenység (vers- és prózamondás, jelenet, előadás stb.)
\tabularnewline
\hline
&
  Színházi előadás megtekintése
\tabularnewline
\hline
\end{longtable}

\hypertarget{elso-elo-idegen-nyelv}{%
\subsection{Első élő idegen nyelv}\label{elso-elo-idegen-nyelv}}

% \hypertarget{evfolyamon-8}{%
% \subsubsection{4. évfolyamon}\label{evfolyamon-8}}

\savebox{\evfbox}{\bfseries\ 4. évf.\ }
\settowidth{\evflength}{\usebox{\evfbox}}
\setlength{\columnlength}{\textwidth}
\addtolength{\columnlength}{-\evflength}

\begin{longtable}[]{p{\evflength}@{\strut}>{\begin{minipage}{\columnlength}\strut}l<{\strut\end{minipage}}}
  \bfseries 4. évf. & \bfseries Első élő idegen nyelv\endhead
  \hline
&
  Megismerkedik az idegen nyelvvel, a nyelvtanulással, és örömmel vesz
  részt az órákon.
\tabularnewline
\hline
&
  Bekapcsolódik a szóbeliséget, írást, szövegértést vagy interakciót
  igénylő alapvető és korának megfelelő játékos, élményalapú élő idegen
  nyelvi tevékenységekbe.
\tabularnewline
\hline
&
  Szóban visszaad szavakat, esetleg rövid, nagyon egyszerű szövegeket
  hoz létre.
\tabularnewline
\hline
&
  Lemásol, leír szavakat és rövid, nagyon egyszerű szövegeket.
\tabularnewline
\hline
&
  Követi a szintjének megfelelő, vizuális vagy nonverbális eszközökkel
  támogatott, ismert célnyelvi óravezetést, utasításokat.
\tabularnewline
\hline
&
  Felismeri és használja a legegyszerűbb, mindennapi nyelvi
  funk-\break ciókat.
\tabularnewline
\hline
&
  Elmondja magáról a legalapvetőbb információkat.
\tabularnewline
\hline
&
  Ismeri az adott célnyelvi kultúrákhoz tartozó országok fontosabb
  jellemzőit és a hozzájuk tartozó alapvető nyelvi elemeket.
\tabularnewline
\hline
&
  Törekszik a tanult nyelvi elemek megfelelő kiejtésére.
\tabularnewline
\hline
&
  Célnyelvi tanulmányain keresztül nyitottabbá, a világ felé
  érdeklődőbbé válik.
\tabularnewline
\hline
&
  Megismétli az élőszóban elhangzó egyszerű szavakat, kifejezéseket
  játékos, mozgást igénylő, kreatív nyelvórai tevékenységek során.
\tabularnewline
\hline
&
  Lebetűzi a nevét.
\tabularnewline
\hline
&
  Lebetűzi a tanult szavakat társaival közösen játékos tevékenységek
  kapcsán, szükség esetén segítséggel.
\tabularnewline
\hline
&
  Célnyelven megoszt egyedül vagy társaival együttműködésben
  megszerzett, alapvető információkat szóban, akár vizuális elemekkel
  támogatva.
\tabularnewline
\hline
&
  Felismeri az anyanyelvén, illetve a tanult idegen nyelven történő
  írásmód és betűkészlet közötti különbségeket.
\tabularnewline
\hline
&
  Ismeri az adott nyelv ábécéjét.
\tabularnewline
\hline
&
  Lemásol tanult szavakat játékos, alkotó nyelvórai tevékenységek során.
\tabularnewline
\hline
&
  Megold játékos írásbeli feladatokat a szavak, szószerkezetek, rövid
  mondatok szintjén.
\tabularnewline
\hline
&
  Részt vesz kooperatív munkaformában végzett kreatív tevékenységekben,
  projektmunkában szavak, szószerkezetek, rövid mondatok leírásával,
  esetleg képi kiegészítéssel.
\tabularnewline
\hline
&
  Írásban megnevezi az ajánlott tématartományokban megjelölt,
  begyakorolt elemeket.
\tabularnewline
\hline
&
  Megérti az élőszóban elhangzó, ismert témákhoz kapcsolódó, verbális,
  vizuális vagy nonverbális eszközökkel segített rövid kijelentéseket,
  kérdéseket.
\tabularnewline
\hline
&
  Beazonosítja az életkorának megfelelő szituációkhoz kapcsolódó, rövid,
  egyszerű szövegben a tanult nyelvi elemeket.
\tabularnewline
\hline
&
  Kiszűri a lényeget az ismert nyelvi elemeket tartalmazó, nagyon rövid,
  egyszerű hangzó szövegből.
\tabularnewline
\hline
&
  Azonosítja a célzott információt a nyelvi szintjének és életkorának
  megfelelő rövid hangzó szövegben.
\tabularnewline
\hline
&
  Támaszkodik az életkorának és nyelvi szintjének megfelelő hangzó
  szövegre az órai alkotó jellegű nyelvi, mozgásos nyelvi és játékos
  nyelvi tevékenységek során.
\tabularnewline
\hline
&
  Felismeri az anyanyelv és az idegen nyelv hangkészletét.
\tabularnewline
\hline
&
  Értelmezi azokat az idegen nyelven szóban elhangzó nyelvórai
  szituációkat, melyeket anyanyelvén már ismer.
\tabularnewline
\hline
&
  Felismeri az anyanyelve és a célnyelv közötti legalapvetőbb
  kiejtésbeli különbségeket.
\tabularnewline
\hline
&
  Figyel a célnyelvre jellemző hangok kiejtésére.
\tabularnewline
\hline
&
  Megkülönbözteti az anyanyelvi és a célnyelvi írott szövegben a betű-
  és jelkészlet közti különbségeket.
\tabularnewline
\hline
&
  Beazonosítja a célzott információt az életkorának megfelelő
  szituációkhoz kapcsolódó, rövid, egyszerű, a nyelvtanításhoz készült,
  illetve eredeti szövegben.
\tabularnewline
\hline
&
  Csendes olvasás keretében feldolgozva megért ismert szavakat
  tartalmazó, pár szóból vagy mondatból álló, akár illusztrációval
  támogatott szöveget.
\tabularnewline
\hline
&
  Megérti a nyelvi szintjének megfelelő, akár vizuális eszközökkel is
  támogatott írott utasításokat és kérdéseket, és ezekre megfelelő
  válaszreakciókat ad.
\tabularnewline
\hline
&
  Kiemeli az ismert nyelvi elemeket tartalmazó, egyszerű, írott, pár
  mondatos szöveg fő mondanivalóját.
\tabularnewline
\hline
&
  Támaszkodik az életkorának és nyelvi szintjének megfelelő írott
  szövegre az órai játékos alkotó, mozgásos vagy nyelvi fejlesztő
  tevékenységek során, kooperatív munkaformákban.
\tabularnewline
\hline
&
  Megtapasztalja a közös célnyelvi olvasás élményét.
\tabularnewline
\hline
&
  Aktívan bekapcsolódik a közös meseolvasásba, a mese tartalmát követi.
\tabularnewline
\hline
&
  A tanórán begyakorolt, nagyon egyszerű, egyértelmű kommunikációs
  helyzetekben a megtanult, állandósult beszédfordulatok alkalmazásával
  kérdez vagy reagál, mondanivalóját segítséggel vagy nonverbális
  eszközökkel kifejezi.
\tabularnewline
\hline
&
  Törekszik arra, hogy a célnyelvet eszközként alkalmazza
  informá-\break ciószerzésre.
\tabularnewline
\hline
&
  Rövid, néhány mondatból álló párbeszédet folytat, felkészülést
  követően.
\tabularnewline
\hline
&
  A tanórán bekapcsolódik a már ismert, szóbeli interakciót igénylő
  nyelvi tevékenységekbe, a begyakorolt nyelvi elemeket tanári
  segítséggel a tevékenység céljainak megfelelően alkalmazza.
\tabularnewline
\hline
&
  Érzéseit egy-két szóval vagy begyakorolt állandósult nyelvi fordulatok
  segítségével kifejezi, főként rákérdezés alapján, nonverbális
  eszközökkel kísérve a célnyelvi megnyilatkozást.
\tabularnewline
\hline
&
  Elsajátítja a tanult szavak és állandósult szókapcsolatok célnyelvi
  normához közelítő kiejtését tanári minta követése által, vagy
  autentikus hangzó anyag, digitális technológia segítségével.
\tabularnewline
\hline
&
  Felismeri és alkalmazza a legegyszerűbb üdvözlésre és elköszönésre
  használt mindennapi nyelvi funkciókat az életkorának és nyelvi
  szintjének megfelelő, egyszerű helyzetekben.
\tabularnewline
\hline
&
  Felismeri és alkalmazza a legegyszerűbb bemutatkozásra használt
  mindennapi nyelvi funkciókat az életkorának és nyelvi szintjének
  megfelelő, egyszerű helyzetekben.
\tabularnewline
\hline
&
  Felismeri és használja a legegyszerűbb megszólításra használt
  mindennapi nyelvi funkciókat az életkorának és nyelvi szintjének
  megfelelő, egyszerű helyzetekben.
\tabularnewline
\hline
&
  Felismeri és használja a köszönet és az arra történő
  reagálás kifejezésére használt legegyszerűbb mindennapi nyelvi funkciókat az
  életkorának és nyelvi szintjének megfelelő, egyszerű helyzetekben.
\tabularnewline
\hline
&
  Felismeri és használja a legegyszerűbb, a tudás és nem tudás
  kifejezésére használt mindennapi nyelvi funkciókat az életkorának és
  nyelvi szintjének megfelelő, egyszerű helyzetekben.
\tabularnewline
\hline
&
  Felismeri és használja a legegyszerűbb, a nem értés, visszakérdezés és
  ismétlés, kérés kifejezésére használt mindennapi nyelvi funkciókat
  életkorának és nyelvi szintjének megfelelő, egyszerű helyzetekben.
\tabularnewline
\hline
&
  Közöl alapvető személyes információkat magáról, egyszerű nyelvi elemek
  segítségével.
\tabularnewline
\hline
&
  Új szavak, kifejezések tanulásakor ráismer a már korábban tanult
  szavakra, kifejezésekre.
\tabularnewline
\hline
&
  Szavak, kifejezések tanulásakor felismeri, ha új elemmel találkozik, és
  rákérdez, vagy megfelelő tanulási stratégiával törekszik a megértésre.
\tabularnewline
\hline
&
  A célok eléréséhez társaival rövid feladatokban együttműködik.
\tabularnewline
\hline
&
  Egy feladat megoldásának sikerességét segítséggel értékelni tudja.
\tabularnewline
\hline
&
  Felismeri az idegen nyelvű írott, olvasott és hallott tartalmakat a
  tanórán kívül is.
\tabularnewline
\hline
&
  Felhasznál és létrehoz rövid, nagyon egyszerű célnyelvi szövegeket
  szabadidős tevékenységek során.
\tabularnewline
\hline
&
  Alapvető célzott információt megszerez a tanult témákban tudásának
  bővítésére.
\tabularnewline
\hline
&
  Megismeri a főbb, az adott célnyelvi kultúrákhoz tartozó országok
  nevét, földrajzi elhelyezkedését, főbb országismereti jellemzőit.
\tabularnewline
\hline
&
  Ismeri a főbb, célnyelvi kultúrához tartozó, ünnepekhez kapcsolódó
  alapszintű kifejezéseket, állandósult szókapcsolatokat és szokásokat.
\tabularnewline
\hline
&
  Megérti a tanult nyelvi elemeket életkorának megfelelő digitális
  tartalmakban, digitális csatornákon olvasott vagy hallott nagyon
  egyszerű szövegekben is.
\tabularnewline
\hline
&
  Létrehoz nagyon egyszerű írott, pár szavas szöveget szóban vagy
  írásban digitális felületen.
\tabularnewline
\hline
\end{longtable}

% \hypertarget{evfolyamon-9}{%
% \subsubsection{5-8. évfolyamon}\label{evfolyamon-9}}


\savebox{\evfbox}{\bfseries\ 5--8. évf.\ }
\settowidth{\evflength}{\usebox{\evfbox}}
\setlength{\columnlength}{\textwidth}
\addtolength{\columnlength}{-\evflength}

\begin{longtable}[]{p{\evflength}@{\strut}>{\begin{minipage}{\columnlength}\strut}l<{\strut\end{minipage}}}
  \bfseries 5--8. évf. & \bfseries Első élő idegen nyelv\endhead
  \hline
&
  Szóban és írásban megold változatos kihívásokat igénylő feladatokat az
  élő idegen nyelven.
\tabularnewline
\hline
&
  Szóban és írásban létrehoz rövid szövegeket, ismert nyelvi
  eszközökkel, a korának megfelelő szövegtípusokban.
\tabularnewline
\hline
&
  Értelmez korának és nyelvi szintjének megfelelő hallott és írott
  célnyelvi szövegeket az ismert témákban és szövegtípusokban.
\tabularnewline
\hline
&
  A tanult nyelvi elemek és kommunikációs stratégiák segítségével
  írásbeli és szóbeli interakciót folytat, valamint közvetít az élő
  idegen nyelven.
\tabularnewline
\hline
&
  Kommunikációs szándékának megfelelően alkalmazza a tanult nyelvi
  funkciókat és a megszerzett szociolingvisztikai, pragmatikai és
  interkulturális jártasságát.
\tabularnewline
\hline
&
  Nyelvtudását egyre inkább képes fejleszteni tanórán kívüli
  helyzetekben is különböző eszközökkel és lehetőségekkel.
\tabularnewline
\hline
&
  Használ életkorának és nyelvi szintjének megfelelő hagyományos és
  digitális alapú nyelvtanulási forrásokat és eszközöket.
\tabularnewline
\hline
&
  Alkalmazza nyelvtudását kommunikációra, közvetítésre, szó-\break rakozásra,
  ismeretszerzésre hagyományos és digitális csator-\break nákon.
\tabularnewline
\hline
&
  Törekszik a célnyelvi normához illeszkedő kiejtés, beszédtempó és
  intonáció megközelítésére.
\tabularnewline
\hline
&
  Érti a nyelvtudás fontosságát, és motivációja a nyelvtanulásra tovább
  erősödik.
\tabularnewline
\hline
&
  Aktívan részt vesz az életkorának és érdeklődésének megfelelő
  gyermek-, illetve ifjúsági irodalmi alkotások közös előadásában.
\tabularnewline
\hline
&
  Egyre magabiztosabban kapcsolódik be történetek kreatív alakításába,
  átfogalmazásába kooperatív munkaformában.
\tabularnewline
\hline
&
  Elmesél rövid történetet, egyszerűsített olvasmányt egyszerű nyelvi
  eszközökkel, önállóan, a cselekményt lineárisan összefűz-\break ve.
\tabularnewline
\hline
&
  Egyszerű nyelvi eszközökkel, felkészülést követően röviden,
  összefüggően beszél az ajánlott tématartományokhoz tartozó témákban,
  élőszóban és digitális felületen.
\tabularnewline
\hline
&
  Képet jellemez röviden, egyszerűen, ismert nyelvi fordulatok
  segítségével, segítő tanári kérdések alapján, önállóan.
\tabularnewline
\hline
&
  Változatos, kognitív kihívást jelentő szóbeli feladatokat old meg
  önállóan vagy kooperatív munkaformában, a tanult nyelvi eszközökkel,
  szükség szerint tanári segítséggel, élőszóban és digitális felületen.
\tabularnewline
\hline
&
  Megold játékos és változatos írásbeli feladatokat rövid szövegek
  szintjén.
\tabularnewline
\hline
&
  Rövid, egyszerű, összefüggő szövegeket ír a tanult nyelvi szerkezetek
  felhasználásával az ismert szövegtípusokban, az ajánlott
  tématartományokban.
\tabularnewline
\hline
&
  Rövid szövegek írását igénylő kreatív munkát hoz létre önállóan.
\tabularnewline
\hline
&
  Rövid, összefüggő, papíralapú vagy IKT-eszközökkel segített írott
  projektmunkát készít önállóan vagy kooperatív munkafor-\break mákban.
\tabularnewline
\hline
&
  A szövegek létrehozásához nyomtatott, illetve digitális alapú
  segédeszközt, szótárt használ.
\tabularnewline
\hline
&
  Megérti a szintjének megfelelő, kevésbé ismert elemekből álló,
  nonverbális vagy vizuális eszközökkel támogatott célnyelvi óravezetést
  és utasításokat, kérdéseket.
\tabularnewline
\hline
&
  Értelmezi az életkorának és nyelvi szintjének megfelelő, egyszerű,
  hangzó szövegben a tanult nyelvi elemeket.
\tabularnewline
\hline
&
  Értelmezi az életkorának megfelelő, élőszóban vagy digitális felületen
  elhangzó szövegekben a beszélők gondolatmenetét.
\tabularnewline
\hline
&
  Megérti a nem kizárólag ismert nyelvi elemeket tartalmazó, élőszóban
  vagy digitális felületen elhangzó rövid szöveg tartalmát.
\tabularnewline
\hline
&
  Kiemel, kiszűr konkrét információkat a nyelvi szintjének megfelelő,
  élőszóban vagy digitális felületen elhangzó szövegből, és azokat
  összekapcsolja egyéb ismereteivel.
\tabularnewline
\hline
&
  Alkalmazza az életkorának és nyelvi szintjének megfelelő hangzó
  szöveget a változatos nyelvórai tevékenységek és a feladatmegoldás
  során.
\tabularnewline
\hline
&
  Értelmez életkorának megfelelő nyelvi helyzeteket hallott szöveg
  alapján.
\tabularnewline
\hline
&
  Felismeri a főbb, életkorának megfelelő hangzószöveg-tí\-pu\-so\-kat.
\tabularnewline
\hline
&
  Hallgat az érdeklődésének megfelelő autentikus szövegeket
  elektronikus, digitális csatornákon, tanórán kívül is, szórakozásra
  vagy ismeretszerzésre.
\tabularnewline
\hline
&
  Értelmezi az életkorának megfelelő szituációkhoz kapcsolódó, írott
  szövegekben megjelenő összetettebb információkat.
\tabularnewline
\hline
&
  Megérti a nem kizárólag ismert nyelvi elemeket tartalmazó rövid írott
  szöveg tartalmát.
\tabularnewline
\hline
&
  Kiemel, kiszűr konkrét információkat a nyelvi szintjének megfelelő
  szövegből, és azokat összekapcsolja más iskolai vagy iskolán kívül
  szerzett ismereteivel.
\tabularnewline
\hline
&
  Megkülönbözteti a főbb, életkorának megfelelő írott szöveg-\break típusokat.
\tabularnewline
\hline
&
  Összetettebb írott instrukciókat értelmez.
\tabularnewline
\hline
&
  Alkalmazza az életkorának és nyelvi szintjének megfelelő írott,
  nyomtatott vagy digitális alapú szöveget a változatos nyelvórai
  tevékenységek és feladatmegoldás során.
\tabularnewline
\hline
&
  A nyomtatott vagy digitális alapú írott szöveget felhasználja
  szórakozásra és ismeretszerzésre önállóan is.
\tabularnewline
\hline
&
  Érdeklődése erősödik a célnyelvi irodalmi alkotások iránt.
\tabularnewline
\hline
&
  Megért és használ szavakat, szókapcsolatokat a célnyelvi, az
  életkorának és érdeklődésének megfelelő hazai és nemzetközi legfőbb
  hírekkel, eseményekkel kapcsolatban.
\tabularnewline
\hline
&
  Kommunikációt kezdeményez egyszerű hétköznapi témában, a beszélgetést
  követi, egyszerű, nyelvi eszközökkel fenntartja és lezárja.
\tabularnewline
\hline
&
  Az életkorának megfelelő mindennapi helyzetekben a tanult nyelvi
  eszközökkel megfogalmazott kérdéseket tesz fel, és válaszol a hozzá
  intézett kérdésekre.
\tabularnewline
\hline
&
  Véleményét, gondolatait, érzéseit egyre magabiztosabban fejezi ki a
  tanult nyelvi eszközökkel.
\tabularnewline
\hline
&
  A tanult nyelvi elemeket többnyire megfelelően használja,
  beszédszándékainak megfelelően, egyszerű spontán helyzetekben.
\tabularnewline
\hline
&
  Váratlan, előre nem kiszámítható eseményekre, jelenségekre és
  történésekre is reagál egyszerű célnyelvi eszközökkel, személyes vagy
  online interakciókban.
\tabularnewline
\hline
&
  Bekapcsolódik a tanórán az interakciót igénylő nyelvi tevékenységekbe,
  azokban társaival közösen részt vesz, a begyakorolt nyelvi elemeket
  tanári segítséggel a játék céljainak megfelelően alkalmazza.
\tabularnewline
\hline
&
  Üzeneteket ír.
\tabularnewline
\hline
&
  Véleményét írásban, egyszerű nyelvi eszközökkel megfogalmazza, és
  arról írásban interakciót folytat.
\tabularnewline
\hline
&
  Rövid, egyszerű, ismert nyelvi eszközökből álló kiselőadást tart
  változatos feladatok kapcsán, hagyományos vagy digitális alapú
  vizuális eszközök támogatásával.
\tabularnewline
\hline
&
  Felhasználja a célnyelvet tudásmegosztásra.
\tabularnewline
\hline
&
  Találkozik az életkorának és nyelvi szintjének megfelelő célnyelvi
  ismeretterjesztő tartalmakkal.
\tabularnewline
\hline
&
  Néhány szóból vagy mondatból álló jegyzetet készít írott szöveg
  alapján.
\tabularnewline
\hline
&
  Egyszerűen megfogalmazza személyes véleményét, másoktól véleményük
  kifejtését kéri, és arra reagál, elismeri vagy cáfolja mások
  állítását, kifejezi egyetértését vagy egyet nem értését.
\tabularnewline
\hline
&
  Kifejez tetszést, nem tetszést, akaratot, kívánságot, tudást és nem
  tudást, ígéretet, szándékot, dicséretet, kritikát.
\tabularnewline
\hline
&
  Információt cserél, információt kér, információt ad.
\tabularnewline
\hline
&
  Kifejez kérést, javaslatot, meghívást, kínálást és ezekre reagálást.
\tabularnewline
\hline
&
  Kifejez alapvető érzéseket, például örömöt, sajnálkozást, bánatot,
  elégedettséget, elégedetlenséget, bosszúságot, csodálkozást, reményt.
\tabularnewline
\hline
&
  Kifejez és érvekkel alátámasztva mutat be szükségességet, lehetőséget,
  képességet, bizonyosságot, bizonytalanságot.
\tabularnewline
\hline
&
  Értelmez és használja az idegen nyelvű írott, olvasott és hallott
  tartalmakat a tanórán kívül is,
\tabularnewline
\hline
&
  Felhasználja a célnyelvet ismeretszerzésre.
\tabularnewline
\hline
&
  Használja a célnyelvet életkorának és nyelvi szintjének megfelelő
  aktuális témákban és a hozzájuk tartozó szituációkban.
\tabularnewline
\hline
&
  Találkozik életkorának és nyelvi szintjének megfelelő célnyelvi
  szórakoztató tartalmakkal.
\tabularnewline
\hline
&
  Összekapcsolja az ismert nyelvi elemeket egyszerű kötőszavakkal
  (például: és, de, vagy).
\tabularnewline
\hline
&
  Egyszerű mondatokat összekapcsolva mond el egymást követő eseményekből
  álló történetet, vagy leírást ad valamilyen témáról.
\tabularnewline
\hline
&
  A tanult nyelvi eszközökkel és nonverbális elemek segítségével
  tisztázza mondanivalójának lényegét.
\tabularnewline
\hline
&
  Ismeretlen szavak valószínű jelentését szövegösszefüggések alapján
  kikövetkezteti az életkorának és érdeklődésének megfelelő, konkrét,
  rövid szövegekben.
\tabularnewline
\hline
&
  Alkalmaz nyelvi funkciókat rövid társalgás megkezdéséhez,
  fenntartásához és befejezéséhez.
\tabularnewline
\hline
&
  Nem értés esetén a meg nem értett kulcsszavak vagy fordulatok
  ismétlését vagy magyarázatát kéri, visszakérdez, betűzést kér.
\tabularnewline
\hline
&
  Megoszt alapvető személyes információkat és szükségleteket magáról
  egyszerű nyelvi elemekkel.
\tabularnewline
\hline
&
  Ismerős és gyakori alapvető helyzetekben, akár telefonon vagy
  digitális csatornákon is, többnyire helyesen és érthetően fejezi ki
  magát az ismert nyelvi eszközök segítségével.
\tabularnewline
\hline
&
  Tudatosan használ alapszintű nyelvtanulási és nyelvhasználati
  stratégiákat.
\tabularnewline
\hline
&
  Hibáit többnyire észreveszi és javítja.
\tabularnewline
\hline
&
  Ismer szavakat, szókapcsolatokat a célnyelven a témakörre jellemző,
  életkorának és érdeklődésének megfelelő más tudásterületen megcélzott
  tartalmakból.
\tabularnewline
\hline
&
  Egy összetettebb nyelvi feladat, projekt végéig tartó célokat tűz ki
  magának.
\tabularnewline
\hline
&
  Céljai eléréséhez megtalálja és használja a megfelelő eszközöket.
\tabularnewline
\hline
&
  Céljai eléréséhez társaival párban és csoportban együttműködik.
\tabularnewline
\hline
&
  Nyelvi haladását többnyire fel tudja mérni,
\tabularnewline
\hline
&
  Társai haladásának értékelésében segítően részt vesz.
\tabularnewline
\hline
&
  A tanórán kívüli, akár játékos nyelvtanulási lehetőségeket felismeri,
  és törekszik azokat kihasználni.
\tabularnewline
\hline
&
  Felhasználja a célnyelvet szórakozásra és játékos nyelvtanulásra.
\tabularnewline
\hline
&
  Digitális eszközöket és felületeket is használ nyelvtudása
  fejlesztésére,
\tabularnewline
\hline
&
  Értelmez egyszerű, szórakoztató kisfilmeket
\tabularnewline
\hline
&
  Megismeri a célnyelvi országok főbb jellemzőit és kulturális
  sajátosságait.
\tabularnewline
\hline
&
  További országismereti tudásra tesz szert.
\tabularnewline
\hline
&
  Célnyelvi kommunikációjába beépíti a tanult interkulturális
  ismereteket.
\tabularnewline
\hline
&
  Találkozik célnyelvi országismereti tartalmakkal.
\tabularnewline
\hline
&
  Találkozik a célnyelvi, életkorának és érdeklődésének megfelelő hazai
  és nemzetközi legfőbb hírekkel, eseményekkel.
\tabularnewline
\hline
&
  Megismerkedik hazánk legfőbb országismereti és történelmi eseményeivel
  célnyelven.
\tabularnewline
\hline
&
  A célnyelvi kultúrákhoz kapcsolódó alapvető tanult nyelvi elemeket
  használja.
\tabularnewline
\hline
&
  Idegen nyelvi kommunikációjában ismeri és használja a célnyelv főbb
  jellemzőit.
\tabularnewline
\hline
&
  Következetesen alkalmazza a célnyelvi betű és jelkészletet.
\tabularnewline
\hline
&
  Egyénileg vagy társaival együttműködve szóban vagy írásban
  projektmunkát vagy kiselőadást készít, és ezeket digitális eszközök
  segítségével is meg tudja valósítani.
\tabularnewline
\hline
&
  Találkozik az érdeklődésének megfelelő akár autentikus szövegekkel
  elektronikus, digitális csatornákon tanórán kívül is.
\tabularnewline
\hline
\end{longtable}

% \hypertarget{evfolyamon-10}{%
% \subsubsection{9-12. évfolyamon}\label{evfolyamon-10}}

\savebox{\evfbox}{\bfseries\ 9--12. évf.\ }
\settowidth{\evflength}{\usebox{\evfbox}}
\setlength{\columnlength}{\textwidth}
\addtolength{\columnlength}{-\evflength}

\begin{longtable}[]{p{\evflength}@{\strut}>{\begin{minipage}{\columnlength}\strut}l<{\strut\end{minipage}}}
  \bfseries 9--12. évf. & \bfseries Első élő idegen nyelv\endhead
  \hline
&
  Szóban és írásban is megold változatos kihívásokat igénylő, többnyire
  valós kommunikációs helyzeteket leképező feladatokat az élő idegen
  nyelven.
\tabularnewline
\hline
&
  Szóban és írásban létrehoz szövegeket különböző szövegtípu-\break sokban.
\tabularnewline
\hline
&
  Értelmez nyelvi szintjének megfelelő hallott és írott célnyelvi
  szövegeket kevésbé ismert témákban és szövegtípusokban is.
\tabularnewline
\hline
&
  A tanult nyelvi elemek és kommunikációs stratégiák segítségével
  írásbeli és szóbeli interakciót folytat és tartalmakat közvetít idegen
  nyelven.
\tabularnewline
\hline
&
  Kommunikációs szándékának megfelelően alkalmazza a nyelvi funkciókat
  és megszerzett szociolingvisztikai, pragmatikai és interkulturális
  jártasságát.
\tabularnewline
\hline
&
  Nyelvtudását képes fejleszteni tanórán kívüli eszközökkel,
  lehetőségekkel és helyzetekben is, valamint a tanultakat és
  gimnáziumban a második idegen nyelv tanulásában is alkalmazza.
\tabularnewline
\hline
&
  Felkészül az aktív nyelvtanulás eszközeivel az egész életen át történő
  tanulásra.
\tabularnewline
\hline
&
  Használ hagyományos és digitális alapú nyelvtanulási forrásokat és
  eszközöket.
\tabularnewline
\hline
&
  Alkalmazza nyelvtudását kommunikációra, közvetítésre, szó-\break rakozásra,
  ismeretszerzésre hagyományos és digitális csator-\break nákon.
\tabularnewline
\hline
&
  Törekszik a célnyelvi normához illeszkedő kiejtés, beszédtempó és
  intonáció megközelítésére.
\tabularnewline
\hline
&
  Beazonosítja nyelvtanulási céljait, és egyéni különbségeinek tudatában,
  ezeknek megfelelően fejleszti nyelvtudását.
\tabularnewline
\hline
&
  Első idegen nyelvéből sikeresen érettségit tesz a céljainak megfelelő
  szinten.
\tabularnewline
\hline
&
  Visszaad tankönyvi vagy más tanult szöveget, elbeszélést, nagyrészt
  folyamatos és érthető történetmeséléssel, a cselekményt logikusan
  összefűzve.
\tabularnewline
\hline
&
  Összefüggően, érthetően és nagyrészt folyékonyan beszél az ajánlott
  tématartományokhoz tartozó és az érettségi témákban a tanult nyelvi
  eszközökkel, felkészülést követően.
\tabularnewline
\hline
&
  Beszámol saját élményen, tapasztalaton alapuló vagy elképzelt
  eseményről a cselekmény, a körülmények, az érzések és gondolatok
  ismert nyelvi eszközökkel történő rövid jellemzésével.
\tabularnewline
\hline
&
  Ajánlott tématartományhoz kapcsolódó képi hatás kapcsán saját
  gondolatait, véleményét és érzéseit is kifejti az ismert nyelvi
  eszközökkel.
\tabularnewline
\hline
&
  Összefoglalja ismert témában nyomtatott vagy digitális alapú ifjúsági
  tartalmak lényegét röviden és érthetően.
\tabularnewline
\hline
&
  Közép- és emelt szintű nyelvi érettségi szóbeli feladatokat old meg.
\tabularnewline
\hline
&
  Összefüggő, folyékony előadásmódú szóbeli prezentációt tart önállóan,
  felkészülést követően, az érettségi témakörök közül szabadon
  választott témában, IKT-eszközökkel támogatva mondanivalóját.
\tabularnewline
\hline
&
  Kreatív, változatos műfajú szövegeket alkot szóban, kooperatív
  munkaformákban.
\tabularnewline
\hline
&
  Beszámol akár az érdeklődési körén túlmutató környezeti eseményről a
  cselekmény, a körülmények, az érzések és gondolatok ismert nyelvi
  eszközökkel történő összetettebb, részletes és világos jellemzésével.
\tabularnewline
\hline
&
  Összefüggően, világosan és nagyrészt folyékonyan beszél az ajánlott
  tématartományhoz tartozó és az idevágó érettségi témákban, akár
  elvontabb tartalmakra is kitérve.
\tabularnewline
\hline
&
  Alkalmazza a célnyelvi normához illeszkedő, természeteshez közelítő
  kiejtést, beszédtempót és intonációt.
\tabularnewline
\hline
&
  Írásban röviden indokolja érzéseit, gondolatait, véleményét már
  elvontabb témákban.
\tabularnewline
\hline
&
  Leír összetettebb cselekvéssort, történetet, személyes élményeket,
  elvontabb témákban.
\tabularnewline
\hline
&
  Információt vagy véleményt közlő és kérő, összefüggő feljegyzéseket,
  üzeneteket ír.
\tabularnewline
\hline
&
  Alkalmazza a formális és informális regiszterhez köthető
  sajátosságokat.
\tabularnewline
\hline
&
  Használ szövegkohéziós és figyelemvezető eszközöket.
\tabularnewline
\hline
&
  Megold változatos írásbeli, feladatokat szövegszinten.
\tabularnewline
\hline
&
  Papíralapú vagy IKT-eszközökkel segített írott projektmunkát készít
  önállóan vagy kooperatív munkaformában.
\tabularnewline
\hline
&
  Összefüggő szövegeket ír önállóan, akár elvontabb témákban.
\tabularnewline
\hline
&
  A szövegek létrehozásához nyomtatott vagy digitális segédeszközt,
  szótárt használ.
\tabularnewline
\hline
&
  Beszámol saját élményen, tapasztalaton alapuló, akár az érdeklődési
  körén túlmutató vagy elképzelt személyes eseményről a cselekmény, a
  körülmények, az érzések és gondolatok ismert nyelvi eszközökkel
  történő összetettebb, részletes és világos jellemzésével.
\tabularnewline
\hline
&
  Beszámol akár az érdeklődési körén túlmutató közügyekkel,
  szórakozással kapcsolatos eseményről a cselekmény, a körülmények, az
  érzések és gondolatok ismert nyelvi eszközökkel történő összetettebb,
  részletes és világos jellemzésével.
\tabularnewline
\hline
&
  A megfelelő szövegtípusok jellegzetességeit követi.
\tabularnewline
\hline
&
  Értelmezi a szintjének megfelelő célnyelvi, komplexebb tanári
  magyarázatokat a nyelvórákon.
\tabularnewline
\hline
&
  Megérti a célnyelvi, életkorának és érdeklődésének megfelelő hazai és
  nemzetközi hírek, események lényegét.
\tabularnewline
\hline
&
  Kikövetkezteti a szövegben megjelenő elvontabb nyelvi elemek
  jelentését az ajánlott témakörökhöz kapcsolódó témákban.
\tabularnewline
\hline
&
  Értelmezi a szövegben megjelenő összefüggéseket.
\tabularnewline
\hline
&
  Megérti, értelmezi és összefoglalja az összetettebb, a
  tématartományhoz kapcsolódó összefüggő hangzó szöveget, és értelmezi a
  szövegben megjelenő összefüggéseket.
\tabularnewline
\hline
&
  Megérti és értelmezi az összetettebb, az ajánlott témakörökhöz
  kapcsolódó összefüggő szövegeket, és értelmezi a szövegben megjelenő
  összefüggéseket.
\tabularnewline
\hline
&
  Megérti az ismeretlen nyelvi elemeket is tartalmazó hangzó szöveg
  lényegi tartalmát.
\tabularnewline
\hline
&
  Megérti a hangzó szövegben megjelenő összetettebb részin-\break formációkat.
\tabularnewline
\hline
&
  Megérti az elvontabb tartalmú hangzószövegek lényegét, valamint a
  beszélők véleményét is.
\tabularnewline
\hline
&
  Alkalmazza a hangzó szövegből nyert információt feladatok megoldása
  során.
\tabularnewline
\hline
&
  Célzottan keresi az érdeklődésének megfelelő autentikus szövegeket
  tanórán kívül is, ismeretszerzésre és szórakozásra.
\tabularnewline
\hline
&
  A tanult nyelvi elemek segítségével megérti a hangzó szöveg lényegét
  számára kevésbé ismert témákban és szituációkban is.
\tabularnewline
\hline
&
  A tanult nyelvi elemek segítségével megérti a hangzó szöveg lényegét
  akár anyanyelvi beszélők köznyelvi kommunikációjában a számára kevésbé
  ismert témákban és szituációkban is.
\tabularnewline
\hline
&
  Megérti és értelmezi a legtöbb televíziós hírműsort.
\tabularnewline
\hline
&
  Megért szokványos tempóban folyó autentikus szórakoztató és
  ismeretterjesztő tartalmakat, változatos csatornákon.
\tabularnewline
\hline
&
  Elolvas és értelmez nyelvi szintjének megfelelő irodalmi
  szöve-\break geket.
\tabularnewline
\hline
&
  Megérti és értelmezi a lényeget az ajánlott tématartományokhoz
  kapcsolódó összefüggő, akár autentikus írott szövegekben.
\tabularnewline
\hline
&
  Megérti és értelmezi az összefüggéseket és a részleteket az ajánlott
  tématartományokhoz kapcsolódó összefüggő, akár autentikus írott
  szövegekben.
\tabularnewline
\hline
&
  Értelmezi a számára ismerős, elvontabb tartalmú szövegekben megjelenő
  ismeretlen nyelvi elemeket.
\tabularnewline
\hline
&
  A szövegkörnyezet alapján kikövetkezteti a szövegben előforduló
  ismeretlen szavak jelentését.
\tabularnewline
\hline
&
  Megérti az ismeretlen nyelvi elemeket is tartalmazó írott szöveg
  tartalmát.
\tabularnewline
\hline
&
  Megérti és értelmezi az írott szövegben megjelenő összetettebb
  részinformációkat.
\tabularnewline
\hline
&
  Kiszűr konkrét információkat nyelvi szintjének megfelelő szövegből, és
  azokat összekapcsolja egyéb ismereteivel.
\tabularnewline
\hline
&
  Alkalmazza az írott szövegből nyert információt feladatok megoldása
  során.
\tabularnewline
\hline
&
  Keresi az érdeklődésének megfelelő, célnyelvi, autentikus szövegeket
  szórakozásra és ismeretszerzésre tanórán kívül is.
\tabularnewline
\hline
&
  Egyre változatosabb, hosszabb, összetettebb és elvontabb szövegeket,
  tartalmakat értelmez és használ.
\tabularnewline
\hline
&
  Részt vesz a változatos szóbeli interakciót és kognitív kihívást
  igénylő nyelvórai tevékenységekben.
\tabularnewline
\hline
&
  Szóban ad át nyelvi szintjének megfelelő célnyelvi tartalmakat valós
  nyelvi interakciót leképező szituációkban.
\tabularnewline
\hline
&
  A társalgásba aktívan, kezdeményezően és egyre magabiztosabban
  bekapcsolódik az érdeklődési körébe tartozó témák esetén vagy az
  ajánlott tématartományokon belül.
\tabularnewline
\hline
&
  Társalgást kezdeményez, a megértést fenntartja, törekszik mások
  bevonására, és szükség esetén lezárja azt az egyes tématartományokon
  belül, akár anyanyelvű beszélgetőtárs esetében is.
\tabularnewline
\hline
&
  A társalgást hatékonyan és udvariasan fenntartja, törekszik mások
  bevonására, és szükség esetén lezárja azt, akár ismeretlen
  beszélgetőtárs esetében is.
\tabularnewline
\hline
&
  Előkészület nélkül részt tud venni személyes jellegű, vagy érdeklődési
  körének megfelelő ismert témáról folytatott társal-\break gásban.
\tabularnewline
\hline
&
  Érzelmeit, véleményét változatos nyelvi eszközökkel szóban
  megfogalmazza, és arról interakciót folytat.
\tabularnewline
\hline
&
  A mindennapi élet különböző területein, a kommunikációs helyzetek
  széles körében tesz fel releváns kérdéseket információszerzés
  céljából, és válaszol megfelelő módon a hozzá intézett célnyelvi
  kérdésekre.
\tabularnewline
\hline
&
  Aktívan, kezdeményezően és magabiztosan vesz részt a változatos
  szóbeli interakciót és kognitív kihívást igénylő nyelvórai
  tevékenységekben.
\tabularnewline
\hline
&
  Társaival a kooperatív munkaformákban és a projektfeladatok megoldása
  során is törekszik a célnyelvi kommunikációra.
\tabularnewline
\hline
&
  Egyre szélesebb körű témákban, nyelvi kommunikációt igénylő
  helyzetekben reagál megfelelő módon, felhasználva általános és nyelvi
  háttértudását, ismereteit, alkalmazkodva a társadalmi normákhoz.
\tabularnewline
\hline
&
  Váratlan, előre nem kiszámítható eseményekre, jelenségekre és
  történésekre jellemzően célnyelvi eszközökkel is reagál tanórai
  szituációkban.
\tabularnewline
\hline
&
  Szóban és írásban, valós nyelvi interakciók során jó
  nyelvhelyességgel, megfelelő szókinccsel, a természeteshez közelítő
  szinten vesz részt az egyes tématartományokban és az idetartozó
  érettségi témákban.
\tabularnewline
\hline
&
  Informális és életkorának megfelelő formális írásos üzeneteket ír,
  digitális felületen is.
\tabularnewline
\hline
&
  Véleményét írásban, tanult nyelvi eszközökkel megfogalmazza, és arról
  írásban interakciót folytat.
\tabularnewline
\hline
&
  Véleményét írásban változatos nyelvi eszközökkel megfogalmazza, és
  arról interakciót folytat.
\tabularnewline
\hline
&
  Írásban átad nyelvi szintjének megfelelő célnyelvi tartalmakat valós
  nyelvi interakciók során.
\tabularnewline
\hline
&
  Írásban és szóban, valós nyelvi interakciók során jó
  nyelvhelyességgel, megfelelő szókinccsel, a természeteshez közelítő
  szinten vesz részt az egyes tématartományokban és az idetartozó
  érettségi témákban.
\tabularnewline
\hline
&
  Összetett információkat ad át és cserél.
\tabularnewline
\hline
&
  Egyénileg vagy kooperáció során létrehozott projektmunkával
  kapcsolatos kiselőadást tart önállóan, összefüggő és folyékony
  előadásmóddal, digitális eszközök segítségével, felkészülést követően.
\tabularnewline
\hline
&
  Használ célnyelvi tartalmakat tudásmegosztásra.
\tabularnewline
\hline
&
  Ismer más tantárgyi tartalmakat, részinformációkat a célnyel-\break ven.
\tabularnewline
\hline
&
  Összefoglal és lejegyzetel, írásban közvetít rövid olvasott vagy
  hallott szövegeket.
\tabularnewline
\hline
&
  Környezeti témákban a kommunikációs helyzetek széles körében
  hatékonyan ad át és cserél információt.
\tabularnewline
\hline
&
  Írott szöveget igénylő projektmunkát készít megadott olva-\break sóközönségnek.
\tabularnewline
\hline
&
  Írásban közvetít célnyelvi tartalmakat valós nyelvi interakciót
  leképező szituációkban.
\tabularnewline
\hline
&
  Tanult kifejezések alkalmazásával és az alapvető nyelvi szokások
  követésével további alapvető érzéseket fejez ki (pl. aggódást,
  félelmet, kételyt).
\tabularnewline
\hline
&
  Tanult kifejezések alkalmazásával és az alapvető nyelvi szokások
  követésével kifejez érdeklődést és érdektelenséget, szemrehányást,
  reklamálást.
\tabularnewline
\hline
&
  Tanult kifejezések alkalmazásával és az alapvető nyelvi szokások
  követésével kifejez kötelezettséget, szándékot, kívánságot,
  engedélykérést, feltételezést.
\tabularnewline
\hline
&
  Tanult kifejezések alkalmazásával és az alapvető nyelvi szokások
  követésével kifejez ítéletet, kritikát, tanácsadást.
\tabularnewline
\hline
&
  Tanult kifejezések alkalmazásával és az alapvető nyelvi szokások
  követésével kifejez segítségkérést, ajánlást és ezekre történő
  reagálást.
\tabularnewline
\hline
&
  Tanult kifejezések alkalmazásával és az alapvető nyelvi szo-\break kások
  követésével kifejez ok---okozat viszonyt vagy cél meg-\break határozását.
\tabularnewline
\hline
&
  Tanult kifejezések alkalmazásával és az alapvető nyelvi szokások
  követésével kifejez emlékezést és nem emlékezést.
\tabularnewline
\hline
&
  Összekapcsolja a mondatokat megfelelő kötőszavakkal, így követhető
  leírást ad, vagy nem kronológiai sorrendben lévő eseményeket is
  elbeszél.
\tabularnewline
\hline
&
  A kohéziós eszközök szélesebb körét alkalmazza szóbeli vagy írásbeli
  megnyilatkozásainak érthetőbb, koherensebb szöveggé szervezéséhez.
\tabularnewline
\hline
&
  Több különálló elemet összekapcsol összefüggő lineáris szem-\break pontsorrá.
\tabularnewline
\hline
&
  Képes rendszerezni kommunikációját: jelzi szándékát, kezdeményez,
  összefoglal és lezár.
\tabularnewline
\hline
&
  Használ kiemelést, hangsúlyozást, helyesbítést.
\tabularnewline
\hline
&
  Körülírással közvetíti a jelentéstartalmat, ha a megfelelő szót nem
  ismeri.
\tabularnewline
\hline
&
  Ismert témákban a szövegösszefüggés alapján kikövetkezteti az
  ismeretlen szavak jelentését, megérti az ismeretlen szavakat is
  tartalmazó mondat jelentését.
\tabularnewline
\hline
&
  Félreértéshez vezető hibáit kijavítja, ha beszédpartnere jelzi a
  problémát; a kommunikáció megszakadása esetén más stratégiát
  alkalmazva újrakezdi a mondandóját.
\tabularnewline
\hline
&
  A társalgás vagy eszmecsere menetének fenntartásához alkalmazza a
  rendelkezésére álló nyelvi és stratégiai eszközöket.
\tabularnewline
\hline
&
  Nem értés esetén képes a tartalom tisztázására.
\tabularnewline
\hline
&
  Mondanivalóját kifejti kevésbé ismerős helyzetekben is nyelvi eszközök
  széles körének használatával.
\tabularnewline
\hline
&
  A tanult nyelvi elemeket adaptálni tudja kevésbé begyakorolt
  helyzetekhez is.
\tabularnewline
\hline
&
  Szóbeli és írásbeli közlései során változatos nyelvi struktúrákat
  használ.
\tabularnewline
\hline
&
  A tanult nyelvi funkciókat és nyelvi eszköztárát életkorának megfelelő
  élethelyzetekben megfelelően alkalmazza.
\tabularnewline
\hline
&
  Szociokulturális ismeretei (például célnyelvi társadalmi szokások,
  testbeszéd) már lehetővé teszik azt, hogy társasági szempontból is
  megfelelő kommunikációt folytasson.
\tabularnewline
\hline
&
  Szükség esetén eltér az előre elgondoltaktól, és mondandóját a
  beszédpartnerekhez, hallgatósághoz igazítja.
\tabularnewline
\hline
&
  Az ismert nyelvi elemeket vizsgahelyzetben is használja.
\tabularnewline
\hline
&
  Megértést nehezítő hibáit önállóan javítani tudja.
\tabularnewline
\hline
&
  Nyelvtanulási céljai érdekében alkalmazza a tanórán kívüli
  nyelvtanulási lehetőségeket.
\tabularnewline
\hline
&
  Célzottan keresi az érdeklődésének megfelelő autentikus szövegeket
  tanórán kívül is, ismeretszerzésre és szórakozásra.
\tabularnewline
\hline
&
  Felhasználja a legfőbb célnyelvű hazai és nemzetközi híreket
  ismeretszerzésre és szórakozásra.
\tabularnewline
\hline
&
  Használ célnyelvi elemeket más tudásterületen megcélzott tartalmakból.
\tabularnewline
\hline
&
  Használ célnyelvi tartalmakat ismeretszerzésre.
\tabularnewline
\hline
&
  Használ ismeretterjesztő anyagokat nyelvtudása fejlesztésére.
\tabularnewline
\hline
&
  Hibáit az esetek többségében önállóan is képes javítani.
\tabularnewline
\hline
&
  Hibáiból levont következtetéseire többnyire épít nyelvtudásának
  fejlesztése érdekében.
\tabularnewline
\hline
&
  Egy összetettebb nyelvi feladat, projekt végéig tartó célokat tűz ki
  magának.
\tabularnewline
\hline
&
  Megfogalmaz hosszú távú nyelvtanulási célokat saját maga számára.
\tabularnewline
\hline
&
  Nyelvtanulási céljai érdekében tudatosabban foglalkozik a cél-\break
  nyelvvel.
\tabularnewline
\hline
&
  Céljai eléréséhez megtalálja és használja a megfelelő eszközöket,
  módokat.
\tabularnewline
\hline
&
  Céljai eléréséhez társaival párban és csoportban is együttmű-\break ködik.
\tabularnewline
\hline
&
  Beazonosít nyelvtanulási célokat, és ismeri az ezekhez tartozó
  nyelvtanulási és nyelvhasználati stratégiákat.
\tabularnewline
\hline
&
  Használja a nyelvtanulási és nyelvhasználati stratégiákat nyelvtudásának
  fenntartására és fejlesztésére.
\tabularnewline
\hline
&
  Hatékonyan alkalmazza a tanult nyelvtanulási és nyelvhasználati
  stratégiákat.
\tabularnewline
\hline
&
  Céljai eléréséhez önszabályozóan is dolgozik.
\tabularnewline
\hline
&
  Az első idegen nyelvből sikeres érettségit tesz legalább
  közép-\break szinten.
\tabularnewline
\hline
&
  Nyelvi haladását fel tudja mérni.
\tabularnewline
\hline
&
  Használ önértékelési módokat nyelvtudása felmérésére.
\tabularnewline
\hline
&
  Egyre tudatosabban használja az ön- és tanári értékelést vagy társai értékelését
  nyelvtudása fenntartására és fejlesztésére.
\tabularnewline
\hline
&
  Használja az ön-, tanári, vagy társai értékelését nyelvtudása
  fenntartására és fejlesztésére.
\tabularnewline
\hline
&
  Hiányosságait, hibáit felismeri, azokat egyre hatékonyabban
  kompenzálja, javítja a tanult stratégiák felhasználásával.
\tabularnewline
\hline
&
  Nyelvtanulási céljai érdekében él a valós nyelvhasználati
  lehetőségekkel.
\tabularnewline
\hline
&
  Használja a célnyelvet életkorának és nyelvi szintjének megfelelő
  aktuális témákban és a hozzájuk tartozó szituációkban.
\tabularnewline
\hline
&
  Az ismert nyelvi elemeket vizsgahelyzetben is használja.
\tabularnewline
\hline
&
  Beszéd- és írásprodukcióját tudatosan megtervezi, hiányosságait
  igyekszik kompenzálni.
\tabularnewline
\hline
&
  Nyelvi produkciójában és recepciójában önállóságot mutat, és egyre
  kevesebb korlát akadályozza.
\tabularnewline
\hline
&
  Törekszik releváns digitális tartalmak használatára beszédkészségének,
  szókincsének és kiejtésének továbbfejlesztése céljából.
\tabularnewline
\hline
&
  Digitális eszközöket és felületeket is magabiztosan használ
  nyelvtudása fejlesztésére.
\tabularnewline
\hline
&
  Digitális eszközöket és felületeket is használ a célnyelven
  ismeretszerzésre és szórakozásra.
\tabularnewline
\hline
&
  Alkalmazza a célnyelvi kultúráról megszerzett ismereteit informális
  kommunikációjában,
\tabularnewline
\hline
&
  Ismeri a célnyelvi országok történelmének és jelenének legfontosabb
  vonásait,
\tabularnewline
\hline
&
  Tájékozott a célnyelvi országok jellemzőiben és kulturális
  sajátosságaiban.
\tabularnewline
\hline
&
  Tájékozott, és alkalmazni is tudja a célnyelvi országokra jellemző
  alapvető érintkezési és udvariassági szokásokat.
\tabularnewline
\hline
&
  Ismeri és keresi a főbb hasonlóságokat és különbségeket saját
  anyanyelvi és a célnyelvi közösség szokásai, értékei, attitűdjei és
  meggyőződései között.
\tabularnewline
\hline
&
  Átadja célnyelven a magyar értékeket.
\tabularnewline
\hline
&
  Ismeri a célnyelvi és saját hazájának kultúrája közötti hasonlóságokat
  és különbségeket.
\tabularnewline
\hline
&
  Interkulturális tudatosságára építve felismeri a célnyelvi és saját
  hazájának kultúrája közötti hasonlóságokat és különbségeket, és a
  magyar értékek átadására képessé válik.
\tabularnewline
\hline
&
  Környezetének kulturális értékeit célnyelven közvetíti.
\tabularnewline
\hline
&
  Kikövetkezteti a célnyelvi kultúrákhoz kapcsolódó egyszerű, ismeretlen
  nyelvi elemeket.
\tabularnewline
\hline
&
  A célnyelvi kultúrákhoz kapcsolódó tanult nyelvi elemeket magabiztosan
  használja.
\tabularnewline
\hline
&
  Interkulturális ismeretei segítségével társasági szempontból is
  megfelelő kommunikációt folytat írásban és szóban.
\tabularnewline
\hline
&
  Felismeri a legfőbb hasonlóságokat és különbségeket az ismert nyelvi
  változatok között.
\tabularnewline
\hline
&
  Megfogalmaz főbb hasonlóságokat és különbségeket az ismert nyelvi
  változatok között.
\tabularnewline
\hline
&
  Alkalmazza a nyelvi változatokról megszerzett ismereteit informális
  kommunikációjában.
\tabularnewline
\hline
&
  Megérti a legfőbb nyelvi dialektusok egyes elemeit is tartalmazó
  szóbeli közléseket.
\tabularnewline
\hline
&
  Digitális eszközökön és csatornákon keresztül is alkot szöveget szóban
  és írásban.
\tabularnewline
\hline
&
  Digitális eszközökön és csatornákon keresztül is megérti az ismert
  témához kapcsolódó írott vagy hallott szövegeket.
\tabularnewline
\hline
&
  Digitális eszközökön és csatornákon keresztül is alkalmazza az ismert
  témához kapcsolódó írott vagy hallott szövegeket.
\tabularnewline
\hline
&
  Digitális eszközökön és csatornákon keresztül is folytat célnyelvi
  interakciót az ismert nyelvi eszközök segítségével.
\tabularnewline
\hline
&
  Digitális eszközökön és csatornákon keresztül is folytat a
  természeteshez közelítő célnyelvi interakciót az ismert nyelvi
  eszközök segítségével.
\tabularnewline
\hline
&
  Digitális eszközökön és csatornákon keresztül is megfelelő nyelvi
  eszközökkel alkot szöveget szóban és írásban.
\tabularnewline
\hline
&
  Alkalmazza az életkorának és érdeklődésének megfelelő digitális
  műfajok főbb jellemzőit.
\tabularnewline
\hline
\end{longtable}

\hypertarget{enek-zene}{%
\subsection{Ének-zene}\label{enek-zene}}

\hypertarget{evfolyamon-11}{%
\subsubsection{1-4. évfolyamon}\label{evfolyamon-11}}

\begin{itemize}
\item
  Csoportosan vagy önállóan, életkorának és hangi sajátosságainak
  \textgreater{} megfelelő hangmagasságban énekel, törekszik a tiszta
  intonációra \textgreater{} és a daloknak megfelelő tempóra.
\item
  Ismer legalább 180 gyermekdalt, magyar népdalt.
\item
  A tanult dalok, zenei részletek éneklésekor változatosan tudja
  \textgreater{} használni hangerejét a zenei kifejezésnek megfelelően.
\item
  Hangszerkíséretes dalokat énekel tanára vagy hangszeren játszó
  \textgreater{} osztálytársa kíséretével.
\item
  A tanult dalokhoz kapcsolódó játékokban, táncokban, dramatizált
  \textgreater{} előadásokban osztálytársaival aktívan részt vesz.
\item
  Fogalmi szinten megkülönbözteti az egyenletes lüktetést és a
  \textgreater{} ritmust.
\item
  Érzékeli és hangoztatja az egyenletes lüktetést, az ütemhangsúlyt a
  \textgreater{} tanult dalokban, zenei szemelvényekben.
\item
  Felismeri és hangoztatja a negyed, nyolcadpár, fél értékű
  \textgreater{} ritmusokat, a negyed és a fél értékű szünetet,
  tájékozódik a \textgreater{} 2/4-es ütemben, felismeri és használja az
  ütemvonalat, \textgreater{} záróvonalat, az ismétlőjelet.
\item
  Felismeri és hangoztatja az összetett ritmusokat (szinkópa, nyújtott
  \textgreater{} és éles ritmus), az egész értékű kottát, a pontozott
  fél értékű \textgreater{} kottát és az egyedül álló nyolcadot azok
  szüneteivel, valamint \textgreater{} tájékozódik a 4/4-es és 3/4-es
  ütemben.
\item
  Megkülönbözteti a páros és a páratlan lüktetést.
\item
  Ritmizálva szólaltat meg mondókákat, gyermekverseket.
\item
  Érzékeli a hangok magasságának változásait, különböző hangszíneket,
  \textgreater{} ellentétes dinamikai szinteket, ezeket felismeri az őt
  körülvevő \textgreater{} világ hangjaiban, tanult dalokban,
  zeneművekben.
\item
  A tanár által énekelt dalokat belső hallással követi.
\item
  Megismeri, énekli és alkalmazza a pentaton hangkészlet hangjait.
\item
  Ismeri a tanult, énekelt zenei anyaghoz köthető szolmizációs
  \textgreater{} hangokat, kézjelről énekel.
\item
  A dalokat tanári segítséggel szolmizálva énekli, kézjelekkel
  \textgreater{} mutatja.
\item
  Tanári segítséggel képes leírni és olvasni egyszerű ritmusokat,
  \textgreater{} dallamfordulatokat.
\item
  Érzékeli, hogy ugyanaz a dallamrészlet különböző magasságokban
  \textgreater{} írható, olvasható.
\item
  Reprodukálja a tanult ritmusokat mozgással, testhangszerrel,
  \textgreater{} valamint egyszerű ritmushangszerekkel.
\item
  A zeneművek befogadásában kreatívan használja képzeletét.
\item
  Adott szempontok alapján figyeli meg a hallgatott zeneművet.
\item
  Egyszerű ritmussorokat rögtönöz.
\item
  Különböző hangszíneket, hangmagasságokat, ellentétes dinamikai
  \textgreater{} szinteket hallás után megfigyel és reprodukál.
\item
  Rövid dallamsorokat rögtönöz.
\item
  Aktívan részt vesz az iskola vagy a helyi közösség hagyományos
  \textgreater{} ünnepein, tematikus projektjein.
\item
  Megismeri a gyermekdalokhoz kapcsolódó játékokat.
\item
  A tanári instrukciók alapján alakít, finomít zenei előadásmódján.
\item
  Életkori sajátosságának megfelelően képessé válik a zeneművek
  \textgreater{} érzelmi és intellektuális befogadására.
\end{itemize}

\hypertarget{evfolyamon-12}{%
\subsubsection{5-8. évfolyamon}\label{evfolyamon-12}}

\begin{itemize}
\item
  Csoportosan vagy önállóan, életkorának és hangi sajátosságainak
  \textgreater{} megfelelő hangmagasságban énekel, törekszik a tiszta
  intonációra, \textgreater{} kifejező, a zene stílusának megfelelő
  előadásra.
\item
  A zenei karaktereket differenciáltan tudja megszólaltatni a népi
  \textgreater{} vagy klasszikus stílusjegyeknek megfelelően egyszerű
  \textgreater{} többszólamúságban is.
\item
  127 új dalt ismer.
\item
  Emlékezetből énekli a himnuszt és a szózatot.
\item
  Változatosan tudja alkalmazni a tempó és dinamikai, előadási
  \textgreater{} utasításokat (tempo giusto, parlando, rubato, piano,
  mezzoforte, \textgreater{} forte).
\item
  Hangszerkíséretes dalokat énekel, tanára vagy hangszeren játszó
  \textgreater{} osztálytársa kíséretével.
\item
  A tanult dalokhoz kapcsolódó dramatizált előadásokban \textgreater{}
  osztálytársaival aktívan részt vesz.
\item
  Ismeri és alkalmazza az alapritmusok relációit (egész-, fél-,
  \textgreater{} negyedérték, nyolcadpár, fél- és negyedszünet) és az
  összetett \textgreater{} ritmusokat (szinkópa, kis- és nagy nyújtott,
  kis- és nagy éles, \textgreater{} triola, tizenhatodos ritmusok),
  grafikai jelüket és értéküket.
\item
  Különböző ritmusképleteket eltérő tempókban is reprodukál.
\item
  Érzékeli a tanult dalokban a váltakozó ütemek lüktetését.
\item
  Hallás útján megfigyeli a tanult zeneművekben a dúr és moll
  \textgreater{} hangzását, melyeket zenei karakterekhez, hangulatokhoz
  kapcsol.
\item
  Ismeri és jártasságot szerez a hétfokú skála szolmizációs hangjainak
  \textgreater{} írásában és olvasásában.
\item
  Ismeri az előjegyzésekhez kapcsolódó abszolút hangneveket.
\item
  Felismerő kottaolvasással követi és értelmezi a módosított hangok
  \textgreater{} szerepét a dalokban.
\item
  Érti a hangköz és hármashangzat fogalmát és fogalmi szinten a
  \textgreater{} hangközök harmóniaalkotó szerepét.
\item
  Fogalmi szinten ismeri a tiszta, kis- és nagy hangközöket (t1-t8),
  \textgreater{} és a fél és egész hangos építkezés logikáját.
\item
  Megnevezi és beazonosítja a kottakép alapvető elemeit, például
  \textgreater{} tempójelzés.
\item
  Ütemmutató, violin- és basszuskulcsok.
\item
  Felismerő kottaolvasással együttesen követi a ritmikai és dallami
  \textgreater{} elemeket.
\item
  Ismeretlen kotta esetében is vizuálisan nyomon tudja követni a
  \textgreater{} hallott ritmikai és dallami folyamatokat.
\item
  Érti és azonosítja a különböző formarészek zenén belüli szerepét.
\item
  Felismeri a homofon és polifon szerkesztést.
\item
  Életkori sajátosságainak megfelelően megadott szempontok
  \textgreater{} segítségével értelmezi a különböző műfajú és stílusú
  zenéket, \textgreater{} követi a lineáris és vertikális zenei
  folyamatokat.
\item
  Azonosítani tudja a zenetörténeti stílusok főbb jellemzőit, a
  \textgreater{} hozzájuk tartozó műfajokat, jellegzetes hangszereket,
  \textgreater{} hangszer-összeállításokat, ismeri történelmi,
  kulturális és \textgreater{} társadalmi hátterüket, mely segíti
  zeneértésüket.
\item
  Megfigyeli, összehasonlítja a zenét és a cselekményt, azonosítja a
  \textgreater{} témát és a szereplőket, hangulatot, műfajt, korszakot.
\item
  Azonosítja a tanult zenéket, megnevezi azok alkotóit és \textgreater{}
  keletkezésüknek ismert körülményeit.
\item
  Követni tudja a zenei elemek összetartozását, egymást kiegészítő
  \textgreater{} vagy egymástól eltérő struktúráját, a zenei
  kifejezésben betöltött \textgreater{} funkcióját.
\item
  Megfigyeli az egyes hangszerek hangszínének a hangszerjáték és az
  \textgreater{} előadásmód különbözőségéből adódó karakterét.
\item
  Kérdéseket vet fel a zenemű üzenetére és kifejezőeszközeire
  \textgreater{} vonatkozóan, többféle szempontot érvényesítve alkot
  véleményt.
\item
  Néhány mondattal összefoglalja a zenemű mondanivalóját (például
  \textgreater{} miért íródott, kinek szól, milyen gondolatokat,
  érzelmeket fejez \textgreater{} ki).
\item
  Megtalálja a kapcsolatot a zeneművek által közvetített élethelyzetek
  \textgreater{} és a saját élethelyzete között.
\item
  A tanult ritmikai és dallami elemeket alkalmazva egyszerű
  \textgreater{} ritmussorokat, dallamokat kiegészít és improvizál.
\item
  A zeneművektől inspirálódva produktívan használja képzeletét,
  \textgreater{} megfogalmazza a zene keltette érzéseit, gondolatait,
  véleményét.
\item
  A zeneművek befogadásának előkészítése során részt vesz olyan közös
  \textgreater{} kreatív zenélési formákban, melyek segítenek a
  remekművek közelébe \textgreater{} jutni, felhasználja énekhangját, az
  akusztikus környezet hangjait, \textgreater{} ütőhangszereket,
  egyszerűbb dallamhangszereket.
\item
  A zeneműveket műfajok és zenei korszakok szerint értelmezi, ismeri
  \textgreater{} történelmi, kulturális és társadalmi hátterüket.
\item
  Ismer néhány magyar és más kultúrákra jellemző hangszert.
\item
  Hangzás és látvány alapján felismeri a legelterjedtebb népi
  \textgreater{} hangszereket és hangszeregyütteseket.
\item
  Megkülönbözteti a giusto, parlando és rubato előadásmódú népdalokat.
\item
  Megkülönbözteti a műzenét, népzenét és a népdalfeldolgozásokat.
\item
  A tanult népdalokat tájegységekhez, azok ma is élő hagyományaihoz,
  \textgreater{} jellegzetes népművészeti motívumaihoz, ételeihez köti.
\item
  Aktívan részt vesz az iskola vagy a helyi közösség hagyományos
  \textgreater{} ünnepein és tematikus projektjeiben.
\item
  Képessé válik a zeneművek érzelmi és intellektuális befogadására.
\item
  Megfigyeli és felismeri az összefüggéseket zene és szövege, zenei
  \textgreater{} eszközök és zenei mondanivaló között.
\item
  A tanári instrukciók alapján alakít, finomít zenei előadásmódján.
\item
  Ismer és használ internetes zenei adatbázisokat, gyűjteményeket.
\item
  Önálló beszámolókat készít internetes és egyéb zenei adatbázisok,
  \textgreater{} gyűjtemények felhasználásával.
\end{itemize}

\hypertarget{evfolyamon-13}{%
\subsubsection{9-10. évfolyamon}\label{evfolyamon-13}}

\begin{itemize}
\item
  Csoportosan vagy önállóan, életkorának és hangi sajátosságainak
  \textgreater{} megfelelő hangmagasságban énekel, törekszik a tiszta
  intonációra.
\item
  A tanult dalokat stílusosan, kifejezően adja elő.
\item
  Emlékezetből és kottakép segítségével énekel régi és új rétegű
  \textgreater{} magyar népdalokat, más népek dalait, műdalokat,
  kánonokat.
\item
  43 új dalt ismer.
\item
  Hangszerkíséretes dalokat énekel, tanára vagy hangszeren játszó
  \textgreater{} osztálytársa kíséretével, a műdalok előadásában
  alkalmazkodni tud \textgreater{} a hangszerkísérethez.
\item
  A tanult dalokhoz kapcsolódó dramatizált előadásokban \textgreater{}
  osztálytársaival aktívan részt vesz.
\item
  Felismerő kottaolvasással követi a ritmikai és dallami elemeket.
\item
  Megnevezi és beazonosítja a kottakép alapvető elemeit, például
  \textgreater{} tempójelzés, ütemmutató, violin- és basszuskulcsok.
\item
  Ismeretlen kotta esetében is vizuálisan nyomon tudja követni a
  \textgreater{} hallott ritmikai és dallami folyamatokat.
\item
  Fogalmi szinten ismeri a hangközök harmóniaalkotó szerepét.
\item
  Érti és azonosítja a különböző formarészek zenén belüli szerepét.
\item
  Különbséget tud tenni korszakok, műfajok között.
\item
  Kezdetben tanári segítséggel, majd önállóan felismeri adott
  \textgreater{} zeneműben a stílus és korszak hatását.
\item
  A zeneműveket tanári segítséggel történelmi, földrajzi és társadalmi
  \textgreater{} kontextusba helyezi.
\item
  Kérdéseket vet fel a zenemű üzenetére és kifejezőeszközeire
  \textgreater{} vonatkozóan.
\item
  Kezdetben tanári segítséggel, majd önállóan azonosítja a zenében
  \textgreater{} megjelenő társadalmi, erkölcsi, vallási, és kulturális
  mintákat.
\item
  Megfigyeli a különböző zenei interpretációk közötti különbségeket, s
  \textgreater{} azokat véleményezi.
\item
  A dalok és a zeneművek befogadásához, azok előadásához felhasználja
  \textgreater{} eddigi ritmikai, dallami, harmóniai ismereteit.
\item
  A zeneműveket összekapcsolja élethelyzetekkel, melyeket saját élete
  \textgreater{} és környezete jelenségeire, problémáira tud
  vonatkoztatni.
\item
  Társítani tudja a zeneművekben megfogalmazott gondolatokat
  \textgreater{} hangszerelési, szerkesztési megoldásokkal, kompozíciós
  \textgreater{} technikákkal, formai megoldásokkal.
\item
  Alapszintű ismereteket alkalmaz a digitális technika zenei
  \textgreater{} felhasználásában.
\item
  Az elsajátított zenei anyagot élményszerűen alkalmazza tematikus
  \textgreater{} projektekben.
\item
  Részt vesz kreatív zenélési formákban.
\item
  Ismer magyar és más kultúrákra jellemző zenei sajátságokat, s ezeket
  \textgreater{} újonnan hallott zeneművekben is felfedezi.
\item
  Értelmezi a népdalok szövegét, mondanivalóját, megtalálja bennük
  \textgreater{} önmagát.
\item
  A tanult népdalokat tájegységekhez, azok ma is élő hagyományaihoz,
  \textgreater{} jellegzetes népművészeti motívumaihoz, ételeihez köti.
\item
  Konkrét művek példáin keresztül tanári segítséggel elkülöníti és
  \textgreater{} egymással kapcsolatba hozza az irodalomhoz és különböző
  művészeti \textgreater{} ágakhoz (film, képzőművészet, tánc) tartozó
  alkotások jellemző \textgreater{} vonásait a nyelvi, vizuális, mozgási
  és multimédiás művészetekben \textgreater{} és a zenében.
\item
  Konkrét, a tanár által választott műalkotásokon keresztül
  \textgreater{} összehasonlítja a különböző művészeti ágak
  kifejezőeszközeit, \textgreater{} például hogyan fejez ki azonos
  érzelmet, mondanivalót a zene és \textgreater{} más művészetek.
\item
  Ismeri a főbb fővárosi zenei intézményeket (például zeneakadémia,
  \textgreater{} magyar állami operaház, müpa), főbb vidéki zenei
  centrumokat \textgreater{} (például kodály központ -- pécs, kodály
  intézet -- kecskemét), \textgreater{} továbbá lakóhelye művelődési
  intézményeit.
\item
  Fogalmi szinten tájékozott a zenének a viselkedésre, fejlődésre és
  \textgreater{} agyműködésre gyakorolt hatásaiban.
\item
  Aktívan részt vesz az iskola vagy a helyi közösség hagyományos
  \textgreater{} ünnepein, tematikus projektjein.
\item
  A zeneműveket zenetörténeti kontextusba tudja helyezni, kapcsolatot
  \textgreater{} talál történelmi, irodalmi, kultúrtörténeti
  vonatkozásokkal, \textgreater{} azonosítja a műfaji jellemzőket
\item
  A hallott zeneműveket érzelmi és intellektuális módon közelíti meg,
  \textgreater{} érti és értékeli a művészeti alkotásokat.
\item
  Órai tapasztalatai és saját ismeretanyaga alapján önálló véleményt
  \textgreater{} alkot a zenemű
\item
  Mondanivalójáról, és azt az adott zeneműből vett példákkal
  \textgreater{} illusztrálja.
\item
  Értelmezi és önállóan véleményezi a zenéhez társított szöveg és a
  \textgreater{} zene kapcsolatát.
\item
  A tanult dalok üzenetét, saját életének, környezetének jelenségeire
  \textgreater{} tudja vonatkoztatni.
\item
  Önálló kutatást végez feladatai megoldásához nyomtatott és digitális
  \textgreater{} forrásokban.
\item
  Zenei ízlése az értékek mentén fejlődik.
\end{itemize}

\hypertarget{etika}{%
\subsection{Etika}\label{etika}}

\hypertarget{evfolyamon-14}{%
\subsubsection{1-4. évfolyamon}\label{evfolyamon-14}}

\begin{itemize}
\item
  Azonosítja saját helyzetét, erősségeit és fejlesztendő területeit, és
  ennek tudatában rövid távú célokat tartalmazó tervet tud kialakítani
  saját egyéni fejlődésére vonatkozóan.
\item
  Felismeri a hatékony kommunikációt segítő és akadályozó verbális és
  nonverbális elemeket és magatartásformákat.
\item
  Törekszik az érzelmek konstruktív és együttérző kifejezésmódjainak
  alkalmazására.
\item
  Felismeri az egyes cselekvésekre vonatkozó erkölcsi szabályokat,
  viselkedése szervezésénél figyelembe veszi ezeket.
\item
  Törekszik szabadságra, alkalmazkodásra, bizalomra, őszinteségre,
  tiszteletre és igazságosságra a különféle közösségekben.
\item
  Felismeri saját lelkiismerete működését.
\item
  Pozitív attitűddel viszonyul a nemzeti és a nemzetiségi hagyományok, a
  nemzeti ünnepek és az egyházi ünnepkörök iránt.
\item
  Érdeklődést tanúsít a gyermeki jogok és kötelezettségek megismerése
  iránt a családban, az iskolában és az iskolán kívüli közösségekben.
\item
  Megfogalmazza gondolatait az élet néhány fontos kérdéséről és a róluk
  tanított elképzelésekkel kapcsolatosan.
\item
  Erkölcsi fogalomkészlete tudatosodik és gazdagodik, az erkölcsi
  értékfogalmak, a segítség, önzetlenség, tisztelet, szeretet,
  felelősség, igazságosság, méltányosság, lelkiismeretesség,
  mértékletesség jelentésével.
\item
  Megismeri az élet tiszteletének és a felelősségvállalásának az elveit,
  hétköznapi szokásai alakításánál tekintettel van társas és fizikai
  környezetére.
\item
  A csoportos tevékenységek keretében felismeri és megjeleníti az
  alapérzelmeket, az alapérzelmeken kívül is felismeri és megnevezi a
  saját érzelmi állapotokat.
\item
  Felismeri, milyen tevékenységeket, helyzeteket kedvel, illetve nem
  kedvel, azonosítja saját viselkedésének jellemző elemeit.
\item
  Célokat tűz ki maga elé, és azonosítja a saját céljai eléréséhez
  szükséges főbb lépéseket.
\item
  Meggyőződése, hogy a hiányosságok javíthatók, a gyengeségek
  fejleszthetők, és ehhez teljesíthető rövid távú célokat tűz maga elé
  saját tudásának és képességeinek fejlesztése céljából.
\item
  Céljai megvalósítása során önkontrollt, siker esetén önjutalmazást
  gyakorol.
\item
  Ismeri a testi-lelki egészség fogalmát és főbb szempontjait , motivált
  a krízisek megelőzésében, és a megoldáskeresésben.
\item
  Rendelkezik a stresszhelyzetben keletkezett negatív érzelmek
  kezeléséhez saját módszerekkel.
\item
  Felismeri az őt ért bántalmazást, ismer néhány olyan segítő bizalmi
  személyt, akihez segítségért fordulhat.
\item
  Megfogalmazza a nehéz élethelyzettel (pl.:új családtag érkezése vagy
  egy családtag eltávozása) kapcsolatos érzéseit.
\item
  Felismeri annak fontosságát, hogy sorsfordító családi események
  kapcsán saját érzelmeit felismerje, megélje, feldolgozza, s azokat
  elfogadható módon kommunikálja a környezete felé.
\item
  Ismeri az életkorának megfelelő beszélgetés alapvető szabályait.
\item
  Mások helyzetébe tudja képzelni magát, és megérti a másik személy
  érzéseit.
\item
  Különbséget tesz verbális és nem verbális jelzések között.
\item
  Megkülönbözteti a felnőttekkel és társakkal folytatott társas
  helyzeteket.
\item
  Megkülönbözteti a sértő és tiszteletteljes közlési módokat fizikai és
  digitális környezetben egyaránt, barátsággá alakuló kapcsolatokat
  kezdeményez.
\item
  Saját érdekeit másokat nem bántó módon fejezi ki, az ehhez illeszkedő
  kifejezésmódokat ismeri és alkalmazza.
\item
  Alkalmazza az asszertív viselkedés elemeit konfliktushelyzetben és
  másokkal kezdeményezett interakcióban, baráti kapcsolatokat tart fenn.
\item
  Felismeri a „jó'' és „rossz'' közötti különbséget a közösen megbeszélt
  eseményekben és történetekben.
\item
  Erkölcsi érzékenységgel reagál az igazmondást, a becsületességet, a
  személyes és szellemi tulajdont, valamint az emberi méltóság
  tiszteletben tartását érintő helyzetekre fizikai és digitális
  környezetben is.
\item
  Megérti a családi szokások jelentőségét és ezek természetes
  különbözőségét (alkalmazkodik az éjszakai pihenéshez, az étkezéshez, a
  testi higiénéhez fűződő, a tanulási és a játékidőt meghatározó családi
  szokásokhoz).
\item
  Megérti az ünneplés jelentőségét, elkülöníti a családi, a nemzeti, az
  állami és egyéb ünnepeket, és az egyházi ünnepköröket, aktív
  résztvevője a közös ünnepek előkészületeinek.
\item
  Szerepet vállal iskolai rendezvényeken, illetve azok előkészítésében,
  vagy iskolán belül szervezett szabadidős programban vesz részt.
\item
  Képes azonosítani a szeretet és az elfogadás jelzéseit.
\item
  A közvetlen lakóhelyéhez kapcsolódó nevezetességeket ismeri, ezekről
  információkat gyűjt fizikai és digitális környezetben, társaival
  együtt meghatározott formában bemutatót készít.
\item
  Érdeklődést mutat magyarország történelmi emlékei iránt, ismer közülük
  néhányat.
\item
  Az életkorához illeszkedő mélységben ismeri a nemzeti, az állami
  ünnepek, egyházi ünnepkörök jelentését, a hozzájuk kapcsolódó
  jelképeket.
\item
  A lakóhelyén élő nemzetiségek tagjai, hagyományai iránt nyitott,
  ezekről információkat gyűjt fizikai és digitális környezetben.
\item
  Tájékozott a testi és érzelmi biztonságra vonatkozó gyermeki jogokról.
\item
  Tájékozott a képességek kibontakoztatását érintő gyermeki jogokról,
  ennek családi, iskolai és iskolán kívüli következményeiről, a
  gyermekek kötelességeiről.
\item
  Irodalmi szemelvények alapján példákat azonosít igazságos és
  igazságtalan cselekedetekre, saját élmény alapján példát hoz ilyen
  helyzetekre, valamint részt vesz ezek megbeszélésében, tanítói
  vezetéssel.
\item
  Magatartásával az igazságosság, a fair play elveinek betartására
  törekszik, és ezáltal igyekszik mások bizalmát elnyerni.
\item
  A bibliai szövegekre támaszkodó történetek megismerése alapján
  értelmezi, milyen vallási eseményhez kapcsolódik egy-egy adott ünnep.
\item
  Más vallások ünnepei közül ismer néhányat.
\item
  Azonosítja az olvasott vagy hallott bibliai tanításokban, mondákban,
  mesékben a megjelenő együttélési szabályokat.
\item
  A bibliai történetekben megnyilvánuló igazságos és megbocsátó
  magatartásra saját életéből példákat hoz, vagy megkezdett történetet a
  megadott szempont szerint fejez be.
\item
  Ismer néhány kihalt vagy kihalófélben lévő élőlényt, tájékozott a
  jelenség magyarázatában, és ezekről információt gyűjt fizikai és
  digitális környezetben is.
\item
  A szabályok jelentőségét különböző kontextusokban azonosítja és
  társaival megvitatja, a szabályszegés lehetséges következményeit
  megfogalmazza.
\item
  Játékvásárlási szokásaiban példát hoz olyan elemekre, amelyek révén
  figyelembe vehetők a környezetvédelmi szempontok, és felhívja társai
  figyelmét is ezekre.
\item
  A naponta használt csomagolóeszközök kiválasztásában megindokolja,
  hogy milyen elvek alkalmazása támogatja a környezetvédelmi szempontok
  érvényesülését, és ezekre társai figyelmét is felhívja.
\end{itemize}

\hypertarget{evfolyamon-15}{%
\subsubsection{5-8. évfolyamon}\label{evfolyamon-15}}

\begin{itemize}
\item
  Fogalmi rendszerében kialakul az erkölcsi jó és rossz, mint minősítő
  kategória, az erkölcsi dilemma és az erkölcsi szabályok fogalma,
  ezeket valós vagy fiktív példákhoz tudja kötni.
\item
  Erkölcsi szempontok alapján vizsgálja személyes helyzetét; etikai
  alapelvek és az alapvető jogok szerint értékeli saját és mások
  élethelyzetét, valamint néhány társadalmi problémát.
\item
  Környezetében felismeri azokat a személyeket, akiknek tevékenysége
  példaként szolgálhat mások számára; ismeri néhány kiemelkedő
  személyiség életművét, és elismeri hozzájárulását a tudomány, a
  technológia, a művészetek gyarapításához, nemzeti és európai
  történelmünk alakításához.
\item
  Azonosítja saját szerepét a családi viszonyrendszerben, felismeri
  azokat az értékeket, amelyek a család összetartozását, harmonikus
  működését és az egyén egészséges fejlődését biztosítják.
\item
  Felismeri saját és mások érzelmi állapotát, az adott érzelmi állapotot
  kezelő viselkedést tud választani; a helyzethez igazodó társas
  konfliktus-megoldási eljárás alkalmazására törekszik.
\item
  Elfogadó attitűdöt tanúsít az eltérő társadalmi, kulturális helyzetű
  vagy különleges bánásmódot igénylő tanulók iránt.
\item
  Véleményt formál a zsidó keresztény, keresztyén értékrenden alapuló
  vallások erkölcsi értékeiről, feltárja, hogy hogyan jelennek meg ezek
  az értékek az emberi viselkedésben a közösségi szabályokban.
\item
  A személyes életben is megvalósítható tevékenységeket végez, ami
  összhangban van a teremtett rend megőrzésével, a fenntartható jövővel.
\item
  Strukturált önmegfigyelésre alapozva megismeri személyiségének egyes
  jellemzőit, saját érdeklődési körét, speciális pályaérdeklődését.
\item
  Megismeri az identitás fogalmát és jellemzőit, azonosítja saját
  identitásának néhány elemét.
\item
  Különbséget tesz a valóságos és a virtuális identitás között, megérti
  a virtuális identitás jellemzőit.
\item
  Megfelelő döntéseket hoz arról, hogy az online térben milyen
  információkat oszthat meg önmagáról.
\item
  Reflektív tanulási gyakorlatot alakít ki, önálló tanulási feladatot
  kezdeményez.
\item
  Dramatikus eszközökkel megjelenített helyzetekben különböző érzelmi
  állapotok által vezérelt viselkedést eljátszik, és ennek a másik
  személyre tett hatását a csoporttal közösen megfogalmazza.
\item
  Valósághűen elmondja, hogy a saját érzelmi állapotai milyen hatást
  gyakorolnak a társas kapcsolatai alakítására, a tanulási
  tevékenységére.
\item
  Kérdőívek használatával felismeri pályaérdeklődését és továbbtanulási
  céljait.
\item
  Ismer testi és mentális egészséget őrző tevékenységeket és felismeri a
  saját egészségét veszélyeztető hatásokat; megfogalmazza saját intim
  terének határait.
\item
  Megismer olyan mintákat és lehetőségeket, amelyek segítségével a
  krízishelyzetek megoldhatók, és tudja, hogy adott esetben hová
  fordulhat segítségért.
\item
  Azonosítja a valós és virtuális térben történő zaklatások fokozatait
  és módjait, van terve a zaklatások elkerülésére, kivédésére, tudja,
  hogy hová fordulhat segítségért.
\item
  Képes a saját véleményétől eltérő véleményekhez tisztelettel
  viszonyulni, a saját álláspontja mellett érvelni, konszenzusra
  törekszik.
\item
  Felismeri a konfliktus kialakulására utaló jelzéseket, vannak
  megoldási javaslatai a konfliktusok békés megoldására.
\item
  Azonosítja a csoportban elfoglalt helyét és szerepét, törekszik a
  személyiségének legjobban megfelelő feladatok ellátására.
\item
  Törekszik mások helyzetének megértésére, felismeri a mások érzelmi
  állapotára és igényeire utaló jelzéseket, a fizikai és a digitális
  környezetben egyaránt.
\item
  Nyitott és segítőkész a nehéz helyzetben lévő személyek iránt.
\item
  Értelmezi a szabadság és önkorlátozás, a tolerancia és a szeretet
  megjelenését és határait egyéni élethelyzeteiben.
\item
  Azonosítja a számára fontos közösségi értékeket, indokolja, hogy ezek
  milyen szerepet játszanak a saját életében.
\item
  Azonosítja, értékeli az etikus és nem etikus cselekvések lehetséges
  következményeit.
\item
  A csoporthoz való csatlakozás vagy az onnan való kiválás esetén
  összeveti a csoportnormákat és saját értékrendjét, mérlegeli az őt érő
  előnyöket és hátrányokat.
\item
  Azonosítja és összehasonlítja a családban betöltött szerepeket és
  feladatokat.
\item
  Érzékeli a családban előforduló, bizalmat érintő konfliktusos
  területeket.
\item
  Felismeri saját családjának viszonyrendszerét, átéli a családot
  összetartó érzelmeket és társas lelkületi értékeket, a különböző
  generációk családot összetartó szerepét.
\item
  Megfogalmazza, hogy a szeretetnek, hűségnek, elkötelezettségnek,
  bizalomnak, tiszteletnek milyen szerepe van a társas lelkületi
  kapcsolatokban.
\item
  Fizikai vagy digitális környezetben információt gyűjt és megosztja
  tudását a sport, a technika vagy a művészetek területén a nemzet és
  európa kultúráját meghatározó kiemelkedő személyiségekről és
  tevékenységükről.
\item
  Ismeri a nemzeti identitást meghatározó kulturális értékeket,
  indokolja miért fontos ezek megőrzése.
\item
  Azonosítja a nemzeti és az európai értékek közös jellemzőit, az
  európai kulturális szellemiség, értékrendszer meghatározó elemeit.
\item
  Összefüggéseket gyűjt a keresztény, keresztyén vallás és az európai,
  nemzeti értékvilágról, a közös jelképekről, szimbólumokról, az egyházi
  ünnepkörökről.
\item
  Ismeri a rá vonatkozó gyermekjogokat, az ezeket szabályozó főbb
  dokumentumokat, értelmezi kötelezettségeit, részt vesz a
  szabályalkotásban.
\item
  Részt vesz a különleges bánásmódot igénylő tanulók megértését segítő
  feladatokban, programokban, kifejti saját véleményét.
\item
  Értelmezi a norma- és szabályszegés következményeit, és etikai
  kérdéseket vet fel velük kapcsolatban.
\item
  Azonosítja az egyéni, családi és a társadalmi boldogulás, érvényesülés
  feltételeit.
\item
  Életkorának megfelelő szinten értelmezi a családi élet mindennapjait
  befolyásoló fontosabb jogszabályok nyújtotta lehetőségeket (például
  családi pótlék, családi adókedvezmény, gyermekvédelmi támogatás).
\item
  Feltárja, hogyan jelenik meg a hétköznapok során a vallás emberi
  életre vonatkozó tanítása.
\item
  Értelmezi a szeretetnek, az élet tisztelete elvének a kultúrára
  gyakorolt hatását.
\item
  Egyéni cselekvési lehetőségeket fogalmaz meg az erkölcsi értékek
  érvényesítésére.
\item
  Saját életét meghatározó világnézeti meggyőződés birtokában a
  kölcsönös tolerancia elveit valósítja meg a tőle eltérő nézetű
  személyekkel való kapcsolata során.
\item
  Folyamatosan frissíti az emberi tevékenység környezetre gyakorolt
  hatásaival kapcsolatos ismereteit fizikai és digitális környezetében,
  mérlegelő szemlélettel vizsgálja a technikai fejlődés lehetőségeit.
\item
  Megismeri és véleményezi a természeti erőforrások felhasználására, a
  környezetszennyeződés, a globális és társadalmi egyenlőtlenségek
  problémájára vonatkozó etikai felvetéseket.
\item
  Értelmezi a teremtett rend, világ, a fenntarthatóság összefüggéseit,
  az emberiség ökológiai cselekvési lehetőségeit.
\end{itemize}

\hypertarget{fizika}{%
\subsection{Fizika}\label{fizika}}

\hypertarget{evfolyamon-16}{%
\subsubsection{9-10. évfolyamon}\label{evfolyamon-16}}

\begin{itemize}
\item
  Azonosítani tudja a fizika körébe tartozó problémákat, a természeti és
  technikai környezet leírására a megfelelő fizikai mennyiségeket
  használja, a jelenségek értelmezése során a megismert fizikai elveket
  alkalmazza.
\item
  A megismert jelenségek kapcsán egyszerű számolásokat végezzen,
  grafikus formában megfogalmazott feladatokat oldjon meg, egyszerű
  méréseket, megfigyeléseket tervezzen, végrehajtson, kiértékeljen,
  ábrákat készítsen.
\item
  Tudjon információkat keresni a vizsgált tudományterülethez
  kapcsolódóan a rendelkezésre álló információforrásokban, elektronikus
  adathordozókon, nyitottan közelítsen az újdonságokhoz folyamatos
  érdeklődés mellett.
\item
  Ismerje meg a fenntartható fejlődés fogalmát és fizikai vonatkozásait,
  elősegítve ezzel a természet és környezet, illetve a fenntartható
  fejlődést segítő életmód iránti felelősségteljes elköteleződés
  kialakulását.
\item
  Felismerjen és megértsen a természettudományok különböző területei
  között fennálló kapcsolatokat konkrét jelenségek kapcsán.
\item
  Eligazodjon a közvetlen természeti és technikai környezetükben,
  illetve a tanultakat alkalmazni tudja a mindennapokban használt
  eszközök működési elvének megértésére, a biztonságos eszközhasználat
  elsajátítására.
\item
  Felismerje az ember és környezetének kölcsönhatásából fakadó előnyöket
  és problémákat, tudatosítsa az emberiség felelősségét a környezet
  megóvásában.
\item
  Fel tudja tárni a megfigyelt jelenségek ok-okozati hátterét.
\item
  Képessé váljon univerzumunkat és az embert kölcsönhatásában szemlélni,
  az emberiség fejlődéstörténetét, jelenét és jövőjét és az univerzum
  történetét összekapcsolni.
\item
  Tisztába kerüljön azzal, hogy a tudomány művelése alapvetően
  társadalmi jelenség.
\item
  Megtanuljon különbséget tenni a valóság és az azt leképező
  természettudományos modellek, leírások és világról alkotott képek
  között.
\item
  Felismerje, hogy a természet egységes egész, szétválasztását
  résztudományokra csak a jobb kezelhetőség, áttekinthetőség indokolja,
  a fizika törvényei általánosak, amelyek a kémia, a biológia, a
  földtudományok és az alkalmazott műszaki tudományok területén is
  érvényesek.
\item
  Fizikai jelenségek megfigyelése, egyszerű értelmezése
\item
  Mozgások a környezetünkben, a közlekedés
\item
  A levegő, a víz, a szilárd anyagok
\item
  Fontosabb mechanikai, hőtani, elektromos és optikai eszközeink
  \textgreater{} működésének alapjai, fűtés és világítás a háztartásban
\item
  Az energia megjelenési formái, megmaradása, energiatermelés és
  \textgreater{} felhasználás
\item
  A föld, a naprendszer és a világegyetem, a föld jövője, megóvása
\item
  A fizikai jelenségek megfigyelése, modellalkotás, értelmezés,
  tudományos érvelés
\item
  Mozgások a környezetünkben, a közlekedés kinematikai és dinamikai
  vonatkozásai
\item
  A halmazállapotok és változásuk, a légnemű, folyékony és szilárd
  anyagok tulajdonságai
\item
  Az emberi test fizikájának elemei
\item
  Fontosabb mechanikai, hőtani és elektromos eszközeink működésének
  alapjai, fűtés és világítás a háztartásban
\item
  A hullámok szerepe a képek és hangok rögzítésében, továbbításában
\item
  Az energia megjelenési formái, megmaradása, energiatermelés és
  -felhasználás
\item
  Az atom szerkezete, fénykibocsátás, radioaktivitás
\item
  A föld, a naprendszer és a világegyetem, a föld jövője, megóvása, az
  űrkutatás eredményei
\item
  Ismeri a helyét a világegyetemben, látja a világegyetem időbeli
  \textgreater{} fejlődését, lehetséges jövőjét, az emberiség és a
  világegyetem \textgreater{} kapcsolatának kulcskérdéseit.
\item
  Tisztában van azzal, hogy a fizika átfogó törvényeket ismer fel,
  \textgreater{} melyek alkalmazhatók jelenségek értelmezésére, egyes
  események \textgreater{} minőségi és mennyiségi előrejelzésére.
\item
  Felismeri, hogyan jelennek meg a fizikai ismeretek a mindennapi
  \textgreater{} tevékenységek során, valamint a gyakran használt
  technikai \textgreater{} eszközök működésében.
\item
  Ismeri a világot leíró legfontosabb természeti jelenségeket, az
  \textgreater{} azokat leíró fizikai mennyiségeket, azok jelentését,
  jellemző \textgreater{} nagyságrendjeit.
\item
  Gyakorlati oldalról ismeri a tudományos megismerési folyamatot:
  \textgreater{} megfigyel, mér, adatait összeveti az egyszerű
  modellekkel, korábbi \textgreater{} ismereteivel. ennek alapján
  következtet, megerősít, cáfol.
\item
  Egyszerű fizikai rendszerek esetén a lényeges elemeket a
  \textgreater{} lényegtelenektől el tudja választani, az egyszerűbb
  számításokat \textgreater{} el tudja végezni és a helyes logikai
  következtetéseket le tudja \textgreater{} vonni, illetve táblázatokat,
  ábrákat, grafikonokat tud értelmezni.
\item
  Tájékozott a földünket és környezetünket fenyegető globális
  \textgreater{} problémákban, ismeri az emberi tevékenység szerepét
  ezek \textgreater{} kialakulásában.
\item
  Látja a fizikai ismeretek bővülése és a társadalmi-gazdasági
  \textgreater{} folyamatok, történelmi események közötti kapcsolatot.
\item
  Tud önállóan fizikai témájú ismeretterjesztő szövegeket olvasni, a
  \textgreater{} lényeget kiemelni, el tudja különíteni a számára
  világos, valamint \textgreater{} a nem érthető, további magyarázatra
  szoruló részeket.
\item
  Tudományos ismereteit érveléssel meg tudja védeni, vita során ki
  \textgreater{} tudja fejteni véleményét, érveit és ellenérveit,
  mérlegelni tudja \textgreater{} egy elképzelés tudományos
  megalapozottságát.
\item
  Egyszerű méréseket, kísérleteket végez, az eredményeket rögzíti.
\item
  Fizikai kísérleteket önállóan is el tud végezni.
\item
  Ismeri a legfontosabb mértékegységek jelentését, helyesen használja a
  mértékegységeket számításokban, illetve az eredmények összehasonlítása
  során.
\item
  A mérések és a kiértékelés során alkalmazza a rendelkezésre álló
  számítógépes eszközöket, programokat.
\item
  Megismételt mérések segítségével, illetve a mérés körülményeinek
  ismeretében következtet a mérés eredményét befolyásoló tényezőkre.
\item
  Egyszerű, a megértést segítő számolási feladatokat old meg,
  táblázatokat, ábrákat, grafikonokat értelmez, következtetést von le,
  összehasonlít.
\item
  El tudja választani egyszerű fizikai rendszerek esetén a lényeges
  elemeket a lényegtelenektől.
\item
  Gyakorlati oldalról ismeri a tudományos megismerési folyamatot:
  megfigyelés, mérés, a tapasztalatok, mérési adatok rögzítése,
  rendszerezése, ezek összevetése valamilyen egyszerű modellel vagy
  matematikai összefüggéssel, a modell (összefüggés) továbbfejlesztése.
\item
  Tudja, hogyan születnek az elismert, új tudományos felismerések,
  ismeri a tudományosság kritériumait.
\item
  Tisztában van azzal, hogy a fizika átfogó törvényeket ismer fel,
  melyek alkalmazhatók jelenségek értelmezésére, egyes események
  minőségi és mennyiségi előrejelzésére.
\item
  Felismeri a tudomány által vizsgálható jelenségeket, azonosítani tudja
  a tudományos érvelést, kritikusan vizsgálja egy elképzelés tudományos
  megalapozottságát.
\item
  Ismeri a fizika főbb szakterületeit, néhány új eredményét.
\item
  Kialakult véleményét mérési eredményekkel, érvekkel támasztja alá.
\item
  Tisztában van a különböző típusú erőművek használatának előnyeivel és
  környezeti kockázatával.
\item
  Érti az atomreaktorok működésének lényegét, a radioaktív hulladékok
  elhelyezésének problémáit.
\item
  Ismeri a környezet szennyezésének leggyakoribb forrásait, fizikai
  vonatkozásait.
\item
  Ismeri a megújuló és a nem megújuló energiaforrások használatának és
  az energia szállításának legfontosabb gyakorlati kérdéseit.
\item
  Az emberiség energiafelhasználásával kapcsolatos adatokat gyűjt, az
  információkat szemléletesen mutatja be.
\item
  Tudja, hogy a föld elsődleges energiaforrása a nap, ismeri a
  napenergia felhasználási lehetőségeit, a napkollektor és a napelem
  mibenlétét, a közöttük lévő különbséget.
\item
  Átlátja az ózonpajzs szerepét a földet ért ultraibolya sugárzással
  kapcsolatban.
\item
  Tisztában van az éghajlatváltozás kérdésével, az üvegházhatás
  jelenségével a természetben, a jelenség erőssége és az emberi
  tevékenység kapcsolatával.
\item
  Ismeri az űrkutatás történetének főbb fejezeteit, jövőbeli
  lehetőségeit, tervezett irányait.
\item
  Tisztában van az űrkutatás ipari-technikai civilizációra gyakorolt
  hatásával, valamint az űrkutatás tágabb értelemben vett céljaival
  (értelmes élet keresése, új nyersanyagforrások felfedezése).
\item
  Néhány konkrét példa alapján felismeri a fizika tudásrendszerének
  fejlődése és a társadalmi-gazdasági folyamatok, történelmi események
  közötti kapcsolatot.
\item
  El tudja helyezni lakóhelyét a földön, a föld helyét a naprendszerben,
  a naprendszer helyét a galaxisunkban és az univerzumban.
\item
  Átlátja az emberiség és a világegyetem kapcsolatának kulcskérdéseit.
\item
  Adatokat gyűjt és dolgoz fel a legismertebb fizikusok életével,
  tevékenységével, annak gazdasági, társadalmi hatásával, valamint
  emberi vonatkozásaival kapcsolatban (galileo galilei, michael faraday,
  james watt, eötvös loránd, marie curie, ernest rutherford, niels bohr,
  albert einstein, szilárd leó, wigner jenő, teller ede).
\item
  Ismeri a legfontosabb természeti jelenségeket (például légköri
  jelenségek, az égbolt változásai, a vízzel kapcsolatos jelenségek),
  azok megfelelően egyszerűsített, a fizikai mennyiségeken és
  törvényeken alapuló magyarázatait.
\item
  Tudja, hogyan jönnek létre a természet színei, és hogyan észleljük
  azokat.
\item
  Ismeri a villámok veszélyét, a villámhárítók működését, a helyes
  magatartást zivataros, villámcsapás-veszélyes időben.
\item
  Ismeri a légnyomás változó jellegét, a légnyomás és az időjárás
  kapcsolatát.
\item
  Érti a legfontosabb közlekedési eszközök -- gépjárművek, légi és vízi
  járművek -- működésének fizikai elveit.
\item
  Átlátja a korszerű lakások és házak hőszabályozásának fizikai
  kérdéseit (fűtés, hűtés, hőszigetelés).
\item
  Ismeri a háztartásban használt fontosabb elektromos eszközöket, az
  elektromosság szerepét azok működésében, szemléletes képe van a
  váltakozó áramról.
\item
  Tisztában van a konyhai tevékenységek (melegítés, főzés, hűtés)
  fizikai vonatkozásaival.
\item
  Átlátja a jelen közlekedése, közlekedésbiztonsága szempontjából
  releváns gyakorlati ismereteket, azok fizikai hátterét.
\item
  Ismeri az egyszerű gépek elvének megjelenését a hétköznapokban,
  mindennapi eszközeinkben.
\item
  Tisztában van az aktuálisan használt világító eszközeink működési
  elvével, energiafelhasználásának sajátosságaival, a korábban
  alkalmazott megoldásokhoz képesti előnyeivel.
\item
  Ismeri a mindennapi életben használt legfontosabb elektromos
  energiaforrásokat, a gépkocsi-, mobiltelefon-akkumulátorok
  legfontosabb jellemzőit.
\item
  Ismeri az elektromágneses hullámok szerepét az információ- (hang-,
  kép-) átvitelben, ismeri a mobiltelefon legfontosabb tartozékait (sim
  kártya, akkumulátor stb.), azok kezelését, funkcióját.
\item
  Ismeri a digitális fényképezőgép működésének elvét.
\item
  Tisztában van az elektromágneses hullámok frekvenciatartományaival, a
  rádióhullámok, mikrohullámok, infravörös hullámok, a látható fény, az
  ultraibolya hullámok, a röntgensugárzás, a gamma-sugárzás gyakorlati
  felhasználásával.
\item
  Tisztában van az elektromos áram veszélyeivel, a veszélyeket csökkentő
  legfontosabb megoldásokkal (gyerekbiztos csatlakozók, biztosíték,
  földvezeték szerepe).
\item
  Ismeri az elektromos fogyasztók használatára vonatkozó balesetvédelmi
  szabályokat.
\item
  Ismeri az elektromos hálózatok kialakítását a lakásokban, épületekben,
  az elektromos kapcsolási rajzok használatát.
\item
  Érti a generátor, a motor és a transzformátor működési elvét,
  gyakorlati hasznát.
\item
  Ismeri az emberi hangérzékelés fizikai alapjait, a hang mint hullám
  jellemzőit, keltésének eljárásait.
\item
  Átlátja a húros hangszerek és a sípok működésének elvét, az ultrahang
  szerepét a gyógyászatban, ismeri a zajszennyezés fogalmát.
\item
  Ismeri az emberi szemet mint képalkotó eszközt, a látás mechanizmusát,
  a gyakori látáshibák (rövid- és távollátás) okát, a szemüveg és a
  kontaktlencse jellemzőit, a dioptria fogalmát.
\item
  Ismeri a radioaktív izotópok néhány orvosi alkalmazását (nyomjelzés).
\item
  Tisztában van az elektromos áram élettani hatásaival, az emberi test
  áramvezetési tulajdonságaival, az idegi áramvezetés jelenségével.
\item
  Ismeri a szervezet energiaháztartásának legfontosabb tényezőit, az
  élelmiszerek energiatartalmának szerepét.
\item
  Átlátja a gyakran alkalmazott orvosdiagnosztikai vizsgálatok, illetve
  egyes kezelések fizikai megalapozottságát, felismeri a sarlatán,
  tudományosan megalapozatlan kezelési módokat.
\item
  Helyesen használja az út, a pálya és a hely fogalmát, valamint a
  sebesség, átlagsebesség, pillanatnyi sebesség, gyorsulás, elmozdulás
  fizikai mennyiségeket a mozgás leírására.
\item
  Tud számításokat végezni az egyenes vonalú egyenletes mozgás esetében:
  állandó sebességű mozgások esetén a sebesség ismeretében meghatározza
  az elmozdulást, a sebesség nagyságának ismeretében a megtett utat, a
  céltól való távolság ismeretében a megérkezéshez szükséges időt.
\item
  Ismeri a szabadesés jelenségét, annak leírását, tud esésidőt számolni,
  mérni, becsapódási sebességet számolni.
\item
  Egyszerű számításokat végez az állandó gyorsulással mozgó testek
  esetében.
\item
  Ismeri a periodikus mozgásokat (ingamozgás, rezgőmozgás) jellemző
  fizikai mennyiségeket, néhány egyszerű esetben tudja mérni a
  periódusidőt, megállapítani az azt befolyásoló tényezőket.
\item
  Ismeri az egyenletes körmozgást leíró fizikai mennyiségeket
  (pályasugár, kerületi sebesség, fordulatszám, keringési idő,
  centripetális gyorsulás), azok jelentését, egymással való kapcsolatát.
\item
  Érti, hogyan alakulnak ki és terjednek a mechanikai hullámok, ismeri a
  hullámhossz és a terjedési sebesség fogalmát.
\item
  Egyszerű esetekben kiszámolja a testek lendületének nagyságát,
  meghatározza irányát.
\item
  Egyszerűbb esetekben alkalmazza a lendület-megmaradás törvényét,
  ismeri ennek általános érvényességét.
\item
  Tisztában van az erő mint fizikai mennyiség jelentésével,
  mértékegységével, ismeri a newtoni dinamika alaptörvényeit, egyszerűbb
  esetekben alkalmazza azokat a gyorsulás meghatározására, a korábban
  megismert mozgások értelmezésére.
\item
  Egyszerűbb esetekben kiszámolja a mechanikai kölcsönhatásokban fellépő
  erőket (nehézségi erő, nyomóerő, fonálerő, súlyerő, súrlódási erők,
  rugóerő), meghatározza az erők eredőjét.
\item
  Ismeri a bolygók, üstökösök mozgásának jellegzetességeit.
\item
  Tudja, mit jelentenek a kozmikus sebességek (körsebesség, szökési
  sebesség).
\item
  Érti a testek súlya és a tömege közötti különbséget, a súlytalanság
  állapotát, a gravitációs mező szerepét a gravitációs erő
  közvetítésében.
\item
  Érti a tömegvonzás általános törvényét, és azt, hogy a gravitációs erő
  bármely két test között hat.
\item
  Ismeri a mechanikai munka fogalmát, kiszámításának módját,
  mértékegységét, a helyzeti energia, a mozgási energia, a rugalmas
  energia, a belső energia fogalmát.
\item
  Konkrét esetekben alkalmazza a munkatételt, a mechanikai energia
  megmaradásának elvét a mozgás értelmezésére, a sebesség kiszámolására.
\item
  Néhány egyszerűbb, konkrét esetben (mérleg, libikóka) a
  forgatónyomatékok meghatározásának segítségével vizsgálja a testek
  egyensúlyi állapotának feltételeit, összeveti az eredményeket a
  megfigyelések és kísérletek tapasztalataival.
\item
  Ismeri a celsius- és az abszolút hőmérsékleti skálát, a gyakorlat
  szempontjából nevezetes néhány hőmérsékletet, a termikus kölcsönhatás
  jellemzőit.
\item
  Gyakorlati példákon keresztül ismeri a hővezetés, hőáramlás és
  hősugárzás jelenségét, a hőszigetelés lehetőségeit, ezek
  anyagszerkezeti magyarázatát.
\item
  Értelmezi az anyag viselkedését hőközlés során, tudja, mit jelent az
  égéshő, a fűtőérték és a fajhő.
\item
  Tudja a halmazállapot-változások típusait (párolgás, forrás,
  lecsapódás, olvadás, fagyás, szublimáció).
\item
  Tisztában van a halmazállapot-változások energetikai viszonyaival,
  anyagszerkezeti magyarázatával, tudja, mit jelent az olvadáshő,
  forráshő, párolgáshő, egyszerű számításokat végez a
  halmazállapot-változásokat kísérő hőközlés meghatározására.
\item
  Ismeri a hőtágulás jelenségét, jellemző nagyságrendjét.
\item
  Ismeri a hőtan első főtételét, és tudja alkalmazni néhány egyszerűbb
  gyakorlati szituációban (palackba zárt levegő, illetve állandó nyomású
  levegő melegítése).
\item
  Tisztában van a megfordítható és nem megfordítható folyamatok közötti
  különbséggel.
\item
  Ismeri a víz különleges tulajdonságait (rendhagyó hőtágulás, nagy
  olvadáshő, forráshő, fajhő), ezek hatását a természetben, illetve
  mesterséges környezetünkben.
\item
  Ismeri az időjárás elemeit, a csapadékformákat, a csapadékok
  kialakulásának fizikai leírását.
\item
  Ismeri a nyomás, hőmérséklet, páratartalom fogalmát, a levegő mint
  ideális gáz viselkedésének legfontosabb jellemzőit. egyszerű
  számításokat végez az állapothatározók megváltozásával kapcsolatban.
\item
  Tisztában van a repülés elvével, a légellenállás jelenségével.
\item
  Ismeri a hidrosztatika alapjait, a felhajtóerő fogalmát, hétköznapi
  példákon keresztül értelmezi a felemelkedés, elmerülés, úszás, lebegés
  jelenségét, tudja az ezt meghatározó tényezőket, ismeri a
  jelenségkörre épülő gyakorlati eszközöket.
\item
  Ismeri az elektrosztatikus alapjelenségeket (dörzselektromosság,
  töltött testek közötti kölcsönhatás, földelés), ezek gyakorlati
  alkalmazásait.
\item
  Átlátja, hogy az elektromos állapot kialakulása a töltések egyenletes
  eloszlásának megváltozásával van kapcsolatban.
\item
  Érti coulomb törvényét, egyszerű esetekben alkalmazza elektromos
  töltéssel rendelkező testek közötti erő meghatározására.
\item
  Tudja, hogy az elektromos kölcsönhatást az elektromos mező közvetíti.
\item
  Tudja, hogy az áram a töltött részecskék rendezett mozgása, és ez
  alapján szemléletes elképzelést alakít ki az elektromos áramról.
\item
  Gyakorlati szinten ismeri az egyenáramok jellemzőit, a feszültség,
  áramerősség és ellenállás fogalmát.
\item
  Érti ohm törvényét, egyszerű esetekben alkalmazza a feszültség,
  áramerősség, ellenállás meghatározására, tudja, hogy az ellenállás
  függ a hőmérséklettől.
\item
  Ki tudja számolni egyenáramú fogyasztók teljesítményét, az általuk
  felhasznált energiát.
\item
  Ismeri az egyszerű áramkör és egyszerűbb hálózatok alkotórészeit,
  felépítését.
\item
  Értelmezni tud egyszerűbb kapcsolási rajzokat, ismeri kísérleti
  vizsgálatok alapján a soros és a párhuzamos kapcsolások legfontosabb
  jellemzőit.
\item
  Elektromágnes készítése közben megfigyeli és alkalmazza, hogy az
  elektromos áram mágneses mezőt hoz létre.
\item
  Megmagyarázza, hogyan működnek az általa megfigyelt egyszerű
  felépítésű elektromos motorok: a mágneses mező erőt fejt ki az árammal
  átjárt vezetőre.
\item
  Ismeri az elektromágneses indukció jelenségének lényegét, fontosabb
  gyakorlati vonatkozásait, a váltakozó áram fogalmát.
\item
  Tudja, hogy a fény elektromágneses hullám, és hogy terjedéséhez nem
  kell közeg.
\item
  Ismeri az elektromágneses hullámok jellemzőit (frekvencia,
  hullámhossz, terjedési sebesség), azt, hogy milyen körülmények
  határozzák meg ezeket. a mennyiségek kapcsolatára vonatkozó egyszerű
  számításokat végez.
\item
  Ismeri a színek és a fény frekvenciája közötti kapcsolatot, a fehér
  fény összetett voltát, a kiegészítő színek fogalmát, a szivárvány
  színeit.
\item
  Ismeri a fénytörés és visszaverődés törvényét, megmagyarázza, hogyan
  alkot képet a síktükör.
\item
  A fókuszpont fogalmának felhasználásával értelmezi, hogyan térítik el
  a fényt a domború és homorú tükrök, a domború és homorú lencsék.
\item
  Ismeri az optikai leképezés fogalmát, a valódi és látszólagos kép
  közötti különbséget. egyszerű kísérleteket tud végezni tükrökkel és
  lencsékkel.
\item
  Megfigyeli a fényelektromos jelenséget, tisztában van annak einstein
  által kidolgozott magyarázatával, a frekvencia (hullámhossz) és a
  foton energiája kapcsolatával.
\item
  Ismeri rutherford szórási kísérletét, mely az atommag felfedezéséhez
  vezetett.
\item
  Ismeri az atomról alkotott elképzelések változásait, a
  rutherford-modellt és bohr-modellt, látja a modellek hiányosságait.
\item
  Megmagyarázza az elektronmikroszkóp működését az elektron
  hullámtermészetének segítségével.
\item
  Átlátja, hogyan használják a vonalas színképet az anyagvizsgálat
  során.
\item
  Ismeri az atommag felépítését, a nukleonok típusait, az izotóp
  fogalmát, a nukleáris kölcsönhatás jellemzőit.
\item
  Átlátja, hogy a maghasadás és magfúzió miért alkalmas
  energiatermelésre, ismeri a gyakorlati megvalósulásuk lehetőségeit, az
  atomerőművek működésének alapelvét, a csillagok energiatermelésének
  lényegét.
\item
  Ismeri a radioaktív sugárzások típusait, az alfa-, béta- és
  gamma-sugárzások leírását és tulajdonságait.
\item
  Ismeri a felezési idő, aktivitás fogalmát, a sugárvédelem
  lehetőségeit.
\item
  Megvizsgálja a naprendszer bolygóin és holdjain uralkodó, a földétől
  eltérő fizikai környezet legjellemzőbb példáit, azonosítja ezen
  eltérések okát, a legfontosabb esetekben megmutatja, hogyan
  érvényesülnek a fizika törvényei a föld és a hold mozgása során.
\item
  Szabad szemmel vagy távcsővel megfigyeli a holdat, a hold felszínének
  legfontosabb jellemzőit, a holdfogyatkozás jelenségét, a látottakat
  fizikai ismeretei alapján értelmezi.
\item
  Ismeri a nap, mint csillag legfontosabb fizikai tulajdonságait, a nap
  várható jövőjét, a csillagok lehetséges fejlődési folyamatait.
\item
  Átlátja és szemlélteti a természetre jellemző fizikai mennyiségek
  nagyságrendjeit (atommag, élőlények, naprendszer, univerzum).
\item
  A legegyszerűbb esetekben azonosítja az alapvető fizikai
  kölcsönhatások és törvények szerepét a világegyetem felépítésében és
  időbeli változásaiban.
\item
  Használ helymeghatározó szoftvereket, a közeli és távoli
  környezetünket leíró adatbázisokat, szoftvereket.
\item
  A vizsgált fizikai jelenségeket, kísérleteket bemutató animációkat,
  videókat keres és értelmez.
\item
  Ismer magyar és idegen nyelvű megbízható fizikai tárgyú honlapokat.
\item
  Készségszinten alkalmazza a különböző kommunikációs eszközöket,
  illetve az internetet a főként magyar, illetve idegen nyelvű, fizikai
  tárgyú tartalmak keresésére.
\item
  Fizikai szövegben, videóban el tudja különíteni a számára világos,
  valamint nem érthető, további magyarázatra szoruló részeket.
\item
  Az interneten talált tartalmakat több forrásból is ellenőrzi.
\item
  A forrásokból gyűjtött információkat számítógépes prezentációban
  mutatja be.
\item
  Az egyszerű vizsgálatok eredményeinek, az elemzések, illetve a
  következtetések bemutatására prezentációt készít.
\item
  A projektfeladatok megoldása során önállóan, illetve a csoporttagokkal
  közösen különböző médiatartalmakat, prezentációkat, rövidebb-hosszabb
  szöveges produktumokat hoz létre a tapasztalatok, eredmények,
  elemzések, illetve következtetések bemutatására.
\item
  A vizsgálatok során kinyert adatokat egyszerű táblázatkezelő szoftver
  segítségével elemzi, az adatokat grafikonok segítségével értelmezi.
\item
  Használ mérésre, adatelemzésre, folyamatelemzésre alkalmas összetett
  szoftvereket (például hang és mozgókép kezelésére alkalmas
  programokat).
\end{itemize}

\hypertarget{foldrajz}{%
\subsection{Földrajz}\label{foldrajz}}

\hypertarget{evfolyamon-17}{%
\subsubsection{9-10. évfolyamon}\label{evfolyamon-17}}

\begin{itemize}
\item
  Tudatosan és kritikusan használja a földrajzi tartalmú nyomtatott és
  elektronikus információforrásokat a tanulásban és tudása önálló
  bővítésekor.
\item
  Ismeretei alapján biztonsággal tájékozódik a valós és a digitális
  eszközök által közvetített virtuális földrajzi térben, földrajzi
  tartalmú adatokban, a különböző típusú térképeken.
\item
  Képes összetettebb földrajzi tartalmú szövegek értelmezésére.
\item
  Adott természeti, társadalmi-gazdasági témához kapcsolódóan írásbeli
  vagy szóbeli beszámolót készít, prezentációt állít össze.
\item
  Összetettebb földrajzi számítási feladatokat megold, az eredmények
  alapján következtetéseket fogalmaz meg.
\item
  Véleményt alkot aktuális társadalmi-gazdasági és környezeti
  kérdésekben, véleménye alátámasztására logikus érveket fogalmaz meg.
\item
  Földrajzi tartalmú projektfeladatokat valósít meg társaival.
\item
  Elkötelezett a természeti és a kulturális értékek, a kulturális
  sokszínűség megőrzése iránt.
\item
  Döntéseit a környezeti szempontok figyelembevételével mérlegeli,
  felelős fogyasztói magatartást tanúsít.
\item
  Nyitott a különböző szintű pénzügyi folyamatok és összefüggések
  megismerése iránt.
\item
  Alkalmazza a más tantárgyak tanulása során megszerzett ismereteit
  földrajzi problémák megoldása során.
\item
  Földrajzi tartalmú adatok, információk alapján következtetéseket von
  le, tendenciákat ismer fel, és várható következményeket (prognózist)
  fogalmaz meg.
\item
  Földrajzi megfigyelést, vizsgálatot, kísérletet tervez és valósít meg,
  az eredményeket értelmezi.
\item
  Feltárja a földrajzi folyamatok, jelenségek közötti hasonlóságokat és
  eltéréseket, különböző szempontok alapján rendszerezi azokat.
\item
  Megkülönbözteti a tényeket a véleményektől, adatokat, információkat
  kritikusan szemlél.
\item
  Önálló, érvekkel alátámasztott véleményt fogalmaz meg földrajzi
  kérdésekben.
\item
  Céljainak megfelelően kiválasztja és önállóan használja a hagyományos,
  illetve digitális információforrásokat, adatbázisokat.
\item
  Földrajzi tartalmú szövegek alapján lényegkiemelő összegzést készít
  szóban és írásban.
\item
  Digitális eszközök segítségével bemutat és értelmez földrajzi
  jelenségeket, folyamatokat, törvényszerűségeket, összefüggéseket.
\item
  Adatokat rendszerez és ábrázol hagyományos és digitális eszközök
  segítségével.
\item
  Megadott szempontok alapján alapvető földrajzi-földtani folyamatokkal,
  tájakkal, országokkal kapcsolatos földrajzi tartalmú szövegeket, képi
  információhordozókat dolgoz fel.
\item
  A közvetlen környezetének földrajzi megismerésére terepvizsgálódást
  tervez és kivitelez.
\item
  Tudatosan használja a földrajzi és a kozmikus térben való tájékozódást
  segítő hagyományos és digitális eszközöket, ismeri a légi- és
  űrfelvételek sajátosságait, alkalmazási területeit.
\item
  Képes problémaközpontú feladatok megoldására, környezeti változások
  összehasonlító elemzésére térképek és légi- vagy űrfelvételek
  párhuzamos használatával.
\item
  Térszemlélettel rendelkezik a csillagászati és a földrajzi térben.
\item
  Érti a világegyetem tér- és időbeli léptékeit, elhelyezi a földet a
  világegyetemben és a naprendszerben.
\item
  Ismeri a föld, a hold és a bolygók jellemzőit, mozgásait és ezek
  következményeit, összefüggéseit.
\item
  Értelmezi a nap és a naprendszer jelenségeit, folyamatait, azok földi
  hatásait.
\item
  Egyszerű csillagászati és időszámítással kapcsolatos feladatokat,
  számításokat végez.
\item
  Ismeri a föld felépítésének törvényszerűségeit.
\item
  Összefüggéseiben mutatja be a lemeztektonika és az azt kísérő
  jelenségek (földrengések, vulkanizmus, hegységképződés) kapcsolatát,
  térbeliségét, illetve magyarázza a kőzetlemez-mozgások lokális és az
  adott helyen túlmutató globális hatásait.
\item
  Felismeri a történelmi és a földtörténeti idő eltérő nagyságrendjét,
  ismeri a geoszférák fejlődésének időbeli szakaszait, meghatározó
  jelentőségű eseményeit.
\item
  Párhuzamot tud vonni a jelenlegi és múltbeli földrajzi folyamatok
  között.
\item
  Felismeri az alapvető ásványokat és kőzeteket, tud példákat említeni
  azok gazdasági és mindennapi életben való hasznosítására.
\item
  Ismeri a kőzetburok folyamataihoz kapcsolódó földtani veszélyek okait,
  következményeit, tér- és időbeli jellemzőit, illetve elemzi az
  alkalmazkodási, kármegelőzési lehetőségeket.
\item
  Érti a különböző kőzettani felépítésű területek eltérő környezeti
  érzékenysége, terhelhetősége közti összefüggéseket.
\item
  Ismeri a légkör szerkezetét, fizikai és kémiai jellemzőit, magyarázza
  az ezekben bekövetkező változások mindennapi életre gyakorolt hatását.
\item
  Összefüggéseiben mutatja be a légköri folyamatokat és jelenségeket,
  illetve összekapcsolja ezeket az időjárás alakulásával.
\item
  Tudja az időjárási térképeket és előrejelzéseket értelmezni egyszerű
  prognózisok készítésére.
\item
  Felismeri a szélsőséges időjárási helyzeteket és tud a helyzetnek
  megfelelően cselekedni.
\item
  A légkör globális változásaival foglalkozó forrásokat kritikusan
  elemzi, érveken alapuló véleményt fogalmaz meg a témával
  összefüggésben.
\item
  Megnevezi a légkör legfőbb szennyező forrásait és a szennyeződés
  következményeit, érti a lokálisan ható légszennyező folyamatok
  globális következményeit.
\item
  Magyarázza az éghajlatváltozás okait, valamint helyi, regionális,
  globális következményeit.
\item
  Ismeri a felszíni és felszín alatti vizek főbb típusait, azok
  jellemzőit, mennyiségi és minőségi viszonyaikat befolyásoló
  tényezőket, a víztípusok közötti összefüggéseket.
\item
  Igazolja a felszíni és felszín alatti vizek egyre fontosabbá váló
  erőforrásszerepét és gazdasági vonatkozásait, bizonyítja a víz
  társadalmi folyamatokat befolyásoló természetét, védelmének
  szükségességét.
\item
  Ismeri a vízburokkal kapcsolatos környezeti veszélyek okait, és
  reálisan számol a várható következményekkel.
\item
  Tudatában van a személyes szerepvállalások értékének a globális
  vízgazdálkodás és éghajlatváltozás rendszerében.
\item
  Összefüggéseiben, kölcsönhatásaiban mutatja be a földrajzi övezetesség
  rendszerének egyes elemeit, a természeti jellemzők
  társadalmi-gazdasági vonatkozásait.
\item
  Összefüggéseiben mutatja be a talajképződés folyamatát, tájékozott a
  talajok gazdasági jelentőségével kapcsolatos kérdésekben, ismeri
  magyarország fontosabb talajtípusait.
\item
  Bemutatja a felszínformálás többtényezős összefüggéseit, ismeri és
  felismeri a különböző felszínformáló folyamatokhoz (szél, víz, jég) és
  kőzettípusokhoz kapcsolódóan kialakuló, felszíni és felszín alatti
  formakincset.
\item
  Érti az ember környezet átalakító szerepét, ember és környezete
  kapcsolatrendszerét, illetve példák alapján igazolja az egyes
  geoszférák folyamatainak, jelenségeinek gazdasági következményeit,
  összefüggéseit.
\item
  Bemutatja a népességszám-változás időbeli és területi különbségeit,
  ismerteti okait és következményeit, összefüggését a fiatalodó és az
  öregedő társadalmak jellemző folyamataival és problémáival.
\item
  Különböző népességi, társadalmi és kulturális jellemzők alapján
  bemutat egy kontinenst, országot, országcsoportot.
\item
  Különböző szempontok alapján csoportosítja és jellemzi az egyes
  településtípusokat, bemutatja szerepkörük és szerkezetük változásait.
\item
  Érti és követi a lakóhelye környékén zajló település- és
  területfejlődési, valamint demográfiai folyamatokat.
\item
  Ismerteti a gazdaság szerveződését befolyásoló telepítő tényezők
  szerepének átalakulását, bemutatja az egyes gazdasági ágazatok
  jellemzőit, értelmezi a gazdasági szerkezetváltás folyamatát.
\item
  Értelmezi és értékeli a társadalmi-gazdasági fejlettség
  összehasonlítására alkalmas mutatók adatait, a társadalmi-gazdasági
  fejlettség területi különbségeit a föld különböző térségeiben.
\item
  Értékeli az eltérő adottságok, erőforrások szerepét a
  társadalmi-gazdasági fejlődésben.
\item
  Modellezi a piacgazdaság működését.
\item
  Megnevezi és értékeli a gazdasági integrációk és a regionális
  együttműködések kialakulásában szerepet játszó tényezőket.
\item
  Ismerteti a világpolitika és a világgazdaság működését befolyásoló
  nemzetközi szervezetek, együttműködések legfontosabb jellemzőit.
\item
  Értelmezi a globalizáció fogalmát, a globális világ kialakulásának és
  működésének feltételeit, jellemző vonásait.
\item
  Példák alapján bemutatja a globalizáció társadalmi-gazdasági és
  környezeti következményeit, mindennapi életünkre gyakorolt hatását.
\item
  Megnevezi a világgazdaság működése szempontjából tipikus térségeket,
  országokat.
\item
  Összehasonlítja az európai, ázsiai és amerikai erőterek gazdaságilag
  meghatározó jelentőségű országainak, országcsoportjainak szerepét a
  globális világban.
\item
  Összefüggéseiben mutatja be a perifériatérség társadalmi-gazdasági
  fejlődésének jellemző vonásait, a felzárkózás nehézségeit.
\item
  Ismerteti az európai unió működésének földrajzi alapjait, példák
  segítségével bemutatja az európai unión belüli társadalmi-gazdasági
  fejlettségbeli különbségeket, és megnevezi a felzárkózást segítő
  eszközöket.
\item
  Példák alapján jellemzi és értékeli magyarország társadalmi-gazdasági
  szerepét annak szűkebb és tágabb nemzetközi környezetében, az európai
  unióban.
\item
  Bemutatja a területi fejlettségi különbségek okait és következményeit
  magyarországon, megfogalmazza a felzárkózás lehetőségeit.
\item
  Értékeli hazánk környezeti állapotát, megnevezi jelentősebb környezeti
  problémáit.
\item
  Magyarázza a monetáris világ működésének alapvető fogalmait,
  folyamatait és azok összefüggéseit, ismer nemzetközi pénzügyi
  szervezeteket.
\item
  Bemutatja a működőtőke- és a pénztőkeáramlás sajátos vonásait,
  magyarázza eltérésük okait.
\item
  Pénzügyi döntéshelyzeteket, aktuális pénzügyi folyamatokat értelmez és
  megfogalmazza a lehetséges következményeket.
\item
  Pénzügyi lehetőségeit mérlegelve egyszerű költségvetést készít,
  értékeli a hitelfelvétel előnyeit és kockázatait.
\item
  Alkalmazza megszerzett ismereteit pénzügyi döntéseiben, belátja a
  körültekintő, felelős pénzügyi tervezés és döntéshozatal fontosságát.
\item
  Felismeri és azonosítja a földrajzi tartalmú természeti,
  társadalmi-gazdasági és környezeti problémákat, megnevezi kialakulásuk
  okait, és javaslatokat fogalmaz meg megoldásukra.
\item
  Rendszerezi a geoszférákat ért környezetkárosító hatásokat, bemutatja
  a folyamatok kölcsönhatásait.
\item
  Példákkal igazolja a természetkárosítás és a természeti, illetve
  környezeti katasztrófák társadalmi következményeit, a
  környezetkárosodás életkörülményekre, életminőségre gyakorolt hatását,
  a lokális szennyeződés globális következményeit.
\item
  Globális problémákhoz vezető, földünkön egy időben jelenlévő,
  különböző természeti és társadalmi-gazdasági eredetű problémákat
  elemez, feltárja azok összefüggéseit, bemutatja mérséklésük lehetséges
  módjait és azok nehézségeit.
\item
  Megfogalmazza az energiahatékony, nyersanyag-takarékos, illetve „zöld''
  gazdálkodás lényegét, valamint példákat nevez meg a környezeti
  szempontok érvényesíthetőségére a termelésben és a fogyasztásban.
\item
  Megkülönbözteti a fogyasztói társadalom és a tudatos fogyasztói
  közösség jellemzőit.
\item
  A lakóhely adottságaiból kiindulva értelmezi a fenntartható fejlődés
  társadalmi, természeti, gazdasági, környezetvédelmi kihívásait.
\item
  Megnevez a környezet védelmében, illetve humanitárius céllal
  tevékenykedő hazai és nemzetközi szervezeteket, példákat említ azok
  tevékenységére, belátja és igazolja a nemzetközi összefogás
  szükségességét.
\item
  Értelmezi a fenntartható gazdaság, a fenntartható gazdálkodás
  fogalmát, érveket fogalmaz meg a fenntarthatóságot szem előtt tartó
  gazdaság, illetve gazdálkodás fontossága mellett.
\item
  Bemutatja az egyén társadalmi szerepvállalásának lehetőségeit, a
  tevékeny közreműködés példáit a környezet védelme érdekében, illetve
  érvényesíti saját döntéseiben a környezeti szempontokat.
\end{itemize}

\hypertarget{hon--es-nepismeret}{%
\subsection{Hon- és népismeret}\label{hon--es-nepismeret}}

\hypertarget{evfolyamon-18}{%
\subsubsection{6. évfolyamon}\label{evfolyamon-18}}

\begin{itemize}
\item
  Képessé válik a nemzedékek közötti párbeszédre.
\item
  Megismeri magyarország és a kárpát-medence kulturális hagyományait,
  tiszteletben tartja más kultúrák értékeit.
\item
  Ismeri és adekvát módon használja az ismeretkör tartalmi
  kulcsfogalmait.
\item
  Az önálló ismeretszerzés során alkalmazni tudja az interjúkészítés, a
  forráselemzés módszerét.
\item
  Képes az együttműködésre csoportmunkában, alkalmazza a tevékenységek
  lezárásakor az önértékelést és a társak értékelését.
\item
  Megismeri a közvetlen környezetében található helyi értékeket,
  felhasználva a digitálisan elérhető adatbázisokat is.
\item
  Megbecsüli szűkebb lakókörnyezetének épített örökségét, természeti
  értékeit, helyi hagyományait.
\item
  Megéli a közösséghez tartozást, nemzeti önazonosságát, kialakul benne
  a haza iránti szeretet és elköteleződés.
\item
  Megtapasztalja a legszűkebb közösséghez, a családhoz, a lokális
  közösséghez tartozás érzését.
\item
  Nyitottá válik a hagyományos családi és közösségi értékek
  befogadására.
\item
  Felismeri a néphagyományok közösségformáló, közösségmegtartó erejét.
\item
  Tiszteletben tartja más kultúrák értékeit.
\item
  Érdeklődő attitűdjével erősíti a nemzedékek közötti párbeszédet.
\item
  Megbecsüli és megismeri az idősebb családtagok tudását,
  tapasztalatait, nyitott a korábbi nemzedékek életmódjának,
  normarendszerének megismerésére.
\item
  Megérti a természeti környezet meghatározó szerepét a más tájakon élő
  emberek életmódbeli különbségében.
\item
  Belátja, hogy a természet kínálta lehetőségek felhasználásának
  elsődleges szempontja a szükségletek kielégítése, a mértéktartás
  alapelvének követése.
\item
  Megismeri a gazdálkodó életmódra jellemző újrahasznosítás elvét, és
  saját életében is megpróbálja alkalmazni.
\item
  Önálló néprajzi gyűjtés során, digitális archívumok, írásos
  dokumentumok tanulmányozásával ismereteket szerez családja, lakóhelye
  múltjáról, hagyományairól.
\item
  Szöveges és képi források, digitalizált archívumok elemzése,
  feldolgozása alapján bővíti tudását.
\item
  Felismeri a csoportban elfoglalt helyét és szerepét, törekszik a
  személyiségének, készségeinek és képességeinek, érdeklődési körének
  legjobban megfelelő feladatok vállalására.
\item
  Tevékenyen részt vesz a csoportmunkában zajló együttműködő alkotási
  folyamatban, digitális források közös elemzésében.
\item
  Meghatározott helyzetekben önálló véleményt tud alkotni, a társak
  véleményének meghallgatását követően álláspontját felül tudja bírálni,
  döntéseit át tudja értékelni.
\end{itemize}

\hypertarget{komplex-termeszettudomany}{%
\subsection{Komplex természettudomány}\label{komplex-termeszettudomany}}

\hypertarget{evfolyamon-19}{%
\subsubsection{9. évfolyamon}\label{evfolyamon-19}}

\begin{itemize}
\item
  Ismeri a tudományos kutatás alapszabályait és azokat alkalmazza
\item
  Önálló tudományos kutatást tervez meg és végez el
\item
  Önálló kutatása összeállításakor tudományos modelleket használ
\item
  Tudományos kutatások során elvégzi a megszerzett adatok feldolgozását
  és értelmezését
\item
  Érti a tudomány szerepét és szükségszerűségét a társadalmi folyamatok
  alakításában
\item
  Hiteles források felhasználásával egy tudományos probléma kritikus
  elemzését adja a megszerzett információk alapján
\item
  Elméleti és gyakorlati eszközöket választ és alkalmaz egy adott
  tudományos probléma ismertetéséhez
\item
  Tudományos kutatási eredményeket érthetően mutat be digitális eszközök
  segítségével
\item
  Önállóan és kiscsoportban biztonságosan végez természettduományos
  kísérleteket
\item
  Felismeri a saját és társai által végzett tudományos kísérletek etikai
  és társadalmi vonatkozásait
\item
  Ismeri és alkalmazza az energia felhasználásának és átalakulásának
  elméleti és gyakorlati lehetőségeit (energiaáramláson alapuló
  ökoszisztémák, a föld saját energiaforrásai stb.)
\item
  Felismeri és kísérletei során alkalmazza azt a tudást, hogy az anyag
  atomi természete határozza meg a fizikai és kémiai tulajdonságokat és
  azok kölcsönhatásából eredő módosulását. (Példák kísérletekre: kémiai
  reakciók, molekuláris biológiai, anyagok körforgása a különböző
  ökoszisztémákban)
\item
  Felismeri és kísérletei során alkalmazza azt a tudást, hogy a
  természet ismert rendszerei előrejelezhető módon változnak (evolúció,
  klímaváltozás, földtörténeti korok, Föld felszíni változása,
  tektonikus mozgások, ember környezeti hatásai stb.)
\item
  Felismeri és kísérletei során alkalmazza azt a tudást, hogy a tárgyak
  mozgása előrejelezhető (erők, égitestek mozgása, molekuláris mozgások,
  hangok, fények mozgása)
\item
  Felismeri és kísérletei során alkalmazza azt a tudást, hogy a világ
  megismerésének egyik alapja a szerveződési szintek és típusok
  megértése (periódusos rendszer, sjetszintű szerveződések, állat és
  növényvilág rendszertana, a világegyetem szervező elvei) -
\item
\item
  Ismeri a föld népességgének aktuális kihívásait beleértve azok
  társadalmi és egészségügyi kockázatait (betegség-megelőzés, járványok,
  élelmezés)
\item
  Ismeri a hulladékgazdálkodás aktuális kihívásait
\item
  Ismeri az energia fogalmát és az egyéni és társadalmi
  energiafelhasználás különböző lehetőségeit
\item
  Megkülönbözteti egymástól a természetes és mesterséges anyagokat és
  felismeri, hogy miként állapítható egy anyag összetétele
\item
  Ismeri a városi és falusi életmód és életterek közötti különbségeket,
  azok környezetre gyakorolt hatását
\item
  Használja a regionalitás fogalmát és érti annak szerepét a gazdasági
  folyamatok alakulásában
\item
  Ismeri a levegő és a víz fizikai és kémiai jellemzőit, felismeri ezek
  élettani hatásait
\item
  Tudja, hogy milyen vízkészletekkel rendelkezik a Föld és azok
  felhasználásának milyen hatása van a környezeti, ipari, turisztikai és
  szállítmányozási folyamatokra
\item
  Ismeri a növények tápanyagigényét és fejlődésük alapjait
\item
  Ismeri a vadon élő állatközösségeket fenyegető veszélyeket és azt a
  kihívást, amit ez az emberekre gyakorol
\item
  Ismeri a különböző kultúrák eltérő gazdasági termelési szokásait
  (növénytermesztés, állattartás)
\item
  Ismeri a talaj természetét és annak megművelésének különböző formáit
\item
  Ismeri a föld légkörét befolyásoló globális folyamatokat (tengerek és
  szelek áramlása, klímaváltozás, üvegházhatású gázok)
\item
  Ismeri az emberi táplálkozás során hasznos ételeket, megkülönbözteti a
  gyógyító és a káros anyagokat, önállóan tud egészséges ételt készíteni
\item
  Ismeri az emberi agy alapvető működési szabályait és az azt
  befolyásoló tényezőket
\item
  Tisztában van az Univerzum létrejöttének ma ismert elméletével,
  valamint a Naprendszer kialakulásának folyamatával
\item
  Ismeri a tanulás és az emberi kommunikáció biológiai alapjait
\item
  Ismeri az emberi szervezet egészségét alapvetően befolyásoló
  tényezőket, a stressz, az öröklött hajlamok és genetikai
  tulajdonságok, valamint a környezeti hatások szerepét
\item
  Ismeri az emberi szexualitás kulturális, társadalmi és biológiai
  alapjait. Önálló véleménye van a nemi szerepek fontosságáról, érti a
  nemi identitás komplex jellegét.
\item
  Ismeri a hálózatok és a hálózatkutatás szerepét modern világunkban, az
  életközösségek, a sejtszintű gondolkodás és az információs
  technológiák területén.
\item
  Ismeri a genetikai információ átadásának alapvető szabályait. \#\#
  Kémia \#\#\# 9-10. évfolyamon
\item
  Érdeklődését felkeltse a környezetben zajló fizikai és kémiai
  változások okai, magyarázata, komplexitása, elméleti háttere iránt.
\item
  Ismerje a mindennapi életben előforduló alapvető vegyülettípusokat,
  legyen tisztában alapvető kémiai fogalmakkal, jelenségekkel és
  reakciókkal, legyen anyagismerete.
\item
  Önálló ismeretszerzési, illetve összefüggés-felismerési készségei
  fejlődjenek a kísérletek, laboratóriumi vizsgálatok, nyomtatott vagy
  digitális információforrások önálló vagy csoportban történő elemzése
  révén, ami megalapozza az értő, önálló munkavégzés lehetőségét.
\item
  Problémaorientált, elemző és mérlegelő gondolkodása alakuljon ki, ami
  nélkülözhetetlen az információs társadalomra jellemző hír- és
  információdömpingben történő eligazodáshoz, a felelős és tudatos
  állampolgári szerepvállaláshoz.
\item
  Tanulmányozza a természetben lejátszódó folyamatokat, valamint
  átgondolja a várható következményeket, cselekedni képes, a
  környezetért felelősséggel tenni akaró magatartást alakítson ki, ezzel
  is hangsúlyozva, hogy az ember egyénként és egy nagyobb közösség
  részeként egyaránt felelős a természeti környezetéért, annak jövőbeni
  állapotáért, felismeri és megérti, hogy a környezettudatos, a
  fenntarthatóságot szem előtt tartó gondolkodás az élhető jövő záloga.
\item
  A köznapi életben használt vegyi anyagok és az azokkal végzett
  felelősségteljes munka alapvető ismereteinek elsajátítása mellett
  tanulja meg a mindennapi életben hasznosítható kémiai ismereteket, és
  alakuljon ki benne az értő, felelős döntési képesség készsége.
\item
  A kísérleti megfigyeléstől a modellalkotásig
\item
  Az anyagi halmazok
\item
  Atomok, molekulák, ionok
\item
  Kémiai reakciók
\item
  Kémia a természetben
\item
  Kémia a mindennapokban
\item
  Az anyagok szerkezete és tulajdonságai
\item
  A kémiai átalakulások
\item
  A szén egyszerű szerves vegyületei
\item
  Az életműködések kémiai alapjai
\item
  Elemek és szervetlen vegyületeik
\item
  Kémia az ipari termelésben és a mindennapokban
\item
  Környezeti kémia és környezetvédelem
\item
  Ismeri a mindennapi életben előforduló fontosabb vegyülettípusokat,
  tisztában van az élettelen és élő természet legfontosabb kémiai
  fogalmaival, jelenségeivel és az azokat működtető reakciótípusokkal.
\item
  Önállóan vagy csoportban el tud végezni egyszerű kémiai kísérleteket
  és megbecsüli azok várható eredményét.
\item
  Alkalmazza a természettudományos problémamegoldás lépéseit egyszerű
  kémiai problémák megoldásában.
\item
  Képes az analógiás, a korrelatív és a mérlegelő gondolkodásra kémiai
  kontextusban.
\item
  Képes számítógépes prezentáció formájában kémiával kapcsolatos
  eredmények, információk bemutatására, megosztására, a mérési adatok
  számítógépes feldolgozására, szemléltetésére.
\item
  Tudja használni a részecskemodellt az anyagok tulajdonságainak és
  átalakulásainak értelmezésére.
\item
  Ismeri a kémiának az egyén és a társadalom életében betöltött
  szerepét.
\item
  Tisztában van a háztartásban leggyakrabban előforduló anyagok
  felhasználásának előnyeivel és veszélyeivel, a biztonságos
  vegyszerhasználat szabályaival.
\item
  Egyedül vagy csoportban elvégez egyszerű kémiai kísérleteket leírás
  vagy szóbeli útmutatás alapján és értékeli azok eredményét.
\item
  Egyedül vagy csoportban elvégez összetettebb,
  halmazállapot-változással és oldódással kapcsolatos kísérleteket és
  megbecsüli azok várható eredményét.
\item
  Kémiai vizsgálatainak tervezése során alkalmazza az analógiás
  gondolkodás alapjait és használja az „egyszerre csak egy tényezőt
  változtatunk'' elvet.
\item
  Ismeri az anyagmennyiség és a mól fogalmát, érti bevezetésük
  szükségességét és egyszerű számításokat végez m, n és m segítségével.
\item
  Analógiás gondolkodással következtet a szerves vegyület tulajdonságára
  a funkciós csoportja ismeretében.
\item
  Használja a fémek redukáló sorát a fémek tulajdonságainak
  megjóslására, tulajdonságaik alátámasztására.
\item
  Ismer megbízható magyar és idegen nyelvű internetes forrásokat kémiai
  tárgyú, elemekkel és vegyületekkel kapcsolatos képek és szövegek
  gyűjtésére.
\item
  Magabiztosan használ magyar és idegen nyelvű mobiltelefonos,
  táblagépes applikációkat kémiai tárgyú információk keresésére.
\item
  A különböző, megbízható forrásokból gyűjtött információkat
  számítógépes prezentációban mutatja be.
\item
  Mobiltelefonos, táblagépes alkalmazások segítségével médiatartalmakat,
  illetve bemutatókat hoz létre.
\item
  Alapvető szinten ismeri a természetes környezetet felépítő légkör,
  vízburok, kőzetburok és élővilág kémiai összetételét.
\item
  Érti a környezetünk megóvásának a jelentőségét az emberi civilizáció
  fennmaradása szempontjából.
\item
  Ismeri a zöld kémia lényegét, a környezetbarát folyamatok előtérbe
  helyezését, példákat mond újonnan előállított, az emberiség jólétét
  befolyásoló anyagokra (pl. új gyógyszerek, lebomló műanyagok,
  intelligens textíliák).
\item
  Ismeri a legfontosabb környezetszennyező forrásokat és anyagokat,
  valamint ezeknek az anyagoknak a környezetre gyakorolt hatását.
\item
  Példákkal szemlélteti egyes kémiai technológiák, illetve bizonyos
  anyagok felhasználásának környezetre gyakorolt pozitív és negatív
  hatásait.
\item
  Ismeri a bioüzemanyagok legfontosabb típusait.
\item
  Példákkal szemlélteti az emberiség legégetőbb globális problémáit
  (globális éghajlatváltozás, ózonlyuk, ivóvízkészlet csökkenése,
  energiaforrások kimerülése), és azok kémiai vonatkozásait.
\item
  Ismeri az emberiség előtt álló legnagyobb kihívásokat, kiemelten azok
  kémiai vonatkozásai (energiahordozók, környezetszennyezés,
  fenntarthatóság, új anyagok előállítása).
\item
  Példákon keresztül szemlélteti az antropogén tevékenységek kémiai
  vonatkozású környezeti következményeit.
\item
  Kiselőadás vagy projektmunka keretében mutatja be a 20. század néhány
  nagy környezeti katasztrófáját és azt, hogy milyen tanulságokat
  vonhatunk le azok megismeréséből.
\item
  Ismeri a légkör kémiai összetételét és az azt alkotó gázok fontosabb
  tulajdonságait, példákat mond a légkör élőlényekre és élettelen
  környezetre gyakorolt hatásaira, ismeri a legfontosabb légszennyező
  gázokat, azok alapvető tulajdonságait, valamint a
  környezetszennyezésének hatásait, ismeri a légkört érintő globális
  környezeti problémák kémiai hátterét és ezen problémák megoldására
  tett erőfeszítéseket.
\item
  Ismeri a természetes vizek típusait, azok legfontosabb kémiai
  összetevőit a víz körforgásának és tulajdonságainak tükrében, példákat
  mond vízszennyező anyagokra, azok forrására, a szennyezés lehetséges
  következményeire, ismeri a víztisztítás folyamatának alapvető
  lépéseit, valamint a tiszta ivóvíz előállításának a módját.
\item
  Érti a kőzetek és a környezeti tényezők talajképző szerepét, példát
  mond alapvető kőzetekre, ásványokra, érti a különbséget a hulladék és
  a szemét fogalmi megkülönböztetése között, ismeri a hulladékok
  típusait, kezelésük módját, környezetre gyakorolt hatásait.
\item
  Érti a különbséget a tudományos és áltudományos információk között,
  konkrét példát mond a köznapi életből tudományos és áltudományos
  ismeretekre, információkra.
\item
  Ismeri a tudományos megközelítés lényegét (objektivitás,
  reprodukálhatóság, ellenőrizhetőség, bizonyíthatóság).
\item
  Látja az áltudományos megközelítés lényegét (feltételezés,
  szubjektivitás, bizonyítatlanság), felismeri az áltudományosságra
  utaló legfontosabb jeleket.
\item
  Ismeri az élelmiszereink legfontosabb összetevőinek, a
  szénhidrátoknak, a fehérjéknek, valamint a zsíroknak és olajoknak a
  molekulaszerkezetét és a tulajdonságait, felsorolja a háztartásban
  megtalálható legfontosabb élelmiszerek tápanyagait, példát mond
  bizonyos összetevők (fehérjék, redukáló cukrok, keményítő)
  kimutatására, ismeri a legfontosabb élelmiszeradalék-csoportokat,
  alapvető szinten értelmezi egy élelmiszer tájékoztató címkéjét.
\item
  Ismeri a gyógyszer fogalmát és a gyógyszerek fontosabb csoportjait
  hatásuk alapján, alapvető szinten értelmezi a gyógyszerek mellékelt
  betegtájékoztatóját.
\item
  Ismeri a leggyakrabban használt élvezeti szerek (szeszes italok,
  dohánytermékek, kávé, energiaitalok, drogok) hatóanyagát, ezen szerek
  használatának veszélyeit, érti az illegális drogok használatával
  kapcsolatos alapvető problémákat, példákat mond illegális drogokra,
  ismeri a doppingszer fogalmát, megérti és értékeli a doppingszerekkel
  kapcsolatos információkat.
\item
  Ismeri a méreg fogalmának jelentését, érti az anyagok mennyiségének
  jelentőségét a mérgező hatásuk tekintetében, példát mond növényi,
  állati és szintetikus mérgekre, ismeri a mérgek szervezetbe jutásának
  lehetőségeit (tápcsatorna, bőr, tüdő), ismeri és felismeri a különböző
  anyagok csomagolásán a mérgező anyag piktogramját, képes ezeknek az
  anyagoknak a felelősségteljes használatára, ismeri a köznapi életben
  előforduló leggyakoribb mérgeket, mérgezéseket (pl. szén-monoxid,
  penészgomba-toxinok, gombamérgezések, helytelen égetés során keletkező
  füst anyagai, drogok, nehézfémek), tudja, hogy a mérgező hatás nem az
  anyag szintetikus eredetének a következménye.
\item
  Ismeri a természetben megtalálható legfontosabb nyersanyagokat.
\item
  Érti az anyagok átalakításának hasznát, valamint konkrét példákat mond
  vegyipari termékek előállítására.
\item
  Ismeri a különböző nyersanyagokból előállítható legfontosabb
  termékeket.
\item
  Érti, hogy az ipari (vegyipari) termelés során különféle, akár a
  környezetre vagy szervezetre káros anyagok is keletkezhetnek, amelyek
  közömbösítése, illetve kezelése fontos feladat.
\item
  Képes az ismeretein alapuló tudatos vásárlással és tudatos
  életvitellel a környezetének megóvására.
\item
  Érti a mészkőalapú építőanyagok kémiai összetételét és átalakulásait
  (mészkő, égetett mész, oltott mész), ismeri a beton alapvető
  összetételét, előállítását és felhasználásának lehetőségeit, ismeri a
  legfontosabb hőszigetelő anyagokat.
\item
  Érti, hogy a fémek többsége a természetben vegyületek formájában van
  jelen, ismeri a legfontosabb redukciós eljárásokat (szenes,
  elektrokémiai redukció), ismeri a legfontosabb ötvözeteket, érti az
  ötvözetek felhasználásának előnyeit.
\item
  Ismeri a fosszilis energiahordozók fogalmát és azok legfontosabb
  képviselőit, érti a kőolaj ipari lepárlásának elvét, ismeri a
  legfontosabb párlatok nevét, összetételét és felhasználási
  lehetőségeit, példát mond motorhajtó anyagokra, ismeri a
  töltőállomásokon kapható üzemanyagok típusait és azok felhasználását.
\item
  Ismeri a műanyag fogalmát és a műanyagok csoportosításának
  lehetőségeit eredetük, illetve hővel szemben mutatott viselkedésük
  alapján, konkrét példákat mond műanyagokra a környezetéből, érti azok
  felhasználásának előnyeit, ismeri a polimerizáció fogalmát, példát ad
  monomerekre és polimerekre, ismeri a műanyagok felhasználásának
  előnyeit és hátrányait, környezetre gyakorolt hatásukat.
\item
  Ismeri a mosó- és tisztítószerek, valamint a fertőtlenítőszerek
  fogalmi megkülönböztetését, példát mond a környezetéből gyakran
  használt mosó-és tisztítószerre és fertőtlenítőszerre, ismeri a
  szappan összetételét és a szappangyártás módját, ismeri a hypo kémiai
  összetételét és felhasználási módját, érti a mosószerek mosóaktív
  komponenseinek (a felületaktív részecskéknek) a mosásban betöltött
  szerepét.
\item
  Ismeri a kemény víz és a lágy víz közötti különbséget, érti a kemény
  víz és egyes mosószerek közötti kölcsönhatás (kicsapódás) folyamatát.
\item
  Ismeri a mindennapi életben előforduló növényvédő szerek használatának
  alapvető szabályait, értelmezi a növényvédő szereknek a leírását,
  felhasználási útmutatóját, példát mond a növényvédő szerekre a múltból
  és a jelenből (bordói lé, korszerű peszticidek), ismeri ezek hatásának
  elvi alapjait.
\item
  Ismeri a legfontosabb (n-, p-, k-tartalmú) műtrágyák kémiai
  összetételét, előállítását és felhasználásának szükségszerűségét.
\item
  Ismeri az atom felépítését, az elemi részecskéket, valamint azok
  jellemzőit, ismeri az izotópok legfontosabb tulajdonságait, érti a
  radioaktivitás lényegét és példát mond a radioaktív izotópok
  gyakorlati felhasználására.
\item
  Ismeri az atom elektronszerkezetének kiépülését a bohr-féle atommodell
  szintjén, tisztában van a vegyértékelektronok kémiai reakciókban
  betöltött szerepével.
\item
  Értelmezi a periódusos rendszer fontosabb adatait (vegyjel, rendszám,
  relatív atomtömeg), alkalmazza a periódusszám és a (fő)csoportszám
  jelentését a héjak és a vegyértékelektronok szempontjából, ismeri a
  periódusos rendszer fontosabb csoportjainak a nevét, és az azokat
  alkotó elemek vegyjelét.
\item
  Ismeri a molekulaképződés szabályait, ismeri az elektronegativitás
  fogalmát, és érti a kötéspolaritás lényegét, a kovalens kötést
  jellemzi száma és polaritása szerint, megalkotja egyszerű molekulák
  szerkezeti képletét, ismeri a legalapvetőbb molekulaalakokat
  (lineáris, síkháromszög, tetraéder, piramis, v-alak), valamint ezek
  meghatározó szerepét a molekulák polaritása szempontjából.
\item
  Meghatározza egyszerű molekulák polaritását, és ennek alapján
  következtet a közöttük kialakuló másodrendű kémiai kötésekre, valamint
  oldhatósági jellemzőikre, érti, hogy a moláris tömeg és a molekulák
  között fellépő másodrendű kötések hogyan befolyásolják az olvadás- és
  forráspontot, ezeket konkrét példákkal támasztja alá.
\item
  Érti a részecske szerkezete és az anyag fizikai és kémiai
  tulajdonságai közötti alapvető összefüggéseket.
\item
  Ismeri az egyszerű ionok atomokból való létrejöttének módját, ezt
  konkrét példákkal szemlélteti, ismeri a fontosabb összetett ionok
  molekulákból való képződésének módját, tudja a nevüket,
  összegképletüket, érti egy ionvegyület képletének a megszerkesztését
  az azt alkotó ionok képlete alapján, érti az ionrács felépülési elvét,
  az ionvegyület képletének jelentését, konkrét példák segítségével
  jellemzi az ionvegyületek fontosabb tulajdonságait.
\item
  Ismeri a fémek helyét a periódusos rendszerben, érti a fémes kötés
  kialakulásának és a fémek kristályszerkezetének a lényegét, érti a
  kapcsolatot a fémek kristályszerkezete és fontosabb tulajdonságai
  között, konkrét példák segítségével (pl. fe, al, cu) jellemzi a fémes
  tulajdonságokat, összehasonlításokat végez.
\item
  Ismeri a fémrács szerkezetét és az ebből adódó alapvető fizikai
  tulajdonságokat.
\item
  Ismeri az anyagok csoportosításának a módját a kémiai összetétel
  alapján, ismeri ezeknek az anyagcsoportoknak a legfontosabb közös
  tulajdonságait, példákat mond minden csoport képviselőire, tudja, hogy
  az oldatok a keverékek egy csoportja.
\item
  Érti a „hasonló a hasonlóban jól oldódik'' elvet, ismeri az oldatok
  töménységével és az oldhatósággal kapcsolatos legfontosabb
  ismereteket, egyszerű számítási feladatokat old meg az oldatok köréből
  (tömegszázalék, anyagmennyiség-koncentráció).
\item
  Adott szempontok alapján összehasonlítja a három halmazállapotba (gáz,
  folyadék, szilárd) tartozó anyagok általános jellemzőit, ismeri
  avogadro gáztörvényét és egyszerű számításokat végez gázok
  térfogatával standard körülmények között, érti a
  halmazállapot-változások lényegét és energiaváltozását.
\item
  Érti a fizikai és kémiai változások közötti különbségeket.
\item
  Ismeri a kémiai reakciók végbemenetelének feltételeit, ismeri, érti és
  alkalmazza a tömegmegmaradás törvényét a kémiai reakciókra.
\item
  Ismeri a kémiai reakciók csoportosítását többféle szempont szerint: a
  reagáló és a képződő anyagok száma, a reakció energiaváltozása,
  időbeli lefolyása, iránya, a reakcióban részt vevő anyagok
  halmazállapota szerint.
\item
  A kémiai reakciókat szimbólumokkal írja le.
\item
  Konkrét reakciókat termokémiai egyenlettel is felír, érti a
  termokémiai egyenlet jelentését, ismeri a reakcióhő fogalmát, a
  reakcióhő ismeretében megadja egy reakció energiaváltozását,
  energiadiagramot rajzol, értelmez, ismeri a termokémia főtételét és
  jelentőségét a többlépéses reakciók energiaváltozásának a
  meghatározásakor.
\item
  Érti a katalizátorok hatásának az elvi alapjait.
\item
  Ismer egyirányú és egyensúlyra vezető kémiai reakciókat, érti a
  dinamikus egyensúly fogalmát, ismeri és alkalmazza az egyensúly
  eltolásának lehetőségeit le châtelier elve alapján.
\item
  Ismeri a fontosabb savakat, bázisokat, azok nevét, képletét, brønsted
  sav-bázis elmélete alapján értelmezi a sav és bázis fogalmát, ismeri a
  savak és bázisok erősségének és értékűségének a jelentését, konkrét
  példát mond ezekre a vegyületekre, érti a víz sav-bázis
  tulajdonságait, ismeri az autoprotolízis jelenségét és a víz
  autoprotolízisének a termékeit.
\item
  Konkrét példákon keresztül értelmezi a redoxireakciókat oxigénfelvétel
  és oxigénleadás alapján, ismeri a redoxireakciók tágabb értelmezését
  elektronátmenet alapján is, konkrét példákon bemutatja a
  redoxireakciót, eldönti egy egyszerű redoxireakció egyenlete
  ismeretében az elektronátadás irányát, az oxidációt és redukciót,
  megadja az oxidálószert és a redukálószert.
\item
  Ismeri az anyagok jellemzésének logikus szempontrendszerét:
  anyagszerkezet -- fizikai tulajdonságok -- kémiai tulajdonságok --
  előfordulás -- előállítás -- felhasználás.
\item
  Ismeri a hidrogén, a halogének, a kalkogének, a nitrogén, a szén és
  fontosabb vegyületeik fizikai és kémiai sajátságait, különös
  tekintettel a köznapi életben előforduló anyagokra.
\item
  Alkalmazza az anyagok jellemzésének szempontjait a hidrogénre,
  kapcsolatot teremt az anyag szerkezete és tulajdonságai között.
\item
  Ismeri a halogének képviselőit, jellemzi a klórt, ismeri a
  hidrogén-klorid és a nátrium-klorid tulajdonságait.
\item
  Ismeri az oxigént és a vizet, ismeri az ózont, mint az oxigén allotróp
  módosulatát, ismeri mérgező hatását (szmogban) és uv-elnyelő hatását
  (ózonpajzsban).
\item
  Ismeri a ként, a kén-dioxidot és a kénsavat.
\item
  Ismeri a nitrogént, az ammóniát, a nitrogén-dioxidot és a
  salétromsavat.
\item
  Ismeri a vörösfoszfort és a foszforsavat, fontosabb tulajdonságaikat
  és a foszfor gyufagyártásban betöltött szerepét.
\item
  Összehasonlítja a gyémánt és a grafit szerkezetét és tulajdonságait,
  különbséget tesz a természetes és mesterséges szenek között, ismeri a
  természetes szenek felhasználását, ismeri a koksz és az aktív szén
  felhasználását, példát mond a szén reakcióira (pl. égés), ismeri a
  szén oxidjainak (co, co2) a tulajdonságait, élettani hatását, valamint
  a szénsavat és sóit, a karbonátokat.
\item
  Ismeri a fontosabb fémek (na, k, mg, ca, al, fe, cu, ag, au, zn)
  fizikai és kémiai tulajdonságait.
\item
  Kísérletek tapasztalatainak ismeretében értelmezi a fémek egymáshoz
  viszonyított reakciókészségét oxigénnel, sósavval, vízzel és más
  fémionok oldatával, érti a fémek redukáló sorának felépülését,
  következtet fémek reakciókészségére a sorban elfoglalt helyük alapján.
\item
  Ismeri a fémek helyét a periódusos rendszerben, megkülönbözteti az
  alkálifémeket, az alkáliföldfémeket, ismeri a vas, az alumínium, a
  réz, valamint a nemesfémek legfontosabb tulajdonságait.
\item
  Ismeri a fémek köznapi szempontból legfontosabb vegyületeit, azok
  alapvető tulajdonságait (nacl, na2co3, nahco3, na3po4, caco3,
  ca3(po4)2, al2o3, fe2o3, cuso4).
\item
  Tisztában van az elektrokémiai áramforrások felépítésével és
  működésével, ismeri a daniell-elem felépítését és az abban végbemenő
  folyamatokat, az elem áramtermelését.
\item
  Érti az elektromos áram és a kémiai reakciók közötti összefüggéseket:
  a galvánelemek áramtermelésének és az elektrolízisnek a lényegét.
\item
  Ismeri az elektrolizáló cella felépítését és az elektrolízis lényegét
  a hidrogén-klorid-oldat grafitelektródos elektrolízise kapcsán, érti,
  hogy az elektromos áram kémiai reakciók végbemenetelét segíti, példát
  ad ezek gyakorlati felhasználására (alumíniumgyártás, galvanizálás).
\item
  Ismer eljárásokat fémek ércekből történő előállítására (vas,
  alumínium).
\item
  Ismeri a szerves vegyületeket felépítő organogén elemeket, érti a
  szerves vegyületek megkülönböztetésének, külön csoportban
  tárgyalásának az okát, az egyszerűbb szerves vegyületeket szerkezeti
  képlettel és összegképlettel jelöli.
\item
  Ismeri a telített szénhidrogének homológ sorának felépülési elvét és
  fontosabb képviselőiket, ismeri a metán fontosabb tulajdonságait,
  jellemzi az anyagok szempontrendszere alapján, ismeri a homológ soron
  belül a forráspont változásának az okát, valamint a szénhidrogének
  oldhatóságát, ismeri és egy-egy kémiai egyenlettel leírja az égés, a
  szubsztitúció és a hőbontás folyamatát.
\item
  Érti az izoméria jelenségét, példákat mond konstitúciós izomerekre.
\item
  Ismeri a telítetlen szénhidrogének fogalmát, az etén és az acetilén
  szerkezetét és fontosabb tulajdonságait, ismeri és
  reakcióegyenletekkel leírja a telítetlen szénhidrogének jellemző
  reakciótípusait, az égést, az addíciót és a polimerizációt.
\item
  Ismeri a legegyszerűbb szerves kémiai reakciótípusokat.
\item
  Felismeri az aromás szerkezetet egy egyszerű vegyületben, ismeri a
  benzol molekulaszerkezetét és fontosabb tulajdonságait, tudja, hogy
  számos illékony aromás szénhidrogén mérgező.
\item
  Példát mond közismert halogéntartalmú szerves vegyületre (pl.
  kloroform, vinil-klorid, freonok, ddt, tetrafluoretén), és ismeri
  felhasználásukat.
\item
  Ismeri, és vegyületek képletében felismeri a legegyszerűbb
  oxigéntartalmú funkciós csoportokat: a hidroxilcsoportot, az
  oxocsoportot, az étercsoportot.
\item
  Ismeri az alkoholok fontosabb képviselőit (metanol, etanol, glikol,
  glicerin), azok fontosabb tulajdonságait, élettani hatásukat és
  felhasználásukat.
\item
  Felismeri az aldehidcsoportot, ismeri a formaldehid tulajdonságait, az
  aldehidek kimutatásának módját, felismeri a ketocsoportot, ismeri az
  aceton tulajdonságait, felhasználását.
\item
  Ismeri, és vegyületek képletében felismeri a karboxilcsoportot és az
  észtercsoportot, ismeri az egyszerűbb és fontosabb karbonsavak
  (hangyasav, ecetsav, zsírsavak) szerkezetét és lényeges
  tulajdonságaikat.
\item
  Az etil-acetát példáján bemutatja a kis szénatomszámú észterek
  jellemző tulajdonságait, tudja, hogy a zsírok, az olajok, a
  foszfatidok, a viaszok egyaránt az észterek csoportjába tartoznak.
\item
  Szerkezetük alapján felismeri az aminok és az amidok egyszerűbb
  képviselőit, ismeri az aminocsoportot és az amidcsoportot.
\item
  Ismeri a biológiai szempontból fontos szerves vegyületek építőelemeit
  (kémiai összetételét, a nagyobbak alkotó molekuláit).
\item
  Ismeri a lipid gyűjtőnevet, tudja, hogy ebbe a csoportba hasonló
  oldhatósági tulajdonságokkal rendelkező vegyületek tartoznak,
  felsorolja a lipidek legfontosabb képviselőit, felismeri azokat
  szerkezeti képlet alapján, ismeri a lipidek csoportjába tartozó
  vegyületek egy-egy fontos szerepét az élő szervezetben.
\item
  Ismeri a szénhidrátok legalapvetőbb csoportjait, példát mond mindegyik
  csoportból egy-két képviselőre, ismeri a szőlőcukor képletét,
  összefüggéseket talál a szőlőcukor szerkezete és tulajdonságai között,
  ismeri a háztartásban található szénhidrátok besorolását a megfelelő
  csoportba, valamint köznapi tulajdonságaikat (ízük, oldhatóságuk) és
  felhasználásukat, összehasonlítja a keményítő és a cellulóz
  molekulaszerkezetét és tulajdonságait, valamint szerepüket a
  szervezetben és a táplálékaink között.
\item
  Tudja, hogy a fehérjék aminosavakból épülnek fel, ismeri az aminosavak
  általános szerkezetét és azok legfontosabb tulajdonságait, ismeri a
  fehérjék elsődleges, másodlagos, harmadlagos és negyedleges
  szerkezetét, érti e fajlagos molekulák szerkezetének a kialakulását,
  példát mond a fehérjék szervezetben és élelmiszereinkben betöltött
  szerepére, ismeri a fehérjék kicsapásának módjait és ennek
  jelentőségét a mérgezések kapcsán.
\end{itemize}

\hypertarget{kornyezetismeret}{%
\subsection{Környezetismeret}\label{kornyezetismeret}}

\hypertarget{evfolyamon-20}{%
\subsubsection{4. évfolyamon}\label{evfolyamon-20}}

\begin{itemize}
\item
  Ismeretei bővítéséhez nyomtatott és digitális forrásokat használ.
\item
  Megfigyelés, mérés és kísérletezés közben szerzett tapasztalatairól
  szóban, rajzban, írásban beszámol.
\item
  Projektmunkában, csoportos tevékenységekben vesz részt.
\item
  Felismeri a helyi természet- és környezetvédelmi problémákat.
\item
  Szöveggel, táblázattal és jelekkel adott információkat értelmez.
\item
  Adott szempontok alapján algoritmus szerint élettelen anyagokon és
  élőlényeken megfigyeléseket végez.
\item
  Felismeri az élőlényeken, élettelen dolgokon az érzékelhető
  tulajdonságokat.
\item
  Összehasonlítja az élőlényeket és az élettelen anyagokat.
\item
  Adott szempontok alapján képes élettelen anyagokat összehasonlítani,
  csoportosítani.
\item
  Adott szempontok alapján képes élőlényeket összehasonlítani,
  csoportosítani.
\item
  Megfigyeléseinek, összehasonlításainak és csoportosításainak
  tapasztalatait szóban, rajzban, írásban rögzíti, megfogalmazza.
\item
  Figyelemmel kísér rövidebb-hosszabb ideig tartó folyamatokat (például
  olvadás, forrás, fagyás, párolgás, lecsapódás, égés, ütközés).
\item
  A megfigyelésekhez, összehasonlításokhoz és csoportosításokhoz
  kapcsolódó ismereteit felidézi.
\item
  Felismeri az élettelen anyagokon és az élőlényeken a mérhető
  tulajdonságokat.
\item
  Algoritmus szerint, előzetes viszonyítás, majd becslés után méréseket
  végez, becsült és mért eredményeit összehasonlítja.
\item
  A méréshez megválasztja az alkalmi vagy szabvány mérőeszközt,
  mértékegységet.
\item
  Az adott alkalmi vagy szabvány mérőeszközt megfelelően használja.
\item
  A mérésekhez kapcsolódó ismereteit felidézi.
\item
  A méréseket és azok tapasztalatait a mindennapi életben alkalmazza.
\item
  Tanítói segítséggel egyszerű kísérleteket végez.
\item
  A kísérletezés elemi lépéseit annak algoritmusa szerint megvalósítja.
\item
  A tanító által felvetett problémákkal kapcsolatosan hipotézist
  fogalmaz meg, a vizsgálatok eredményét összeveti hipotézisével.
\item
  Az adott kísérlethez választott eszközöket megfelelően használja.
\item
  A kísérletek tapasztalatait a mindennapi életben alkalmazza.
\item
  Figyelemmel kísér rövidebb-hosszabb ideig tartó folyamatokat (például
  a természet változásai, időjárási elemek).
\item
  A vizsgálatok tapasztalatait megfogalmazza, rajzban, írásban rögzíti.
\item
  A feladatvégzés során társaival együttműködik.
\item
  Megfelelően eligazodik az időbeli relációkban, ismeri és használja az
  életkorának megfelelő időbeli relációs szókincset.
\item
  Megfelelő sorrendben sorolja fel a napszakokat, a hét napjait, a
  hónapokat, az évszakokat, ismeri ezek időtartamát, relációit.
\item
  Felismeri a napszakok, évszakok változásai, valamint a föld mozgásai
  közötti összefüggéseket.
\item
  Az évszakokra vonatkozó megfigyeléseket végez, tapasztalatait rögzíti,
  és az adatokból következtetéseket von le.
\item
  Figyelemmel kísér rövidebb-hosszabb ideig tartó folyamatokat (például
  víz körforgása, emberi élet szakaszai, növények csírázása,
  növekedése).
\item
  Naptárt használ, időintervallumokat számol, adott eseményeket időrend
  szerint sorba rendez.
\item
  Napirendet tervez és használ.
\item
  Analóg és digitális óráról leolvassa a pontos időt.
\item
  Ismeri és használja az életkorának megfelelő térbeli relációs
  szókincset.
\item
  Megnevezi és iránytű segítségével megállapítja a fő- és
  mellékvilágtájakat.
\item
  Irányokat ad meg viszonyítással.
\item
  Megkülönböztet néhány térképfajtát: domborzati, közigazgatási,
  turista, autós.
\item
  Felismeri és használja az alapvető térképjeleket: felszínformák,
  vizek, települések, útvonalak, államhatárok.
\item
  A tanterméről, otthona valamely helyiségéről egyszerű alaprajzot
  készít és leolvas.
\item
  Tájékozódik az iskola környékéről és településéről készített
  térképvázlattal és térképpel, az iskola környezetéről egyszerű
  térképvázlatot készít.
\item
  Felismeri a különböző domborzati formákat, felszíni vizeket, ismeri
  jellemzőiket, ezeket terepasztalon vagy saját készítésű modellen
  előállítja.
\item
  Felismeri lakóhelyének jellegzetes felszínformáit.
\item
  Domborzati térképen felismeri a felszínformák és vizek jelölését.
\item
  Térkép segítségével megnevezi hazánk szomszédos országait, megyéit,
  saját megyéjét, megyeszékhelyét, környezetének nagyobb településeit,
  hazánk fővárosát, és ezeket megtalálja a térképen is.
\item
  Térkép segítségével megnevezi magyarország jellemző felszínformáit
  (síkság, hegy, hegység, domb, dombság), vizeit (patak, folyó, tó),
  ezeket terepasztalon vagy saját készítésű modellen előállítja.
\item
  Térkép segítségével megmutatja hazánk nagytájait, felismeri azok
  jellemző felszínformáit.
\item
  Felismeri, megnevezi és megfigyeli az életfeltételeket,
  életjelenségeket.
\item
  Felismeri, megnevezi és megfigyeli egy konkrét növény választott
  részeit, algoritmus alapján a részek tulajdonságait, megfogalmazza, mi
  a növényi részek szerepe a növény életében.
\item
  Növényt ültet és gondoz, megfigyeli a fejlődését, tapasztalatait
  rajzos formában rögzíti.
\item
  Felismeri, megnevezi és megfigyeli egy konkrét állat választott
  részeit, algoritmus alapján a részek tulajdonságait, megfogalmazza, mi
  a megismert rész szerepe az állat életében.
\item
  Algoritmus alapján megfigyeli és összehasonlítja a saját
  lakókörnyezetében fellelhető, jellemző növények és állatok jellemzőit,
  a megfigyelt tulajdonságok alapján csoportokba rendezi azokat.
\item
  Ismeri a lakóhelyéhez közeli életközösségek (erdő, mező-rét,
  víz-vízpart) főbb jellemzőit.
\item
  Algoritmus alapján megfigyeli és összehasonlítja hazánk természetes és
  mesterséges élőhelyein, életközösségeiben élő növények és állatok
  jellemzőit, a megfigyelt jellemzőik alapján csoportokba rendezi
  azokat.
\item
  Megnevezi a megismert életközösségekre jellemző élőlényeket, használja
  az életközösségekhez kapcsolódó kifejezéseket.
\item
  Konkrét példán keresztül megfigyeli és felismeri az élőhely, életmód
  és testfelépítés kapcsolatát.
\item
  Felismeri a lakóhelyéhez közeli életközösségek és az ott élő élőlények
  közötti különbségeket (pl. természetes -- mesterséges életközösség,
  erdő -- mező, rét -- víz, vízpart -- park, díszkert -- zöldséges,
  gyümölcsöskert esetében).
\item
  Megfigyeléseit mérésekkel (például időjárási elemek, testméret),
  modellezéssel, egyszerű kísérletek végzésével (például láb- és
  csőrtípusok) egészíti ki.
\item
  Felismeri, hogy az egyes fajok környezeti igényei eltérőek.
\item
  Felismeri az egyes életközösségek növényei és állatai közötti
  jellegzetes kapcsolatokat.
\item
  Példákkal mutatja be az emberi tevékenység természeti környezetre
  gyakorolt hatását, felismeri a természetvédelem jelentőségét.
\item
  Egyéni és közösségi környezetvédelmi cselekvési formákat ismer meg és
  gyakorol közvetlen környezetében.
\item
  Megnevezi az ember életszakaszait.
\item
  Ismeri az emberi szervezet fő életfolyamatait.
\item
  Megnevezi az emberi test részeit, fő szerveit, ismeri ezek működését,
  szerepét.
\item
  Megnevezi az érzékszerveket és azok szerepét a megismerési
  folyamatokban.
\item
  Belátja az érzékszervek védelmének fontosságát, és ismeri ezek
  eszközeit, módjait.
\item
  Ismer betegségeket, felismeri a legjellemzőbb betegségtüneteket, a
  betegségek megelőzésének alapvető módjait.
\item
  Felismeri az egészséges, gondozott környezet jellemzőit,
  megfogalmazza, milyen hatással van a környezet az egészségére.
\item
  Tisztában van az egészséges életmód alapelveivel, összetevőivel, az
  emberi szervezet egészséges testi és lelki fejlődéséhez szükséges
  szokásokkal,azokat igyekszik betartani.
\item
  Felismeri, mely anyagok szennyezhetik környezetünket a mindennapi
  életben, mely szokások vezetnek környezetünk károsításához, egyéni és
  közösségi környezetvédelmi cselekvési formákat ismer meg és gyakorol
  közvetlen környezetében (pl. madárbarát kert, iskolakert kiépítésében,
  fenntartásában való részvétel, iskolai környezet kialakításában,
  rendben tartásában való részvétel, települési természet- és
  környezetvédelmi tevékenységben való részvétel).
\item
  Elsajátít olyan szokásokat és viselkedésformákat, amelyek a
  károsítások megelőzésére irányulnak (pl. hulladékminimalizálás --
  anyagtakarékosság, újrahasználat és -felhasználás, tömegközlekedés,
  gyalogos vagy kerékpáros közlekedés előnyben részesítése,
  energiatakarékosság).
\item
  Felelősségtudattal rendelkezik a szűkebb, illetve tágabb környezete
  iránt.
\item
  Felismeri az élőlényeken, élettelen anyagokon az érzékelhető és
  mérhető tulajdonságokat.
\item
  Adott szempontok alapján élettelen anyagokat és élőlényeket
  összehasonlít, csoportosít.
\item
  Azonosítja az anyagok halmazállapotait, megnevezi és összehasonlítja
  azok alapvető jellemzőit.
\item
  Egyszerű kísérletek során megfigyeli a halmazállapot-változásokat:
  fagyás, olvadás, forrás, párolgás, lecsapódás.
\item
  Megismeri és modellezi a víz természetben megtett útját, felismeri a
  folyamat ciklikus jellegét.
\item
  Tanítói segítséggel égéssel kapcsolatos egyszerű kísérleteket végez,
  csoportosítja a megvizsgált éghető és éghetetlen anyagokat.
\item
  Megfogalmazza a tűz és az égés szerepét az ember életében.
\item
  Megnevezi az időjárás fő elemeit, időjárási megfigyeléseket tesz,
  méréseket végez.
\item
  Megfigyeli a mozgások sokféleségét, csoportosítja a mozgásformákat:
  hely- és helyzetváltoztató mozgás.
\item
  Megfigyeli a növények csírázásának és növekedésének feltételeit,
  ezekre vonatkozóan egyszerű kísérletet végez.
\end{itemize}

\hypertarget{magyar-nyelv-es-irodalom}{%
\subsection{Magyar nyelv és irodalom}\label{magyar-nyelv-es-irodalom}}

\hypertarget{evfolyamon-21}{%
\subsubsection{1-4. évfolyamon}\label{evfolyamon-21}}

\begin{itemize}
\item
  Az életkorának és egyéni adottságainak megfelelő, hallott és olvasott
  szövegeket megérti.
\item
  Felkészülés után tagolt szöveget érthetően és pontosan olvas hangosan.
\item
  Életkorának megfelelően és adottságaihoz mérten kifejezően, érthetően,
  az élethelyzethez igazodva kommunikál.
\item
  Az életkorának és egyéni képességeinek megfelelően alkot szövegeket
  szóban és írásban.
\item
  Segítséggel egyéni érdeklődésének megfelelő olvasmányt választ,
  amelyről beszámol.
\item
  Érdeklődésének megfelelően, hagyományos és digitális szövegek által
  bővíti ismereteit.
\item
  Megfogalmazza saját álláspontját, véleményét.
\item
  Egyéni sajátosságaihoz mérten törekszik a rendezett írásképre,
  esztétikus füzetvezetésre.
\item
  A tanult nyelvi, nyelvtani, helyesírási ismereteket képességeihez
  mérten alkalmazza.
\item
  Élményeket és tapasztalatokat szerez változatos irodalmi szövegek
  megismerésével, olvasásával.
\item
  Részt vesz a testséma-tudatosságot fejlesztő tevékenységekben
  (szem-kéz koordináció, térérzékelés, irányok, arányok, jobb-bal oldal
  összehangolása, testrészek tudatosítása) és érzékelő játékokban.
\item
  Megérti és használja a tér- és időbeli tájékozódáshoz szükséges
  szókincset.
\item
  Észleli, illetve megérti a nyelv alkotóelemeit, hangot, betűt,
  szótagot, szót, mondatot, szöveget, és azokra válaszokat fogalmaz meg.
\item
  Beszédlégzése és artikulációja megfelelő; figyelmet fordít a hangok
  időtartamának helyes ejtésére, a beszéd helyes ritmusára, hangsúlyára,
  tempójára, az élethelyzetnek megfelelő hangerőválasztásra.
\item
  A szavakat hangokra, szótagokra bontja.
\item
  Hangokból, szótagokból szavakat épít.
\item
  Biztosan ismeri az olvasás jelrendszerét.
\item
  Felismeri, értelmezi a szövegben a számára ismeretlen szavakat,
  kifejezéseket; digitális forrásokat is használ.
\item
  Egyszerű magyarázat, szemléltetés (szóbeli, képi, dramatikus
  tevékenység) alapján megérti az új kifejezés jelentését.
\item
  A megismert szavakat, kifejezéseket a nyelvi fejlettségi szintjén
  alkalmazza.
\item
  Használ életkorának megfelelő digitális és hagyományos szótárakat.
\item
  Adottságaihoz mérten, életkorának megfelelően szöveget hangos vagy
  néma olvasás útján megért.
\item
  Részt vesz népmesék és műmesék, regék, mondák, történetek közös
  olvasásában és feldolgozásában.
\item
  Rövid meséket közösen olvas, megért, feldolgoz.
\item
  Néma olvasás útján megérti az írott utasításokat, közléseket,
  kérdéseket, azokra adekvát módon reflektál.
\item
  Megérti a közösen olvasott rövid szövegeket, részt vesz azok
  olvasásában, feldolgozásában.
\item
  Önállóan, képek, grafikai szervezők (kerettörténet, történettérkép,
  mesetáblázat, karakter-térkép, történetpiramis stb.) segítségével vagy
  tanítói segédlettel a szöveg terjedelmétől függően összefoglalja a
  történetet.
\item
  Értő figyelemmel követi a tanító, illetve társai felolvasását.
\item
  Felkészülés után tagolt szöveget érthetően olvas hangosan.
\item
  A szöveg megértését igazoló feladatokat végez.
\item
  Önállóan, képek vagy tanítói segítség alapján a szöveg terjedelmétől
  függően kiemeli annak lényeges elemeit, összefoglalja azt.
\item
  Alkalmaz alapvető olvasási stratégiákat.
\item
  Az olvasott szöveghez illusztrációt készít, a hiányos illusztrációt
  kiegészíti, vagy a meglévőt társítja a szöveggel.
\item
  Az olvasott szövegekben kulcsszavakat azonosít, a főbb szerkezeti
  egységeket önállóan vagy segítséggel elkülöníti.
\item
  Egyszerű, játékos formában megismerkedik a szövegek különböző
  modalitásával, médiumok szövegalkotó sajátosságainak alapjaival.
\item
  Megérti a szóbeli utasításokat, kérdéseket, az adottságainak és
  életkorának megfelelő szöveg tartalmát.
\item
  Mozgósítja a hallott szöveg tartalmával kapcsolatos ismereteit,
  élményeit, tapasztalatait, és összekapcsolja azokat.
\item
  Megérti az életkorának megfelelő nyelvi és nem nyelvi üzeneteket, és
  azokra a kommunikációs helyzetnek megfelelően reflektál.
\item
  Részt vesz nagymozgást és finommotorikát fejlesztő tevékenységekben,
  érzékelő játékokban.
\item
  Tér- és síkbeli tájékozódást fejlesztő feladatokat megold.
\item
  Saját tempójában elsajátítja az anyanyelvi írás jelrendszerét.
\item
  Szavakat, szószerkezeteket, 3-4 szavas mondatokat leír megfigyelés,
  illetve diktálás alapján.
\item
  Az egyéni sajátosságaihoz mérten olvashatóan ír, törekszik a rendezett
  írásképre, esztétikus füzetvezetésre.
\item
  Adottságaihoz mérten, életkorának megfelelően érthetően, az
  élethelyzethez igazodva kommunikál.
\item
  Részt vesz a kortársakkal és felnőttekkel való kommunikációban,
  beszélgetésben, vitában, és alkalmazza a megismert kommunikációs
  szabályokat.
\item
  Használja a kapcsolat-felvételi, kapcsolattartási, kapcsolatlezárási
  formákat: köszönés, kérés, megszólítás, kérdezés; testtartás,
  testtávolság, tekintettartás, hangsúly, hanglejtés, hangerő, hangszín,
  megköszönés, elköszönés.
\item
  Élményeiről segítséggel vagy önállóan beszámol.
\item
  Megadott szempontok alapján szóban mondatokat és 3-4 mondatos szöveget
  alkot.
\item
  Bekapcsolódik párbeszédek, dramatikus helyzetgyakorlatok, szituációs
  játékok megalkotásába.
\item
  A tanult verseket, mondókákat, rövidebb szövegeket szöveghűen,
  érthetően tolmácsolja.
\item
  A hallás és olvasás alapján megfigyelt szavakat, szószerkezeteket,
  mondatokat önállóan leírja.
\item
  Egyéni képességeinek megfelelően alkot szövegeket írásban.
\item
  Gondolatait, érzelmeit, véleményét a kommunikációs helyzetnek
  megfelelően, néhány mondatban írásban is megfogalmazza.
\item
  A szövegalkotáskor törekszik a megismert helyesírási szabályok
  alkalmazására, meglévő szókincsének aktivizálására.
\item
  Tanítói segítséggel megadott rímpárokból, különböző témákban 2--4
  soros verset alkot.
\item
  Megadott szempontok alapján rövid mesét ír, kiegészít vagy átalakít.
\item
  A megismert irodalmi szövegekhez, iskolai eseményekhez plakátot,
  meghívót, saját programjaihoz meghívót készít hagyományosan és
  digitálisan.
\item
  Alapvető hagyományos és digitális kapcsolattartó formákat alkalmaz.
\item
  Nyitott az irodalmi művek befogadására.
\item
  Könyvet kölcsönöz a könyvtárból, és azt el is olvassa, élményeit,
  gondolatait megosztja.
\item
  Ajánlással, illetve egyéni érdeklődésének és az életkori
  sajátosságainak megfelelően választott irodalmi alkotást ismer meg.
\item
  Részt vesz az adott közösség kultúrájának megfelelő gyermekirodalmi mű
  közös olvasásában, és nyitott annak befogadására.
\item
  Verbális és vizuális módon vagy dramatikus eszközökkel reflektál a
  szövegre, megfogalmazza a szöveg alapján benne kialakult képet.
\item
  Részt vesz népmesék és műmesék közös olvasásában, feldolgozásában.
\item
  Különböző célú, rövidebb tájékoztató, ismeretterjesztő szövegeket
  olvas hagyományos és digitális felületen.
\item
  Ismer és használ az életkorának megfelelő nyomtatott és digitális
  forrásokat az ismeretei bővítéséhez, rendszerezéséhez.
\item
  A mesék, történetek szereplőinek cselekedeteiről kérdéseket fogalmaz
  meg, véleményt alkot.
\item
  Megfogalmazza, néhány érvvel alátámasztja saját álláspontját;
  meghallgatja társai véleményét.
\item
  Különbséget tesz mesés és valószerű történetek között.
\item
  Megfigyeli és összehasonlítja a történetek tartalmát és a saját
  élethelyzetét.
\item
  Részt vesz dramatikus játékokban.
\item
  A feladatvégzéshez szükséges személyes élményeit, előzetes tudását
  felidézi.
\item
  Képzeletét a megértés érdekében mozgósítja.
\item
  Ismer és alkalmaz néhány alapvető tanulási technikát.
\item
  Gyakorolja az ismeretfeldolgozás egyszerű technikáit.
\item
  Információkat, adatokat gyűjt a szövegből, kiemeli a bekezdések
  lényegét; tanítói segítséggel vagy önállóan megfogalmazza azt.
\item
  Bővíti a témáról szerzett ismereteit egyéb források feltárásával,
  gyűjtőmunkával, könyvtárhasználattal, filmek, médiatermékek
  megismerésével.
\item
  Írásbeli munkáját segítséggel vagy önállóan ellenőrzi és javítja.
\item
  Megfigyeli, és tapasztalati úton megkülönbözteti egymástól a
  magánhangzókat és a mássalhangzókat, valamint időtartamukat.
\item
  Különbséget tesz az egyjegyű, a kétjegyű és a háromjegyű betűk között.
\item
  A hangjelölés megismert szabályait jellemzően helyesen alkalmazza a
  tanult szavakban.
\item
  A mondatot nagybetűvel kezdi, alkalmazza a mondat hanglejtésének, a
  beszélő szándékának megfelelő mondatvégi írásjeleket.
\item
  Biztosan ismeri a kis- és nagybetűs ábécét, azonos és különböző
  betűkkel kezdődő szavakat betűrendbe sorol; a megismert szabályokat
  alkalmazza digitális felületen való kereséskor is.
\item
  Felismeri, jelentésük alapján csoportosítja, és önállóan vagy
  segítséggel helyesen leírja az élőlények, tárgyak, gondolati dolgok
  nevét.
\item
  A több hasonló élőlény, tárgy nevét kis kezdőbetűvel írja.
\item
  A személyneveket, állatneveket és a lakóhelyhez kötődő helyneveket
  nagy kezdőbetűvel írja le.
\item
  Törekszik a tanult helyesírási ismeretek alkalmazására.
\item
  Kérdésre adott válaszában helyesen toldalékolja a szavakat.
\item
  Önállóan felismeri és elkülöníti az egytövű ismert szavakban a
  szótövet és a toldalékot.
\item
  Biztosan szótagol, alkalmazza az elválasztás szabályait.
\item
  Helyesen alkalmazza a szóbeli és írásbeli szövegalkotásában az idő
  kifejezésének nyelvi eszközeit.
\item
  A kiejtéssel megegyező rövid szavak leírásában követi a helyesírás
  szabályait.
\item
  A kiejtéstől eltérő ismert szavakat megfigyelés, szóelemzés
  alkalmazásával megfelelően leírja.
\item
  Ellentétes jelentésű és rokon értelmű kifejezéseket gyűjt, azokat a
  beszédhelyzetnek megfelelően használja az írásbeli és szóbeli
  szövegalkotásban.
\item
  Megkülönbözteti a szavak egyes és többes számát.
\item
  Felismeri és önállóan vagy segítséggel helyesen leírja a tulajdonságot
  kifejező szavakat és azok fokozott alakjait.
\item
  Felismeri, önállóan vagy segítséggel helyesen leírja az ismert
  cselekvést kifejező szavakat.
\item
  Megkülönbözteti a múltban, jelenben és jövőben zajló cselekvéseket,
  történéseket.
\item
  Ismer és ért számos egyszerű közmondást és szólást, szóláshasonlatot,
  közmondást, találós kérdést, nyelvtörőt, kiszámolót, mondókát.
\item
  Megérti és használja az ismert állandósult szókapcsolatokat.
\item
  Különféle módokon megjeleníti az ismert szólások, közmondások
  jelentését.
\item
  Élményeket és tapasztalatokat szerez változatos irodalmi szövegtípusok
  és műfajok -- magyar klasszikus, kortárs magyar alkotások --
  megismerésével.
\item
  Élményeket és tapasztalatokat szerez néhány szövegtípusról és
  műfajról, szépirodalmi és ismeretközlő szövegről.
\item
  Részt vesz különböző műfajú és megjelenésű szövegek olvasásában és
  feldolgozásában.
\item
  A közös olvasás, szövegfeldolgozás során megismer néhány életkorának
  megfelelő mesét, elbeszélést.
\item
  Megtapasztalja az életkorának, érdeklődésének megfelelő szövegek
  befogadásának és előadásának élményét és örömét.
\item
  Megfigyeli a költői nyelv sajátosságait; élményeit az általa
  választott módon megfogalmazza, megjeleníti.
\item
  Részt vesz ismert szövegek (magyar népi mondókák, kiszámolók,
  nyelvtörők, népdalok, klasszikus és kortárs magyar gyerekversek,
  mesék) mozgásos-játékos feldolgozásában, dramatikus elemekkel történő
  élményszerű megjelenítésében, érzületileg, lelkületileg átérzi azokat.
\item
  Segítséggel vagy önállóan előad ritmuskísérettel verseket.
\item
  Olvas és megért rövidebb nép- és műköltészeti alkotásokat, rövidebb
  epikai műveket, verseket.
\item
  Megtapasztalja a vershallgatás, a versmondás, a versolvasás örömét és
  élményét.
\item
  Érzékeli és átéli a vers ritmusát és hangulatát.
\item
  A versek hangulatát kifejezi különféle érzékszervi tapasztalatok
  segítségével (színek, hangok, illatok, tapintási élmények stb.).
\item
  A tanító vagy társai segítségével, együttműködésével verssorokat,
  versrészleteket memorizál.
\item
  Felismeri, indokolja a cím és a szöveg közötti összefüggést,
  azonosítja a történetekben, elbeszélő költeményekben a helyszínt, a
  szereplőket, a konfliktust és annak megoldását.
\item
  Szövegszerűen felidézi kölcsey ferenc: himnusz, vörösmarty mihály:
  szózat, petőfi sándor: nemzeti dal című verseinek részleteit.
\item
  Részt vesz rövid mesék, történetek dramatikus, bábos és egyéb
  vizuális, digitális eszközökkel történő megjelenítésében, saját
  gondolkodási és nyelvi szintjén megfogalmazza a szöveg hatására benne
  kialakult képet.
\item
  Megismer néhány mesét és történetet a magyar és más népek irodalmából.
\item
  Megismer néhány klasszikus verset a magyar irodalomból.
\item
  Élményt és tapasztalatot szerez különböző ritmikájú lírai művek
  megismerésével a kortárs és a klasszikus magyar gyermeklírából és a
  népköltészeti alkotásokból.
\item
  Segítséggel, majd önállóan szöveghűen felidéz néhány könnyen
  tanulható, rövidebb verset, mondókát, versrészletet, prózai és
  dramatikus szöveget, szövegrészletet.
\item
  A tanult verseket, mondókákat, rövidebb szövegeket szöveghűen,
  érthetően tolmácsolja.
\item
  Részt vesz legalább két hosszabb terjedelmű magyar gyermekirodalmi
  alkotás feldolgozásában.
\item
  Jellemző és ismert részletek alapján azonosítja a nemzeti ünnepeken
  elhangzó költemények részleteit, szerzőjüket megnevezi.
\item
  Megéli és az általa választott formában megjeleníti a közösséghez
  tartozás élményét.
\item
  Megismer a jeles napokhoz, ünnepekhez kapcsolódó szövegeket, dalokat,
  szokásokat, népi gyermekjátékokat.
\item
  Megfigyeli az ünnepek, hagyományok éves körforgását.
\item
  Nyitottá válik a magyarság értékeinek megismerésére, megalapozódik
  nemzeti identitástudata, történelmi szemlélete.
\item
  Képes családjából származó közösségi élményeit megfogalmazni,
  összevetni az iskolai élet adottságaival, a témakört érintő
  beszélgetésekben aktívan részt venni.
\item
  Törekszik a világ tapasztalati úton történő megismerésére, értékeinek
  tudatos megóvására.
\item
  Ismeri a keresztény, keresztyén ünnepköröket (karácsony, húsvét,
  pünkösd), jelképeket, nemzeti és állami ünnepeket (március 15.,
  augusztus 20., október 23.), népszokásokat (márton-nap, luca-nap,
  betlehemezés, húsvéti locsolkodás, pünkösdölés).
\item
  Ismerkedik régi magyar mesterségekkel, irodalmi művek olvasásával és
  gyűjtőmunkával.
\item
  Megismer gyermekirodalmi alkotás alapján készült filmet,
  médiaterméket.
\item
  Részt vesz gyerekeknek szóló kiállítások megismerésében, alkotásaival
  hozzájárul létrehozásukhoz.
\item
  Megismer a szűkebb környezetéhez kötődő irodalmi és kulturális
  emlékeket, emlékhelyeket.
\item
  Megismeri saját lakóhelyének irodalmi és kulturális értékeit.
\end{itemize}

\hypertarget{evfolyamon-22}{%
\subsubsection{5-8. évfolyamon}\label{evfolyamon-22}}

\begin{itemize}
\item
  Elkülöníti a nyelv szerkezeti egységeit, megnevezi a tanult elemeket.
\item
  Ismeri a magyar hangrendszer főbb jellemzőit és a hangok kapcsolódási
  szabályait. írásban helyesen jelöli.
\item
  Funkciójuk alapján felismeri és megnevezi a szóelemeket és szófajokat.
\item
  A szövegben felismer és funkciójuk alapján azonosít alapvető és
  gyakori szószerkezeteket (alanyos, határozós, jelzős, tárgyas).
\item
  Szerkezetük alapján megkülönbözteti az egyszerű és összetett
  mondatokat.
\item
  Felismeri és elemzi a főbb szóelemek mondat- és szövegbeli szerepét,
  törekszik helyes alkalmazásukra.
\item
  Megfigyeli és elemzi a mondat szórendjét, a szórendi változatok,
  valamint a környező szöveg kölcsönhatását.
\item
  Érti és megnevezi a tanult nyelvi egységek szövegbeli szerepét.
\item
  Felismeri és megnevezi a főbb szóalkotási módokat: szóösszetétel,
  szóképzés, néhány ritkább szóalkotási mód.
\item
  Tanári irányítással, néhány szempont alapján összehasonlítja az
  anyanyelv és a tanult idegen nyelv sajátosságait.
\item
  Megfigyeli, elkülöníti és funkciót társítva értelmezi a környezetében
  előforduló nyelvváltozatokat (nyelvjárások, csoportnyelvek,
  rétegnyelvek).
\item
  Felismeri és megnevezi a nyelvi és nem nyelvi kommunikáció elemeit.
\item
  Használja a digitális kommunikáció eszközeit, megnevezi azok főbb
  jellemző tulajdonságait.
\item
  Megfigyeli és értelmezi a tömegkommunikáció társadalmat befolyásoló
  szerepét.
\item
  Felismeri a kommunikáció főbb zavarait, alkalmaz korrekciós
  lehetőségeket.
\item
  Alkalmazza az általa tanult nyelvi, nyelvtani, helyesírási,
  nyelvhelyességi ismereteket.
\item
  Ismeri és alkalmazza helyesírásunk alapelveit: kiejtés, szóelemzés,
  hagyomány, egyszerűsítés.
\item
  Társai és saját munkájában a tanult formáktól eltérő, gyakran
  előforduló helyesírási hibákat felismeri és javítja.
\item
  A tanuló etikusan és kritikusan használja a hagyományos papíralapú,
  illetve a világhálón található és egyéb digitális adatbázisokat.
\item
  Elolvassa a kötelező olvasmányokat, és saját örömére is olvas.
\item
  Megérti az irodalmi mű szövegszerű és elvont jelentéseit.
\item
  A tanult fogalmakat használva beszámol a megismert műről.
\item
  Megismeri és elkülöníti a műnemeket, illetve a műnemekhez tartozó főbb
  műfajokat, felismeri és megnevezi azok poétikai és retorikai
  jellemzőit.
\item
  Összekapcsol irodalmi műveket különböző szempontok alapján (téma,
  műfaj, nyelvi kifejezőeszközök).
\item
  Elkülöníti az irodalmi művek főbb szerkezeti egységeit.
\item
  Összehasonlít egy adott irodalmi művet annak adaptációival (film,
  festmény, stb.).
\item
  Felismeri, és bemutatja egy adott műfaj főbb jellemzőit.
\item
  Az irodalmi szövegek befogadása során felismer és értelmez néhány
  alapvető nyelvi-stilisztikai eszközt: szóképek, alakzatok stb..
\item
  A megismert epikus művek cselekményét összefoglalja, fordulópontjait
  önállóan ismerteti.
\item
  Elkülöníti és jellemzi a fő- és mellékszereplőket, megkülönbözteti a
  helyszíneket, az idősíkokat, azonosítja az előre- és visszautalásokat.
\item
  Felismeri és a szövegből vett példával igazolja az elbeszélő és a
  szereplő nézőpontja közötti eltérést.
\item
  A szövegből vett idézetekkel támasztja alá, és saját szavaival
  fogalmazza meg a lírai szöveg hangulati jellemzőit, a befogadás során
  keletkezett érzéseit és gondolatait.
\item
  Azonosítja a versben a lírai én különböző megszólalásait, és
  elkülöníti a mű szerkezeti egységeit.
\item
  Felismeri a tanult alapvető rímképleteket (páros rím, keresztrím,
  bokorrím), és azonosítja az olvasott versek ritmikai, hangzásbeli
  hasonlóságait és különbségeit.
\item
  Ismer és felismer néhány alapvető lírai műfajt.
\item
  Szöveghűen, értőn mondja el a memoriterként tanult lírai műveket.
\item
  Drámai műveket olvas és értelmez, előadásokat, feldolgozásokat megnéz
  (pl.: mesedráma, vígjáték, ifjúsági regény adaptációja).
\item
  A nemzeti hagyomány szempontjából meghatározó néhány mű esetén
  bemutatja a szerzőhöz és a korszakhoz kapcsolódó legfőbb jellemzőket.
\item
  Megérti az életkorának megfelelő hallott és olvasott szövegeket.
\item
  Kifejezően tudja olvasni és értelmezni az életkorának megfelelő
  különböző műfajú és megjelenésű szövegeket. a tanuló felismeri és a
  tanár segítségével értelmezi a számára ismeretlen kifejezéseket.
\item
  Felismeri és szükség szerint a tanár segítségével értelmezi a
  szövegben számára ismeretlen kifejezéseket.
\item
  A lábjegyzetek, a digitális és nyomtatott szótárak használatával
  önállóan értelmezi az olvasott szöveget.
\item
  A pedagógus irányításával kiválasztja a rendelkezésre álló digitális
  forrásokból a megfelelő információkat.
\item
  Különbséget tesz a jelentésszerkezetben a szó szerinti és metaforikus
  értelmezés között.
\item
  Alkalmazza a különböző olvasási típusokat és szöveg-feldolgozási
  módszereket.
\item
  Összekapcsolja ismereteit a szöveg tartalmával, és reflektál azok
  összefüggéseire.
\item
  Az olvasott szövegeket szerkezeti egységekre tagolja.
\item
  Szóbeli vagy képi módszerekkel megfogalmazza, megjeleníti a szöveg
  alapján kialakult érzéseit, gondolatait.
\item
  Az életkorának megfelelő szöveg alapján jegyzetet, vázlatot készít.
\item
  A tanulási tevékenységében hagyományos és digitális forrásokat
  használ, ezt mérlegelő gondolkodással és etikusan teszi.
\item
  Érthetően, a kommunikációs helyzetnek megfelelően beszél.
\item
  Gondolatait, érzelmeit, véleményét a kommunikációs helyzetnek
  megfelelően, érvekkel alátámasztva fogalmazza meg, és mások véleményét
  is figyelembe veszi.
\item
  A tanult szövegeket szöveghűen és mások számára követhetően
  tolmácsolja.
\item
  Az általa tanult hagyományos és digitális szövegtípusok megfelelő
  tartalmi és műfaji követelményeinek megfelelően alkot szövegeket.
\item
  A szövegalkotás során alkalmazza a tanult helyesírási és szerkesztési
  szabályokat, használja a hagyományos és a digitális helyesírási
  szabályzatot és szótárt.
\item
  Az egyéni sajátosságaihoz mérten tagolt, rendezett, áttekinthető
  írásképpel, egyértelmű javításokkal alkot szöveget.
\item
  Tanári segítséggel kreatív szöveget alkot a megismert műhöz
  kapcsolódóan hagyományos és digitális formában.
\item
  Egyszerű rímes és rímtelen verset alkot.
\item
  Életkorának megfelelő irodalmi szövegeket olvas.
\item
  Érthetően, kifejezően és pontosan olvas.
\item
  Egy általa elolvasott művet ajánl kortársainak.
\item
  A tanult szövegeket szöveghűen és mások számára követhetően
  tolmácsolja.
\item
  Megfogalmazza vagy társaival együttműködve drámajátékban megjeleníti
  egy mű megismerése során szerzett tapasztalatait, élményeit.
\item
  Személyes véleményt alakít ki a szövegek és művek által felvetett
  problémákról (pl. döntési helyzetek, motivációk, konfliktusok), és
  véleményét indokolja.
\item
  Megérti mások álláspontját, elfogadja azt, vagy a sajátja mellett
  érveket fogalmaz meg.
\item
  Személyes tapasztalatait összeköti a művekben megismert
  konfliktusokkal, érzelmi állapotokkal.
\item
  A feladatvégzés során hatékony közös munkára, együttműködésre
  törekszik.
\end{itemize}

\hypertarget{evfolyamon-23}{%
\subsubsection{9-12. évfolyamon}\label{evfolyamon-23}}

\begin{itemize}
\item
  Az anyanyelvről szerzett ismereteit alkalmazva képes a
  kommunikációjában a megfelelő nyelvváltozat kiválasztására,
  használatára.
\item
  Felismeri a kommunikáció zavarait, kezelésükre stratégiát dolgoz ki.
\item
  Felismeri és elemzi a tömegkommunikáció befolyásoló eszközeit, azok
  céljait és hatásait.
\item
  Reflektál saját kommunikációjára, szükség esetén változtat azon.
\item
  Ismeri az anyanyelvét, annak szerkezeti felépítését, nyelvhasználata
  tudatos és helyes.
\item
  Ismeri a magyar nyelv hangtanát, alaktanát, szófajtanát, mondattanát,
  ismeri és alkalmazza a tanult elemzési eljárásokat.
\item
  Felismeri és megnevezi a magyar és a tanult idegen nyelv közötti
  hasonlóságokat és eltéréseket.
\item
  Ismeri a szöveg fogalmát, jellemzőit, szerkezeti sajátosságait,
  valamint a különféle szövegtípusokat és megjelenésmódokat.
\item
  Felismeri és alkalmazza a szövegösszetartó grammatikai és jelentésbeli
  elemeket, szövegépítése arányos és koherens.
\item
  Ismeri a stílus fogalmát, a stíluselemeket, a stílushatást, a
  stíluskorszakokat, stílusrétegeket, ismereteit a szöveg befogadása és
  alkotása során alkalmazza.
\item
  Szövegelemzéskor felismeri az alakzatokat és a szóképeket, értelmezi
  azok hatását, szerepét, megnevezi típusaikat.
\item
  Ismeri a nyelvhasználatban előforduló különféle nyelvváltozatokat
  (nyelvjárások, csoportnyelvek, rétegnyelvek), összehasonlítja azok
  főbb jellemzőit.
\item
  Alkalmazza az általa tanult nyelvi, nyelvtani, helyesírási,
  nyelvhelyességi ismereteket.
\item
  A retorikai ismereteit a gyakorlatban is alkalmazza.
\item
  Ismeri és érti a nyelvrokonság fogalmát, annak kritériumait.
\item
  Ismeri a magyar nyelv eredetének hipotéziseit, és azok tudományosan
  megalapozott bizonyítékait.
\item
  Érti, hogy nyelvünk a történelemben folyamatosan változik, ismeri a
  magyar nyelvtörténet nagy korszakait, kiemelkedő jelentőségű
  nyelvemlékeit.
\item
  Ismeri a magyar nyelv helyesírási, nyelvhelyességi szabályait.
\item
  Tud helyesen írni, szükség esetén nyomtatott és digitális helyesírási
  segédleteket használ.
\item
  Etikusan és kritikusan használja a hagyományos, papír alapú, illetve a
  világhálón található és egyéb digitális adatbázisokat.
\item
  Elolvassa a kötelező olvasmányokat, és saját örömére is olvas.
\item
  Felismeri és elkülöníti a műnemeket, illetve a műnemekhez tartozó
  műfajokat, megnevezi azok poétikai és retorikai jellemzőit.
\item
  Megérti, elemzi az irodalmi mű jelentésszerkezetének szintjeit.
\item
  Értelmezésében felhasználja irodalmi és művészeti, történelmi,
  művelődéstörténeti ismereteit.
\item
  Összekapcsolja az irodalmi művek szerkezeti felépítését, nyelvi
  sajátosságait azok tartalmával és értékszerkezetével.
\item
  Az irodalmi mű értelmezése során figyelembe veszi a mű
  keletkezéstörténeti hátterét, a műhöz kapcsolható filozófiai,
  eszmetörténeti szempontokat is.
\item
  Összekapcsolja az irodalmi művek szövegének lehetséges értelmezéseit
  azok társadalmi-történelmi szerepével, jelentőségével.
\item
  Összekapcsol irodalmi műveket különböző szempontok alapján (motívumok,
  történelmi, erkölcsi kérdésfelvetések, művek és parafrázisaik).
\item
  Összehasonlít egy adott irodalmi művet annak adaptációival (film,
  festmény, zenemű, animáció, stb.), összehasonlításkor figyelembe veszi
  az adott művészeti ágak jellemző tulajdonságait.
\item
  Epikai és drámai művekben önállóan értelmezi a cselekményszálak, a
  szerkezet, az időszerkezet (lineáris, nem lineáris), a helyszínek és a
  jellemek összefüggéseit.
\item
  Epikai és drámai művekben rendszerbe foglalja a szereplők viszonyait,
  valamint összekapcsolja azok motivációját és cselekedeteit.
\item
  Epikai művekben értelmezi a különböző elbeszélésmódok szerepét
  (tudatábrázolás, egyenes és függő beszéd, mindentudó és korlátozott
  elbeszélő stb.).
\item
  A drámai mű értelmezésében alkalmazza az általa tanult drámaelméleti
  és drámatörténeti fogalmakat (pl. analitikus és abszurd dráma, epikus
  színház, elidegenedés).
\item
  A líra mű értelmezésében alkalmazza az általa tanult líraelméleti és
  líratörténeti fogalmakat (pl. lírai én, beszédhelyzetek, beszédmódok,
  ars poetica, szereplíra).
\item
  A tantárgyhoz kapcsolódó fogalmakkal bemutatja a lírai mű hangulati és
  hangnemi sajátosságait, hivatkozik a mű verstani felépítésére.
\item
  Szükség esetén a mű értelmezéséhez felhasználja történeti ismereteit.
\item
  A mű értelmezésében összekapcsolja a szöveg poétikai tulajdonságait a
  mű nemzeti hagyományban betöltött szerepével.
\item
  Tájékozottságot szerez régiója magyar irodalmáról.
\item
  Tanulmányai során ismereteket szerez a kulturális intézmények (múzeum,
  könyvtár, színház) és a nyomtatott, illetve digitális formában
  megjelenő kulturális folyóiratok, adatbázisok működéséről.
\item
  Különböző megjelenésű, típusú, műfajú, korú és összetettségű
  szövegeket olvas, értelmez.
\item
  A különböző olvasási típusokat és a szövegfeldolgozási stratégiákat a
  szöveg típusának és az olvasás céljának megfelelően választja ki és
  kapcsolja össze.
\item
  A megismert szöveg tartalmi és nyelvi minőségéről érvekkel
  alátámasztott véleményt alkot.
\item
  Hosszabb terjedelmű szöveg alapján többszintű vázlatot vagy részletes
  gondolattérképet készít.
\item
  Azonosítja a szöveg szerkezeti elemeit, és figyelembe veszi azok
  funkcióit a szöveg értelmezésekor.
\item
  Egymással összefüggésben értelmezi a szöveg tartalmi elemeit és a
  hozzá kapcsolódó illusztrációkat, ábrákat.
\item
  Különböző típusú és célú szövegeket hallás alapján értelmez és
  megfelelő stratégia alkalmazásával értékel és összehasonlít.
\item
  Összefüggő szóbeli szöveg (előadás, megbeszélés, vita) alapján
  önállóan vázlatot készít.
\item
  Felismeri és értelmezésében figyelembe veszi a hallott és az írott
  szövegek közötti funkcionális és stiláris különbségeket.
\item
  Folyamatos és nem folyamatos, hagyományos és digitális szövegeket
  olvas és értelmez maga által választott releváns szempontok alapján.
\item
  Feladatai megoldásához önálló kutatómunkát végez nyomtatott és
  digitális forrásokban, ezek eredményeit szintetizálja.
\item
  Felismeri és értelmezi a szövegben a kétértelműséget és a félrevezető
  információt, valamint elemzi és értelmezi a szerző szándékát.
\item
  Megtalálja a közös és eltérő jellemzőket a hagyományos és a digitális
  technikával előállított, tárolt szövegek között, és véleményt formál
  azok sajátosságairól.
\item
  Törekszik arra, hogy a különböző típusú, stílusú és regiszterű
  szövegekben megismert, számára új kifejezéseket beépítse szókincsébe,
  azokat adekvát módon használja.
\item
  Önállóan értelmezi az ismeretlen kifejezéseket a szövegkörnyezet vagy
  digitális, illetve nyomtatott segédeszközök használatával.
\item
  Ismeri a tanult tantárgyak, tudományágak szakszókincsét, azokat a
  beszédhelyzetnek megfelelően használja.
\item
  Megadott szempontrendszer alapján szóbeli feleletet készít.
\item
  Képes eltérő műfajú szóbeli szövegek alkotására: felelet, kiselőadás,
  hozzászólás, felszólalás.
\item
  Rendelkezik korának megfelelő retorikai ismeretekkel.
\item
  Felismeri és megnevezi a szóbeli előadásmód hatáskeltő eszközeit,
  hatékonyan alkalmazza azokat.
\item
  Írásbeli és szóbeli nyelvhasználata, stílusa az adott kommunikációs
  helyzetnek megfelelő. írásképe tagolt, beszéde érthető, artikulált.
\item
  A tanult szövegtípusoknak megfelelő tartalommal és szerkezettel
  önállóan alkot különféle írásbeli szövegeket.
\item
  Az írásbeli szövegalkotáskor alkalmazza a tanult szerkesztési,
  stilisztikai ismereteket és a helyesírási szabályokat.
\item
  Érvelő esszét alkot megadott szempontok vagy szövegrészletek alapján.
\item
  Ismeri, érti és etikusan alkalmazza a hagyományos, digitális és
  multimédiás szemléltetést.
\item
  Különböző, a munka világában is használt hivatalos szövegeket alkot
  hagyományos és digitális felületeken (pl. kérvény, beadvány,
  nyilatkozat, egyszerű szerződés, meghatalmazás, önéletrajz, motivációs
  levél).
\item
  Megadott vagy önállóan kiválasztott szempontok alapján az irodalmi
  művekről elemző esszét ír.
\item
  A kötelező olvasmányokat elolvassa, és saját örömére is olvas.
\item
  Tudatosan keresi a történeti és esztétikai értékekkel rendelkező
  olvasmányokat, műalkotásokat.
\item
  Olvasmányai kiválasztásakor figyelembe veszi az alkotások kulturális
  regiszterét.
\item
  Társai érdeklődését figyelembe véve ajánl olvasmányokat.
\item
  Választott olvasmányaira is vonatkoztatja a tanórán megismert
  kontextusteremtő eljárások tanulságait.
\item
  Önismeretét irodalmi művek révén fejleszti.
\item
  Részt vesz irodalmi mű kreatív feldolgozásában, bemutatásában (pl.
  animáció, dramaturgia, átirat).
\item
  A környező világ jelenségeiről, szövegekről, műalkotásokról véleményt
  alkot, és azt érvekkel támasztja alá.
\item
  Megnyilvánulásaiban, a vitákban alkalmazza az érvelés alapvető
  szabályait.
\item
  Vitahelyzetben figyelembe veszi mások álláspontját, a lehetséges
  ellenérveket is.
\item
  Feladatai megoldásához önálló kutatómunkát végez nyomtatott és
  digitális forrásokban, a források tartalmát mérlegelő módon gondolja
  végig.
\item
  A feladatokat komplex szempontoknak megfelelően oldja meg, azokat
  kiegészíti saját szempontjaival.
\item
  A kommunikációs helyzetnek és a célnak megfelelően tudatosan
  alkalmazza a beszélt és írott nyelvet, reflektál saját és társai
  nyelvhasználatára.
\end{itemize}

\hypertarget{matematika}{%
\subsection{Matematika}\label{matematika}}

\hypertarget{evfolyamon-24}{%
\subsubsection{1-4. évfolyamon}\label{evfolyamon-24}}

\begin{itemize}
\item
  Tudatos megfigyeléseket tesz a környező világ tárgyaira, ezek
  viszonyára vonatkozóan.
\item
  Tájékozódik a környező világ mennyiségi és formai világában.
\item
  Megérti a tanult ismereteket és használja azokat a feladatok megoldása
  során.
\item
  A környezetében lévő dolgokat szétválogatja, összehasonlítja és
  rendszerezi egy-két szempont alapján.
\item
  Jártas a mérőeszközök használatában, a mérési módszerekben.
\item
  Helyes képzete van a természetes számokról, érti a számnevek és
  számjelek épülésének rendjét.
\item
  Helyesen értelmezi az alapműveleteket tevékenységekkel, szövegekkel,
  és jártas azok elvégzésében fejben és írásban.
\item
  Megfigyeli jelenségek matematikai tartalmát, és le tudja ezeket írni
  számokkal, műveletekkel vagy geometriai alakzatokkal.
\item
  Életkorának megfelelően eligazodik környezetének térbeli és időbeli
  viszonyaiban.
\item
  Érti a korának megfelelő, matematikai tartalmú hallott és olvasott
  szövegeket.
\item
  Megszerzett ismereteit digitális eszközökön is alkalmazza.
\item
  Megkülönböztet, azonosít egyedi konkrét látott, hallott, mozgással,
  tapintással érzékelhető tárgyakat, dolgokat, helyzeteket, jeleket.
\item
  Válogatásokat végez saját szempont szerint személyek, tárgyak, dolgok,
  számok között.
\item
  Felismeri a mások válogatásában együvé kerülő dolgok közös és a
  különválogatottak eltérő tulajdonságát.
\item
  Folytatja a megkezdett válogatást felismert szempont szerint.
\item
  Személyek, tárgyak, dolgok, szavak, számok közül kiválogatja az adott
  tulajdonsággal rendelkező összes elemet.
\item
  Azonosítja a közös tulajdonsággal rendelkező dolgok halmazába nem való
  elemeket.
\item
  Megnevezi egy adott tulajdonság szerint ki nem válogatott elemek közös
  tulajdonságát a tulajdonság tagadásával.
\item
  Barkochbázik valóságos és elképzelt dolgokkal is, kerüli a felesleges
  kérdéseket.
\item
  Halmazábrán is elhelyez elemeket adott címkék szerint.
\item
  Adott, címkékkel ellátott halmazábrán elhelyezett elemekről eldönti,
  hogy a megfelelő helyre kerültek-e; a hibás elhelyezést javítja.
\item
  Talál megfelelő címkéket halmazokba rendezett elemekhez.
\item
  Megfogalmaz adott halmazra vonatkozó állításokat; értelemszerűen
  használja a „mindegyik'', „nem mindegyik'', „van köztük\ldots{}'',
  „egyik sem\ldots'' és a velük rokon jelentésű kifejezéseket.
\item
  Két szempontot is figyelembe vesz egyidejűleg.
\item
  Két meghatározott tulajdonság egyszerre történő figyelembevételével
  szétválogat adott elemeket: tárgyakat, személyeket, szavakat,
  számokat, alakzatokat.
\item
  Felsorol elemeket konkrét halmazok közös részéből.
\item
  Megfogalmazza a halmazábra egyes részeibe kerülő elemek közös,
  meghatározó tulajdonságát; helyesen használja a logikai „nem'' és a
  logikai „és'' szavakat, valamint velük azonos értelmű kifejezéseket.
\item
  Megítéli, hogy adott halmazra vonatkozó állítás igaz-e, vagy hamis.
\item
  Keresi az okát annak, ha a halmazábra valamelyik részébe nem kerülhet
  egyetlen elem sem.
\item
  Hiányos állításokat igazzá tevő elemeket válogat megadott
  alaphalmazból.
\item
  Összehasonlít véges halmazokat az elemek száma szerint.
\item
  Ismeri két halmaz elemeinek kölcsönösen egyértelmű megfeleltetését
  (párosítását) az elemszámok szerinti összehasonlításra.
\item
  Helyesen alkalmazza a feladatokban a több, kevesebb, ugyanannyi
  fogalmakat 10 000-es számkörben.
\item
  Helyesen érti és alkalmazza a feladatokban a „valamennyivel'' több,
  kevesebb fogalmakat.
\item
  Adott elemeket elrendez választott és megadott szempont szerint is.
\item
  Sorba rendezett elemek közé elhelyez további elemeket a felismert
  szempont szerint.
\item
  Két, három szempont szerint elrendez adott elemeket többféleképpen is;
  segédeszközként használja a táblázatos elrendezést és a fadiagramot.
\item
  Megkeresi egyszerű esetekben a két, három feltételnek megfelelő összes
  elemet, alkotást.
\item
  Megfogalmazza a rendezés felismert szempontjait.
\item
  Megkeresi két, három szempont szerint teljes rendszert alkotó,
  legfeljebb 48 elemű készlet hiányzó elemeit, felismeri az elemek által
  meghatározott rendszert.
\item
  A tevékenysége során felmerülő problémahelyzetben megoldást keres.
\item
  Megfogalmazott problémát tevékenységgel, megjelenítéssel,
  átfogalmazással értelmez.
\item
  Az értelmezett problémát megoldja.
\item
  A problémamegoldás során a sorrendben végzett tevékenységeket szükség
  szerint visszafelé is elvégzi.
\item
  Megoldását értelmezi, ellenőrzi.
\item
  Kérdést tesz fel a megfogalmazott probléma kapcsán.
\item
  Értelmezi, elképzeli, megjeleníti a szöveges feladatban megfogalmazott
  hétköznapi szituációt.
\item
  Szöveges feladatokban megfogalmazott hétköznapi problémát megold
  matematikai ismeretei segítségével.
\item
  Tevékenység, ábrarajzolás segítségével megold egyszerű,
  következtetéses, szöveges feladatokat.
\item
  Megkülönbözteti az ismert és a keresendő (ismeretlen) adatokat.
\item
  Megkülönbözteti a lényeges és a lényegtelen adatokat.
\item
  Az értelmezett szöveges feladathoz hozzákapcsol jól megismert
  matematikai modellt.
\item
  A megválasztott modellen belül meghatározza a keresett adatokat.
\item
  A modellben kapott megoldást értelmezi az eredeti problémára; arra
  vonatkoztatva ellenőrzi a megoldást.
\item
  Választ fogalmaz meg a felvetett kérdésre.
\item
  Tudatosan emlékezetébe vési az észlelt tárgyakat, személyeket,
  dolgokat, és ezek jellemző tulajdonságait, elrendezését, helyzetét.
\item
  Tudatosan emlékezetébe vés szavakat, számokat, utasítást, adott
  helyzetre vonatkozó megfogalmazást.
\item
  Kérésre, illetve problémahelyzetben felidézi a kívánt, szükséges
  emlékképet.
\item
  Alkalmazza a felismert törvényszerűségeket analógiás esetekben.
\item
  Összekapcsolja az azonos matematikai tartalmú tevékenységek során
  szerzett tapasztalatait.
\item
  Példákat gyűjt konkrét tapasztalatai alapján matematikai állítások
  alátámasztására.
\item
  Egy állításról ismeretei alapján eldönti, hogy igaz vagy hamis.
\item
  Ismeretei alapján megfogalmaz önállóan is egyszerű állításokat.
\item
  Egy- és többszemélyes logikai játékban döntéseit mérlegelve előre
  gondolkodik.
\item
  Érti és helyesen használja a több, kevesebb, ugyanannyi relációkat
  halmazok elemszámával kapcsolatban, valamint a kisebb, nagyobb,
  ugyanakkora relációkat a megismert mennyiségekkel (hosszúság, tömeg,
  űrtartalom, idő, terület, pénz) kapcsolatban 10 000-es számkörben.
\item
  Használja a kisebb, nagyobb, egyenlő kifejezéseket a természetes
  számok körében.
\item
  Kis darabszámokat ránézésre felismer többféle rendezett alakban.
\item
  Megszámlál és leszámlál; adott (alkalmilag választott vagy szabványos)
  egységgel meg- és kimér a 10 000-es számkörben; oda-vissza számlál
  kerek tízesekkel, százasokkal, ezresekkel.
\item
  Ismeri a következő becslési módszereket: közelítő számlálás, közelítő
  mérés, mérés az egység többszörösével; becslését finomítja
  újrabecsléssel.
\item
  Nagyság szerint sorba rendez számokat, mennyiségeket.
\item
  Megadja és azonosítja számok sokféle műveletes alakját.
\item
  Megtalálja a számok helyét, közelítő helyét egyszerű számegyenesen,
  számtáblázatokban, a számegyenesnek ugyanahhoz a pontjához rendeli a
  számokat különféle alakjukban a 10 000-es számkörben.
\item
  Megnevezi a 10 000-es számkör számainak egyes, tízes, százas, ezres
  szomszédjait, tízesekre, százasokra, ezresekre kerekített értékét.
\item
  Számokat jellemez tartalmi és formai tulajdonságokkal.
\item
  Számot jellemez más számokhoz való viszonyával.
\item
  Helyesen írja az arab számjeleket.
\item
  Ismeri a római számjelek közül az i, v, x jeleket, hétköznapi
  helyzetekben felismeri az ezekkel képzett számokat.
\item
  Összekapcsolja a tízes számrendszerben a számok épülését a különféle
  számrendszerekben végzett tevékenységeivel.
\item
  Érti a számok ezresekből, százasokból, tízesekből és egyesekből való
  épülését, ezresek, százasok, tízesek és egyesek összegére való
  bontását.
\item
  Érti a számok számjegyeinek helyi, alaki, valódi értékét.
\item
  Helyesen írja és olvassa a számokat a tízes számrendszerben 10 000-ig.
\item
  Megbecsül, mér alkalmi és szabványos mértékegységekkel hosszúságot,
  tömeget, űrtartalmat és időt.
\item
  Helyesen alkalmazza a mérési módszereket, használ skálázott
  mérőeszközöket, helyes képzete van a mértékegységek nagyságáról.
\item
  Helyesen használja a hosszúságmérés, az űrtartalommérés és a
  tömegmérés szabványegységei közül a következőket: mm, cm, dm, m, km;
  ml, cl, dl, l; g, dkg, kg.
\item
  Ismeri az időmérés szabványegységeit: az órát, a percet, a
  másodpercet, a napot, a hetet, a hónapot, az évet.
\item
  Ismer hazai és külföldi pénzcímleteket 10 000-es számkörben.
\item
  Alkalmazza a felváltást és beváltást különböző pénzcímletek között.
\item
  Összeveti azonos egységgel mért mennyiség és mérőszáma nagyságát,
  összeveti ugyanannak a mennyiségnek a különböző egységekkel való
  mérésekor kapott mérőszámait.
\item
  Megméri különböző sokszögek kerületét különböző egységekkel.
\item
  Területet mér különböző egységekkel lefedéssel vagy darabolással.
\item
  Ismer a terület és kerület mérésére irányuló tevékenységeket.
\item
  Helyesen értelmezi a 10 000-es számkörben az összeadást, a kivonást, a
  szorzást, a bennfoglaló és az egyenlő részekre osztást.
\item
  Helyesen használja a műveletek jeleit.
\item
  Hozzákapcsolja a megfelelő műveletet adott helyzethez, történéshez,
  egyszerű szöveges feladathoz.
\item
  Értelmezi a műveleteket megjelenítéssel, modellezéssel, szöveges
  feladattal.
\item
  Megérti a következő kifejezéseket: tagok, összeg, kisebbítendő,
  kivonandó, különbség, tényezők, szorzandó, szorzó, szorzat, osztandó,
  osztó, hányados, maradék.
\item
  Számolásaiban felhasználja a műveletek közti kapcsolatokat, számolásai
  során alkalmazza konkrét esetekben a legfontosabb műveleti
  tulajdonságokat.
\item
  Megold hiányos műveletet, műveletsort az eredmény ismeretében, a
  műveletek megfordításával is.
\item
  Alkalmazza a műveletekben szereplő számok (kisebbítendő, kivonandó és
  különbség; tagok és összeg; tényezők és szorzat; osztandó, osztó és
  hányados) változtatásának következményeit.
\item
  Szöveghez, valós helyzethez kapcsolva zárójelet tartalmazó műveletsort
  értelmez, elvégez.
\item
  Alkalmazza a számolást könnyítő eljárásokat.
\item
  Fejben pontosan összead és kivon a 100-as számkörben.
\item
  Érti a szorzó- és bennfoglaló táblák kapcsolatát.
\item
  Emlékezetből tudja a kisegyszeregy és a megfelelő bennfoglalások,
  egyenlő részekre osztások eseteit a számok tízszereséig.
\item
  Fejben pontosan számol a 100-as számkörben egyjegyűvel való szorzás és
  maradék nélküli osztás során.
\item
  Fejben pontosan számol a 10 000-es számkörben a 100-as számkörben
  végzett műveletekkel analóg esetekben.
\item
  Érti a 10-zel, 100-zal, 1000-rel való szorzás, osztás kapcsolatát a
  helyiérték-táblázatban való jobbra, illetve balra tolódással, fejben
  pontosan számol a 10 000-es számkörben a számok 10-zel, 100-zal,
  1000-rel történő szorzásakor és maradék nélküli osztásakor.
\item
  Elvégzi a feladathoz szükséges észszerű becslést, mérlegeli a becslés
  során kapott eredményt.
\item
  Teljes négyjegyűek összegét, különbségét százasokra kerekített
  értékekkel megbecsüli, teljes kétjegyűek két- és egyjegyűvel való
  szorzatát megbecsüli.
\item
  Helyesen végzi el az írásbeli összeadást, kivonást.
\item
  Helyesen végzi el az írásbeli szorzást egy- és kétjegyű szorzóval, az
  írásbeli osztást egyjegyű osztóval.
\item
  Tevékenységekkel megjelenít egységtörteket és azok többszöröseit
  különféle mennyiségek és többféle egységválasztás esetén.
\item
  A kirakást, mérést és a rajzot mint modellt használja a törtrészek
  összehasonlítására.
\item
  A negatív egész számokat irányított mennyiségként (hőmérséklet,
  tengerszint alatti magasság, idő) és hiányként (adósság) értelmezi.
\item
  Nagyság szerint összehasonlítja a természetes számokat és a negatív
  egész számokat a használt modellen belül.
\item
  Szabadon épít, kirak formát, mintát adott testekből, síklapokból.
\item
  Minta alapján létrehoz térbeli, síkbeli alkotásokat.
\item
  Sormintát, síkmintát felismer, folytat.
\item
  Alkotásában követi az adott feltételeket.
\item
  Testeket épít élekből, lapokból; elkészíti a testek élvázát, hálóját;
  testeket épít képek, alaprajzok alapján; elkészíti egyszerű testek
  alaprajzát.
\item
  Síkidomokat hoz létre különféle eszközök segítségével.
\item
  Alaklemezt, vonalzót, körzőt használ alkotáskor.
\item
  Megtalálja az összes, több feltételnek megfelelő építményt, síkbeli
  kirakást.
\item
  Megfogalmazza az alkotásai közti különbözőséget.
\item
  Megkülönbözteti és szétválogatja szabadon választott vagy
  meghatározott geometriai tulajdonságok szerint a gyűjtött, megalkotott
  testeket, síkidomokat.
\item
  Megfigyeli az alakzatok közös tulajdonságát, megfelelő címkéket talál
  megadott és halmazokba rendezett alakzatokhoz.
\item
  Megtalálja a közös tulajdonsággal nem rendelkező alakzatokat.
\item
  Megnevezi a tevékenységei során előállított, válogatásai során
  előkerülő alakzatokon megfigyelt tulajdonságokat.
\item
  Különbséget tesz testek és síkidomok között.
\item
  Megnevezi a sík és görbült felületeket, az egyenes és görbe vonalakat,
  szakaszokat tapasztalati ismeretei alapján.
\item
  Kiválasztja megadott síkidomok közül a sokszögeket.
\item
  Megnevezi a háromszögeket, négyszögeket, köröket.
\item
  Megkülönböztet tükrösen szimmetrikus és tükrösen nem szimmetrikus
  síkbeli alakzatokat.
\item
  Megszámlálja az egyszerű szögletes test lapjait.
\item
  Megnevezi a téglatest lapjainak alakját, felismeri a téglatesten az
  egybevágó lapokat, megkülönbözteti a téglatesten az éleket, csúcsokat.
\item
  Tudja a téglalap oldalainak és csúcsainak számát, összehajtással
  megmutatja a téglalap szögeinek egyenlőségét.
\item
  Megmutatja a téglalap azonos hosszúságú oldalait és elhelyezkedésüket,
  megmutatja és megszámlálja a téglalap átlóit és szimmetriatengelyeit.
\item
  Megfigyeli a kocka mint speciális téglatest és a négyzet mint
  speciális téglalap tulajdonságait.
\item
  Megnevezi megfigyelt tulajdonságai alapján a téglatestet, kockát,
  téglalapot, négyzetet.
\item
  Megfigyelt tulajdonságaival jellemzi a létrehozott síkbeli és térbeli
  alkotást, mintázatot.
\item
  Tapasztalattal rendelkezik mozgással, kirakással a tükörkép
  előállításáról.
\item
  Szimmetrikus alakzatokat hoz létre térben, síkban különböző
  eszközökkel; felismeri a szimmetriát valóságos dolgokon, síkbeli
  alakzatokon.
\item
  Megépíti, kirakja, megrajzolja hálón, jelölés nélküli lapon sablonnal,
  másolópapír segítségével alakzat tükörképét, eltolt képét.
\item
  Ellenőrzi a tükrözés, eltolás helyességét tükör vagy másolópapír
  segítségével.
\item
  Követi a sormintában vagy a síkmintában lévő szimmetriát.
\item
  Térben, síkban az eredetihez hasonló testeket, síkidomokat alkot
  nagyított vagy kicsinyített elemekből; az eredetihez hasonló
  síkidomokat rajzol hálón.
\item
  Helyesen használja az irányokat és távolságokat jelölő kifejezéseket
  térben és síkon.
\item
  Tájékozódik lakóhelyén, bejárt terepen: bejárt útvonalon visszatalál
  adott helyre, adott utca és házszám alapján megtalál házat.
\item
  Térképen, négyzethálón megtalál pontot két adat segítségével.
\item
  Részt vesz memóriajátékokban különféle tulajdonságok szerinti párok
  keresésében.
\item
  Megfogalmazza a személyek, tárgyak, dolgok, időpontok, számok, testek,
  síklapok közötti egyszerű viszonyokat, kapcsolatokat.
\item
  Érti a problémákban szereplő adatok viszonyát.
\item
  Megfogalmazza a felismert összefüggéseket.
\item
  Összefüggéseket keres sorozatok elemei között.
\item
  Megadott szabály szerint sorozatot alkot; megértett probléma
  értelmezéséhez, megoldásához sorozatot, táblázatot állít elő
  modellként.
\item
  Tárgyakkal, logikai készletek elemeivel kirakott periodikus
  sorozatokat folytat.
\item
  Elsorolja az évszakokat, hónapokat, napokat, napszakokat egymás után,
  tetszőleges kezdőponttól is.
\item
  Ismert műveletekkel alkotott sorozat, táblázat szabályát felismeri;
  ismert szabály szerint megkezdett sorozatot, táblázatot helyesen,
  önállóan folytat.
\item
  Tárgyakkal, számokkal kapcsolatos gépjátékhoz szabályt alkot;
  felismeri az egyszerű gép megfordításával nyert gép szabályát.
\item
  Felismer kapcsolatot elempárok, elemhármasok tagjai között.
\item
  Szabályjátékok során létrehoz a felismert kapcsolat alapján további
  elempárokat, elemhármasokat.
\item
  A sorozatban, táblázatban, gépjátékokban felismert összefüggést
  megfogalmazza saját szavaival, nyíljelöléssel vagy nyitott mondattal.
\item
  Adatokat gyűjt a környezetében.
\item
  Adatokat rögzít későbbi elemzés céljából.
\item
  Gyűjtött adatokat táblázatba rendez, diagramon ábrázol.
\item
  Adatokat gyűjt ki táblázatból, adatokat olvas le diagramról.
\item
  Jellemzi az összességeket.
\item
  Részt vesz olyan játékokban, kísérletekben, melyekben a véletlen
  szerepet játszik.
\item
  Különbséget tesz tapasztalatai alapján a „biztos'', „lehetetlen'',
  „lehetséges, de nem biztos'' események között.
\item
  Megítéli „biztos'', „lehetetlen'', „lehetséges, de nem biztos''
  eseményekkel kapcsolatos állítások igazságát.
\item
  Tapasztalatai alapján tippet fogalmaz meg arról, hogy két esemény
  közül melyik esemény valószínűbb olyan véletlentől függő szituációk
  során, melyekben a két esemény valószínűsége között jól belátható a
  különbség.
\item
  Tetszőleges vagy megadott módszerrel összeszámlálja az egyes
  kimenetelek előfordulásait olyan egyszerű játékokban, kísérletekben,
  amelyekben a véletlen szerepet játszik.
\item
  A valószínűségi játékokban, kísérletekben megfogalmazott előzetes
  sejtését, tippjét összeveti a megfigyelt előfordulásokkal.
\item
  Önállóan értelmezi a hallott, olvasott matematikai tartalmú szöveget.
\item
  Helyesen használja a mennyiségi viszonyokat kifejező szavakat,
  nyelvtani szerkezeteket.
\item
  Szöveges feladatokban a különböző kifejezésekkel megfogalmazott
  műveleteket megérti.
\item
  Megfelelő szókincset és jeleket használ mennyiségi viszonyok
  kifejezésére szóban és írásban.
\item
  Megfelelően használja szóban és írásban a nyelvtani szerkezeteket
  matematikai tartalmuk szerint.
\item
  Szöveget, ábrát alkot matematikai jelekhez, műveletekhez.
\item
  Játékos feladatokban személyeket, tárgyakat, számokat, formákat néhány
  meghatározó tulajdonsággal jellemez.
\item
  Kérdést fogalmaz meg, ha munkája során nehézségbe ütközik.
\item
  Nyelvi szempontból megfelelő választ ad a feladatokban megjelenő
  kérdésekre.
\item
  Használ matematikai képességfejlesztő számítógépes játékokat,
  programokat.
\item
  Alkalmazza a tanult infokommunikációs ismereteket matematikai
  problémák megoldása során.
\end{itemize}

\hypertarget{evfolyamon-25}{%
\subsubsection{5-8. évfolyamon}\label{evfolyamon-25}}

\begin{itemize}
\item
  Rendelkezik a matematikai problémamegoldáshoz szükséges
  eszközrendszerrel, melyet az adott problémának megfelelően tud
  alkalmazni.
\item
  Felismeri a hétköznapi helyzetekben a matematikai vonatkozásokat, és
  ezek leírására megfelelő modellt használ.
\item
  Megfogalmaz sejtéseket, és logikus érveléssel ellenőrzi azokat.
\item
  Helyesen használja a matematikai jelöléseket írásban.
\item
  Olvassa és érti az életkorának megfelelő matematikai tartalmú
  szövegeket.
\item
  Tanulási módszerei változatosak: szóbeli közlés, írott szöveg és
  digitális csatornák útján egyaránt képes az ismeretek elsajátítására.
\item
  Matematikai ismereteit össze tudja kapcsolni más tanulásterületeken
  szerzett tapasztalatokkal.
\item
  Matematikai ismereteit alkalmazza a pénzügyi tudatosság területét
  érintő feladatok megoldásában.
\item
  Különböző szövegekhez megfelelő modelleket készít.
\item
  Számokat, számhalmazokat, halmazműveleti eredményeket számegyenesen
  ábrázol.
\item
  Konkrét szituációkat szemléltet gráfok segítségével.
\item
  Egyismeretlenes elsőfokú egyenletet lebontogatással és mérlegelvvel
  megold.
\item
  Állítások logikai értékét (igaz vagy hamis) megállapítja.
\item
  Igaz és hamis állításokat fogalmaz meg.
\item
  Tanult minták alapján néhány lépésből álló bizonyítási gondolatsort
  megért és önállóan összeállít.
\item
  A logikus érvelésben a matematikai szaknyelvet következetesen
  alkalmazza társai meggyőzésére.
\item
  Elemeket halmazba rendez több szempont alapján.
\item
  Részhalmazokat konkrét esetekben felismer és ábrázol.
\item
  Véges halmaz kiegészítő halmazát (komplementerét), véges halmazok
  közös részét (metszetét), egyesítését (unióját) képezi és ábrázolja
  konkrét esetekben.
\item
  A természetes számokat osztóik száma alapján és adott számmal való
  osztási maradékuk szerint csoportosítja.
\item
  Síkbeli tartományok közül kiválasztja a szögtartományokat, nagyság
  szerint összehasonlítja, méri, csoportosítja azokat.
\item
  Csoportosítja a háromszögeket szögeik és oldalaik szerint.
\item
  Ismeri a speciális négyszögek legfontosabb tulajdonságait, ezek
  alapján elkészíti a halmazábrájukat.
\item
  Valószínűségi játékokat, kísérleteket végez, ennek során az adatokat
  tervszerűen gyűjti, rendezi és ábrázolja digitálisan is.
\item
  Összeszámlálási feladatok megoldása során alkalmazza az összes eset
  áttekintéséhez szükséges módszereket.
\item
  A háromszögek és a speciális négyszögek tulajdonságait alkalmazza
  feladatok megoldásában.
\item
  A kocka, a téglatest, a hasáb, a gúla, a gömb tulajdonságait
  alkalmazza feladatok megoldásában.
\item
  Konkrét adatsor esetén átlagot számol, megállapítja a leggyakoribb
  adatot (módusz), a középső adatot (medián), és ezeket összehasonlítja.
\item
  Matematikából, más tantárgyakból és a mindennapi életből vett egyszerű
  szöveges feladatokat következtetéssel vagy egyenlettel megold.
\item
  Gazdasági, pénzügyi témájú egyszerű szöveges feladatokat
  következtetéssel vagy egyenlettel megold.
\item
  Gyakorlati problémák megoldása során előforduló mennyiségeknél
  becslést végez.
\item
  Megoldását ellenőrzi.
\item
  Ismeri a százalék fogalmát, gazdasági, pénzügyi és mindennapi élethez
  kötődő százalékszámítási feladatokat megold.
\item
  Érti és alkalmazza a számok helyi értékes írásmódját nagy számok
  esetén.
\item
  Érti és alkalmazza a számok helyi értékes írásmódját tizedes törtek
  esetén.
\item
  Ismeri a római számjelek közül az l, c, d, m jeleket, felismeri az
  ezekkel képzett számokat a hétköznapi helyzetekben.
\item
  Ismeri az egész számokat.
\item
  Meghatározza konkrét számok ellentettjét, abszolút értékét.
\item
  Ábrázol törtrészeket, meghatároz törtrészeknek megfelelő törtszámokat.
\item
  Megfelelteti egymásnak a racionális számok közönséges tört és tizedes
  tört alakját.
\item
  Ismeri a racionális számokat, tud példát végtelen nem szakaszos
  tizedes törtre.
\item
  Meghatározza konkrét számok reciprokát.
\item
  Ismeri és helyesen alkalmazza a műveleti sorrendre és a zárójelezésre
  vonatkozó szabályokat fejben, írásban és géppel számolás esetén is a
  racionális számok körében.
\item
  Ismeri és alkalmazza a 2-vel, 3-mal, 4-gyel, 5-tel, 6-tal, 9-cel,
  10-zel, 100-zal való oszthatóság szabályait.
\item
  Ismeri a prímszám és az összetett szám fogalmakat; el tudja készíteni
  összetett számok prímtényezős felbontását 1000-es számkörben.
\item
  Meghatározza természetes számok legnagyobb közös osztóját és legkisebb
  közös többszörösét.
\item
  Pozitív egész számok pozitív egész kitevőjű hatványát kiszámolja.
\item
  Négyzetszámok négyzetgyökét meghatározza.
\item
  Egyszerű betűs kifejezésekkel összeadást, kivonást végez, és
  helyettesítési értéket számol.
\item
  Egy- vagy kéttagú betűs kifejezést számmal szoroz, két tagból közös
  számtényezőt kiemel.
\item
  Írásban összead, kivon és szoroz.
\item
  Gyakorlati feladatok megoldása során legfeljebb kétjegyű egész számmal
  írásban oszt. a hányadost megbecsüli.
\item
  Gyakorlati feladatok megoldása során tizedes törtet legfeljebb
  kétjegyű egész számmal írásban oszt. a hányadost megbecsüli.
\item
  A műveleti szabályok ismeretében ellenőrzi számolását. a kapott
  eredményt észszerűen kerekíti.
\item
  A gyakorlati problémákban előforduló mennyiségeket becsülni tudja,
  feladatmegoldásához ennek megfelelő tervet készít.
\item
  Elvégzi az alapműveleteket a racionális számok körében, eredményét
  összeveti előzetes becslésével.
\item
  Ismeri az idő, a tömeg, a hosszúság, a terület, a térfogat és az
  űrtartalom szabványmértékegységeit, használja azokat mérések és
  számítások esetén.
\item
  Egyenes hasáb, téglatest, kocka alakú tárgyak felszínét és térfogatát
  méréssel megadja, egyenes hasáb felszínét és térfogatát képlet
  segítségével kiszámolja; a képleteket megalapozó összefüggéseket érti.
\item
  Idő, tömeg, hosszúság, terület, térfogat és űrtartalom
  mértékegységeket átvált helyi értékes gondolkodás alapján, gyakorlati
  célszerűség szerint.
\item
  Meghatározza háromszögek és speciális négyszögek kerületét, területét.
\item
  A kocka, a téglatest, a hasáb és a gúla hálóját elkészíti.
\item
  Testeket épít képek, nézetek, alaprajzok, hálók alapján.
\item
  Ismeri a gömb tulajdonságait.
\item
  Ismeri a kocka, a téglatest, a hasáb és a gúla következő
  tulajdonságait: határoló lapok típusa, száma, egymáshoz viszonyított
  helyzete; csúcsok, élek száma; lapátló, testátló.
\item
  Ismeri a tengelyesen szimmetrikus háromszöget.
\item
  Ismeri a speciális négyszögeket: trapéz, paralelogramma, téglalap,
  deltoid, rombusz, húrtrapéz, négyzet.
\item
  Tapasztalatot szerez a síkbeli mozgásokról gyakorlati helyzetekben.
\item
  Felismeri a síkban az egybevágó alakzatokat.
\item
  Felismeri a kicsinyítést és a nagyítást hétköznapi helyzetekben.
\item
  Ismeri a kör részeit; különbséget tesz egyenes, félegyenes és szakasz
  között.
\item
  Ismeri a háromszögek tulajdonságait: belső és külső szögek összege,
  háromszög-egyenlőtlenség.
\item
  Ismeri a pitagorasz-tételt és alkalmazza számítási feladatokban.
\item
  Ismeri a négyszögek tulajdonságait: belső és külső szögek összege,
  konvex és konkáv közti különbség, átló fogalma.
\item
  A szerkesztéshez tervet, előzetes ábrát készít.
\item
  Ismeri az alapszerkesztéseket: szakaszfelező merőlegest, szögfelezőt,
  merőleges és párhuzamos egyeneseket szerkeszt, szöget másol.
\item
  Megszerkeszti alakzatok tengelyes és középpontos tükörképét.
\item
  Geometriai ismereteinek felhasználásával pontosan szerkeszt több adott
  feltételnek megfelelő ábrát.
\item
  Konkrét halmazok elemei között megfeleltetést hoz létre.
\item
  Felismeri az egyenes és a fordított arányosságot konkrét helyzetekben.
\item
  Tájékozódik a koordináta-rendszerben: koordinátáival adott pontot
  ábrázol, megadott pont koordinátáit leolvassa.
\item
  Értéktáblázatok adatait grafikusan ábrázolja.
\item
  Egyszerű grafikonokat jellemez.
\item
  Felismeri és megalkotja az egyenes arányosság grafikonját.
\item
  Sorozatokat adott szabály alapján folytat.
\item
  Néhány tagjával adott sorozat esetén felismer és megfogalmaz képzési
  szabályt.
\item
  Valószínűségi játékokban érti a lehetséges kimeneteleket, játékában
  stratégiát követ.
\item
  Ismeri a gyakoriság és a relatív gyakoriság fogalmát. ismereteit
  felhasználja a „lehetetlen'', a „biztos'' és a „kisebb, nagyobb
  eséllyel lehetséges'' kijelentések megfogalmazásánál.
\item
  Helyesen használja a tanult matematikai fogalmakat megnevező
  szakkifejezéseket.
\item
  Adatokat táblázatba rendez, diagramon ábrázol hagyományos és digitális
  eszközökkel is.
\item
  Különböző típusú diagramokat megfeleltet egymásnak.
\item
  Megadott szempont szerint adatokat gyűjt ki táblázatból, olvas le
  hagyományos vagy digitális forrásból származó diagramról, majd
  rendszerezés után következtetéseket fogalmaz meg.
\item
  Konkrét esetekben halmazokat felismer és ábrázol.
\item
  Értelmezi a táblázatok adatait, az adatoknak megfelelő ábrázolási
  módot kiválasztja, és az ábrát elkészíti.
\item
  Ismer táblázatkezelő programot, tud adatokat összehasonlítani,
  elemezni.
\item
  A fejszámoláson és az írásban végzendő műveleteken túlmutató számolási
  feladatokhoz és azok ellenőrzéséhez számológépet használ.
\item
  Ismer és használ dinamikus geometriai szoftvereket, tisztában van
  alkalmazási lehetőségeikkel.
\item
  Ismer és használ digitális matematikai játékokat, programokat.
\item
  Alkalmazza a tanult infokommunikációs ismereteket matematikai
  problémák megoldása során.
\end{itemize}

\hypertarget{evfolyamon-26}{%
\subsubsection{9-12. évfolyamon}\label{evfolyamon-26}}

\begin{itemize}
\item
  Ismeretei segítségével, a megfelelő modell alkalmazásával megold
  hétköznapi és matematikai problémákat, a megoldást ellenőrzi és
  értelmezi.
\item
  Megérti a környezetében jelen lévő logikai, mennyiségi, függvényszerű,
  térbeli és statisztikai kapcsolatokat.
\item
  Sejtéseket fogalmaz meg és logikus lépésekkel igazolja azokat.
\item
  Adatokat gyűjt, rendez, ábrázol, értelmez.
\item
  A matematikai szakkifejezéseket és jelöléseket helyesen használja
  írásban és szóban egyaránt.
\item
  Megérti a hallott és olvasott matematikai tartalmú szövegeket.
\item
  Felismeri a matematika különböző területei közötti kapcsolatokat.
\item
  A matematika tanulása során digitális eszközöket és különböző
  információforrásokat használ.
\item
  A matematikát más tantárgyakhoz kapcsolódó témákban is használja.
\item
  Matematikai ismereteit alkalmazza a pénzügyi tudatosság területét
  érintő feladatok megoldásában.
\item
  Adott halmazt diszjunkt részhalmazaira bont, osztályoz.
\item
  Matematikai vagy hétköznapi nyelven megfogalmazott szövegből a
  matematikai tartalmú információkat kigyűjti, rendszerezi.
\item
  Felismeri a matematika különböző területei közötti kapcsolatot.
\item
  Látja a halmazműveletek és a logikai műveletek közötti kapcsolatokat.
\item
  Halmazokat különböző módokon megad.
\item
  Halmazokkal műveleteket végez, azokat ábrázolja és értelmezi.
\item
  Véges halmazok elemszámát meghatározza.
\item
  Alkalmazza a logikai szita elvét.
\item
  Adott állításról eldönti, hogy igaz vagy hamis.
\item
  Alkalmazza a tagadás műveletét egyszerű feladatokban.
\item
  Ismeri és alkalmazza az „és'', a (megengedő és kizáró) „vagy'' logikai
  jelentését.
\item
  Megfogalmazza adott állítás megfordítását.
\item
  Megállapítja egyszerű „ha~..., akkor~...'' és „akkor és csak akkor''
  típusú állítások logikai értékét.
\item
  Helyesen használja a „minden'' és „van olyan'' kifejezéseket.
\item
  Tud egyszerű állításokat indokolni és tételeket bizonyítani.
\item
  Megold sorba rendezési és kiválasztási feladatokat.
\item
  Konkrét szituációkat szemléltet és egyszerű feladatokat megold gráfok
  segítségével.
\item
  Adott problémához megoldási stratégiát, algoritmust választ, készít.
\item
  A problémának megfelelő matematikai modellt választ, alkot.
\item
  A kiválasztott modellben megoldja a problémát.
\item
  A modellben kapott megoldását az eredeti problémába
  visszahelyettesítve értelmezi, ellenőrzi és az észszerűségi
  szempontokat figyelembe véve adja meg válaszát.
\item
  Geometriai szerkesztési feladatoknál vizsgálja és megállapítja a
  szerkeszthetőség feltételeit.
\item
  Ismeri és alkalmazza a következő egyenletmegoldási módszereket:
  mérlegelv, grafikus megoldás, szorzattá alakítás.
\item
  Megold elsőfokú egyismeretlenes egyenleteket és egyenlőtlenségeket,
  elsőfokú kétismeretlenes egyenletrendszereket.
\item
  Megold másodfokú egyismeretlenes egyenleteket és egyenlőtlenségeket;
  ismeri és alkalmazza a diszkriminánst, a megoldóképletet és a
  gyöktényezős alakot.
\item
  Megold egyszerű, a megfelelő definíció alkalmazását igénylő
  exponenciális egyenleteket, egyenlőtlenségeket.
\item
  Egyenletek megoldását behelyettesítéssel, értékkészlet-vizsgálattal
  ellenőrzi.
\item
  Ismeri a mérés alapelvét, alkalmazza konkrét alap- és származtatott
  mennyiségek esetén.
\item
  Ismeri a hosszúság, terület, térfogat, űrtartalom, idő mértékegységeit
  és az átváltási szabályokat. származtatott mértékegységeket átvált.
\item
  Sík- és térgeometriai feladatoknál a problémának megfelelő
  mértékegységben adja meg válaszát.
\item
  Ismeri és alkalmazza az oszthatóság alapvető fogalmait.
\item
  Összetett számokat felbont prímszámok szorzatára.
\item
  Meghatározza két természetes szám legnagyobb közös osztóját és
  legkisebb közös többszörösét, és alkalmazza ezeket egyszerű gyakorlati
  feladatokban.
\item
  Ismeri és alkalmazza az oszthatósági szabályokat.
\item
  Érti a helyi értékes írásmódot 10-es és más alapú számrendszerekben.
\item
  Ismeri a számhalmazok épülésének matematikai vonatkozásait a
  természetes számoktól a valós számokig.
\item
  A kommutativitás, asszociativitás, disztributivitás műveleti
  azonosságokat helyesen alkalmazza különböző számolási helyzetekben.
\item
  Racionális számokat tizedes tört és közönséges tört alakban is felír.
\item
  Ismer példákat irracionális számokra.
\item
  Ismeri a valós számok és a számegyenes kapcsolatát.
\item
  Ismeri és alkalmazza az abszolút érték, az ellentett és a reciprok
  fogalmát.
\item
  A számolással kapott eredményeket nagyságrendileg megbecsüli, és így
  ellenőrzi az eredményt.
\item
  Valós számok közelítő alakjaival számol, és megfelelően kerekít.
\item
  Ismeri és alkalmazza a négyzetgyök fogalmát és azonosságait.
\item
  Ismeri és alkalmazza az n-edik gyök fogalmát.
\item
  Ismeri és alkalmazza a normálalak fogalmát.
\item
  Ismeri és alkalmazza az egész kitevőjű hatvány fogalmát és a
  hatványozás azonosságait.
\item
  Ismeri és alkalmazza a racionális kitevőjű hatvány fogalmát és a
  hatványozás azonosságait.
\item
  Ismeri és alkalmazza a logaritmus fogalmát.
\item
  Műveleteket végez algebrai kifejezésekkel.
\item
  Ismer és alkalmaz egyszerű algebrai azonosságokat.
\item
  Átalakít algebrai kifejezéseket összevonás, szorzattá alakítás,
  nevezetes azonosságok alkalmazásával.
\item
  Ismeri és alkalmazza az egyenes és a fordított arányosságot.
\item
  Ismeri és alkalmazza a százalékalap, -érték, -láb, -pont fogalmát.
\item
  Ismeri és használja a pont, egyenes, sík (térelemek) és szög fogalmát.
\item
  Ismeri és feladatmegoldásban alkalmazza a térelemek kölcsönös
  helyzetét, távolságát és hajlásszögét.
\item
  Ismeri és alkalmazza a nevezetes szögpárok tulajdonságait.
\item
  Ismeri az alapszerkesztéseket, és ezeket végre tudja hajtani
  hagyományos vagy digitális eszközzel.
\item
  Ismeri és alkalmazza a háromszögek oldalai, szögei, oldalai és szögei
  közötti kapcsolatokat; a speciális háromszögek tulajdonságait.
\item
  Ismeri és alkalmazza a háromszög nevezetes vonalaira, pontjaira és
  köreire vonatkozó fogalmakat és tételeket.
\item
  Ismeri és alkalmazza a pitagorasz-tételt és megfordítását.
\item
  Kiszámítja háromszögek területét.
\item
  Ismeri és alkalmazza speciális négyszögek tulajdonságait, területüket
  kiszámítja.
\item
  Ismeri és alkalmazza a szabályos sokszög fogalmát; kiszámítja a konvex
  sokszög belső és külső szögeinek összegét.
\item
  Átdarabolással kiszámítja sokszögek területét.
\item
  Ki tudja számolni a kör és részeinek kerületét, területét.
\item
  Ismeri a kör érintőjének fogalmát, kapcsolatát az érintési pontba
  húzott sugárral.
\item
  Ismeri és alkalmazza a thalész-tételt és megfordítását.
\item
  Ismer példákat geometriai transzformációkra.
\item
  Ismeri és alkalmazza a síkbeli egybevágósági transzformációkat és
  tulajdonságaikat; alakzatok egybevágóságát.
\item
  Ismeri és alkalmazza a középpontos hasonlósági transzformációt, a
  hasonlósági transzformációt és az alakzatok hasonlóságát.
\item
  Ismeri és alkalmazza a hasonló síkidomok kerületének és területének
  arányára vonatkozó tételeket.
\item
  Megszerkeszti egy alakzat tengelyes, illetve középpontos tükörképét,
  pont körüli elforgatottját, párhuzamos eltoltját hagyományosan és
  digitális eszközzel.
\item
  Ismeri a vektorokkal kapcsolatos alapvető fogalmakat.
\item
  Ismer és alkalmaz egyszerű vektorműveleteket.
\item
  Alkalmazza a vektorokat feladatok megoldásában.
\item
  Ismeri hegyesszögek szögfüggvényeinek definícióját a derékszögű
  háromszögben.
\item
  Ismeri tompaszögek szögfüggvényeinek származtatását a hegyesszögek
  szögfüggvényei alapján.
\item
  Ismeri a hegyes- és tompaszögek szögfüggvényeinek összefüggéseit.
\item
  Alkalmazza a szögfüggvényeket egyszerű geometriai számítási
  feladatokban.
\item
  A szögfüggvény értékének ismeretében meghatározza a szöget.
\item
  Ismeri és alkalmazza a szinusz- és a koszinusztételt.
\item
  Ismeri és alkalmazza a hasáb, a henger, a gúla, a kúp, a gömb, a
  csonkagúla, a csonkakúp (speciális testek) tulajdonságait.
\item
  Lerajzolja a kocka, téglatest, egyenes hasáb, egyenes körhenger,
  egyenes gúla, forgáskúp hálóját.
\item
  Kiszámítja a speciális testek felszínét és térfogatát egyszerű
  esetekben.
\item
  Ismeri és alkalmazza a hasonló testek felszínének és térfogatának
  arányára vonatkozó tételeket.
\item
  Megad pontot és vektort koordinátáival a derékszögű
  koordináta-rendszerben.
\item
  Koordináta-rendszerben ábrázol adott feltételeknek megfelelő
  ponthalmazokat.
\item
  Koordináták alapján számításokat végez szakaszokkal, vektorokkal.
\item
  Ismeri és alkalmazza az egyenes egyenletét.
\item
  Egyenesek egyenletéből következtet az egyenesek kölcsönös helyzetére.
\item
  Kiszámítja egyenesek metszéspontjainak koordinátáit az egyenesek
  egyenletének ismeretében.
\item
  Megadja és alkalmazza a kör egyenletét a kör sugarának és a középpont
  koordinátáinak ismeretében.
\item
  Megad hétköznapi életben előforduló hozzárendeléseket.
\item
  Adott képlet alapján helyettesítési értékeket számol, és azokat
  táblázatba rendezi.
\item
  Táblázattal megadott függvény összetartozó értékeit ábrázolja
  koordináta-rendszerben.
\item
  Képlettel adott függvényt hagyományosan és digitális eszközzel
  ábrázol.
\item
  Adott értékkészletbeli elemhez megtalálja az értelmezési tartomány
  azon elemeit, amelyekhez a függvény az adott értéket rendeli.
\item
  A grafikonról megállapítja függvények alapvető tulajdonságait.
\item
  Számtani és mértani sorozatokat adott szabály alapján felír, folytat.
\item
  A számtani, mértani sorozat n-edik tagját felírja az első tag és a
  különbség (differencia)/hányados (kvóciens) ismeretében.
\item
  A számtani, mértani sorozatok első n tagjának összegét kiszámolja.
\item
  Mértani sorozatokra vonatkozó ismereteit használja gazdasági,
  pénzügyi, természettudományi és társadalomtudományi problémák
  megoldásában.
\item
  Adott cél érdekében tudatos adatgyűjtést és rendszerezést végez.
\item
  Hagyományos és digitális forrásból származó adatsokaság alapvető
  statisztikai jellemzőit meghatározza, értelmezi és értékeli.
\item
  Adatsokaságból adott szempont szerint oszlop- és kördiagramot készít
  hagyományos és digitális eszközzel.
\item
  Ismeri és alkalmazza a sodrófa (box-plot) diagramot adathalmazok
  jellemzésére, összehasonlítására.
\item
  Felismer grafikus manipulációkat diagramok esetén.
\item
  Tapasztalatai alapján véletlen jelenségek jövőbeni kimenetelére
  észszerűen tippel.
\item
  Ismeri és alkalmazza a klasszikus valószínűségi modellt és a
  laplace-képletet.
\item
  Véletlen kísérletek adatait rendszerezi, relatív gyakoriságokat
  számol, nagy elemszám esetén számítógépet alkalmaz.
\item
  Konkrét valószínűségi kísérletek esetében az esemény, eseménytér,
  elemi esemény, relatív gyakoriság, valószínűség, egymást kizáró
  események, független események fogalmát megkülönbözteti és alkalmazza.
\item
  Ismeri és egyszerű esetekben alkalmazza a valószínűség geometriai
  modelljét.
\item
  Meghatározza a valószínűséget visszatevéses, illetve visszatevés
  nélküli mintavétel esetén.
\item
  A megfelelő matematikai tankönyveket, feladatgyűjteményeket,
  internetes tartalmakat értőn olvassa, a matematikai tartalmat
  rendszerezetten kigyűjti és megérti.
\item
  A matematikai fogalmakat és jelöléseket megfelelően használja.
\item
  Önállóan kommunikál matematika tartalmú feladatokkal kapcsolatban.
\item
  Matematika feladatok megoldását szakszerűen prezentálja írásban és
  szóban a szükséges alapfogalmak, azonosságok, definíciók és tételek
  segítségével.
\item
  Szöveg alapján táblázatot, grafikont készít, ábrát, kapcsolatokat
  szemléltető gráfot rajzol, és ezeket kombinálva prezentációt készít és
  mutat be.
\item
  Ismer a tananyaghoz kapcsolódó matematikatörténeti vonatkozásokat.
\item
  Számológép segítségével alapműveletekkel felírható számolási
  eredményt; négyzetgyököt; átlagot; szögfüggvények értékét, illetve
  abból szöget; logaritmust; faktoriálist; binomiális együtthatót;
  szórást meghatároz.
\item
  Digitális környezetben matematikai alkalmazásokkal dolgozik.
\item
  Megfelelő informatikai alkalmazás segítségével szöveget szerkeszt,
  táblázatkezelő programmal diagramokat készít.
\item
  Ismereteit digitális forrásokból kiegészíti, számítógép segítségével
  elemzi és bemutatja.
\item
  Prezentációhoz informatív diákat készít, ezeket logikusan és
  következetesen egymás után fűzi és bemutatja.
\item
  Kísérletezéshez, sejtés megfogalmazásához, egyenlet grafikus
  megoldásához és ellenőrzéshez dinamikus geometriai, grafikus és
  táblázatkezelő szoftvereket használ.
\item
  Szerkesztési feladatok euklideszi módon történő megoldásához dinamikus
  geometriai szoftvert használ.
\end{itemize}

\hypertarget{masodik-idegen-nyelv}{%
\subsection{Második idegen nyelv}\label{masodik-idegen-nyelv}}

\hypertarget{evfolyamon-27}{%
\subsubsection{9-12. évfolyamon}\label{evfolyamon-27}}

\begin{itemize}
\item
  Megismerkedik az idegen nyelvvel, a nyelvtanulással és örömmel vesz
  részt az órákon.
\item
  Bekapcsolódik a szóbeliséget, írást, szövegértést vagy interakciót
  igénylő alapvető és korának megfelelő játékos, élményalapú élő idegen
  nyelvi tevékenységekbe.
\item
  Szóban visszaad szavakat, esetleg rövid, nagyon egyszerű szövegeket
  hoz létre.
\item
  Lemásol, leír szavakat és rövid, nagyon egyszerű szövegeket.
\item
  Követi a szintjének megfelelő, vizuális vagy nonverbális eszközökkel
  támogatott, ismert célnyelvi óravezetést, utasításokat.
\item
  Felismeri és használja a legegyszerűbb, mindennapi nyelvi funkciókat.
\item
  Elmondja magáról a legalapvetőbb információkat.
\item
  Ismeri az adott célnyelvi kultúrákhoz tartozó országok fontosabb
  jellemzőit és a hozzájuk tartozó alapvető nyelvi elemeket.
\item
  Törekszik a tanult nyelvi elemek megfelelő kiejtésére.
\item
  Célnyelvi tanulmányain keresztül nyitottabbá, a világ felé
  érdeklődőbbé válik.
\item
  Megismétli az élőszóban elhangzó egyszerű szavakat, kifejezéseket
  játékos, mozgást igénylő, kreatív nyelvórai tevékenységek során.
\item
  Lebetűzi a nevét.
\item
  Lebetűzi a tanult szavakat társaival közösen játékos tevékenységek
  kapcsán, szükség esetén segítséggel.
\item
  Célnyelven megoszt egyedül vagy társaival együttműködésben
  megszerzett, alapvető információkat szóban, akár vizuális elemekkel
  támogatva.
\item
  Felismeri az anyanyelvén, illetve a tanult idegen nyelven történő
  írásmód és betűkészlet közötti különbségeket.
\item
  Ismeri az adott nyelv ábécéjét.
\item
  Lemásol tanult szavakat játékos, alkotó nyelvórai tevékenységek során.
\item
  Megold játékos írásbeli feladatokat a szavak, szószerkezetek, rövid
  mondatok szintjén.
\item
  Részt vesz kooperatív munkaformában végzett kreatív tevékenységekben,
  projektmunkában szavak, szószerkezetek, rövid mondatok leírásával,
  esetleg képi kiegészítéssel.
\item
  Írásban megnevezi az ajánlott tématartományokban megjelölt,
  begyakorolt elemeket.
\item
  Megérti az élőszóban elhangzó, ismert témákhoz kapcsolódó, verbális,
  vizuális vagy nonverbális eszközökkel segített rövid kijelentéseket,
  kérdéseket.
\item
  Beazonosítja az életkorának megfelelő szituációkhoz kapcsolódó, rövid,
  egyszerű szövegben a tanult nyelvi elemeket.
\item
  Kiszűri a lényeget az ismert nyelvi elemeket tartalmazó, nagyon rövid,
  egyszerű hangzó szövegből.
\item
  Azonosítja a célzott információt a nyelvi szintjének és életkorának
  megfelelő rövid hangzó szövegben.
\item
  Támaszkodik az életkorának és nyelvi szintjének megfelelő hangzó
  szövegre az órai alkotó jellegű nyelvi, mozgásos nyelvi és játékos
  nyelvi tevékenységek során.
\item
  Felismeri az anyanyelv és az idegen nyelv hangkészletét.
\item
  Értelmezi azokat az idegen nyelven szóban elhangzó nyelvórai
  szituációkat, melyeket anyanyelvén már ismer.
\item
  Felismeri az anyanyelve és a célnyelv közötti legalapvetőbb
  kiejtésbeli különbségeket.
\item
  Figyel a célnyelvre jellemző hangok kiejtésére.
\item
  Megkülönbözteti az anyanyelvi és a célnyelvi írott szövegben a betű-
  és jelkészlet közti különbségeket.
\item
  Beazonosítja a célzott információt az életkorának megfelelő
  szituációkhoz kapcsolódó, rövid, egyszerű, a nyelvtanításhoz készült,
  illetve eredeti szövegben.
\item
  Csendes olvasás keretében feldolgozva megért ismert szavakat
  tartalmazó, pár szóból vagy mondatból álló, akár illusztrációval
  támogatott szöveget.
\item
  Megérti a nyelvi szintjének megfelelő, akár vizuális eszközökkel is
  támogatott írott utasításokat és kérdéseket, és ezekre megfelelő
  válaszreakciókat ad.
\item
  Kiemeli az ismert nyelvi elemeket tartalmazó, egyszerű, írott, pár
  mondatos szöveg fő mondanivalóját.
\item
  Támaszkodik az életkorának és nyelvi szintjének megfelelő írott
  szövegre az órai játékos alkotó, mozgásos vagy nyelvi fejlesztő
  tevékenységek során, kooperatív munkaformákban.
\item
  Megtapasztalja a közös célnyelvi olvasás élményét.
\item
  Aktívan bekapcsolódik a közös meseolvasásba, a mese tartalmát követi.
\item
  A tanórán begyakorolt, nagyon egyszerű, egyértelmű kommunikációs
  helyzetekben a megtanult, állandósult beszédfordulatok alkalmazásával
  kérdez vagy reagál, mondanivalóját segítséggel vagy nonverbális
  eszközökkel kifejezi.
\item
  Törekszik arra, hogy a célnyelvet eszközként alkalmazza
  információszerzésre.
\item
  Rövid, néhány mondatból álló párbeszédet folytat, felkészülést
  követően.
\item
  A tanórán bekapcsolódik a már ismert, szóbeli interakciót igénylő
  nyelvi tevékenységekbe, a begyakorolt nyelvi elemeket tanári
  segítséggel a tevékenység céljainak megfelelően alkalmazza.
\item
  Érzéseit egy-két szóval vagy begyakorolt állandósult nyelvi fordulatok
  segítségével kifejezi, főként rákérdezés alapján, nonverbális
  eszközökkel kísérve a célnyelvi megnyilatkozást.
\item
  Elsajátítja a tanult szavak és állandósult szókapcsolatok célnyelvi
  normához közelítő kiejtését tanári minta követése által, vagy
  autentikus hangzó anyag, digitális technológia segítségével.
\item
  Felismeri és alkalmazza a legegyszerűbb, üdvözlésre és elköszönésre
  használt mindennapi nyelvi funkciókat az életkorának és nyelvi
  szintjének megfelelő, egyszerű helyzetekben.
\item
  Felismeri és alkalmazza a legegyszerűbb, bemutatkozásra használt
  mindennapi nyelvi funkciókat az életkorának és nyelvi szintjének
  megfelelő, egyszerű helyzetekben.
\item
  Felismeri és használja a legegyszerűbb, megszólításra használt
  mindennapi nyelvi funkciókat az életkorának és nyelvi szintjének
  megfelelő, egyszerű helyzetekben.
\item
  Felismeri és használja a legegyszerűbb, a köszönet és az arra történő
  reagálás kifejezésére használt mindennapi nyelvi funkciókat az
  életkorának és nyelvi szintjének megfelelő, egyszerű helyzetekben.
\item
  Felismeri és használja a legegyszerűbb, a tudás és nem tudás
  kifejezésére használt mindennapi nyelvi funkciókat az életkorának és
  nyelvi szintjének megfelelő, egyszerű helyzetekben.
\item
  Felismeri és használja a legegyszerűbb, a nem értés, visszakérdezés és
  ismétlés, kérés kifejezésére használt mindennapi nyelvi funkciókat
  életkorának és nyelvi szintjének megfelelő, egyszerű helyzetekben.
\item
  Közöl alapvető személyes információkat magáról, egyszerű nyelvi elemek
  segítségével.
\item
  Új szavak, kifejezések tanulásakor ráismer a már korábban tanult
  szavakra, kifejezésekre.
\item
  Szavak, kifejezések tanulásakor felismeri, ha új elemmel találkozik és
  rákérdez, vagy megfelelő tanulási stratégiával törekszik a megértésre.
\item
  A célok eléréséhez társaival rövid feladatokban együttműködik.
\item
  Egy feladat megoldásának sikerességét segítséggel értékelni tudja.
\item
  Felismeri az idegen nyelvű írott, olvasott és hallott tartalmakat a
  tanórán kívül is.
\item
  Felhasznál és létrehoz rövid, nagyon egyszerű célnyelvi szövegeket
  szabadidős tevékenységek során.
\item
  Alapvető célzott információt megszerez a tanult témákban tudásának
  bővítésére.
\item
  Megismeri a főbb, az adott célnyelvi kultúrákhoz tartozó országok
  nevét, földrajzi elhelyezkedését, főbb országismereti jellemzőit.
\item
  Ismeri a főbb, célnyelvi kultúrához tartozó, ünnepekhez kapcsolódó
  alapszintű kifejezéseket, állandósult szókapcsolatokat és szokásokat.
\item
  Megérti a tanult nyelvi elemeket életkorának megfelelő digitális
  tartalmakban, digitális csatornákon olvasott vagy hallott nagyon
  egyszerű szövegekben is.
\item
  Létrehoz nagyon egyszerű írott, pár szavas szöveget szóban vagy
  írásban digitális felületen.
\item
  Szóban és írásban megold változatos kihívásokat igénylő feladatokat az
  élő idegen nyelven.
\item
  Szóban és írásban létrehoz rövid szövegeket, ismert nyelvi
  eszközökkel, a korának megfelelő szövegtípusokban.
\item
  Értelmez korának és nyelvi szintjének megfelelő hallott és írott
  célnyelvi szövegeket az ismert témákban és szövegtípusokban.
\item
  A tanult nyelvi elemek és kommunikációs stratégiák segítségével
  írásbeli és szóbeli interakciót folytat, valamint közvetít az élő
  idegen nyelven.
\item
  Kommunikációs szándékának megfelelően alkalmazza a tanult nyelvi
  funkciókat és a megszerzett szociolingvisztikai, pragmatikai és
  interkulturális jártasságát.
\item
  Nyelvtudását egyre inkább képes fejleszteni tanórán kívüli
  helyzetekben is különböző eszközökkel és lehetőségekkel.
\item
  Használ életkorának és nyelvi szintjének megfelelő hagyományos és
  digitális alapú nyelvtanulási forrásokat és eszközöket.
\item
  Alkalmazza nyelvtudását kommunikációra, közvetítésre, szórakozásra,
  ismeretszerzésre hagyományos és digitális csatornákon.
\item
  Törekszik a célnyelvi normához illeszkedő kiejtés, beszédtempó és
  intonáció megközelítésére.
\item
  Érti a nyelvtudás fontosságát, és motivációja a nyelvtanulásra tovább
  erősödik.
\item
  Aktívan részt vesz az életkorának és érdeklődésének megfelelő
  gyermek-, illetve ifjúsági irodalmi alkotások közös előadásában.
\item
  Egyre magabiztosabban kapcsolódik be történetek kreatív alakításába,
  átfogalmazásába kooperatív munkaformában.
\item
  Elmesél rövid történetet, egyszerűsített olvasmányt egyszerű nyelvi
  eszközökkel, önállóan, a cselekményt lineárisan összefűzve.
\item
  Egyszerű nyelvi eszközökkel, felkészülést követően röviden,
  összefüggően beszél az ajánlott tématartományokhoz tartozó témákban,
  élőszóban és digitális felületen.
\item
  Képet jellemez röviden, egyszerűen, ismert nyelvi fordulatok
  segítségével, segítő tanári kérdések alapján, önállóan.
\item
  Változatos, kognitív kihívást jelentő szóbeli feladatokat old meg
  önállóan vagy kooperatív munkaformában, a tanult nyelvi eszközökkel,
  szükség szerint tanári segítséggel, élőszóban és digitális felületen.
\item
  Megold játékos és változatos írásbeli feladatokat rövid szövegek
  szintjén.
\item
  Rövid, egyszerű, összefüggő szövegeket ír a tanult nyelvi szerkezetek
  felhasználásával az ismert szövegtípusokban, az ajánlott
  tématartományokban.
\item
  Rövid szövegek írását igénylő kreatív munkát hoz létre önállóan.
\item
  Rövid, összefüggő, papíralapú vagy ikt-eszközökkel segített írott
  projektmunkát készít önállóan vagy kooperatív munkaformákban.
\item
  A szövegek létrehozásához nyomtatott, illetve digitális alapú
  segédeszközt, szótárt használ.
\item
  Megérti a szintjének megfelelő, kevésbé ismert elemekből álló,
  nonverbális vagy vizuális eszközökkel támogatott célnyelvi óravezetést
  és utasításokat, kérdéseket.
\item
  Értelmezi az életkorának és nyelvi szintjének megfelelő, egyszerű,
  hangzó szövegben a tanult nyelvi elemeket.
\item
  Értelmezi az életkorának megfelelő, élőszóban vagy digitális felületen
  elhangzó szövegekben a beszélők gondolatmenetét.
\item
  Megérti a nem kizárólag ismert nyelvi elemeket tartalmazó, élőszóban
  vagy digitális felületen elhangzó rövid szöveg tartalmát.
\item
  Kiemel, kiszűr konkrét információkat a nyelvi szintjének megfelelő,
  élőszóban vagy digitális felületen elhangzó szövegből, és azokat
  összekapcsolja egyéb ismereteivel.
\item
  Alkalmazza az életkorának és nyelvi szintjének megfelelő hangzó
  szöveget a változatos nyelvórai tevékenységek és a feladatmegoldás
  során.
\item
  Értelmez életkorának megfelelő nyelvi helyzeteket hallott szöveg
  alapján.
\item
  Felismeri a főbb, életkorának megfelelő hangzószöveg-típusokat.
\item
  Hallgat az érdeklődésének megfelelő autentikus szövegeket
  elektronikus, digitális csatornákon, tanórán kívül is, szórakozásra
  vagy ismeretszerzésre.
\item
  Értelmezi az életkorának megfelelő szituációkhoz kapcsolódó, írott
  szövegekben megjelenő összetettebb információkat.
\item
  Megérti a nem kizárólag ismert nyelvi elemeket tartalmazó rövid írott
  szöveg tartalmát.
\item
  Kiemel, kiszűr konkrét információkat a nyelvi szintjének megfelelő
  szövegből, és azokat összekapcsolja más iskolai vagy iskolán kívül
  szerzett ismereteivel.
\item
  Megkülönbözteti a főbb, életkorának megfelelő írott szövegtípusokat.
\item
  Összetettebb írott instrukciókat értelmez.
\item
  Alkalmazza az életkorának és nyelvi szintjének megfelelő írott,
  nyomtatott vagy digitális alapú szöveget a változatos nyelvórai
  tevékenységek és feladatmegoldás során.
\item
  A nyomtatott vagy digitális alapú írott szöveget felhasználja
  szórakozásra és ismeretszerzésre önállóan is.
\item
  Érdeklődése erősödik a célnyelvi irodalmi alkotások iránt.
\item
  Megért és használ szavakat, szókapcsolatokat a célnyelvi, az
  életkorának és érdeklődésének megfelelő hazai és nemzetközi legfőbb
  hírekkel, eseményekkel kapcsolatban
\item
  Kommunikációt kezdeményez egyszerű hétköznapi témában, a beszélgetést
  követi, egyszerű, nyelvi eszközökkel fenntartja és lezárja.
\item
  Az életkorának megfelelő mindennapi helyzetekben a tanult nyelvi
  eszközökkel megfogalmazott kérdéseket tesz fel, és válaszol a hozzá
  intézett kérdésekre.
\item
  Véleményét, gondolatait, érzéseit egyre magabiztosabban fejezi ki a
  tanult nyelvi eszközökkel.
\item
  A tanult nyelvi elemeket többnyire megfelelően használja,
  beszédszándékainak megfelelően, egyszerű spontán helyzetekben.
\item
  Váratlan, előre nem kiszámítható eseményekre, jelenségekre és
  történésekre is reagál egyszerű célnyelvi eszközökkel, személyes vagy
  online interakciókban.
\item
  Bekapcsolódik a tanórán az interakciót igénylő nyelvi tevékenységekbe,
  abban társaival közösen részt vesz, a begyakorolt nyelvi elemeket
  tanári segítséggel a játék céljainak megfelelően alkalmazza.
\item
  Üzeneteket ír,
\item
  Véleményét írásban, egyszerű nyelvi eszközökkel megfogalmazza, és
  arról írásban interakciót folytat.
\item
  Rövid, egyszerű, ismert nyelvi eszközökből álló kiselőadást tart
  változatos feladatok kapcsán, hagyományos vagy digitális alapú
  vizuális eszközök támogatásával.
\item
  Felhasználja a célnyelvet tudásmegosztásra.
\item
  Találkozik az életkorának és nyelvi szintjének megfelelő célnyelvi
  ismeretterjesztő tartalmakkal.
\item
  Néhány szóból vagy mondatból álló jegyzetet készít írott szöveg
  alapján.
\item
  Egyszerűen megfogalmazza személyes véleményét, másoktól véleményük
  kifejtését kéri, és arra reagál, elismeri vagy cáfolja mások
  állítását, kifejezi egyetértését vagy egyet nem értését.
\item
  Kifejez tetszést, nem tetszést, akaratot, kívánságot, tudást és nem
  tudást, ígéretet, szándékot, dicséretet, kritikát.
\item
  Információt cserél, információt kér, információt ad.
\item
  Kifejez kérést, javaslatot, meghívást, kínálást és ezekre reagálást.
\item
  Kifejez alapvető érzéseket, például örömöt, sajnálkozást, bánatot,
  elégedettséget, elégedetlenséget, bosszúságot, csodálkozást, reményt.
\item
  Kifejez és érvekkel alátámasztva mutat be szükségességet, lehetőséget,
  képességet, bizonyosságot, bizonytalanságot.
\item
  Értelmez és használja az idegen nyelvű írott, olvasott és hallott
  tartalmakat a tanórán kívül is,
\item
  Felhasználja a célnyelvet ismeretszerzésre.
\item
  Használja a célnyelvet életkorának és nyelvi szintjének megfelelő
  aktuális témákban és a hozzájuk tartozó szituációkban.
\item
  Találkozik életkorának és nyelvi szintjének megfelelő célnyelvi
  szórakoztató tartalmakkal.
\item
  Összekapcsolja az ismert nyelvi elemeket egyszerű kötőszavakkal
  (például: és, de, vagy).
\item
  Egyszerű mondatokat összekapcsolva mond el egymást követő eseményekből
  álló történetet, vagy leírást ad valamilyen témáról.
\item
  A tanult nyelvi eszközökkel és nonverbális elemek segítségével
  tisztázza mondanivalójának lényegét.
\item
  Ismeretlen szavak valószínű jelentését szövegösszefüggések alapján
  kikövetkezteti az életkorának és érdeklődésének megfelelő, konkrét,
  rövid szövegekben.
\item
  Alkalmaz nyelvi funkciókat rövid társalgás megkezdéséhez,
  fenntartásához és befejezéséhez.
\item
  Nem értés esetén a meg nem értett kulcsszavak vagy fordulatok
  ismétlését vagy magyarázatát kéri, visszakérdez, betűzést kér.
\item
  Megoszt alapvető személyes információkat és szükségleteket magáról
  egyszerű nyelvi elemekkel.
\item
  Ismerős és gyakori alapvető helyzetekben, akár telefonon vagy
  digitális csatornákon is, többnyire helyesen és érthetően fejezi ki
  magát az ismert nyelvi eszközök segítségével.
\item
  Tudatosan használ alapszintű nyelvtanulási és nyelvhasználati
  stratégiákat.
\item
  Hibáit többnyire észreveszi és javítja.
\item
  Ismer szavakat, szókapcsolatokat a célnyelven a témakörre jellemző,
  életkorának és érdeklődésének megfelelő más tudásterületen megcélzott
  tartalmakból.
\item
  Egy összetettebb nyelvi feladat, projekt végéig tartó célokat tűz ki
  magának.
\item
  Céljai eléréséhez megtalálja és használja a megfelelő eszközöket.
\item
  Céljai eléréséhez társaival párban és csoportban együttműködik.
\item
  Nyelvi haladását többnyire fel tudja mérni,
\item
  Társai haladásának értékelésében segítően részt vesz.
\item
  A tanórán kívüli, akár játékos nyelvtanulási lehetőségeket felismeri,
  és törekszik azokat kihasználni.
\item
  Felhasználja a célnyelvet szórakozásra és játékos nyelvtanulásra.
\item
  Digitális eszközöket és felületeket is használ nyelvtudása
  fejlesztésére,
\item
  Értelmez egyszerű, szórakoztató kisfilmeket
\item
  Megismeri a célnyelvi országok főbb jellemzőit és kulturális
  sajátosságait.
\item
  További országismereti tudásra tesz szert.
\item
  Célnyelvi kommunikációjába beépíti a tanult interkulturális
  ismereteket.
\item
  Találkozik célnyelvi országismereti tartalmakkal.
\item
  Találkozik a célnyelvi, életkorának és érdeklődésének megfelelő hazai
  és nemzetközi legfőbb hírekkel, eseményekkel.
\item
  Megismerkedik hazánk legfőbb országismereti és történelmi eseményeivel
  célnyelven.
\item
  A célnyelvi kultúrákhoz kapcsolódó alapvető tanult nyelvi elemeket
  használja.
\item
  Idegen nyelvi kommunikációjában ismeri és használja a célnyelv főbb
  jellemzőit.
\item
  Következetesen alkalmazza a célnyelvi betű és jelkészletet
\item
  Egyénileg vagy társaival együttműködve szóban vagy írásban
  projektmunkát vagy kiselőadást készít, és ezeket digitális eszközök
  segítségével is meg tudja valósítani.
\item
  Találkozik az érdeklődésének megfelelő akár autentikus szövegekkel
  elektronikus, digitális csatornákon tanórán kívül is.
\end{itemize}

\hypertarget{mozgokepkultura-es-mediaismeret}{%
\subsection{Mozgóképkultúra és
médiaismeret}\label{mozgokepkultura-es-mediaismeret}}

\hypertarget{evfolyamon-28}{%
\subsubsection{12. évfolyamon}\label{evfolyamon-28}}

\begin{itemize}
\item
  Rendelkezik átfogó egyetemes és magyar médiatörténeti ismeretekkel.
  ismeri a különböző médiumok specifikumait, a fontosabb műfajokat,
  szerzőket, műveket.
\item
  Képes különböző médiatermékek értő befogadására. ismeri a különböző
  médiumok formanyelvének alapvető eszköztárát, érzékeli ezek hatását az
  értelmezés folyamatára, valamint képes ezeknek az eszközöknek
  alapszintű alkalmazására is saját környezetében.
\item
  Rövid média-alkotások tervezése és kivitelezése révén szert tesz a
  különböző médiumok szövegalkotási folyamatainak elemi tapasztalataira,
  érti a különféle szövegtípusok eltérő működési elvét, s tud azok
  széles spektrumával kreatív befogadói viszonyt létrehozni.
\item
  Ismereteket szerez a filmkultúra területéről: érti a szerzői kultúra
  és a tömegkultúra eltérő működésmódját; felismeri az elterjedtebb
  filmműfajok jegyeit; különbséget tud tenni a különböző
  stílusirányzatokhoz tartozó alkotások között.
\item
  Ismeri a magyar filmművészet főbb szerzőit, jellegzetességeit és
  értékeit.
\item
  Tisztában van a sztárfogalom kialakulásával és módosulásával.
  azonosítani tudja a sztárjelenséget a filmen és a médiában. képes
  ismert filmszereplők és médiaszemélyiségek imázsának elemzésére, a
  háttérben fellelhető archetípusok meghatározására.
\item
  Ismeri a modern tömegkommunikáció fő működési elveit, jellemző
  vonásait, érti, milyen társadalmi és kulturális következményekkel
  járnak a kommunikációs rendszerben bekövetkező változások, ezek
  hatásait saját környezetében is észreveszi.
\item
  Értelmezni tudja a valóság és a média által a valóságról kialakított
  „kép'' viszonyát, ismeri a reprezentáció fogalmát, és rendelkezik a
  médiatudatosság képességével.
\item
  Ismeri a médiareprezentáció, valószerűség és hitelesség kritériumait,
  a fikciós műfajok illetve a dokumentumjelleg különbségeit, a
  sztereotípia, tematizáció, valóságábrázolás, hitelesség, hír
  fogalmait. képes saját értékelő viszonyt kialakítani ezekkel
  kapcsolatban.
\item
  Tisztában van a médiaipar, médiafogyasztás és -befogadás jellemzőivel.
  ismeri a kereskedelmi, közszolgálati és nonprofit média, az alkotói
  szándék, célcsoport, a közönség, mint vevő és áru, a médiafogyasztási
  szokások jellegzetességeit, a médiafüggőség tüneteit.
\item
  Ismeri az internet-világ jelenségeit, a globális kommunikáció adta
  lehetőségeket. érti a hálózati kommunikáció és a közösségi média
  működési módját, képes abban felelősen részt venni, ismeri és
  megfelelően használja annak alapvető szövegtípusait. képes igényes
  önálló tartalmak alkotására és részt venni a lokális nyilvánosságban.
\item
  Ismeretekkel rendelkezik a médiaetika és a médiaszabályozás főbb
  kérdéseit illetően, saját álláspontot tud megfogalmazni azokkal
  kapcsolatban. érti a média etikai környezetének és jogi
  szabályozásának tétjeit.
\item
  Tisztában van a média véleményformáló szerepével, az alkotók és
  felhasználók felelősségével, az egyének és közösségek jogaival. ismeri
  a norma és normaszegés fogalmait a médiában.
\item
  Tudatosítja magában a nyilvános megszólalás szabadságával és
  felelősségével járó lehetőségeket és lehetséges következményeket,
  tisztában van a digitális zaklatás veszélyeivel. tudatos
  médiahasználóként állást tud foglalni a média által közvetített
  értékek minőségével kapcsolatban.
\end{itemize}

\hypertarget{penzugyi-es-vallalkozoi-ismeretek}{%
\subsection{Pénzügyi és vállalkozói
ismeretek}\label{penzugyi-es-vallalkozoi-ismeretek}}

\hypertarget{evfolyamon-29}{%
\subsubsection{10. évfolyamon}\label{evfolyamon-29}}

\begin{itemize}
\item
  A tanuló érti a nemzetgazdaság szereplőinek (háztartások, vállalatok,
  állam, pénzintézetek) feladatait, a köztük lévő kapcsolatrendszer
  sajátosságait.
\item
  Tudja értelmezni az állam gazdasági szerepvállalásának jelentőségét,
  ismeri főbb feladatait, azok hatásait.
\item
  Tisztában van azzal, hogy az adófizetés biztosítja részben az állami
  feladatok ellátásnak pénzügyi fedezetét.
\item
  Ismeri a mai bankrendszer felépítését, az egyes pénzpiaci szereplők
  főbb feladatait.
\item
  Képes választani az egyes banki lehetőségek közül.
\item
  Tisztában van az egyes banki ügyletek előnyeivel, hátrányaival,
  kockázataival.
\item
  A bankok kínálatából bankot, bankszámla csomagot tud választani.
\item
  Tud érvelni a családi költségvetés mellett, a tudatos, hatékony
  pénzgazdálkodás érdekében.
\item
  Önismereti tesztek, játékok segítségével képes átgondolni milyen
  foglalkozások, tevékenységek illeszkednek személyiségéhez.
\item
  Tisztában van az álláskeresés folyamatával, a munkaviszonnyal
  kapcsolatos jogaival, kötelezettségeivel.
\item
  Ismer vállalkozókat, vállalatokat, össze tudja hasonlítani az
  alkalmazotti, és a vállalkozói személyiségjegyeket.
\item
  Érti a leggyakoribb vállalkozási formák jellemzőit, előnyeit,
  hátrányait.
\item
  Tisztában van a nem nyereségérdekelt szervezetek gazdaságban betöltött
  szerepével.
\item
  Ismeri a vállalkozásalapítás, -működtetés legfontosabb lépéseit, képes
  önálló vállalkozói ötlet kidolgozására.
\item
  Meg tudja becsülni egy vállalkozás lehetséges költségeit, képes adott
  időtartamra költségkalkulációt tervezni.
\item
  Tisztában van az üzleti tervezés szükségességével, mind egy új
  vállalkozás alapításakor, mind már meglévő vállalkozás működése
  esetén.
\item
  Tájékozott az üzleti terv tartalmi elemeiről.
\item
  Megismeri a nem üzleti (társadalmi, kulturális, egyéb civil)
  kezdeményezések pénzügyi-gazdasági igényeit, lehetőségeit.
\item
  Felismeri a kezdeményezőkészség jelentőségét az állampolgári
  felelősségvállalásban.
\item
  Felismeri a sikeres vállalkozás jellemzőit, képes azonosítani az
  esetleges kudarc okait, javaslatot tud tenni a problémák megoldására.
  \#\# Szoftverfejlesztés és -tesztelés \#\#\# 9-10. évfolyamon
\item
  Adott kapcsolási rajz alapján egyszerűbb áramköröket épít próbapanel
  segítségével vagy forrasztásos technológiával.
\item
  Ismeri az elektronikai alapfogalmakat, kapcsolódó fizikai törvényeket,
  alapvető alkatrészeket és kapcsolásokat.
\item
  A funkcionalitás biztosítása mellett törekszik az esztétikus
  kialakításra (pl. minőségi forrasztás, egyenletes alkatrész sűrűség,
  olvashatóság).
\item
  Az elektromos berendezésekre vonatkozó munka- és balesetvédelmi
  szabályokat a saját és mások testi épsége érdekében betartja és
  betartatja.
\item
  Alapvető villamos méréseket végez önállóan a megépített áramkörökön.
\item
  Ismeri az elektromos mennyiségek mérési metódusait, a mérőműszerek
  használatát.
\item
  Elvégzi a számítógépen és a mobil eszközökön az operációs rendszer
  (pl. Windows, Linux, Android, iOS), valamint az alkalmazói szoftverek
  telepítését, frissítését és alapszintű beállítását. Grafikus
  felületen, valamint parancssorban használja a Windows, és Linux
  operációs rendszerek alapszintű parancsait és szolgáltatásait (pl.
  állomány- és könyvtárkezelési műveletek, jogosultságok beállítása,
  szövegfájlokkal végzett műveletek, folyamatok kezelése).
\item
  Ismeri a számítógépen és a mobil informatikai eszközökön használt
  operációs rendszerek telepítési és frissítési módjait, alapvető
  parancsait és szolgáltatásait, valamint alapvető beállítási
  lehetőségeit.
\item
  Törekszik a felhasználói igényekhez alkalmazkodó szoftverkörnyezet
  kialakítására.
\item
  Önállóan elvégzi a kívánt szoftverek telepítését, szükség esetén
  gondoskodik az eszközön korábban tárolt adatok biztonsági mentéséről.
\item
  Elvégzi a PC perifériáinak csatlakoztatását, szükség esetén új
  alkatrészt szerel be vagy alkatrészt cserél egy számítógépben.
\item
  Ismeri az otthoni és irodai informatikai környezetet alkotó
  legáltalánosabb összetevők (PC, nyomtató, mobiltelefon, WiFi router
  stb.) szerepét, alapvető működési módjukat. Ismeri a PC és a mobil
  eszközök főbb alkatrészeit (pl. alaplap, CPU, memória) és azok
  szerepét.
\item
  Törekszik a végrehajtandó műveletek precíz és előírásoknak megfelelő
  elvégzésére.
\item
  Az informatikai berendezésekre vonatkozó munka- és balesetvédelmi
  szabályokat a saját és mások testi épsége érdekében betartja és
  betartatja.
\item
  Alapvető karbantartási feladatokat lát el az általa megismert
  informatikai és távközlési berendezéseken (pl. szellőzés és
  csatlakozások ellenőrzése, tisztítása).
\item
  Tisztában van vele, hogy miért szükséges az informatikai és távközlési
  eszközök rendszeres és eseti karbantartása. Ismeri legalapvetőbb
  karbantartási eljárásokat.
\item
  A hibamentes folyamatos működés elérése érdekében fontosnak tartja a
  megelőző karbantartások elvégzését.
\item
  Otthoni vagy irodai hálózatot alakít ki WiFi router segítségével,
  elvégzi WiFi router konfigurálását, a vezetékes- és vezeték nélküli
  eszközök (PC, mobiltelefon, set-top box stb.), csatlakoztatását és
  hálózati beállítását.
\item
  Ismeri az informatikai hálózatok felépítését, alapvető technológiáit
  (pl. Ethernet), protokolljait (pl. IP, HTTP) és szabványait (pl.
  802.11-es WiFi szabványok). Ismeri az otthoni és irodai hálózatok
  legfontosabb összetevőinek (kábelezés, WiFi router, PC, mobiltelefon
  stb.) szerepét, jellemzőit, csatlakozási módjukat és alapszintű
  hálózati beállításait.
\item
  Törekszik a felhasználói igények megismerésére, megértésére, és szem
  előtt tartja azokat a hálózat kialakításakor.
\item
  Néhány alhálózatból álló kis- és közepes vállalati hálózatot alakít ki
  forgalomirányító és kapcsoló segítségével, elvégzi az eszközök
  alapszintű hálózati beállításait (pl. forgalomirányító interfészeinek
  IP-cím beállítása, alapértelmezett átjáró beállítása).
\item
  Ismeri a kis- és közepes vállalati hálózatok legfontosabb
  összetevőinek (pl. kábelrendező szekrény, kapcsoló, forgalomirányító)
  szerepét, jellemzőit, csatlakozási módjukat és alapszintű hálózati
  beállításait.
\item
  Alkalmazza a hálózatbiztonsággal kapcsolatos legfontosabb irányelveket
  (pl. erős jelszavak használata, vírusvédelem alkalmazása, tűzfal
  használat).
\item
  Ismeri a fontosabb hálózatbiztonsági elveket, szabályokat, támadás
  típusokat, valamint a szoftveres és hardveres védekezési módszereket.
\item
  Megkeresi és elhárítja az otthoni és kisvállalati informatikai
  környezetben jelentkező hardveres és szoftveres hibákat.
\item
  Ismeri az otthoni és kisvállalati informatikai környezetben
  leggyakrabban felmerülő hibákat (pl. hibás IP-beállítás, kilazult
  csatlakozó) és azok elhárításának módjait.
\item
  Önállóan behatárolja a hibát. Egyszerűbb problémákat önállóan,
  összetettebbeket szakmai irányítással hárít el.
\item
  Internetes források és tudásbázisok segítségével követi, valamint
  feladatainak elvégzéséhez lehetőség szerint alkalmazza a legmodernebb
  információs technológiákat és trendeket (virtualizáció,
  felhőtechnológia, IoT, mesterséges intelligencia, gépi tanulás stb.).
\item
  Naprakész információkkal rendelkezik a legmodernebb információs
  technológiákkal és trendekkel kapcsolatban.
\item
  Nyitott és érdeklődő a legmodernebb információs technológiák és
  trendek iránt.
\item
  Önállóan szerez információkat a témában releváns szakmai
  platformokról.
\item
  Szabványos, reszponzív megjelenítést biztosító weblapokat hoz létre és
  formáz meg stíluslapok segítségével.
\item
  Ismeri a HTML5, a CSS3 alapvető elemeit, a stíluslapok fogalmát,
  felépítését. Érti a reszponzív megjelenítéshez használt módszereket,
  keretrendszerek előnyeit, a reszponzív webdizájn alapelveit.
\item
  A felhasználói igényeknek megfelelő funkcionalitás és design
  összhangjára törekszik.
\item
  Önállóan létrehozza és megformázza a weboldalt.
\item
  Munkája során jelentkező problémák kezelésére vagy folyamatok
  automatizálására egyszerű alkalmazásokat készít Python programozási
  nyelv segítségével.
\item
  Ismeri a Python nyelv elemeit, azok céljait (vezérlési szerkezetek,
  adatszerkezetek, változók, aritmetikai és logikai kifejezések,
  függvények, modulok, csomagok). Ismeri az algoritmus fogalmát, annak
  szerepét.
\item
  Jól átlátható kódszerkezet kialakítására törekszik.
\item
  Önállóan készít egyszerű alkalmazásokat.
\item
  Git verziókezelő rendszert, valamint fejlesztést és csoportmunkát
  támogató online eszközöket és szolgáltatásokat (pl.: GitHub, Slack,
  Trello, Microsoft Teams, Webex Teams) használ.
\item
  Ismeri a Git, valamint a csoportmunkát támogató eszközök és online
  szolgáltatások célját, működési módját, legfontosabb funkcióit.
\item
  Törekszik a feladatainak megoldásában a hatékony csoportmunkát
  támogató online eszközöket kihasználni.
\item
  A Git verziókezelőt, valamint a csoportmunkát támogató eszközöket és
  szolgáltatásokat önállóan használja.
\item
  Társaival hatékonyan együttműködve, csapatban dolgozik egy
  informatikai projekten. A projektek végrehajtása során társaival
  tudatosan és célirányosan kommunikál.
\item
  Ismeri a projektmenedzsment lépéseit (kezdeményezés, követés,
  végrehajtás, ellenőrzés, dokumentáció, zárás).
\item
  Más munkáját és a csoport belső szabályait tiszteletben tartva,
  együttműködően vesz részt a csapatmunkában.
\item
  A projektekben irányítás alatt, társaival közösen dolgozik. A
  ráosztott feladatrészt önállóan végzi el.
\item
  Munkája során hatékonyan használja az irodai szoftvereket.
\item
  Ismeri az irodai szoftverek főbb funkcióit, felhasználási területeit.
\item
  Az elkészült termékhez prezentációt készít és bemutatja, előadja azt
  munkatársainak, vezetőinek, ügyfeleinek.
\item
  Ismeri a hatékony prezentálás szabályait, a prezentációs szoftverek
  lehetőségeit.
\item
  Törekszik a tömör, lényegre törő, de szakszerű bemutató
  összeállítására.
\item
  A projektcsapat tagjaival egyeztetve, de önállóan elkészíti az
  elvégzett munka eredményét bemutató prezentációt. \#\#\# 11-13.
  évfolyamon
\item
  Használja a Git verziókezelő rendszert, valamint a fejlesztést
  támogató csoportmunkaeszközöket és szolgáltatásokat (pl. GitHub,
  Slack, Trello, Microsoft Teams, Webex Teams).
\item
  Ismeri a legelterjedtebb csoportmunkaeszközöket, valamint a Git
  verziókezelőrendszer szolgáltatásait.
\item
  Igyekszik munkatársaival hatékonyan, igazi csapatjátékosként együtt
  dolgozni. Törekszik a csoporton belül megkapott feladatok precíz,
  határidőre történő elkészítésére, társai segítésére.
\item
  Szoftverfejlesztési projektekben irányítás alatt dolgozik, a rábízott
  részfeladatok megvalósításáért felelősséget vállal.
\item
  Az általa végzett szoftverfejlesztési feladatok esetében kiválasztja a
  legmegfelelőbb technikákat, eljárásokat és módszereket.
\item
  Elegendő ismerettel rendelkezik a meghatározó szoftverfejlesztési
  technológiák (programozási nyelvek, keretrendszerek, könyvtárak stb.),
  illetve módszerek erősségeiről és hátrányairól.
\item
  Nyitott az új technológiák megismerésére, tudását folyamatosan
  fejleszti.
\item
  Önállóan dönt a fejlesztés során használt technológiákról és
  eszközökről.
\item
  A megfelelő kommunikációs forma (e-mail, chat, telefon, prezentáció
  stb.) kiválasztásával munkatársaival és az ügyfelekkel hatékonyan
  kommunikál műszaki és egyéb információkról magyarul és angolul.
\item
  Ismeri a különböző kommunikációs formákra (e-mail, chat, telefon,
  prezentáció stb.) vonatkozó etikai és belső kommunikációs szabályokat.
\item
  Angol nyelvismerettel rendelkezik (KER B1 szint). Ismeri a gyakran
  használt szakmai kifejezéseket angolul.
\item
  Kommunikációjában konstruktív, együttműködő, udvarias. Feladatainak a
  felhasználói igényeknek leginkább megfelelő, minőségi megoldására
  törekszik.
\item
  Felelősségi körébe tartozó feladatokkal kapcsolatban a vállalati
  kommunikációs szabályokat betartva, önállóan kommunikál az ügyfelekkel
  és munkatársaival.
\item
  Szabványos, reszponzív megjelenítést biztosító weblapokat hoz létre és
  formáz meg stíluslapok segítségével. Kereső optimalizálási
  beállításokat alkalmaz.
\item
  Ismeri a HTML5 és a CSS3 szabvány alapvető nyelvi elemeit és eszközeit
  (strukturális és szemantikus HTML-elemek, attribútumok, listák,
  táblázatok, stílus jellemzők és függvények). Ismeri a a reszponzív
  webdizájn alapelveit és a Bootstrap keretrendszer alapvető
  szolgáltatásait.
\item
  Törekszik a weblapok igényes és a használatot megkönnyítő
  kialakítására.
\item
  Kisebb webfejlesztési projekteken önállóan, összetettebbekben
  részfeladatokat megvalósítva, irányítás mellett dolgozik.
\item
  Egyszerűbb webhelyek dinamikus viselkedését (eseménykezelés, animáció
  stb.) biztosító kódot, készít JavaScript nyelven.
\item
  Alkalmazási szinten ismeri a JavaScript alapvető nyelvi elemeit,
  valamint az aszinkron programozás és az AJAX technológia működési
  elvét. Tisztában van a legfrissebb ECMAScript változatok (ES6 vagy
  újabb) hatékonyság növelő funkcióival.
\item
  Egyszerűbb JavaScript programozási feladatokat önállóan végez el.
\item
  RESTful alkalmazás kliens oldali komponensének fejlesztését végzi
  JavaScript nyelven.
\item
  Tisztában van a REST szoftverarchitektúra elvével, alkalmazás szintjén
  ismeri az AJAX technológiát.
\item
  A tiszta kód elveinek megfelelő, megfelelő mennyiségű megjegyzéssel
  ellátott, kellőképpen tagolt, jól átlátható, kódot készít.
\item
  Ismeri a tiszta kód készítésének alapelveit.
\item
  Törekszik arra, hogy az elkészített kódja jól átlátható, és mások
  számára is értelmezhető legyen.
\item
  Adatbázis-kezelést is végző konzolos vagy grafikus felületű asztali
  alkalmazást készít magas szintű programozási nyelvet (C\#, Java)
  használva.
\item
  Ismeri a választott magas szintű programozási nyelv alapvető nyelvi
  elemeit, illetve a hozzá tartozó fejlesztési környezetet.
\item
  Törekszik a felhasználó számára minél könnyebb használatot biztosító
  felhasználói felület és működési mód kialakítására.
\item
  Kisebb asztali alkalmazás-fejlesztési projekteken önállóan,
  összetettebbekben részfeladatokat megvalósítva, irányítás mellett
  dolgozik.
\item
  Adatkezelő alkalmazásokhoz relációs adatbázist tervez és hoz létre,
  többtáblás lekérdezéseket készít.
\item
  Tisztában van a relációs adatbázis-tervezés és -kezelés alapelveivel.
  Haladó szinten ismeri a különböző típusú SQL lekérdezéseket, azok
  nyelvi elemeit és lehetőségeit.
\item
  Törekszik a redundanciamentes, világos szerkezetű, legcélravezetőbb
  kialakítású adatbázis szerkezet megvalósítására.
\item
  Kisebb projektekhez néhány táblás adatbázist önállóan tervez meg,
  nagyobb projektekben a biztosított adatbáziskörnyezetet használva
  önállóan valósít meg lekérdezéseket.
\item
  Önálló- vagy komplex szoftverrendszerek részét képző kliens oldali
  alkalmazásokat fejleszt mobil eszközökre.
\item
  Ismeri a választott mobil alkalmazás fejlesztésére alkalmas nyelvet és
  fejlesztői környezetet. Tisztában van a mobil alkalmazásfejlesztés
  alapelveivel.
\item
  Törekszik a felhasználó számára minél könnyebb használatot biztosító
  felhasználói felület és működési mód kialakítására.
\item
  Kisebb projektek mobil eszközökre optimalizált kliens oldali
  alkalmazását önállóan megvalósítja meg.
\item
  Webes környezetben futtatható kliens oldali (frontend) alkalmazást
  készít JavaScript keretrendszer (pl. React, Vue, Angular)
  segítségével.
\item
  Érti a frontend fejlesztésre szolgáló JavaScript keretrendszerek
  célját. Meg tudja nevezni a 3-4 legelterjedtebb keretrendszert.
  Alkalmazás szintjén ismeri a könyvtárak és modulok kezelését végző
  csomagkezelő rendszereket (package manager, pl. npm, yarn). Ismeri a
  választott JavaScript keretrendszer működési elvét, nyelvi és
  strukturális elemeit.
\item
  Törekszik maximálisan kihasználni a választott keretrendszer előnyeit,
  követi az ajánlott fejlesztési mintákat.
\item
  Kisebb frontend alkalmazásokat önállóan készít el, nagyobb
  projektekben irányítás mellett végzi el a kijelölt komponensek
  fejlesztését.
\item
  RESTful alkalmazás adatbázis-kezelési feladatokat is ellátó
  szerveroldali komponensének (backend) fejlesztését végzi erre alkalmas
  nyelv vagy keretrendszer segítségével (pl. Node.js, Spring, Laravel).
\item
  Érti a RESTful szoftverarchitektúra lényegét. Tisztában van legalább
  egy backend készítésére szolgáló nyelv vagy keretrendszer működési
  módjával, nyelvi és strukturális elemeivel. Alkalmazás szintjén ismeri
  az objektum-relációs leképzés technológiát (ORM).
\item
  Igyekszik backend működését leíró precíz, a frontend fejlesztők
  számára könnyen értelmezhető dokumentáció készítésére.
\item
  Kisebb backend alkalmazásokat önállóan készít el, nagyobb projektekben
  részletes specifikációt követve, irányítás mellett végzi el a kijelölt
  komponensek fejlesztését.
\item
  Objektum orientált (OOP) programozási módszertant alkalmazó asztali,
  webes és mobil alkalmazást készít.
\item
  Ismeri az objektumorientált programozás elvét, tisztában van az
  öröklődés, a polimorfizmus, a metódus/konstruktor túlterhelés
  fogalmával.
\item
  Törekszik az OOP technológia nyújtotta előnyök kihasználására,
  valamint igyekszik követni az OOP irányelveket és ajánlásokat.
\item
  Kisebb projektekben önállóan tervezi meg a szükséges osztályokat,
  nagyobb projektekben irányítás mellett, a projektben a projektcsapat
  által létrehozott osztálystruktúrát használva, illetve azt kiegészítve
  végzi a fejlesztést.
\item
  Tartalomkezelő rendszer (CMS, pl. WordPress) segítségével webhelyet
  készít, egyéni problémák megoldására saját beépülőket hoz létre.
\item
  Ismeri a tartalomkezelő-rendszerek célját és alapvető szolgáltatásait.
  Ismeri a beépülők célját és alkalmazási területeit.
\item
  Törekszik az igényes kialakítású és a felhasználók számára könnyű
  használatot biztosító webhelyek kialakításra.
\item
  Kevésbé összetett portálokat igényes vizuális megjelenést biztosító
  sablonok, valamint magas funkcionalitást biztosító beépülők
  használatával önállóan valósít meg. Összetettebb projekteken irányítás
  mellett, grafikus tervezőkkel, UX szakemberekkel és más fejlesztőkkel
  együttműködve dolgozik.
\item
  Manuális és automatizált szoftvertesztelést végezve ellenőrzi a
  szoftver hibátlan működését, dokumentálja a tesztek eredményét.
\item
  Ismeri a unit tesztelés, valamint más tesztelési, hibakeresési
  technikák alapelveit és alapvető eszközeit.
\item
  Törekszik a mindenre kiterjedő, az összes lehetséges hibát felderítő
  tesztelésre, valamint a tesztek körültekintő dokumentálására.
\item
  Saját fejlesztésként megvalósított kisebb projektekben önállóan végzi
  a tesztelést, tesztelői szerepben nagyobb projektekben irányítás
  mellett végez meghatározott tesztelési feladatokat.
\item
  Szoftverfejlesztés vagy -tesztelés során felmerülő problémákat old meg
  és hibákat hárít el webes kereséssel és internetes tudásbázisok
  használatával (pl. Stack Overflow).
\item
  Ismeri a hibakeresés szisztematikus módszereit, a problémák
  elhárításának lépéseit.
\item
  Ismeri a munkájához kapcsolódó internetes keresési módszereket és
  tudásbázisokat.
\item
  Törekszik a hibák elhárítására, megoldására, és arra, hogy azokkal
  lehetőség szerint ne okozzon újabb hibákat.
\item
  Internetes információszerzéssel önállóan old meg problémákat és hárít
  el hibákat.
\item
  Munkája során hatékonyan használja az irodai szoftvereket, műszaki
  tartalmú dokumentumokat és bemutatókat készít.
\item
  Ismeri az irodai szoftverek haladó szintű szolgáltatásait.
\item
  Precízen készíti el a műszaki tartalmú dokumentációkat,
  prezentációkat. Törekszik arra, hogy a dokumentumok könnyen
  értelmezhetők és mások által is szerkeszthetők legyenek.
\item
  Felelősséget vállal az általa készített műszaki tartalmú
  dokumentációkért.
\item
  Munkája során cél szerint alkalmazza a legmodernebb információs
  technológiákat és trendeket (virtaulizáció, felhőtechnológia, IoT,
  mesterséges intelligencia, gépi tanulás stb.).
\item
  Alapszintű alkalmazási szinten ismeri a legmodernebb információs
  technológiákat és trendeket (virtualizáció, felhőtechnológia, IoT,
  mesterséges intelligencia, gépi tanulás stb.).
\item
  Nyitott az új technológiák megismerésére, és törekszik azok hatékony,
  a felhasználói igényeknek és a költséghatékonysági elvárásoknak
  megfelelő felhasználására a szoftverfejlesztési feladatokban.
\item
  Részt vesz szoftverrendszerek ügyfeleknél történő bevezetésében, a
  működési környezetet biztosító IT-környezet telepítésében és
  beállításában.
\item
  Ismeri a számítógép és a mobil informatikai eszközök felépítését (főbb
  komponenseket, azok feladatait) és működését. Ismeri az eszközök
  operációs rendszerének és alkalmazói szoftvereinek telepítési és
  beállítási lehetőségeit.
\item
  A szoftverrendszerek bevezetése és a működési környezet kialakítása
  során törekszik az ügyfelek elvárásainak megfelelni, valamint
  tiszteletben tartja az ügyfél vállalati szabályait.
\item
  Az elvégzett eszköz- és szoftvertelepítésekért felelősséget vállal.
\item
  A szoftverfejlesztés és tesztelési munkakörnyezetének kialakításához
  beállítja a hálózati eszközöket, elvégzi a vezetékes és vezetéknélküli
  eszközök csatlakoztatását és hálózatbiztonsági beállítását. A
  fejlesztett szoftverben biztonságos, HTTPS protokollt használó webes
  kommunikációt valósít meg.
\item
  Ismeri az IPv4 és IPv6 címzési rendszerét és a legalapvetőbb hálózati
  protokollok szerepét és működési módját (IP, TCP, UDP, DHCP, HTTP,
  HTTPS, telnet, ssh, SMTP, POP3, IMAP4, DNS, TLS/SSL stb.). Ismeri a
  végponti berendezések IP-beállítási és hibaelhárítási lehetőségeit.
  Ismeri az otthoni és kisvállalati hálózatokban működő multifunkciós
  forgalomirányítók szolgáltatásait, azok beállításának módszereit. \#\#
  Technika és tervezés \#\#\# 1-4. évfolyamon
\item
  Elkülöníti a természeti és mesterséges környezet jellemzőit.
\item
  Felismeri, hogy tevékenységei során változtatni tud a közvetlen
  környezetén.
\item
  Kitartó a munkavégzésben, szükség esetén segítséget kér, segítséget
  ad.
\item
  Szöveg vagy rajz alapján épít, tárgyakat készít, alkalmazza a tanult
  munkafolyamatokat, terveit megosztja.
\item
  Munkafolyamatokat, technológiákat segítséggel algoritmizál.
\item
  Megadott szempontok mentén értékeli saját, a társak, a csoport
  munkáját, viszonyítja a kitűzött célokhoz.
\item
  Alkotótevékenysége során megéli, megismeri a jeles napokat, ünnepeket,
  hagyományokat mint értékeket.
\item
  Tevékenysége során munkatapasztalatot szerez, megéli az alkotás
  örömét, az egyéni és csapatsiker élményét.
\item
  Felismeri az egymásért végzett munka fontosságát, a munkamegosztás
  értékét.
\item
  Az anyagok tulajdonságairól érzékszervi úton, önállóan szerez
  ismereteket -- szín, alak, átlátszóság, szag, keménység, rugalmasság,
  felületi minőség.
\item
  Alkotótevékenysége során figyelembe veszi az anyag tulajdonságait,
  felhasználhatóságát.
\item
  Adott szempontok alapján egyszerűbb tárgyakat önállóan tervez, készít,
  alkalmazza a tanult munkafolyamatokat.
\item
  Egyszerű szöveges, rajzos és képi utasításokat hajt végre a
  tevékenysége során.
\item
  Alkotótevékenysége során előkészítő, alakító, szerelő és felületkezelő
  műveleteket végez el.
\item
  Saját és társai tevékenységét a kitűzött célok mentén, megadott
  szempontok szerint reálisan értékeli.
\item
  Értékelés után megfogalmazza tapasztalatait, következtetéseket von le
  a későbbi eredményesebb munkavégzés érdekében.
\item
  Felismeri, hogy tevékenysége során tud változtatni közvetlen
  környezetén, megóvhatja, javíthat annak állapotán.
\item
  Rendet tart a környezetében.
\item
  Törekszik a takarékos anyagfelhasználásra.
\item
  Szelektíven gyűjti a hulladékot.
\item
  Rendelkezik az életkorának megfelelő szintű probléma-felismerési,
  probléma-megoldási képességgel.
\item
  Ismeri a családellátó tevékenységeket, melyek keretében vállalt
  feladatait az iskolai önellátás során munkamegosztásban végzi --
  terítés, rendrakás, öltözködés, növények, állatok gondozása stb..
\item
  Otthoni és iskolai környezetének, tevékenységeinek balesetveszélyes
  helyzeteit felismeri, és ismeri megelőzésük módját.
\item
  Takarékosan gazdálkodik az anyaggal, energiával, idővel.
\item
  Ismeri a tudatos vásárlás néhány fontos elemét.
\item
  Ismeri az egészségmegőrzés tevékenységeit.
\item
  Ismeri és használni, alkalmazni tudja a legfontosabb közlekedési
  lehetőségeket, szabályokat, viselkedési elvárásokat.
\item
  Tudatosan megtartja az egészséges és biztonságos munkakörnyezetét.
\item
  Az elvárt feladatokban önállóan dolgozik -- elvégzi a műveletet.
\item
  Társaival munkamegosztás szerint együttműködik a csoportos munkavégzés
  során.
\item
  Felismeri az egymásért végzett munka fontosságát, a munkamegosztás
  értékét.
\item
  Ismeri a környezetében fellelhető, megfigyelhető szakmák, hivatások
  jellemzőit.
\end{itemize}

\hypertarget{evfolyamon-30}{%
\subsubsection{5-7. évfolyamon}\label{evfolyamon-30}}

\begin{itemize}
\item
  Ismeri a felhasznált anyagok vizsgálati lehetőségeit és módszereit,
  tulajdonságait, és céljainak megfelelően választ a rendelkezésre álló
  anyagokból.
\item
  Tevékenységét megtervezi, terveit másokkal megosztja.
\item
  Ismeri és betartja az alapvető munkavédelmi szabályokat.
\item
  Tervek mentén folytatja alkotótevékenységét.
\item
  Célszerűen választja ki és rendeltetésszerűen használja a szükséges
  szerszámokat, eszközöket, digitális alkalmazásokat.
\item
  Törekszik a balesetmentes tevékenységre, a munkaterületen rendet tart.
\item
  Munkavégzéskor szabálykövető, kooperatív magatartást követ.
\item
  Ismeri az egyes műveletek jelentőségét a munka biztonságának,
  eredményességének vonatkozásában.
\item
  A tevékenység során társaival együttműködik, feladatmegosztás szerint
  tevékenykedik.
\item
  Az elkészült produktumot használatba veszi, a tervhez viszonyítva
  értékeli saját, mások és a csoport munkáját.
\item
  Értékeli az elvégzett munkákat, az értékelésben elhangzottakat
  felhasználja a későbbi munkavégzés során.
\item
  Értékként tekint saját és mások alkotásaira, a létrehozott
  produktumokra.
\item
  Felismeri az emberi cselekvés jelentőségét és felelősségét a
  környezetalakításban.
\item
  Önállóan szerez információt megfigyelés, vizsgálat, adatgyűjtés útján.
\item
  Környezeti, fenntarthatósági szempontokat is mérlegelve, céljainak
  megfelelően választ a rendelkezésre álló anyagokból.
\item
  Tevékenységét önállóan vagy társakkal együttműködve tervezi.
\item
  Terveit a műszaki kommunikáció alkalmazásával osztja meg.
\item
  A terv szerinti lépések megtartásával, önellenőrzéssel halad
  alkotótevékenységében.
\item
  Alkalmazza a forma és funkció összefüggéseit, önállóan választ
  szerszámot, eszközt.
\item
  A megismert szerszámokat és eszközöket önállóan, az újakat tanári
  útmutatással használja.
\item
  Részt vesz a munkavégzési szabályok megalkotásában, betartja azokat.
\item
  Csoportmunkában feladatot vállal, részt vesz a döntéshozatalban, és a
  döntésnek megfelelően tevékenykedik.
\item
  Felméri és tervezi a tevékenység munkavédelmi szabályait.
\item
  A csoportban feladata szerint tevékenykedik, tudását megosztja.
\item
  Adott szempontok mentén értékeli saját és mások munkáját.
\item
  A használatbavétel során, az eltéréseket kiindulópontként alkalmazva
  javaslatot tesz produktuma továbbfejlesztésére.
\item
  Megérti az egyén felelősségét a közös értékteremtésben.
\item
  Szempontokat határoz meg a környezeti állapot felméréséhez, bizonyos
  eltéréseket számszerűsít.
\item
  Érti és értékeli a globális változásokat érintő lehetséges megoldások
  és az emberi tevékenység szerepét, jelentőségét.
\item
  Tevékenységének tervezésénél és értékelésénél figyelembe vesz
  környezeti szempontokat.
\item
  Felismeri a technikai fejlődés és a társadalmi, gazdasági fejlődés
  kapcsolatát.
\item
  A problémamegoldás során önállóan vagy társakkal együtt fogalmaz meg
  megoldási alternatívákat.
\item
  Komplex szempontrendszer mentén választ stratégiát, optimalizál.
\item
  Holisztikus szemlélettel rendelkezik, az összefüggések megértésére
  törekszik.
\item
  Döntéseit tudatosság jellemzi, alternatívákat mérlegel.
\item
  Felismeri a személyes cselekvés jelentőségét a globális problémák
  megoldásában.
\item
  Felismeri saját felelősségét életvezetése megtervezésében és
  megszervezésében, tudatosan gazdálkodik a rendelkezésre álló anyagi és
  nem anyagi erőforrásokkal.
\item
  Rendszerszinten végzi az elemzést és az alkalmazást.
\item
  Tisztában van a saját, a családi és a társadalmi erőforrásokkal és az
  azokkal való hatékony és tudatos gazdálkodás módjaival.
\item
  Egészség- és környezettudatosan dönt és tevékenykedik.
\item
  Terv szerint tevékenykedik, probléma esetén észszerű kockázatokat
  felvállal.
\item
  Önismeretére építve vállal feladatokat, szem előtt tartva a csapat
  eredményességét.
\item
  Alkalmazkodik a változó munkafeladatokhoz, szerepelvárásokhoz;
  vezetőként tudatosan vezeti a csoport döntési folyamatát.
\item
  Alkalmazza a döntés-előkészítés, döntéshozatal eljárásait, hibás
  döntésein változtat.
\item
  Az egyes részfeladatokat rendszerszinten szemléli.
\item
  Érti a társadalmi munkamegosztás lényegét, az egyes foglalkoztatási
  ágazatok jelentőségét.
\item
  Ismeri az egyes modulokhoz kapcsolódó foglalkozások jellemzőit,
  ezekkel kapcsolatban megfogalmazza saját preferenciáit.
\item
  A fizikai és digitális környezetből információt gyűjt a számára vonzó
  foglalkozások alkalmassági és képesítési feltételeiről, keresi a
  vállalkozási lehetőségeket, a jövedelmezőséget és a jellemző tanulási
  utakat.
\end{itemize}

\hypertarget{termeszettudomany}{%
\subsection{Természettudomány}\label{termeszettudomany}}

\hypertarget{evfolyamon-31}{%
\subsubsection{4. évfolyamon}\label{evfolyamon-31}}

\begin{itemize}
\item
  Ráébredjen a természeti rendszerek és a természetben zajló folyamatok
  komplexitására, alapvető okaira és magyarázataira.
\item
  Képessé váljon az önálló ismeretszerzésre, az összefüggések
  felismerésére és az egyszerű elemzések elvégzésére a tanulói
  kísérletek, terepi megfigyelések és vizsgálatok révén, azzal, hogy a
  távlati cél a felsőbb évfolyamokon való értő és önálló munkavégzés
  lehetőségének megalapozása.
\item
  Elsajátítsa a természettudományok egységét szem előtt tartó
  szintetizáló gondolkodásmódot, legyen képes folyamatokat rendszerben
  szemlélni.
\item
  Tudjon kritikusan gondolkodni az adott természeti, környezeti
  problémáról, illetve hogy felismerje az áltudományos információkat,
  amely nagyban hozzájárul a felelős és tudatos állampolgári
  szerepvállalás kialakításához.
\item
  Hozzáférjen a mindennapi életben hasznosítható természettudományos
  tudáshoz, amelynek révén a mindennapi életükben előforduló egyszerűbb
  problémákat tudjon megoldani, és kialakuljon benne az értő, felelős
  döntéshozás képessége.
\item
  A természetben lejátszódó folyamatok vizsgálatával, a várható
  következmények megértésével cselekvőképes, a környezetért
  felelősséggel tenni akaró állampolgárrá váljon, ezzel is hangsúlyozva,
  hogy az ember egyénként és egy nagyobb közösség részeként egyaránt
  felelős természeti környezetéért, annak jövőbeni állapotáért.
\item
  Felismerje és megértse, hogy az élhető jövő záloga a környezettudatos,
  fenntarthatóságot szem előtt tartó gondolkodás.
\item
  Tudatos eszközhasználóvá váljon az infokommunikációs eszközök
  használata és a digitális kompetenciák fejlesztése révén.
\item
  Segítséget kapjon a későbbi műszaki vagy természettudományos
  pályaválasztáshoz.
\item
  Megfigyelési és mérési módszerek
\item
  Tájékozódás az időben
\item
  Tájékozódás a térben
\item
  Élő környezetünk
\item
  Anyagok és folyamatok
\end{itemize}

\hypertarget{evfolyamon-32}{%
\subsubsection{5-6. évfolyamon}\label{evfolyamon-32}}

\begin{itemize}
\item
  Megfigyeléseket, összehasonlításokat, csoportosításokat, méréseket és
  kísérleteket végez.
\item
  Megfigyeléseiről, tapasztalatairól szóban, írásban, rajzban beszámol.
\item
  A szöveggel, táblázattal és jelekkel kapott információt önállóan
  értelmezi, azokból következtetéseket von le.
\item
  Ismeretei bővítéséhez tanári útmutatás mellett kutatásokat végez a
  nyomtatott és digitális források felhasználásával.
\item
  Kialakul benne a szűkebb, illetve tágabb környezet iránti
  felelősségtudat.
\item
  Kialakul benne a természettudomány iránti érdeklődés.
\item
  Felismeri és megfigyeli a környezetben előforduló élő és élettelen
  anyagokat, megadott vagy önállóan kitalált szempontok alapján
  csoportosítja azokat.
\item
  Felismer és megfigyel különböző természetes és mesterséges anyagokat,
  ismeri azok tulajdonságait, felhasználhatóságukat, ismeri a
  természetes és mesterséges környezetre gyakorolt hatásukat.
\item
  Önállóan végez becsléseket, méréseket és használ mérőeszközöket
  különféle fizikai paraméterek meghatározására.
\item
  Önállóan végez egyszerű kísérleteket.
\item
  Korábbi tapasztalatai és megfigyelései révén felismeri a víz különböző
  tulajdonságait, különböző szempontok alapján rendszerezi a vizek
  fajtáit.
\item
  Bizonyítja, és hétköznapi példákkal alátámasztja a víz fagyásakor
  történő térfogat-növekedést.
\item
  Megfigyeli a különböző halmazállapot-változásokhoz (olvadás, fagyás,
  párolgás, forrás, lecsapódás) kapcsolódó folyamatokat, példákat gyűjt
  hozzájuk a természetben, a háztartásban és az iparban.
\item
  Kísérletek során megfigyeli a különböző halmazállapotú anyagok vízben
  oldódásának folyamatát.
\item
  Felismeri az olvadás és az oldódás közötti különbséget kísérleti
  tapasztalatok alapján.
\item
  Elsajátítja a tűzveszélyes anyagokkal való bánásmódot, ismeri tűz
  esetén a szükséges teendőket.
\item
  Megfigyeli a talaj élő és élettelen alkotóelemeit, tulajdonságait,
  összehasonlít különböző típusú talajféleségeket, valamint
  következtetések révén felismeri a talajnak, mint rendszernek a
  komplexitását.
\item
  Korábbi tapasztalatai és megfigyelései révén felismeri a levegő egyes
  tulajdonságait.
\item
  Vizsgálat révén azonosítja a tipikus lágyszárú és fásszárú növények
  részeit.
\item
  Megkülönbözteti a hely- és helyzetváltoztatást és példákat keres
  ezekre megadott szempontok alapján.
\item
  Önállóan végez becsléseket, méréseket és használ mérőeszközöket a
  hőmérséklet, a hosszúság, a tömeg, az űrtartalom és az idő
  meghatározására.
\item
  Észleli, méri az időjárási elemeket, a mért adatokat rögzíti,
  ábrázolja.
\item
  Magyarországra vonatkozó adatok alapján kiszámítja a napi
  középhőmérsékletet, a napi és évi közepes hőingást.
\item
  Leolvassa és értékeli a magyarországra vonatkozó éghajlati diagramok
  és éghajlati térképek adatait.
\item
  Megfigyeli a mágneses kölcsönhatásokat, kísérlettel igazolja a vonzás
  és a taszítás jelenségét, példákat ismer a mágnesesség gyakorlati
  életben való felhasználására.
\item
  Megfigyeli a testek elektromos állapotát és a köztük lévő
  kölcsönhatásokat, ismeri ennek gyakorlati életben való megjelenését.
\item
  Megfigyeléseken és kísérleten keresztül megismeri az
  energiatermelésben szerepet játszó anyagokat és az energiatermelés
  folyamatát.
\item
  Kísérletekkel igazolja a növények életfeltételeit.
\item
  Kísérleti úton megfigyeli az időjárás alapvető folyamatait, magyarázza
  ezek okait és következményeit.
\item
  Felismeri az idő múlásával bekövetkező változásokat és ezek
  összefüggéseit az élő és élettelen környezet elemein.
\item
  Tudja értelmezni az időt különböző dimenziójú skálákon.
\item
  Tervet készít saját időbeosztására vonatkozóan.
\item
  Megfigyeli a természet ciklikus változásait.
\item
  Megérti a föld mozgásai és a napi, évi időszámítás közötti
  összefüggéseket.
\item
  Modellezi a nap és a föld helyzetét a különböző napszakokban és
  évszakokban.
\item
  Meghatározza az irányt a valós térben.
\item
  Érti a térkép és a valóság közötti viszonyt.
\item
  Tájékozódik a térképen és a földgömbön.
\item
  Mágneses kölcsönhatásként értelmezi az iránytű működését.
\item
  Felismeri a felszínformák ábrázolását a térképen.
\item
  Megérti a méretarány és az ábrázolás részletessége közötti
  összefüggéseket.
\item
  Fő- és mellékégtájak segítségével meghatározza különböző földrajzi
  objektumok egymáshoz viszonyított helyzetét.
\item
  Felismeri és használja a térképi jelrendszert és a térképfajtákat
  (domborzati térkép, közigazgatási térkép, autós térkép, turista
  térkép).
\item
  Felismeri a földrészeket és az óceánokat a különböző méretarányú és
  ábrázolásmódú térképeken.
\item
  Felismeri a nevezetes szélességi köröket a térképen.
\item
  Megfogalmazza európa és magyarország tényleges és viszonylagos
  földrajzi fekvését.
\item
  Ismeri a főfolyó, a mellékfolyó és a torkolat térképi ábrázolását.
\item
  Felismeri és megnevezi a legjelentősebb hazai álló- és folyóvizeket.
\item
  Bejelöli a térképen budapestet és a saját lakóhelyéhez közeli
  fontosabb nagyvárosokat és a szomszédos országokat.
\item
  A valóságban megismert területről egyszerű, jelrendszerrel ellátott
  útvonaltervet, térképet készít.
\item
  Tájékozódik a terepen térképvázlat, iránytű és gps segítségével.
\item
  Meghatározott szempontok alapján útvonalat tervez a térképen.
\item
  Használni tud néhány egyszerű térinformatikai alkalmazást.
\item
  Komplex rendszerként értelmezi az élő szervezeteket és az ezekből
  felépülő élőlénytársulásokat.
\item
  Tisztában van az életfeltételek és a testfelépítés közti kapcsolattal.
\item
  Tisztában van azzal, hogy az élő rendszerekbe történő beavatkozás
  káros hatásokkal járhat.
\item
  Felismeri és megnevezi a növények életfeltételeit, életjelenségeit.
\item
  Összehasonlít ismert hazai termesztett vagy vadon élő növényeket adott
  szempontok (testfelépítés, életfeltételek, szaporodás) alapján.
\item
  Felismeri és megnevezi a növények részeit, megfigyeli jellemzőiket,
  megfogalmazza ezek funkcióit.
\item
  Összehasonlítja ismert hazai termesztett vagy vadon élő növények
  részeit megadott szempontok alapján.
\item
  Ismert hazai termesztett vagy vadon élő növényeket különböző
  szempontok szerint csoportosít.
\item
  Azonosítja a lágyszárú és a fás szárú növények testfelépítése közötti
  különbségeket.
\item
  Felismeri és megnevezi az állatok életfeltételeit és életjelenségeit.
\item
  Összehasonlít ismert hazai házi vagy vadon élő állatokat adott
  szempontok (testfelépítés, életfeltételek, szaporodás) alapján.
\item
  Felismeri és megnevezi az állatok testrészeit, megfigyeli
  jellemzőiket, megfogalmazza ezek funkcióit.
\item
  Az állatokat különböző szempontok szerint csoportosítja.
\item
  Azonosítja a gerinctelen és a gerinces állatok testfelépítése közötti
  különbségeket.
\item
  Mikroszkóp segítségével megfigyel egysejtű élőlényeket.
\item
  Megfigyeli hazánk erdei élőlénytársulásainak főbb jellemzőit.
\item
  Életközösségként értelmezi az erdőt, mezőt, vizes élőhelyeket.
\item
  Felismeri és magyarázza az élőhely -- életmód - testfelépítés
  összefüggéseit az erdők életközössége esetén.
\item
  Példákkal bizonyítja, rendszerezi és következtetéseket von le az erdei
  élőlények környezethez történő alkalmazkodására vonatkozóan.
\item
  Táplálékláncokat és azokból táplálékhálózatot állít össze a megismert
  erdei növény- és állatfajokból.
\item
  Példákon keresztül bemutatja az erdőgazdálkodási tevékenységek
  életközösségre gyakorolt hatásait.
\item
  Tisztában van az erdő természetvédelmi értékével, fontosnak tartja
  annak védelmét.
\item
  Megfigyeli hazánk fátlan élőlénytársulásainak főbb jellemzőit.
\item
  Megadott szempontok alapján összehasonlítja a rétek és a szántóföldek
  életközösségeit.
\item
  Felismeri és magyarázza az élőhely - életmód - testfelépítés
  összefüggéseit a rétek életközössége esetén.
\item
  Példákkal bizonyítja, rendszerezi és következtetéseket von le a mezei
  élőlények környezethez történő alkalmazkodására vonatkozóan.
\item
  Táplálékláncokat és azokból táplálékhálózatot állít össze a megismert
  mezei növény- és állatfajokból.
\item
  Példákon keresztül mutatja be a mezőgazdasági tevékenységek
  életközösségre gyakorolt hatásait.
\item
  Tisztában van a fátlan társulások természetvédelmi értékével,
  fontosnak tartja annak védelmét.
\item
  Megfigyeli hazánk vízi és vízparti élőlénytársulásainak főbb
  jellemzőit.
\item
  Összehasonlítja a vízi és szárazföldi élőhelyek környezeti tényezőit.
\item
  Felismeri és magyarázza az élőhely -- életmód - testfelépítés
  összefüggéseit a vízi és vízparti életközösségek esetén.
\item
  Példákkal bizonyítja, rendszerezi és következtetéseket von le a vízi
  élőlények környezethez történő alkalmazkodására vonatkozóan.
\item
  Táplálékláncokat és ezekből táplálékhálózatot állít össze a megismert
  vízi és vízparti növény- és állatfajokból.
\item
  Példákon keresztül bemutatja a vízhasznosítás és a vízszennyezés
  életközösségre gyakorolt hatásait.
\item
  Tisztában van a vízi társulások természetvédelmi értékével, fontosnak
  tartja annak védelmét.
\item
  Érti, hogy a szervezet rendszerként működik.
\item
  Tisztában van a testi és lelki egészség védelmének fontosságával.
\item
  Tisztában van az egészséges környezet és az egészségmegőrzés közti
  összefüggéssel.
\item
  Felismeri és megnevezi az emberi test fő részeit, szerveit.
\item
  Látja az összefüggéseket az egyes szervek működése között.
\item
  Érti a kamaszkori testi és lelki változások folyamatát, élettani
  hátterét.
\item
  Tisztában van az egészséges életmód alapelveivel, azokat igyekszik
  betartani.
\item
  Összetett rendszerként értelmezi az egyes földi szférák működését.
\item
  Ismeri a természeti erőforrások energiatermelésben betöltött szerepét.
\item
  Tisztában van a természeti erők szerepével a felszínalakításban.
\item
  Csoportosítja az energiahordozókat különböző szempontok alapján.
\item
  Példákat hoz a megújuló és a nem megújuló energiaforrások
  felhasználására.
\item
  Megismeri az energiatermelés hatását a természetes és a mesterséges
  környezetre.
\item
  Megállapítja, összehasonlítja és csoportosítja néhány jellegzetes
  hazai kőzet egyszerűen vizsgálható tulajdonságait.
\item
  Példákat hoz a kőzetek tulajdonságai és a felhasználásuk közötti
  összefüggésekre.
\item
  Tisztában van azzal, hogy a talajpusztulás világméretű probléma.
\item
  Ismer olyan módszereket, melyek a talajpusztulás ellen hatnak
  (tápanyag-visszapótlás, komposztkészítés, ökológiai kertművelés).
\item
  Felismeri és összehasonlítja a gyűrődés, a vetődés, a földrengés és a
  vulkáni tevékenység hatásait.
\item
  Magyarázza a felszín lejtése, a folyó vízhozama, munkavégző képessége
  és a felszínformálás közti összefüggéseket.
\item
  Magyarázza az éghajlat és a folyók vízjárása közötti összefüggéseket.
\item
  Megnevezi az éghajlat fő elemeit.
\item
  Jellemezi és összehasonlítja az egyes éghajlati övezeteket (forró,
  mérsékelt, hideg).
\item
  Értelmezi az évszakok változását.
\item
  Értelmezi az időjárás-jelentést.
\item
  Piktogramok alapján megfogalmazza a várható időjárást.
\end{itemize}

\hypertarget{testneveles-es-egeszsegfejlesztes}{%
\subsection{Testnevelés és
egészségfejlesztés}\label{testneveles-es-egeszsegfejlesztes}}

\hypertarget{evfolyamon-33}{%
\subsubsection{1-4. évfolyamon}\label{evfolyamon-33}}

\begin{itemize}
\item
  Életkorának és testi adottságának megfelelően fejlődött motoros
  teljesítőképessége a hozzá kapcsolódó ismeretekkel olyan mérvű, hogy
  képes a saját teljesítménye tudatos befolyásolására.
\item
  Mozgáskultúrája olyan szintre fejlődött, hogy képes a hatékony
  mozgásos cselekvéstanulásra, testedzésre.
\item
  Ismeri a testnevelés életkorához igazodó elméleti ismeretanyagát,
  szakkifejezéseit, helyes terminológiáját, érti azok szükségességét.
\item
  Megismeri az elsősegélynyújtás jelentőségét, felismeri a baleseti és
  egészségkárosító veszélyforrásokat, képes azonnali segítséget kérni.
\item
  Önismerete, érzelmi-akarati készségei és képességei a testmozgás, a
  testnevelés és a sport eszközei által megfelelően fejlődtek.
\item
  A tanult mozgásformákat összefüggő cselekvéssorokban, jól koordináltan
  kivitelezi.
\item
  A tanult mozgásforma könnyed és pontos kivitelezésének elsajátításáig
  fenntartja érzelmi-akarati erőfeszítéseit.
\item
  A sportjátékok, a testnevelési és népi játékok művelése során egyaránt
  törekszik a szabályok betartására.
\item
  Nyitott az alapvető mozgásformák újszerű és alternatív környezetben
  történő alkalmazására, végrehajtására.
\item
  Megfelelő motoros képességszinttel rendelkezik az alapvető
  mozgásformák viszonylag önálló és tudatos végrehajtásához.
\item
  Olyan szintű relatív erővel rendelkezik, amely lehetővé teszi
  összefüggő cselekvéssorok kidolgozását, az elemek közötti összhang
  megteremtését.
\item
  Megfelelő általános állóképesség-fejlődést mutat.
\item
  Az alapvető mozgásformákat külsőleg meghatározott ritmushoz, a társak
  mozgásához igazított sebességgel és dinamikával képes végrehajtani.
\item
  Mozgásműveltsége szintjénél fogva pontosan hajtja végre a keresztező
  mozgásokat.
\item
  Futását összerendezettség, lépésszabályozottság, ritmusosság jellemzi.
\item
  Különböző mozgásai jól koordináltak, a hasonló mozgások szimultán és
  egymást követő végrehajtása jól tagolt.
\item
  Az egyszerűbb mozgásformákat jól koordináltan hajtja végre, a hasonló
  mozgások szimultán és egymást követő végrehajtásában megfelelő szintű
  tagoltságot mutat.
\item
  A különböző ugrásmódok alaptechnikáit és előkészítő mozgásformáit a
  vezető műveletek ismeretében tudatosan és koordináltan hajtja végre.
\item
  A különböző dobásmódok alaptechnikáit és előkészítő mozgásformáit a
  vezető műveletek ismeretében tudatosan és koordináltan hajtja végre.
\item
  A támasz- és függésgyakorlatok végrehajtásában a testtömegéhez igazodó
  erő- és egyensúlyozási képességgel rendelkezik.
\item
  A funkcionális hely- és helyzetváltoztató mozgásformáinak kombinációit
  változó feltételek között koordináltan hajtja végre.
\item
  Labdás ügyességi szintje lehetővé teszi az egyszerű taktikai
  helyzetekre épülő folyamatos, célszerű játéktevékenységet.
\item
  A megtanultak birtokában örömmel, a csapat teljes jogú tagjaként vesz
  részt a játékokban.
\item
  Vállalja a társakkal szembeni fizikai kontaktust, sportszerű test-test
  elleni küzdelmet valósít meg.
\item
  Ismeri és képes megnevezni a küzdőfeladatok, esések, tompítások játék-
  és baleset-megelőzési szabályait.
\item
  Ellenőrzött tevékenység keretében mozog a szabad levegőn, egyúttal
  tudatosan felkészül az időjárás kellemetlen hatásainak elviselésére
  sportolás közben.
\item
  Az elsajátított egy (vagy több) úszásnemben helyes technikával úszik.
\item
  A testnevelési és népi játékokban tudatosan, célszerűen alkalmazza az
  alapvető mozgásformákat.
\item
  Játék közben az egyszerű alaptaktikai elemek tudatos alkalmazására
  törekszik, játék- és együttműködési készsége megmutatkozik.
\item
  Célszerűen alkalmaz sportági jellegű mozgásformákat
  sportjáték-előkészítő kisjátékokban.
\item
  Játéktevékenysége közben a tanult szabályokat betartja.
\item
  A versengések és a versenyek közben toleráns a csapattársaival és az
  ellenfeleivel szemben.
\item
  A versengések és a versenyek tudatos szereplője, a közösséget
  pozitívan alakító résztvevő.
\item
  A szabályjátékok közben törekszik az egészséges versenyszellem
  megőrzésére.
\item
  Felismeri a sportszerű és sportszerűtlen magatartásformákat, betartja
  a sportszerű magatartás alapvető szabályait.
\item
  Felismeri a különböző veszély- és baleseti forrásokat, elkerülésükhöz
  tanári segítséget kér.
\item
  Ismeri a keringési, légzési és mozgatórendszerét fejlesztő alapvető
  mozgásformákat.
\item
  Tanári segítséggel megvalósít a biomechanikailag helyes testtartás
  kialakítását elősegítő gyakorlatokat.
\item
  Tanári irányítással, ellenőrzött formában végzi a testnevelés --
  számára nem ellenjavallt -- mozgásanyagát.
\item
  Aktívan vesz részt az uszodában végzett mozgásformák elsajátításában,
  gyakorlásában.
\item
  A családi háttere és a közvetlen környezete adta lehetőségeihez mérten
  rendszeresen végez testmozgást.
\item
  Az öltözködés és a higiéniai szokások terén teljesen önálló, adott
  esetben segíti társait.
\item
  Megismeri az életkorának megfelelő sporttáplálkozás alapelveit, képes
  különbséget tenni egészséges és egészségtelen tápanyagforrások között.
\item
  A szabadban végzett foglalkozások során nem csupán ügyel környezete
  tisztaságára és rendjére, hanem erre felhívja társai figyelmét is.
\item
  Ismeri a helyes testtartás egészségre gyakorolt pozitív hatásait.
\end{itemize}

\hypertarget{evfolyamon-34}{%
\subsubsection{5-8. évfolyamon}\label{evfolyamon-34}}

\begin{itemize}
\item
  Életkorának és testi adottságának megfelelően fejlődött motoros
  teljesítőképessége olyan mérvű, hogy képes a saját teljesítménye és
  fittségi szintje tudatos befolyásolására.
\item
  Sokoldalú mozgásműveltségének birtokában eredményesen tanul összetett
  mozgásformákat.
\item
  Ismeri és használja a testnevelés életkorához igazodó elméleti
  ismeretanyagát, szakkifejezéseit, helyes terminológiáját.
\item
  Önismerete, érzelmi-akarati készségei és képességei a testmozgás, a
  testnevelés és a sport eszközei által megfelelően fejlődtek.
\item
  Képes értelmezni az életben adódó baleseti forrásokat és az egészséget
  károsító, veszélyes szokásokat, tevékenységeket.
\item
  A tanult alapvető mozgásformák kombinációiból álló cselekvéssorokat
  változó térbeli, időbeli, dinamikai feltételek mellett készségszinten
  kivitelezi.
\item
  A tanult mozgásforma készségszintű kivitelezése közben fenntartja
  érzelmi-akarati erőfeszítéseit.
\item
  Minden sporttevékenységében forma- és szabálykövető attitűddel
  rendelkezik, ez tevékenységének automatikus részévé válik.
\item
  Nyitott az alapvető mozgásformák újszerű és alternatív környezetben
  történő felhasználására, végrehajtására.
\item
  A motoros képességeinek fejlődési szintje révén képes az összhang
  megteremtésére a cselekvéssorainak elemei között.
\item
  Relatív erejének birtokában képes a sportágspecifikus mozgástechnikák
  koordinált, készségszintű kivitelezésére.
\item
  Az alapvető mozgásainak koordinációjában megfelelő begyakorlottságot
  mutat, és képes a változó környezeti feltételekhez célszerűen
  illeszkedő végrehajtásra.
\item
  A (meg)tanult erő-, gyorsaság-, állóképesség- és ügyességfejlesztő
  eljárásokat tanári irányítással tudatosan alkalmazza.
\item
  Futótechnikája -- összefüggő cselekvéssor részeként -- eltérést mutat
  a vágta- és a tartós futás közben.
\item
  A rajttechnikákat a játékok, a versengések és a versenyek közben
  készségszinten használja.
\item
  Magabiztosan alkalmazza a távol- és magasugrás, valamint a
  kislabdahajítás és súlylökés -- számára megfelelő -- technikáit.
\item
  Segítségadással képes egy-egy általa kiválasztott tornaelem
  bemutatására és a tanult elemekből önállóan alkotott gyakorlatsor
  kivitelezésére.
\item
  A torna, a ritmikus gimnasztika, tánc és aerobik jellegű
  mozgásformákon keresztül tanári irányítás mellett fejleszti
  esztétikai-művészeti tudatosságát és kifejezőképességét.
\item
  A testnevelési és sportjáték közben célszerű, hatékony játék- és
  együttműködési készséget mutat.
\item
  A tanári irányítást követve, a mozgás sebességét növelve hajt végre
  önvédelmi fogásokat, ütéseket, rúgásokat, védéseket és
  ellentámadásokat.
\item
  Ellenőrzött tevékenység keretében rendszeresen mozog, edz, sportol a
  szabad levegőn, egyúttal tudatosan felkészül az időjárás kellemetlen
  hatásainak elviselésére sportolás közben.
\item
  Az elsajátított egy (vagy több) úszásnemben helyes technikával,
  készségszinten úszik.
\item
  A tanult testnevelési és népi játékok mellett folyamatosan, jól
  koordináltan végzi a választott sportjátékokat.
\item
  A sportjátékok előkészítő kisjátékaiban tudatosan és célszerűen
  alkalmazza a technikai és taktikai elemeket.
\item
  A küzdő jellegű feladatokban életkorának megfelelő asszertivitást
  mutatva tudatosan és célszerűen alkalmazza a támadó és védő
  szerepeknek megfelelő technikai és taktikai elemeket.
\item
  A versengések és a versenyek közben toleráns a csapattársaival és az
  ellenfeleivel szemben, ezt tőlük is elvárja.
\item
  A versengések és a versenyek közben közösségformáló, csapatkohéziót
  kialakító játékosként viselkedik.
\item
  Egészséges versenyszellemmel rendelkezik, és tanári irányítás vagy
  ellenőrzés mellett képes a játékvezetésre.
\item
  Megoldást keres a különböző veszély- és baleseti források
  elkerülésére.
\item
  Tanári segítséggel, egyéni képességeihez mérten, tervezetten,
  rendezetten és rendszeresen fejleszti keringési, légzési és
  mozgatórendszerét.
\item
  Tervezetten, rendezetten és rendszeresen végez a biomechanikailag
  helyes testtartás kialakítását elősegítő gyakorlatokat.
\item
  A mindennapi sporttevékenységébe tudatosan beépíti a korrekciós
  gyakorlatokat.
\item
  Tudatosan, összehangoltan végzi a korrekciós gyakorlatait és uszodai
  tevékenységét, azok megvalósítása automatikussá, mindennapi életének
  részévé válik.
\item
  Rendszeresen végez számára megfelelő vízi játékokat, és hajt végre
  úszástechnikákat.
\item
  Ismeri a tanult mozgásformák gerinc- és ízületvédelmi szempontból
  helyes végrehajtását.
\item
  Megnevez és bemutat egyszerű relaxációs gyakorlatokat.
\item
  A családi háttere és a közvetlen környezete adta lehetőségeihez mérten
  tervezetten, rendezetten és rendszeresen végez testmozgást.
\item
  A higiéniai szokások terén teljesen önálló, adott esetben segíti
  társait.
\item
  Az életkorának és alkati paramétereinek megfelelően tervezett,
  rendezett és rendszeres, testmozgással összefüggő táplálkozási
  szokásokat alakít ki.
\item
  A szabadban végzett foglalkozások során nem csupán ügyel környezete
  tisztaságára és rendjére, hanem erre felhívja társai figyelmét is.
\item
  A helyes testtartás egészségre gyakorolt pozitív hatásai ismeretében
  önállóan is kezdeményez ilyen tevékenységet.
\end{itemize}

\hypertarget{evfolyamon-35}{%
\subsubsection{9-12. évfolyamon}\label{evfolyamon-35}}

\begin{itemize}
\item
  Alkotó módon használja a testnevelés életkorához igazodó
  ismeretanyagát, szakkifejezéseit, helyes terminológiáját.
\item
  Megismeri és mindennapjai részévé teszi a mozgáshoz kapcsolódó helyes
  attitűdöket, a fizikailag aktív életmód és a társas-érzelmi jóllét
  élethosszig tartó jótékony hatásait.
\item
  Képes elhárítani a baleseti és veszélyforrásokat, magabiztosan
  segíteni és elsősegélyt nyújtani embertársainak.
\item
  Társas-közösségi kapcsolatai, valamint stressztűrő és -kezelő
  képességei megfelelő szintre fejlődtek.
\item
  Toleráns a testi és más fogyatékossággal élő személyek iránt,
  megismeri és tiszteletben tartja a szexuális kultúra alapelveit,
  elfogadja az egészségügyi szűrések és a környezetvédelem fontosságát.
\item
  A tanult mozgásformákat alkotó módon, a testedzés és a sportolás
  minden területén használja.
\item
  A testedzéshez, a sportoláshoz kívánatosnak tartott jellemzőknek
  megfelelően (fegyelmezetten, határozottan, lelkiismeretesen,
  innovatívan és kezdeményezően) törekszik végrehajtani az elsajátított
  mozgásformákat.
\item
  Sporttevékenységében spontán, automatikus forma- és szabálykövető
  attitűdöt követ.
\item
  Nyitott az alapvető és sportágspecifikus mozgásformák újszerű és
  alternatív környezetben történő felhasználására, végrehajtására.
\item
  Olyan szintű motoros képességekkel rendelkezik, amelyek lehetővé
  teszik a tanult mozgásformák alkotó módon történő végrehajtását.
\item
  Relatív erejének birtokában a tanult mozgásformákat változó környezeti
  feltételek mellett, hatékonyan és készségszinten kivitelezi.
\item
  A különböző sportágspecifikus mozgásformákat változó környezeti
  feltételek mellett, hatékonyan és készségszinten hajtja végre.
\item
  A (meg)tanult erő-, gyorsaság-, állóképesség- és ügyességfejlesztő
  eljárásokat önállóan, tanári ellenőrzés nélkül alkalmazza.
\item
  Tanári ellenőrzés mellett digitálisan méri és értékeli a kondicionális
  és koordinációs képességeinek változásait, ezekből kiindulva felismeri
  saját motoros képességbeli hiányosságait, és ezeket a képességeket
  tudatosan és rendszeresen fejleszti.
\item
  A korábbi évfolyamokon elért eredményeihez képest folyamatosan javítja
  futóteljesítményét, amelyet önmaga is tudatosan nyomon követ.
\item
  A rajtolási módokat a játékok, versenyek, versengések közben
  hatékonyan, kreatívan alkalmazza.
\item
  Képes a kiválasztott ugró- és dobótechnikákat az ilyen jellegű
  játékok, versengések és versenyek közben, az eredményesség érdekében,
  egyéni sajátosságaihoz formálva hatékonyan alkalmazni.
\item
  Önállóan képes az általa kiválasztott elemkapcsolatokból
  tornagyakorlatot összeállítani, majd bemutatni.
\item
  A torna, ritmikus gimnasztika, aerobik és tánc jellegű mozgásformákon
  keresztül fejleszti esztétikai-művészeti tudatosságát és
  kifejezőképességét.
\item
  A zenei ütemnek megfelelően, készségszintű koordinációval végzi a
  kiválasztott ritmikus gimnasztika, illetve aerobik mozgásformákat.
\item
  Önállóan képes az életben adódó, elkerülhetetlen veszélyhelyzetek
  célszerű hárítására.
\item
  A különböző eséstechnikák készségszintű elsajátítása mellett a
  választott küzdősport speciális mozgásformáit célszerűen alkalmazza.
\item
  Rendszeresen mozog, edz, sportol a szabad levegőn, erre −
  lehetőségeihez mérten − társait is motiválja.
\item
  Az elsajátított egy (vagy több) úszásnemben vízbiztosan,
  készségszinten úszik, a természetes vizekben is.
\item
  Önállóan képes az elkerülhetetlen vízi veszélyhelyzetek célszerű
  kezelésére.
\item
  A tanult testnevelési, népi és sportjátékok összetett technikai és
  taktikai elemeit kreatívan, az adott játékhelyzetnek megfelelően,
  célszerűen, készségszinten alkalmazza.
\item
  Játéktevékenységét kreativitást mutató játék- és együttműködési
  készség jellemzi.
\item
  A versengések és a versenyek közben toleráns a csapattársaival és az
  ellenfeleivel szemben, ezt tőlük is elvárja.
\item
  A versengések és a versenyek közben közösségformáló, csapatkohéziót
  kialakító játékosként viselkedik.
\item
  A szabályjátékok alkotó részese, képes szabálykövető játékvezetésre.
\item
  Megoldást keres a különböző veszély- és baleseti források
  elkerülésére, erre társait is motiválja.
\item
  Az egyéni képességeihez mérten, mindennapi szokásrendszerébe építve
  fejleszti keringési, légzési és mozgatórendszerét.
\item
  Belső igénytől vezérelve rendszeresen végez a biomechanikailag helyes
  testtartás kialakítását elősegítő gyakorlatokat.
\item
  Mindennapi tevékenységének tudatos részévé válik a korrekciós
  gyakorlatok végzése.
\item
  A szárazföldi és az uszodai korrekciós gyakorlatait készségszinten
  sajátítja el, azokat tudatosan rögzíti.
\item
  Önállóan, de tanári ellenőrzés mellett végez számára megfelelő uszodai
  tevékenységet.
\item
  A családi háttere és a közvetlen környezete adta lehetőségeihez
  mérten, belső igénytől vezérelve, alkotó módon, rendszeresen végez
  testmozgást.
\item
  Ismer és alkalmaz alapvető relaxációs technikákat.
\item
  Mindennapi életének részeként kezeli a testmozgás, a sportolás közbeni
  higiéniai és tisztálkodási szabályok betartását.
\item
  Az életkorának és alkati paramétereinek megfelelő pozitív,
  egészégtudatos, testmozgással összefüggő táplálkozási szokásokat
  alakít ki.
\item
  A szabadban végzett foglalkozások során nem csupán ügyel környezete
  tisztaságára és rendjére, hanem erre felhívja társai figyelmét is.
\item
  Megoldást keres a testtartási rendellenesség kialakulásának
  megakadályozására, erre társait is motiválja.
\end{itemize}

\hypertarget{tortenelem}{%
\subsection{Történelem}\label{tortenelem}}

\hypertarget{evfolyamon-36}{%
\subsubsection{5-8. évfolyamon}\label{evfolyamon-36}}

\begin{itemize}
\item
  Alapvető ismeretekkel rendelkezik a magyar nemzet, magyarország és az
  európai civilizáció és földünk legfontosabb régióinak múltjáról és
  jelenéről.
\item
  Képes a múlt és jelen alapvető társadalmi, gazdasági, politikai és
  kulturális folyamatairól, jelenségeiről véleményt alkotni.
\item
  Megérti, hogy minden történelmi eseménynek és folyamatnak okai és
  következményei vannak.
\item
  Ismeri a közös magyar nemzeti, az európai, valamint az egyetemes
  emberi civilizáció kulturális örökségének, kódrendszerének
  legalapvetőbb elemeit.
\item
  Különbséget tud tenni a múltról szóló fiktív történetek és a
  történelmi tények között.
\item
  Megérti és méltányolja, hogy a múlt viszonyai, az emberek
  gondolkodása, értékítélete eltért a maitól.
\item
  Alapvető ismereteket szerez a demokratikus államszervezetről, a
  társadalmi együttműködés szabályairól és a piacgazdaságról.
\item
  Kialakul a múlt iránti érdeklődés.
\item
  Megerősödik benne a nemzeti identitás és hazaszeretet érzése; büszke
  népe múltjára, ápolja hagyományait, és méltón emlékezik meg hazája
  nagyjairól.
\item
  Kialakulnak az európai civilizációs identitás alapelemei.
\item
  Kialakulnak a társadalmi felelősség és normakövetés, az egyéni
  kezdeményezőkészség és a hazája, közösségei és embertársai iránti
  felelősségvállalás, az aktív állampolgárság, valamint a demokratikus
  elkötelezettség alapelemei.
\item
  Ismeri és fel tudja idézni a magyar és az európai történelmi
  hagyományhoz kapcsolódó legfontosabb mítoszokat, mondákat,
  történeteket, elbeszéléseket.
\item
  Be tudja mutatni a különböző korok életmódjának és kultúrájának főbb
  vonásait és az egyes történelmi korszakokban élt emberek életét
  befolyásoló tényezőket.
\item
  Tisztában van a kereszténység kialakulásának főbb állomásaival, ismeri
  a legfontosabb tanításait és hatását az európai civilizációra és
  magyarországra.
\item
  Ismeri a magyar történelem kiemelkedő alakjait, cselekedeteiket,
  illetve szerepüket a magyar nemzet történetében.
\item
  Fel tudja idézni a magyar történelem legfontosabb eseményeit,
  jelenségeit, folyamatait és fordulópontjait a honfoglalástól
  napjainkig.
\item
  Képes felidézni a magyar nemzet honvédő és szabadságharcait, példákat
  hoz a hazaszeretet, önfeláldozás és hősiesség megnyilvánulásaira.
\item
  Tisztában van a középkor és újkor világképének fő vonásaival, a 19. és
  20. század fontosabb politikai eszméivel és azok hatásaival.
\item
  Ismeri és be tudja mutatni a 19. és 20. századi modernizáció
  gazdasági, társadalmi és kulturális hatásait magyarországon és a
  világban.
\item
  Ismeri a különböző korok hadviselési szokásait, az első és a második
  világháború legfontosabb eseményeit, jellemzőit, valamint napjainkra
  is hatással bíró következményeit.
\item
  Fel tudja idézni az első és második világháború borzalmait, érveket
  tud felsorakoztatni a békére való törekvés mellett.
\item
  Ismeri a nemzetiszocialista és a kommunista diktatúrák főbb
  jellemzőit, az emberiség ellen elkövetett bűneiket, ellentmondásaikat
  és ezek következményeit, továbbá a velük szembeni ellenállás példáit.
\item
  Felismeri a különbségeket a demokratikus és a diktatórikus
  berendezkedések között, érvel a demokrácia értékei mellett.
\item
  Példákat tud felhozni arra, hogy a történelem során miként járultak
  hozzá a magyarok európa és a világ kulturális, tudományos és politikai
  fejlődéséhez.
\item
  Ismeri a magyarság, illetve a kárpát-medence népei együttélésének
  jellemzőit néhány történelmi korszakban, beleértve a határon kívüli
  magyarság sorsát, megmaradásáért folytatott küzdelmét, példákat hoz a
  magyar nemzet és a közép-európai régió népeinek kapcsolatára és
  együttműködésére.
\item
  Valós képet alkotva képes elhelyezni magyarországot a globális
  folyamatokban a történelem során és napjainkban.
\item
  Ismeri hazája államszervezetét.
\item
  Képes ismereteket szerezni személyes beszélgetésekből, tárgyak,
  épületek megfigyeléséből, olvasott és hallott, valamint a különböző
  médiumok által felkínált szöveges és képi anyagokból.
\item
  Kiemel lényeges információkat (kulcsszavakat, tételmondatokat)
  elbeszélő vagy leíró, illetve rövidebb magyarázó írott és hallott
  szövegekből és az ezek alapján megfogalmazott kérdésekre egyszerű
  válaszokat adni.
\item
  Megadott szempontok alapján, tanári útmutatás segítségével történelmi
  információkat gyűjt különböző médiumokból és forrásokból (könyvek,
  atlaszok, kronológiák, könyvtárak, múzeumok, médiatárak, filmek;
  nyomtatott és digitális, szöveges és vizuális források).
\item
  Képes élethelyzetek, magatartásformák megfigyelése által értelmezni
  azokat.
\item
  Megadott szempontok alapján tudja értelmezni és rendszerezni a
  történelmi információkat.
\item
  Felismeri, hogy melyik szöveg, kép, egyszerű ábra, grafikon vagy
  diagram kapcsolódik az adott történelmi témához.
\item
  Képen, egyszerű ábrán, grafikonon, diagramon ábrázolt folyamatot,
  jelenséget saját szavaival le tud írni.
\item
  Képes egyszerű esetekben forráskritikát végezni, valamint különbséget
  tenni források között típus és szövegösszefüggés alapján.
\item
  Össze tudja vetni a forrásokban található információkat az
  ismereteivel, párhuzamot tud vonni különböző típusú (pl. szöveges és
  képi) történelmi források tartalma között.
\item
  Meg tudja vizsgálni, hogy a történet szerzője résztvevője vagy
  kortársa volt-e az eseményeknek.
\item
  Egyszerű következtetéseket von le, és véleményt tud alkotni különböző
  források hitelességéről és releváns voltáról.
\item
  Ismeri a nagy történelmi korszakok elnevezését és időhatárait, néhány
  kiemelten fontos esemény, jelenség és történelmi folyamat időpontját.
\item
  Biztonsággal használja az idő tagolására szolgáló kifejezéseket,
  történelmi eseményre, jelenségre, folyamatra, korszakra való utalással
  végez időmeghatározást.
\item
  Ismeretei segítségével időrendbe tud állítani történelmi eseményeket,
  képes az idő ábrázolására pl. időszalag segítségével.
\item
  A tanult történelmi eseményeket, jelenségeket, személyeket, ikonikus
  szimbólumokat, tárgyakat, képeket hozzá tudja rendelni egy adott
  történelmi korhoz, régióhoz, államhoz.
\item
  Biztonsággal használ különböző történelmi térképeket a fontosabb
  történelmi események helyszíneinek azonosítására, egyszerű jelenségek,
  folyamatok leolvasására, értelmezésére, vaktérképen való
  elhelyezésére.
\item
  Egyszerű alaprajzokat, modelleket, térképvázlatokat (pl. települések,
  épületek, csaták) tervez és készít.
\item
  Önállóan, folyamatos beszéddel képes eseményeket, történeteket
  elmondani, történelmi személyeket bemutatni, saját véleményt
  megfogalmazni.
\item
  Össze tudja foglalni saját szavaival hosszabb elbeszélő vagy leíró,
  valamint rövidebb magyarázó szövegek tartalmát.
\item
  Az általa gyűjtött történelmi adatokból, szövegekből rövid tartalmi
  ismertetőt tud készíteni.
\item
  Képes önálló kérdések megfogalmazására a tárgyalt történelmi témával,
  eseményekkel, folyamatokkal kapcsolatban.
\item
  Képes rövid fogalmazások készítésére egy-egy történetről, történelmi
  témáról.
\item
  Különböző történelmi korszakok, történelmi és társadalmi kérdések
  tárgyalása során szakszerűen alkalmazza az értelmező és tartalmi
  kulcsfogalmakat, továbbá használja a témához kapcsolódó történelmi
  fogalmakat.
\item
  Tud egyszerű vizuális rendezőket készíteni és kiegészíteni hagyományos
  vagy digitális módon (táblázatok, ábrák, tablók, rajzok, vázlatok) egy
  történelmi témáról.
\item
  Egyszerű történelmi témáról tanári útmutatás segítségével kiselőadást
  és digitális prezentációt állít össze és mutat be.
\item
  Egyszerű történelmi kérdésekről tárgyilagos véleményt tud
  megfogalmazni, állításait alátámasztja.
\item
  Meghallgatja és megérti -- adott esetben elfogadja -- mások
  véleményét, érveit.
\item
  Tanári segítséggel dramatikusan, szerepjáték formájában tud
  megjeleníteni történelmi eseményeket, jelenségeket, személyiségeket.
\item
  Adott történetben különbséget tud tenni fiktív és valós, irreális és
  reális elemek között.
\item
  Képes megfigyelni, értelmezni és összehasonlítani a történelemben
  előforduló különböző emberi magatartásformákat és élethelyzeteket.
\item
  A történelmi eseményekkel, folyamatokkal és személyekkel kapcsolatban
  önálló kérdéseket fogalmaz meg.
\item
  Feltételezéseket fogalmaz meg történelmi személyek cselekedeteinek
  mozgatórugóiról, és adatokkal, érvekkel alátámasztja azokat.
\item
  A történelmi szereplők megnyilvánulásainak szándékot tulajdonít,
  álláspontjukat azonosítja.
\item
  Önálló véleményt képes megfogalmazni történelmi szereplőkről,
  eseményekről, folyamatokról.
\item
  Felismeri és értékeli a különböző korokra és régiókra jellemző
  tárgyakat, alkotásokat, életmódokat, szokásokat, változásokat, képes
  azokat összehasonlítani egymással, illetve a mai korral.
\item
  Társadalmi és erkölcsi problémákat azonosít adott történetek,
  történelmi események, különböző korok szokásai alapján.
\item
  Példákat hoz a történelmi jelenségekre, folyamatokra.
\item
  Feltételezéseket fogalmaz meg néhány fontos történelmi esemény és
  folyamat feltételeiről, okairól és következményeiről, és tényekkel
  alátámasztja azokat.
\item
  Több szempontból képes megkülönböztetni a történelmi jelenségek és
  események okait és következményeit (pl. hosszú vagy rövid távú,
  gazdasági, társadalmi vagy politikai).
\item
  Felismeri, hogy az emberi cselekedet és annak következménye között
  szoros kapcsolat van.
\end{itemize}

\hypertarget{evfolyamon-37}{%
\subsubsection{9-12. évfolyamon}\label{evfolyamon-37}}

\begin{itemize}
\item
  Megbízható ismeretekkel bír az európai, valamint az egyetemes
  történelem és mélyebb tudással rendelkezik a magyar történelem
  fontosabb eseményeiről, történelmi folyamatairól, fordulópontjairól.
\item
  Képes a múlt és jelen társadalmi, gazdasági, politikai és kulturális
  folyamatairól, jelenségeiről többszempontú, tárgyilagos érveléssel
  alátámasztott véleményt alkotni, ezekkel kapcsolatos problémákat
  megfogalmazni.
\item
  Ismeri a közös magyar nemzeti és európai, valamint az egyetemes emberi
  civilizáció kulturális örökségének, kódrendszerének lényeges elemeit.
\item
  Különbséget tud tenni történelmi tények és történelmi interpretáció,
  illetve vélemény között.
\item
  Képes következtetni történelmi események, folyamatok és jelenségek
  okaira és következményeire.
\item
  Képes a tanulási célhoz megfelelő információforrást választani, a
  források között szelektálni, azokat szakszerűen feldolgozni és
  értelmezni.
\item
  Kialakul a hiteles és tárgyilagos forráshasználat és kritika igénye.
\item
  Képes a múlt eseményeit és jelenségeit a saját történelmi
  összefüggésükben értelmezni, illetve a jelen viszonyait kapcsolatba
  hozni a múltban történtekkel.
\item
  Ismeri a demokratikus államszervezet működését, a társadalmi
  együttműködés szabályait, a piacgazdaság alapelveit; autonóm és
  felelős állampolgárként viselkedik.
\item
  Kialakul és megerősödik a történelmi múlt, illetve a társadalmi,
  politikai, gazdasági és kulturális kérdések iránti érdeklődés.
\item
  Kialakulnak a saját értékrend és történelemszemlélet alapjai.
\item
  Elmélyül a nemzeti identitás és hazaszeretet, büszke népe múltjára,
  ápolja hagyományait, és méltón emlékezik meg hazája nagyjairól.
\item
  Megerősödnek az európai civilizációs identitás alapelemei.
\item
  Megerősödik és elmélyül a társadalmi felelősség és normakövetés, az
  egyéni kezdeményezőkészség, a hazája, közösségei és embertársai iránt
  való felelősségvállalás, valamint a demokratikus elkötelezettség.
\item
  Ismeri az ókori civilizációk legfontosabb jellemzőit, valamint az
  athéni demokrácia és a római állam működését, hatásukat az európai
  civilizációra.
\item
  Felidézi a monoteista vallások kialakulását, legfontosabb
  jellemzőiket, tanításaik főbb elemeit, és bemutatja terjedésüket.
\item
  Bemutatja a keresztény vallás civilizációformáló hatását, a középkori
  egyházat, valamint a reformáció és a katolikus megújulás folyamatát és
  kulturális hatásait; érvel a vallási türelem, illetve a
  vallásszabadság mellett.
\item
  Képes felidézni a középkor gazdasági és kulturális jellemzőit,
  világképét, a kor meghatározó birodalmait és bemutatni a rendi
  társadalmat.
\item
  Ismeri a magyar nép őstörténetére és a honfoglalásra vonatkozó
  tudományos elképzeléseket és tényeket, tisztában van legfőbb vitatott
  kérdéseivel, a különböző tudományterületek kutatásainak főbb
  eredményeivel.
\item
  Értékeli az államalapítás, valamint a kereszténység felvételének
  jelentőségét.
\item
  Felidézi a középkori magyar állam történetének fordulópontjait,
  legfontosabb uralkodóink tetteit.
\item
  Ismeri a magyarság törökellenes küzdelmeit, fordulópontjait és hőseit;
  felismeri, hogy a magyar és az európai történelem alakulását
  meghatározóan befolyásolta a török megszállás.
\item
  Be tudja mutatni a kora újkor fő gazdasági és társadalmi folyamatait,
  ismeri a felvilágosodás eszméit, illetve azok kulturális és politikai
  hatását, valamint véleményt formál a francia forradalom európai
  hatásáról.
\item
  Összefüggéseiben és folyamatában fel tudja idézni, miként hatott a
  magyar történelemre a habsburg birodalomhoz való tartozás, bemutatja
  az együttműködés és konfrontáció megnyilvánulásait, a függetlenségi
  törekvéseket és értékeli a rákóczi-szabadságharc jelentőségét.
\item
  Ismeri és értékeli a magyar nemzetnek a polgári átalakulás és nemzeti
  függetlenség elérésére tett erőfeszítéseit a reformkor, az 1848/49-es
  forradalom és szabadságharc, illetve az azt követő időszakban; a kor
  kiemelkedő magyar politikusait és azok nézeteit, véleményt tud
  formálni a kiegyezésről.
\item
  Fel tudja idézni az ipari forradalom szakaszait, illetve azok
  gazdasági, társadalmi, kulturális és politikai hatásait; képes
  bemutatni a modern polgári társadalom és állam jellemzőit és a 19.
  század főbb politikai eszméit, valamint felismeri a hasonlóságot és
  különbséget azok mai formái között.
\item
  Fel tudja idézni az első világháború előzményeit, a háború jellemzőit
  és fontosabb fordulópontjait, értékeli a háborúkat lezáró békék
  tartalmát, és felismeri a háborúnak a 20. század egészére gyakorolt
  hatását.
\item
  Bemutatja az első világháború magyar vonatkozásait, a háborús vereség
  következményeit; példákat tud hozni a háborús helytállásra.
\item
  Képes felidézni azokat az okokat és körülményeket, amelyek a
  történelmi magyarország felbomlásához vezettek.
\item
  Tisztában van a trianoni békediktátum tartalmával és
  következményeivel, be tudja mutatni az ország talpra állását, a
  horthy-korszak politikai, gazdasági, társadalmi és kulturális
  viszonyait, felismeri a magyar külpolitika mozgásterének
  korlátozottságát.
\item
  Össze tudja hasonlítani a nemzetiszocialista és a kommunista
  ideológiát és diktatúrát, példák segítségével bemutatja a rendszerek
  embertelenségét és a velük szembeni ellenállás formáit.
\item
  Képes felidézni a második világháború okait, a háború jellemzőit és
  fontosabb fordulópontjait, ismeri a holokausztot és a hozzávezető
  okokat.
\item
  Bemutatja magyarország revíziós lépéseit, a háborús részvételét, az
  ország német megszállását, a magyar zsidóság tragédiáját, a szovjet
  megszállást, a polgári lakosság szenvedését, a hadifoglyok embertelen
  sorsát.
\item
  Össze tudja hasonlítani a nyugati demokratikus világ és a kommunista
  szovjet blokk politikai és társadalmi berendezkedését, képes
  jellemezni a hidegháború időszakát, bemutatni a gyarmati rendszer
  felbomlását és az európai kommunista rendszerek összeomlását.
\item
  Bemutatja a kommunista diktatúra magyarországi kiépítését, működését
  és változatait, az 1956-os forradalom és szabadságharc okait,
  eseményeit és hőseit, összefüggéseiben szemléli a rendszerváltoztatás
  folyamatát, felismerve annak történelmi jelentőségét.
\item
  Bemutatja a gyarmati rendszer felbomlásának következményeit, india,
  kína és a közel-keleti régió helyzetét és jelentőségét.
\item
  Ismeri és reálisan látja a többpólusú világ jellemzőit napjainkban,
  elhelyezi magyarországot a globális világ folyamataiban.
\item
  Bemutatja a határon túli magyarság helyzetét, a megmaradásért való
  küzdelmét trianontól napjainkig.
\item
  Ismeri a magyar cigányság történetének főbb állomásait, bemutatja
  jelenkori helyzetét.
\item
  Ismeri a magyarság, illetve a kárpát-medence népei együttélésének
  jellemzőit, példákat hoz a magyar nemzet és a közép-európai régió
  népeinek kapcsolatára, különös tekintettel a visegrádi
  együttműködésre.
\item
  Ismeri hazája államszervezetét, választási rendszerét.
\item
  Önállóan tud használni általános és történelmi, nyomtatott és
  digitális információforrásokat (tankönyv, kézikönyvek, szakkönyvek,
  lexikonok, képzőművészeti alkotások, könyvtár és egyéb adatbázisok,
  filmek, keresők).
\item
  Önállóan információkat tud gyűjteni, áttekinteni, rendszerezni és
  értelmezni különböző médiumokból és írásos vagy képi forrásokból,
  statisztikákból, diagramokból, térképekről, nyomtatott és digitális
  felületekről.
\item
  Tud forráskritikát végezni és különbséget tenni a források között
  hitelesség, típus és szövegösszefüggés alapján.
\item
  Képes azonosítani a különböző források szerzőinek a szándékát,
  bizonyítékok alapján értékeli egy forrás hitelességét.
\item
  Képes a szándékainak megfelelő információkat kiválasztani különböző
  műfajú forrásokból.
\item
  Összehasonlítja a forrásokban talált információkat saját ismereteivel,
  illetve más források információival és megmagyarázza az eltérések
  okait.
\item
  Képes kiválasztani a megfelelő forrást valamely történelmi állítás,
  vélemény alátámasztására vagy cáfolására.
\item
  Ismeri a magyar és az európai történelem tanult történelmi korszakait,
  időszakait, és képes azokat időben és térben elhelyezni.
\item
  Az egyes események, folyamatok idejét konkrét történelmi korhoz,
  időszakhoz kapcsolja vagy viszonyítja, ismeri néhány kiemelten fontos
  esemény, jelenség időpontját, kronológiát használ és készít.
\item
  Össze tudja hasonlítani megadott szempontok alapján az egyes
  történelmi korszakok, időszakok jellegzetességeit az egyetemes és a
  magyar történelem egymáshoz kapcsolódó eseményeit.
\item
  Képes azonosítani a tanult egyetemes és magyar történelmi
  személyiségek közül a kortársakat.
\item
  Felismeri, hogy a magyar történelem az európai történelem része, és
  példákat tud hozni a magyar és európai történelem kölcsönhatásaira.
\item
  Egyszerű történelmi térképvázlatot alkot hagyományos és digitális
  eljárással.
\item
  A földrajzi környezet és a történeti folyamatok összefüggéseit
  példákkal képes alátámasztani.
\item
  Képes különböző időszakok történelmi térképeinek összehasonlítására, a
  történelmi tér változásainak és a történelmi mozgások követésére
  megadott szempontok alapján a változások hátterének feltárásával.
\item
  Képes a történelmi jelenségeket általános és konkrét történelmi
  fogalmak, tartalmi és értelmező kulcsfogalmak felhasználásával
  értelmezni és értékelni.
\item
  Fel tud ismerni fontosabb történelmi fogalmakat meghatározás alapján.
\item
  Képes kiválasztani, rendezni és alkalmazni az azonos korhoz, témához
  kapcsolható fogalmakat.
\item
  Össze tudja foglalni rövid és egyszerű szaktudományos szöveg
  tartalmát.
\item
  Képes önállóan vázlatot készíteni és jegyzetelni.
\item
  Képes egy-egy korszakot átfogó módon bemutatni.
\item
  Történelmi témáról kiselőadást, digitális prezentációt alkot és mutat
  be.
\item
  Történelmi tárgyú folyamatábrákat, digitális táblázatokat, diagramokat
  készít, történelmi, gazdasági, társadalmi és politikai modelleket
  vizuálisan is meg tud jeleníteni.
\item
  Megadott szempontok alapján történelmi tárgyú szerkesztett szöveget
  (esszét) tud alkotni, amelynek során tételmondatokat fogalmaz meg,
  állításait több szempontból indokolja és következtetéseket von le.
\item
  Társaival képes megvitatni történelmi kérdéseket, amelynek során
  bizonyítékokon alapuló érvekkel megindokolja a véleményét, és
  választékosan reflektál mások véleményére, árnyalja saját
  álláspontját.
\item
  Képes felismerni, megfogalmazni és összehasonlítani különböző
  társadalmi és történelmi problémákat, értékrendeket, jelenségeket,
  folyamatokat.
\item
  A tanult ismereteket problémaközpontúan tudja rendezni.
\item
  Hipotéziseket alkot történelmi személyek, társadalmi csoportok és
  intézmények viselkedésének mozgatórugóiról.
\item
  Önálló kérdéseket fogalmaz meg történelmi folyamatok, jelenségek és
  események feltételeiről, okairól és következményeiről.
\item
  Önálló véleményt tud alkotni történelmi eseményekről, folyamatokról,
  jelenségekről és személyekről.
\item
  Képes különböző élethelyzetek, magatartásformák megfigyelése által
  következtetések levonására, erkölcsi kérdéseket is felvető történelmi
  helyzetek felismerésére és megítélésére.
\item
  A változás és a fejlődés fogalma közötti különbséget ismerve képes
  felismerni és bemutatni azokat azonos korszakon belül, vagy azokon
  átívelően.
\item
  Képes összevetni, csoportosítani és súlyozni az egyes történelmi
  folyamatok, jelenségek, események okait, következményeit, és ítéletet
  alkotni azokról, valamint a benne résztvevők szándékairól.
\item
  Összehasonlít különböző, egymáshoz hasonló történeti helyzeteket,
  folyamatokat, jelenségeket.
\item
  Képes felismerni konkrét történelmi helyzetekben, jelenségekben és
  folyamatokban valamely általános szabályszerűség érvényesülését.
\item
  Összehasonlítja és kritikusan értékeli az egyes történelmi
  folyamatokkal, eseményekkel és személyekkel kapcsolatos eltérő
  álláspontokat.
\item
  Feltevéseket fogalmaz meg, azok mellett érveket gyűjt, illetve
  mérlegeli az ellenérveket.
\item
  Felismeri, hogy a jelen társadalmi, gazdasági, politikai és kulturális
  viszonyai a múltbeli események, tényezők következményeiként alakultak
  ki.
\end{itemize}

\hypertarget{vizualis-kultura}{%
\subsection{Vizuális kultúra}\label{vizualis-kultura}}

\hypertarget{evfolyamon-38}{%
\subsubsection{1-4. évfolyamon}\label{evfolyamon-38}}

\begin{itemize}
\item
  Alkotó tevékenység közben bátran kísérletezik.
\item
  Különböző eszközöket rendeltetésszerűen használ.
\item
  Megérti és végrehajtja a feladatokat.
\item
  Példák alapján különbséget tesz a hétköznapi és a művészi között.
\item
  Gondolatait vizuálisan is érthetően megmagyarázza.
\item
  Példák alapján azonosítja a médiafogyasztás mindennapi jelenségeit.
\item
  Önálló döntéseket hoz a környezet alakításának szempontjából.
\item
  Csoportban végzett feladatmegoldás során részt vállal a
  feladatmegoldásban, és figyelembe veszi társai álláspontját.
\item
  Csoportban végzett feladatmegoldás közben képes érvelésre.
\item
  Feladatmegoldás során betartja az előre ismertetett szabályokat.
\item
  Egyszerű, begyakorolt feladatokat önállóan is elvégez.
\item
  Véleményét önállóan megfogalmazza.
\item
  Érzékszervi tapasztalatokat pontosan megfogalmaz mérettel, formával,
  színnel, felülettel, illattal, hanggal, mozgással kapcsolatban.
\item
  Közvetlen vizuális megfigyeléssel leolvasható egyszerű jellemzők
  alapján vizuális jelenségeket, képeket méret, irány, elhelyezkedés,
  mennyiség, szín szempontjából azonosít, kiválaszt, rendez, szövegesen
  pontosan leír és összehasonlít.
\item
  Látványt, vizuális jelenségeket, képeket viszonylagos pontossággal
  emlékezetből azonosít, kiválaszt, megnevez, különböző szempontok
  szerint rendez.
\item
  Vizuális jelenségeket, egyszerű látott képi elemeket különböző
  vizuális eszközökkel megjelenít: rajzol, fest, nyomtat, formáz, épít.
\item
  Élmények, elképzelt vagy hallott történetek, szövegek részleteit
  különböző vizuális eszközökkel egyszerűen megjeleníti: rajzol, fest,
  nyomtat, fotóz, formáz, épít.
\item
  Rövid szövegekhez, egyéb tananyagtartalmakhoz síkbeli és térbeli
  vizuális illusztrációt készít különböző vizuális eszközökkel: rajzol,
  fest, nyomtat, fotóz, formáz, épít és a képet, tárgyat szövegesen
  értelmezi.
\item
  Egyszerű eszközökkel és anyagokból elképzelt teret rendez, alakít,
  egyszerű makettet készít egyénileg vagy csoportmunkában, és az
  elképzelést szövegesen is bemutatja, magyarázza.
\item
  Elképzelt történeteket, irodalmi alkotásokat bemutat, dramatizál,
  ehhez egyszerű eszközöket: bábot, teret, díszletet, kelléket, egyszerű
  jelmezt készít csoportmunkában, és élményeit szövegesen megfogalmazza.
\item
  Saját és társai vizuális munkáit szövegesen értelmezi, kiegészíti,
  magyarázza.
\item
  Saját munkákat, képeket, műalkotásokat, mozgóképi részleteket
  szereplők karaktere, szín-, fényhatás, kompozíció, kifejezőerő
  szempontjából szövegesen elemez, összehasonlít.
\item
  Képek, műalkotások, mozgóképi közlések megtekintése után önállóan
  megfogalmazza és indokolja tetszésítéletét.
\item
  Képek, műalkotások, mozgóképi közlések megtekintése után adott
  szempontok szerint következtetést fogalmaz meg, megállapításait
  társaival is megvitatja.
\item
  Különböző alakzatokat, motívumokat, egyszerű vizuális megjelenéseket
  látvány alapján, különböző vizuális eszközökkel, viszonylagos
  pontossággal megjelenít: rajzol, fest, nyomtat, formáz, épít.
\item
  Adott cél érdekében fotót vagy rövid mozgóképet készít.
\item
  Alkalmazza az egyszerű tárgykészítés legfontosabb technikáit: vág,
  ragaszt, tűz, varr, kötöz, fűz, mintáz.
\item
  Adott cél érdekében alkalmazza a térbeli formaalkotás különböző
  technikáit egyénileg és csoportmunkában.
\item
  Gyűjtött természeti vagy mesterséges formák egyszerűsítésével, vagy a
  magyar díszítőművészet általa megismert mintakincsének
  felhasználásával mintát tervez.
\item
  A tanulás során szerzett tapasztalatait, saját céljait, gondolatait
  vizuális megjelenítés segítségével magyarázza, illusztrálja egyénileg
  és csoportmunkában.
\item
  Saját és mások érzelmeit, hangulatait segítséggel megfogalmazza és
  egyszerű dramatikus eszközökkel eljátssza, vizuális eszközökkel
  megjeleníti.
\item
  Korábban átélt eseményeket, tapasztalatokat, élményeket különböző
  vizuális eszközökkel, élményszerűen megjelenít: rajzol, fest, nyomtat,
  formáz, épít, fotóz és magyarázza azt.
\item
  Valós vagy digitális játékélményeit vizuálisan és dramatikusan
  feldolgozza: rajzol, fest, formáz, nyomtat, eljátszik, elmesél.
\item
  A vizuális nyelv elemeinek értelmezésével és használatával
  kísérletezik.
\item
  Az adott életkornak megfelelő rövid mozgóképi közléseket segítséggel
  elemez a vizuális kifejezőeszközök használatának tudatosítása
  érdekében.
\item
  Egyszerű, mindennapok során használt jeleket felismer.
\item
  Pontosan ismeri államcímerünk és nemzeti zászlónk felépítését,
  összetevőit, színeit.
\item
  Az adott életkornak megfelelő tájékoztatást, meggyőzést,
  figyelemfelkeltést szolgáló, célzottan kommunikációs szándékú vizuális
  közléseket segítséggel értelmez.
\item
  Azonosítja a gyerekeknek szóló vagy fogyasztásra ösztönző, célzottan
  kommunikációs szándékú vizuális közléseket.
\item
  Azonosítja a nonverbális kommunikáció eszközeit: mimika, gesztus,
  \textgreater{} ezzel kapcsolatos tapasztalatait közlési és kifejezési
  \textgreater{} helyzetekben használja.
\item
  Adott cél érdekében egyszerű vizuális kommunikációt szolgáló
  \textgreater{} megjelenéseket -- jel, meghívó, plakát -- készít
  egyénileg vagy \textgreater{} csoportmunkában.
\item
  Saját kommunikációs célból egyszerű térbeli tájékozódást segítő
  \textgreater{} ábrát -- alaprajz, térkép -- készít.
\item
  Időbeli történéseket egyszerű vizuális eszközökkel, segítséggel
  \textgreater{} megjelenít.
\item
  Saját történetet alkot, és azt vizuális eszközökkel is tetszőlegesen
  \textgreater{} megjeleníti.
\item
  Adott álló- vagy mozgóképi megjelenéseket egyéni elképzelés szerint
  \textgreater{} átértelmez.
\item
  Különböző egyszerű anyagokkal kísérletezik, szabadon épít, saját
  \textgreater{} célok érdekében konstruál.
\end{itemize}

\hypertarget{evfolyamon-39}{%
\subsubsection{5-8. évfolyamon}\label{evfolyamon-39}}

\begin{itemize}
\item
  Alkotó tevékenység közben önállóan kísérletezik, különböző megoldási
  utakat keres.
\item
  Eszközhasználata begyakorlott, szakszerű.
\item
  Anyaghasználata gazdaságos, átgondolt.
\item
  Feladatokat önállóan megold, eredményeit érthetően bemutatja.
\item
  Adott tanulási helyzetben a tanulási előrehaladás érdekében adekvát
  kérdést tesz fel.
\item
  Feladatmegoldásai során felidézi és alkalmazza a korábban szerzett
  ismereteket, illetve kapcsolódó információkat keres.
\item
  Felismeri a vizuális művészeti műfajok példáit.
\item
  Használja és megkülönbözteti a különböző vizuális médiumok
  kifejezőeszközeit.
\item
  Megkülönböztet művészettörténeti korszakokat, stílusokat.
\item
  Felismeri az egyes témakörök szemléltetésére használt műalkotásokat,
  alkotókat az ajánlott képanyag alapján.
\item
  Használja a térbeli és az időbeli viszonyok megjelenítésének különböző
  vizuális lehetőségeit.
\item
  Példák alapján megérti a képmanipuláció és befolyásolás
  összefüggéseit.
\item
  Példák alapján meg tudja magyarázni a tervezett, épített környezet és
  a funkció összefüggéseit.
\item
  Érti a magyar és egyetemes kulturális örökség és hagyomány
  jelentőségét.
\item
  Csoportban végzett feladatmegoldás során részt vállal a
  feladatmegoldásban, önállóan megtalálja saját feladatát, és figyelembe
  veszi társai álláspontját is.
\item
  Feladatmegoldás során megállapodásokat köt, szabályt alkot, és
  betartja a közösen meghatározott feltételeket.
\item
  Önállóan véleményt alkot, és azt röviden indokolja.
\item
  Látványt, vizuális jelenségeket, műalkotásokat önállóan is pontosan,
  részletgazdagon szövegesen jellemez, bemutat.
\item
  Látványok, vizuális jelenségek, alkotások lényeges, egyedi jellemzőit
  kiemeli, bemutatja.
\item
  Alkotómunka során felhasználja a már látott képi inspirációkat.
\item
  Adott témával, feladattal kapcsolatos vizuális információkat és képi
  inspirációkat keres többféle forrásból.
\item
  Megfigyeléseit, tapasztalatait, gondolatait vizuálisan rögzíti, mások
  számára érthető vázlatot készít.
\item
  Különböző érzetek kapcsán belső képeinek, képzeteinek megfigyelésével
  tapasztalatait vizuálisan megjeleníti.
\item
  Elvont fogalmakat, művészeti tartalmakat belső képek
  összekapcsolásával bemutat, magyaráz és különböző vizuális eszközökkel
  megjelenít.
\item
  Adott tartalmi keretek figyelembevételével karaktereket, tereket,
  tárgyakat, helyzeteket, történeteket részletesen elképzel, fogalmi és
  vizuális eszközökkel bemutat és megjelenít, egyénileg és
  csoportmunkában is.
\item
  A valóság vagy a vizuális alkotások, illetve azok elemei által
  felidézett asszociatív módon generált képeket, történeteket szövegesen
  megfogalmaz, vizuálisan megjelenít, egyénileg és csoportmunkában is.
\item
  Szöveges vagy egyszerű képi inspiráció alapján elképzeli és
  megjeleníti a látványt, egyénileg és csoportmunkában is.
\item
  Látványok, képek részeinek, részleteinek alapján elképzeli a látvány
  egészét, fogalmi és vizuális eszközökkel bemutatja és megjeleníti,
  rekonstruálja azt.
\item
  A látványokkal kapcsolatos objektív és szubjektív észrevételeket
  pontosan szétválasztja.
\item
  Vizuális problémák vizsgálata során összegyűjtött információkat,
  gondolatokat különböző szempontok szerint rendez és összehasonlít, a
  tapasztalatait különböző helyzetekben a megoldás érdekében
  felhasználja.
\item
  Vizuális megjelenések, képek, mozgóképek, médiaszövegek vizsgálata,
  összehasonlítása során feltárt következtetéseit megfogalmazza, és
  alkotó tevékenységében, egyénileg és csoportmunkában is felhasználja.
\item
  Különböző korok és kultúrák szimbólumai és motívumai közül adott cél
  érdekében gyűjtést végez, és alkotó tevékenységében felhasználja a
  gyűjtés eredményeit.
\item
  Különböző művészettörténeti korokban, stílusokban készült alkotásokat,
  építményeket összehasonlít, megkülönböztet és összekapcsol más
  jelenségekkel, fogalmakkal, alkotásokkal, melyek segítségével
  alkotótevékenysége során újrafogalmazza a látványt.
\item
  Adott koncepció figyelembevételével, tudatos anyag- és
  eszközhasználattal tárgyakat, tereket tervez és hoz létre, egyénileg
  vagy csoportmunkában is.
\item
  Adott cél és szempontok figyelembevételével térbeli, időbeli
  viszonyokat, változásokat, eseményeket, történeteket egyénileg és
  csoportmunkában is rögzít, megjelenít.
\item
  Ismeri a térábrázolás alapvető módszereit (pl. axonometria,
  perspektíva) és azokat alkotómunkájában felhasználja.
\item
  Gondolatait, terveit, észrevételeit, véleményét változatos vizuális
  eszközök segítségével prezentálja.
\item
  Tetszésítélete alapján alkotásokról információkat gyűjt, kifejezőerő
  és a közvetített hatás szempontjából csoportosítja, és megállapításait
  felhasználja más szituációban.
\item
  Megfogalmazza személyes viszonyulását, értelmezését adott vagy
  választott művész alkotásai, társadalmi reflexiói kapcsán.
\item
  Látványok, képek, médiaszövegek, történetek, szituációk feldolgozása
  kapcsán személyes módon kifejezi, megjeleníti felszínre kerülő
  érzéseit, gondolatait, asszociációit.
\item
  Vizuális megjelenítés során egyénileg és csoportmunkában is használja
  a kiemelés, figyelemirányítás, egyensúlyteremtés vizuális eszközeit.
\item
  Egyszerű tájékoztató, magyarázó rajzok, ábrák, jelek, szimbólumok
  tervezése érdekében önállóan információt gyűjt.
\item
  Célzottan vizuális kommunikációt szolgáló megjelenéseket értelmez és
  tervez a kommunikációs szándék és a hatáskeltés szempontjait kiemelve.
\item
  A helyzetek, történetek ábrázolása, dokumentálása során egyénileg vagy
  csoportmunkában is felhasználja a kép és szöveg, a kép és hang
  viszonyában rejlő lehetőségeket.
\item
  Adott témát, időbeli, térbeli folyamatokat, történéseket közvetít
  újabb médiumok képírási formáinak segítségével egyénileg vagy
  csoportmunkában is.
\item
  Nem vizuális információkat (pl. számszerű adat, absztrakt fogalom)
  különböző célok (pl. tudományos, gazdasági, turisztikai) érdekében
  vizuális, képi üzenetté alakít.
\item
  Adott téma vizuális feldolgozása érdekében problémákat vet fel,
  megoldási lehetőségeket talál, javasol, a probléma megoldása érdekében
  kísérletezik.
\item
  Nem konvencionális feladatok kapcsán egyéni elképzeléseit, ötleteit
  rugalmasan alkalmazva megoldást talál.
\end{itemize}

\hypertarget{evfolyamon-40}{%
\subsubsection{9-10. évfolyamon}\label{evfolyamon-40}}

\begin{itemize}
\item
  Feladatmegoldás közben kísérletezik, különböző megoldási utakat keres,
  és törekszik az egyéni megoldás igényes kivitelezésére.
\item
  Eszköz- és anyaghasználat során adekvát döntést hoz.
\item
  Adott tanulási helyzetben a tanulási előrehaladás érdekében
  problémákat keres, kérdéseket tesz fel, és ezekre önállóan is keresi a
  megoldásokat és válaszokat.
\item
  Feladatmegoldásai során a felhasználás érdekében hatékonyan
  szelektálja a korábban szerzett ismereteket.
\item
  Feladatmegoldásai során saját ötleteit és eredményeit bátran
  bemutatja.
\item
  Példák alapján érti a művészet kultúraközvetítő szerepét.
\item
  Példák alapján művészettörténeti korszakokat, stílusokat felismer és
  egymáshoz képest időben viszonylagos pontossággal elhelyez.
\item
  Példák alapján felismeri és értelmezi a kortárs művészetben a
  társadalmi reflexiókat.
\item
  Érti és meg tudja magyarázni a célzott vizuális közlések
  hatásmechanizmusát.
\item
  Ismeri néhány példáját a digitális képalkotás közösségi médiában
  használt lehetőségének.
\item
  Felismeri a designgondolkodás sajátosságait az őt körülvevő tárgy- és
  környezetkultúra produktumaiban.
\item
  A fenntarthatóság érdekében felelős döntéseket hoz a saját tervezett,
  épített környezetével kapcsolatban.
\item
  Csoportban végzett feladatmegoldás során részt vállal a
  feladatmegoldásban, önállóan megtalálja saját feladatát, figyelembe
  veszi társai álláspontját, de az optimális eredmény érdekében
  hatékonyan érvényesíti érdekeit.
\item
  Önálló véleményt alkot, és azt meggyőzően indokolja.
\item
  Adott területen megtalálja a számára érdekes és fontos kihívásokat.
\item
  A látható világ vizuális összefüggéseinek megfigyeléseit ok-okozati
  viszonyoknak megfelelően rendszerezi.
\item
  Alkotó és befogadó tevékenységei során érti, és komplex módon
  használja a vizuális nyelv eszközeit.
\item
  A vizuális megjelenések mintáinak önálló megfigyelése és felismerése
  által konstrukciókat alkot, e megfigyelések szempontjainak
  összekapcsolásával definiál és következtet, mindezt társaival
  együttműködve alkotótevékenységébe is beilleszti.
\item
  Adott feladatmegoldás érdekében meglévő vizuális ismeretei között
  megfelelően szelektál, a további szakszerű információszerzés érdekében
  adekvátan keres.
\item
  Az alkotótevékenység során szerzett tapasztalatait önálló
  feladatmegoldás során beépíti, és az eredményes feladatmegoldás
  érdekében szükség szerint továbbfejleszti.
\item
  Alkotó feladatmegoldásai során az elraktározott, illetve a
  folyamatosan újraalkotott belső képeit, képzeteit szabadon párosítja a
  felkínált tartalmi elemek és látványok újrafogalmazásakor, amelyet
  indokolni is tud.
\item
  Új ötleteket is felhasznál képek, tárgyak, terek megjelenítésének,
  átalakításának, rekonstruálásának megvalósításánál síkbeli, térbeli és
  időbeli produktumok létrehozása esetében.
\item
  A vizuális megjelenések elemzése és értelmezése során a befogadó és az
  alkotó szerepkört egyaránt megismerve reflexióit szemléletesen és
  szakszerűen fogalmazza meg szövegesen és képi megjelenítéssel is.
\item
  A művészi hatás megértése és magyarázata érdekében összehasonlít, és
  következtetéseket fogalmaz meg a különböző művészeti ágak kifejezési
  formáival kapcsolatban.
\item
  Adott és választott vizuális művészeti témában önállóan gyűjtött képi
  és szöveges információk felhasználásával részletesebb helyzetfeltáró,
  elemző, összehasonlító projektmunkát végez.
\item
  Megfelelő érvekkel alátámasztva, mérlegelő szemlélettel viszonyul az
  őt körülvevő kulturális környezet vizuális értelmezéseinek mediális
  csatornáihoz, amit társaival is megvitat.
\item
  Különböző mediális produktumokat vizuális jelrendszer, kommunikációs
  szándék és hatáskeltés szempontjából elemez, összehasonlít, és
  következtetéseit társaival is megvitatja.
\item
  Érti és megkülönbözteti a klasszikus és a modern művészet
  kultúrtörténeti összetevőit, közlésformáinak azonosságait és
  különbségeit.
\item
  Adott vagy választott kortárs művészeti üzenetet személyes viszonyulás
  alapján, a társadalmi reflexiók kiemelésével értelmez.
\item
  Adott szempontok alapján érti és megkülönbözteti a történeti korok és
  a modern társadalmak tárgyi és épített környezetének legfontosabb
  jellemzőit, miközben értelmezi kulturális örökségünk jelentőségét is.
\item
  Személyes élményei alapján elemzi a tárgy- és környezetkultúra,
  valamint a fogyasztói szokások mindennapi életre gyakorolt hatásait és
  veszélyeit, és ezeket társaival megvitatja.
\item
  Az adott vagy a választott célnak megfelelően környezetátalakítás
  érdekében, társaival együttműködésben, környezetfelméréssel
  alátámasztva tervet készít, amelyet indokolni is tud.
\item
  Bemutatás, felhívás, történetmesélés céljából térbeli és időbeli
  \textgreater{} folyamatokat, történéseket, cselekményeket különböző
  eszközök \textgreater{} segítségével rögzít.
\item
  Adott feladatnak megfelelően alkalmazza az analóg és a digitális
  \textgreater{} prezentációs technikákat, illetve az ezekhez
  kapcsolható álló- és \textgreater{} mozgóképi lehetőségeket.
\item
  Tervezési folyamat során a gondolkodás szemléltetése érdekében
  \textgreater{} gondolatait mások számára is érthetően, szövegesen és
  képpel \textgreater{} dokumentálja.
\item
  Képalkotás és tárgyformálás során autonóm módon felhasználja
  \textgreater{} személyes tapasztalatait a hiteles kifejezési szándék
  érdekében a \textgreater{} választott médiumnak is megfelelően.
\item
  Saját munkáit bátran újraértelmezi és felhasználja további
  \textgreater{} alkotótevékenység során.
\item
  Vizuális megjelenéseket, alkotásokat újraértelmez, áttervez és
  \textgreater{} módosított kifejezési szándék vagy funkció érdekében
  újraalkot.
\item
  Valós célokat szolgáló, saját kommunikációs helyzetnek megfelelő,
  \textgreater{} képes és szöveges üzenetet felhasználó vizuális közlést
  hoz létre \textgreater{} társaival együttműködésben is.
\item
  Szabadon választott témában, társaival együtt ok-okozati
  \textgreater{} összefüggéseken alapuló történetet alkot, amelynek
  részleteit \textgreater{} vizuális eszközökkel is magyarázza,
  bemutatja.
\item
  Adott téma újszerű megjelenítéséhez illő technikai lehetőségeket
  \textgreater{} kiválaszt, és adott vizuális feladatmegoldás érdekében
  megfelelően \textgreater{} felhasznál.
\item
  Technikai képnél és a számítógépes környezetben felismeri a
  \textgreater{} manipuláció lehetőségét, és érti a befolyásolás
  vizuális \textgreater{} eszközeinek jelentőségét.
\item
  Adott feladatmegoldás érdekében ötleteiből rendszert alkot, a célok
  \textgreater{} érdekében alkalmas kifejezési eszközöket és technikákat
  választ, \textgreater{} az újszerű ötletek megvalósítása érdekében
  szabályokat újraalkot.
\item
  Egyéni munkáját hajlandó a közösségi alkotás érdekei alá rendelni, a
  \textgreater{} hatékonyság érdekében az együttműködésre törekszik.
\item
  A leghatékonyabb megoldás megtalálása érdekében felméri a
  \textgreater{} lehetőségeket és azok feltételeit, amelyek komplex
  mérlegelésével \textgreater{} döntést hoz az adott feladatokban.
\end{itemize}
