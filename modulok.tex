\hypertarget{tanulasi-tanitasi-egysegek-a-modulok}{%
\section{Tanulási-tanítási egységek, a
modulok}\label{tanulasi-tanitasi-egysegek-a-modulok}}

A \emph{tanulási-tanítási egységek}, a Budapest School Modell
szóhasználatában a \emph{modulok} a tanulásszervezés
\emph{alapegységei}: olyan foglalkozások megtervezett sorozata, amelyek
során egy meghatározott időn belül a gyerekek valamely képességüket
fejlesztik, valamilyen ismeretet elsajátítanak, vagy valamilyen
produktumot létrehoznak. A tanítási egységek célja sokféle lehet, de
kötelező elvárás, hogy a résztvevők a portfóliójukba való bejegyzésre érdemes
eredményt hozzanak létre, és hogy legyen egyértelmű céljuk.

A modulok tehát önálló céllal, fókusszal, kerettel bíró
foglalkozásegységek. Akár szakköröknek, workshopoknak,
mikrotantárgyaknak, projekteknek vagy foglalkozásoknak is hívhatnánk őket. A BPS modell a
legsemlegesebb szót akarja az használni, hogy a tanárok és a gyerekek
tudják feltölteni tartalommal a modulokat.

\hypertarget{a-modulok-a-tanmenet-egysegei}{%
\subsubsection{A modulok a tanmenet
egységei}\label{a-modulok-a-tanmenet-egysegei}}

A modulok fogalmat a NAT és az Nkt.\ is ismeri. Ezekben a jogszabályokban
tantárgyi témaköri egységet jelentenek, bár maga a NAT nem definiálja a
fogalmat. A BPS modellben a modul csak annyit jelent: összefüggő
foglalkozások egysége, azaz a tanulási-tanítási egységet és a modul
szavakat szinonimaként kezeli. Ha a diszciplináris tanítás egysége
diszciplináris modul, a multidiszciplináris tanítás egysége
multidiszciplináris modul.

A BPS modellben kétféle tervezéstípust ötvözünk:

\begin{enumerate}
\def\labelenumi{\arabic{enumi}.}
\tightlist
\item
  a hosszú távú célokból kiinduló, majd azt a szakaszosan a mindennapokra
  lebontó felülről lefelé tervezést ötvözzük
\item
  az elsődlegesen a gyerekek élményére fókuszáló és abból építkező
  lentről felfelé való tervezési megközelítéssel.
\end{enumerate}

Az előbbi megközelítésben a tantervi célokból alakulnak a tantárgyak,
azokat tematikára, tematikus egységekre bontjuk. Ebből alakulnak ki
tanítási egységek, majd foglakozásokra bontva alakul ki a gyerekek
élményének megtervezése. Ezt hívhatjuk felülről-lefele (angolul
top-down) tervezésnek. A lentről-felfelé (bottom-up) tervezéssel pedig
a kívánt gyerek-élménytől indulunk ki, és a tanulási-tanítási
egységekből építjük újra fel a tantárgyak tartalmát, illetve az elvárt
tanulási eredményeket.

Például \emph{A koronavírus terjedése} című foglalkozás elvezethet oda,
hogy a gyerek \emph{mértani sorozatokra vonatkozó ismereteit használja
gazdasági, pénzügyi, természettudományi és társadalomtudományi problémák
megoldásában.} Fentről-lefelé tervezés ehhez a foglalkozáshoz úgy
jutott, hogy az 9--12. évfolyam Matematika tantárgy céljaiból levezette,
hogy foglalkozni kell a mértani sorozatokkal társadalmi változások
elemzéséhez. A tanár úgy gondolta, hogy nem a kamatos kamattal
foglalkozik, hanem a gyerekek számára éppen érdekesebb koronavírussal,
azaz cél eléréséhez a gyerekek érdeklődése mentén talált utat.

A lentről felfelé tervezés pedig a gyerekek érdeklődéséből indul ki, ami
jelen esetben a koronavírus. És esetleg később veszi észre a tanár és a
gyerek, hogy már az 9--12. Matematika elvárásainak egy részét is
teljesítették. Hisz ez előre nem volt tervezve, mert a cél nem a
tantárgyi követelmények teljesítése volt, hanem a téma feldolgozása.

A két tervezési módot ötvözi a BPS modell. Az agilis tervezés
eredményeképp feltehetőleg az alulról felfelé tervezés indul meg: itt és
most mire van igényük a gyerekeknek. Idővel, például az év vége fele,
vagy az érettségi vizsgához közeledve egyre erősebb lesz a fentről
lefelé tervezés, és ez így van helyén. Ott és akkor az a kérdés merül
fel, hogy mire van még szükség, hogy azokat a célokat is elérjük. De az
is előfordulhat, hogy egy közösség inkább fentről lefelé tervezne:
biztos, ami biztos, érjük el a tantárgyi célokat, és az úton legyünk
kreatívak, vagy épp akkor, amikor már a szükséges eredményeket biztos
elértük.

\emph{A fentről lefelé vagy lentről felfelé tervezés során létrejövő
tanulási-ta\-ní\-tá\-si egységeket, a tanmeneteket építő egységeket hívja a
BPS modell moduloknak.}

\hypertarget{foglalkozasok-modulok---szohasznalat-definiciok}{%
\subsection{Foglalkozások, modulok - szóhasználat,
definíciók}\label{foglalkozasok-modulok---szohasznalat-definiciok}}

Bár a BPS modell a modulok szintjén írja le a tanulás-tanítás élményét,
a tanulási élmény leginkább meghatározó elemei mégis a
\emph{foglalkozások}. Az a dedikált idő, amit a (szak)tanár és a gyerek
azért alakítanak ki, hogy a gyerek tanuljon. A modul nem más, mint
egybefüggő foglalkozások sorozata. Ezért a BPS modellben, ahol a \emph{modul}
szót használjuk, ott a \emph{foglalkozás} szót is használhatnák.

A BPS modell nem a tanóra szót használja, hogy kihangsúlyozza: egy
foglalkozás célja nem csak egy adott tantárgy következő témakörének
feldolgozása lehet. Foglalkozás többek közt lehet

\begin{enumerate}
\def\labelenumi{\arabic{enumi}.}
\tightlist
\item
  tantárgycsoportok összevont foglalkozása;
\item
  témanapok, témahetek eseményei;
\item
  projektekben való dolgozás;
\item
  önálló tanulás, gyakorlás;
\item
  közös önálló tanulás, tanulószobai tevékenység;
\item
  vagy épp egy tantárgy témakörének feldolgozása.
\end{enumerate}

A foglalkozások tanügyigazgatási értelemben megegyeznek a tanórákkal. A
foglalkozások hossza különböző lehet a törvények által megadott keretek
között. A foglalkozások tanórának számolhatók el, ha tantárgyi tartalom
elsajátítása a céljuk. Ha több tantárgy tartalmát érinti a foglalkozás,
akkor több tantárgy tanórájának számolandó el.

\hypertarget{modulok---a-tantargyak-kozott-szabadon-mozoghatunk}{%
\subsubsection{Modulok --- a tantárgyak között szabadon
mozoghatunk}\label{modulok---a-tantargyak-kozott-szabadon-mozoghatunk}}

A Budapest School Modell hangsúlyozza a tanulási-tanítási egység, a
modulok szerepét, ezért nevezi a tanmenetet \emph{moduláris
tanmenetnek}. Ugyanazt az eredményt éri el egy tanár, aki egy tantárgy
tanóráinak egyes egységeire külön foglalkozásokat tervez (és akár
külső előadókat, tanárokat is meghív besegíteni), mint az a tanár, aki
tanulási-tanítási egységeket, modulokat tervez, és ezekből építi fel a
tanévet.

Egy tanulási-tanítási egység több tantárgy tananyagtartalmát is
lefedheti.

A mindennapi tanulás a tanulás-tanítási egységek, a modulok elvégzésével
történik, ezzel biztosítva, hogy rugalmas keretek között, pontosan
megfogalmazott célok mentén, a gyerekek számára érthető, átlátható és
sajátnak megélt tartalommal történjen.

A tanulási-tanítási egységeket, vagyis a tanulás tartalmának és
formájának alapegységét a tanulásszervezők három kötelező összetevőből
állítják össze:

\begin{enumerate}
\def\labelenumi{\arabic{enumi}.}
\item
  a NAT tantárgyainak tartalmából;
\item
  a gyerekek, tanárok érdeklődéséből, aktuális tudásából;
\item
  és a környezetük és a világ aktuális kihívásaiból.
\end{enumerate}

A három komponensből a legelső a legstatikusabb, hiszen a NAT
meghatározza a tantárgyakat és azok tartalmát, valamint azt, hogy milyen
lehetséges eredmények elérését várjuk az ezekben való fejlődéstől. Az
egyes tanítási egységekban ezek személyre, illetve a csoport igényeire
szabhatóak, hiszen az elérhető eredményeket különféle gyakorlati és
elméleti tanulási módszerekkel el lehet érni.

A gyerekek és tanárok érdeklődése --- ami a sajátként megélt cél és a
minél nagyobb fokú bevonódás alapfeltétele --- alakítja ki a tanítási
egységek témáját, a projekteket, és a gyerekek egyéni tanulási idejét is
meghatározhatja.

Mindemellett az iskola szándéka, hogy a tanárok, gyerekek reagáljanak a
környezetükre, a világ aktuális kihívásaira, kérdéseire. A NAT
meghatározza például, hogy a gyerek \emph{„Problémákat old meg
táblázatkezelő\break
program segítségével.''}. Az azonban, hogy a gyerekek
milyen táblázatokat szerkesztenek szívesen, csak a tanítási egységek
összeállításakor és a tanítási egységek elvégzése során derül ki. Nagyon
hasonló táblázatkezelési képességeket lehet fejleszteni, ha valaki az
önvezető autóktól várt csökkenő számú balesetekről , vagy ha az egy főre
jutó károsanyag-kibocsátás és a GDP-növekedés alakulásának arányáról
készít táblázatot.

A tanulási-tanítási egységekben épülő moduláris tanmenet fő célja, hogy
egyszerre képes legyen alkalmazkodni a menet közben felmerülő tanulási
igényekhez, adjon átlátható struktúrát a tanulásnak, és hogy a
tanulóközösség minél rugalmasabban tudja támogatni a tanulást úgy, hogy
a saját, a közösségi és a társadalmi célok harmóniába kerülhessenek.

Ez is mutatja, hogy bár közösek a kereteink, végtelen az elképzelhető
tanítási egységek (a tanulási utak építőkövei, és így a különböző
tanulási utak) száma. Ezért tartja fontosabbnak a Budapest School Modell
annak meghatározását, hogy hogyan kell a tanítási egységeket, a
modulokat létrehozni, mint azt, hogy a tanítási egységeket, a modulokat
tételesen felsorolja.

\hypertarget{modulok-sokfelek-lehetnek}{%
\subsubsection{A modulok sokfélék
lehetnek}\label{modulok-sokfelek-lehetnek}}

Egy-egy tanítási-tanulási egység, azaz egy modul során a gyerekek
képesek

\begin{itemize}
\tightlist
\item
  produktum létrehozására szerveződő projektben részt venni;
\item
  felfedezni, feltalálni, kutatni, vizsgálni, azaz kérdésekre választ
  keresni;
\item
  egy jelenséget több nézőpontból megismerni;
\item
  valamely képességüket, készségüket fejleszteni;
\item
  adott vizsgára gyakorló feladatokkal felkészülni;
\item
  közösségi programokban részt venni;
\item
  az önismeretükkel, a tudatosságukkal, a testi-lelki jóllétükkel
  foglalkozni.
\end{itemize}

\hypertarget{tantargyi-es-multidiszciplinaris-modulok}{%
\subsubsection{Tantárgyi és multidiszciplináris
modulok}\label{tantargyi-es-multidiszciplinaris-modulok}}

Egyes modulok, a \emph{tantárgyi modulok} célja egy tantárgy
ismereteinek elsajátítása, vagy egy-egy tantárgyi vizsgára, pl.
érettségi vizsgára való felkészülés. A tantárgyi modulokat tantárgyi
szaktanárok tartják.

A \emph{multidiszciplináris modulok} esetén több tantárgy ismereteinek
integrálását igénylő (multidiszciplináris) téma kerül a középpontba. A
multidiszciplináris modulok tervezéséhez erős szempontot adnak a
% \href{/tanulas-megkozelitese/kiemelt-fejlesztesi-teruletek.md}
{\emph{kiemelt
fejlesztési területek}}.

\hypertarget{orabontasok-csoportbontasok-osztalyok}{%
\subsection{Órabontások, csoportbontások,
osztályok}\label{orabontasok-csoportbontasok-osztalyok}}

Egy foglalkozás megszervezhető különböző évfolyamok, különböző osztályok
tanulóiból álló csoportok részére is, ahogy azt a 20/2020. EMMI rendelet
13 § (1) is kimondja. Ebben az esetben a foglalkozások differenciált és
kooperatív tanulásszervezést igényelnek.

Ebből következik, hogy egy modul, azaz a foglalkozások sorozata is
megszervezhető különböző évfolyamok, különböző osztályok tanulóiból álló
csoportok részére is.

\hypertarget{a-modulok-meghirdetese}{%
\subsection{A modulok meghirdetése}\label{a-modulok-meghirdetese}}

A tanulási-tanítási egységek kiválasztása, felkínálása a
tanulásszervezők feladata, hiszen ők figyelnek és reagálnak a gyerekek,
szülők céljaira és igényeire. A meghirdetett tanítási egységekből áll
össze a tanulás trimeszterenkénti tanulási rendje.

A tanulásszervezők az egyes tanulási-tanítási egységek tematikáját,\break
hosszát és feladatát a gyerekek tanulási céljainak megismerését követően
és a NAT-ban meghatározott tantárgyi tanulási eredményeket figyelembe véve
határozzák meg.

A tanulási-tanítási egységekbe, modulokba való csatlakozásról a mentor,
a szülő és a gyerek közösen dönt, mindig szem előtt tartva, hogy
folyamatos előrelépés legyen a már elért egyéni és tantárgyi
eredményekben is. Egy modul megkezdésének lehet feltétele egy korábbi
modul elvégzése, a gyerek képességszintje, a jelentkezők száma, és lehet
egyedüli feltétele a gyerekek érdeklődése.

Egy szaktanár különféle tematikájú tanítási egységeket tarthat attól
függően, hogy a saját célok, a tantárgyi eredmények mit kívánnak, és
hogy a
tanulásszervezők, valamint a szaktanárok kapacitása mit enged meg.

Amikor egy gyerek moduljai befejeződnek, és újat vesz fel, a
tanulásszervező feladata a gyereket segíteni abban, hogy az érdeklődési
körének, tanulási céljainak, és a soron következő, még el nem ért
tantárgyi eredményekben való fejlődéshez megfelelő tanítási egységek
közül választhasson.

A tanulásszervezők feladata a tantárgyi eredményelvárások nyomon
követése is. A tanítási egységek kidolgozásához és azok megtartásához
külsős szakembereket is meghívhatnak, azonban ilyenkor is a
tanulásszervezők felelnek azért, hogy a tanítási egységekkal elérni
kívánt tanulási célok teljesüljenek.

\hypertarget{a-modulok-egysegek-formatuma}{%
\subsection{A modulok, egységek
formátuma}\label{a-modulok-egysegek-formatuma}}

A moduláris rendszer elég nagy szabadságot ad a tanároknak abban, hogy
hogyan szervezik a mindennapokat. Ezért is fontos, hogy már a modul
meghirdetése előtt néhány szempont szerint kialakítsák a tanítási
egységek kereteit.

\hypertarget{celok}{%
\paragraph{Célok}\label{celok}}

Minden tanulási-tanítási egységnek, modulnak előre meg kell határozni a
célját. A tanulási szerződések célkitűzéseihez hasonlóan itt is minél
specifikusabban és mérhetőbben kell megfogalmazni a célokat. Javasolt az
OKR {\autocite{Doerr2018}} vagy a SMART {\autocite{Doran1981}} technika
alkalmazása, hogy minél specifikusabb, teljesíthetőbb, tervezhetőbb és
köny-\break
nyen mérhető célokat tűzzenek ki.

\hypertarget{ertekeles}{%
\paragraph{Értékelés}\label{ertekeles}}

A tanulási-tanítási egység végén minden résztvevő személyes, több
szempont alapján készült értékelést, visszajelzést kap a
tanulási-ta\-ní\-tá\-si egységgel kapcsolatos tevékenységére és elért
eredményeire. A visszajelzés struktúráját előre meg kell határozni, és
még a modul kezdete előtt meg kell osztani a résztvevőkkel.

Természetesen a visszajelzés szempontjai változhatnak a
tanulási-ta\-ní\-tá\-si egység során, ha változik a tanulási-tanítási egység
tartalma, szempontjai. Ebben az esetben ezt mindenki számára nyilvánvalóvá
kell tenni.

\hypertarget{idotartam}{%
\paragraph{Időtartam}\label{idotartam}}

Egy-egy tanulási-tanítási egység hossza és a tanulási-tanítási egységhez
kapcsolódó foglalkozások száma és gyakorisága változó: egy alkalomtól
legfeljebb egy teljes trimeszteren keresztül tarthat. A
tanulási-tanítási egység végén a tanulásszervező és a gyerek(ek) a
modult lezárják, értékelik és az elért eredményeket rögzítik a
(tanulási) portfólióban. Egy tanulási-tanítási egység folytatásaként a
következő trimeszterben új tanulási-tanítási egységet alakítanak ki a
tanárok.

\hypertarget{modszertan-formatum}{%
\paragraph{Módszertan, formátum}\label{modszertan-formatum}}

A tanulási-tanítási egységek nemcsak témájukban, céljaikban,
időtartamukban, hanem módszertanukban, folyamataikban is különbözhetnek:
bizonyos tanulási-tanítási egységekben a felfedeztető (inquiry based)
módszer, másokban az ismétlő (repetitív) gyakorlás a célravezető. Így
mindig a tanulási-tanítási egység céljához, a tanárok és a gyerekek
képességeihez és igényeihez választható a legjobb módszer.
Tanulási-tanítási egységenként változhat, hogy a folyamatot a gyerekek
vagy a tanárok befolyásolják-e, és milyen mértékben. Két példa az
eltérésre:

\begin{enumerate}
\def\labelenumi{\arabic{enumi}.}
\item
  Egy digitális kézműves tanulási-tanítási egység célja, hogy építsünk
  valamit, ami programozható. Annak kitalálása, hogy mit és hogyan
  építünk, a gyerekek feladata. Itt a tanár csak támogatja a tanulás
  folyamatát, azaz \emph{facilitál}.
\item
  Egy „\emph{A vizuális kommunikáció fejlődése a XX. század második
  felében}'' tanulási-tanítási egység esetén a tanár előre felépíti a
  tanmenetet, például hogy mely alkotók munkásságát, alkotásait fogja
  bemutatni, és ezeket a gyerekekkel sorban végigveszi. Ilyenkor is
  bővülhet azonban a tematika a gyerekek érdeklődése, felvetései mentén.
\end{enumerate}

\hypertarget{a-tanulasi-tanitasi-egysegek-helyszine}{%
\subsection{A tanulási-tanítási egységek
helyszíne}\label{a-tanulasi-tanitasi-egysegek-helyszine}}

A tanulás az egyes tanulóközösségek helyszínén, egy másik Budapest\break
School-épületben, a tanár által kiválasztott külső helyszínen, vagy akár
online, virtuális térben történik. A tanulásra úgy tekintünk, mint az
élethez szorosan kapcsolódó holisztikus fejlődési igényre, melynek
jegyében az elsődleges szocializációs térről és formáról, a szülői,
családi környezetről sem akarjuk a tanulást leválasztani. Az élethosszig
tartó tanulás jegyében a tanulás tere az iskolai időszak után és az
iskola terein kívül is folytatódik.

A gyerekek több ok miatt is tanulnak az iskolán kívül:

\begin{enumerate}
\def\labelenumi{\arabic{enumi}.}
\item
  tanulási-tanítási egységek foglalkozásai szervezhetők külső
  helyszínekre, úgymint múzeumokba, erdei iskolákba, parkokba,
  vállalatokhoz, vagy tölthetik az idejüket „kint a társadalomban''.
\item
  Amennyiben ez saját céljuk elérését nem veszélyezteti, és a folyamatos
  fejlődés biztosított, a mentoruk tudomásával a gyerekek az
  önirányított tanulás elvére figyelemmel a telephelyen kívüli egyéb
  helyszínen is elvégezhetnek egy-egy tanulási-tanítási egységet.
\end{enumerate}

A tanulási-tanítási egység lezárásaként a gyerekek és tanárok
visszajelzést adnak egymásnak, aminek része, hogy megosztják saját
élményeiket, reflektálnak a közös időre, összegyűjtik és értékelik az
elért eredményt, és kitérnek az esetleges fejlődési lehetőségekre.

\hypertarget{kulsos-tanarok-aranya}{%
\subsection{Külsős tanárok aránya}\label{kulsos-tanarok-aranya}}

A pedagógiai program egy fontos megkötést ad a tanulási-tanítási
egységek megtartására: a tanulóközösség tanulásszervezőinek kell
vezetniük a gyerekek moduljainak nagy részét.  Másképp fogalmazva:
korlátozva van a külsős, nem tanulásszervezők által tartott modulok
óraszáma, ahogy ezt a lenti táblázat mutatja. Ennek a megkötésnek az az
oka, hogy

\begin{itemize}
\item
  Kisebb korban szeretnénk, ha kevesebb, állandóan jelenlévő felnőtt
  vezetné a gyerekek tanulását (a kéttanítós rendszer mintájára).
\item
  Biztosított legyen, hogy a gyerekek a tanulóközösség
  tanulásszervezőiel elegendő időt töltsenek.
\item
  Érettségihez közeledve legyen lehetőség minél több külsős, akár\break
  speciális szaktudással bíró embertől tanulni.
\end{itemize}

\begin{longtable}[]{@{}llllll@{}}
\toprule
Évfolyamszint & 1--2 & 3--4 & 5--8 & 9--11 & 11--12\tabularnewline
\midrule
\endhead
„belsős'' modulok min. aránya & 70\% & 60\% & 55\% & 50\% &
40\%\tabularnewline
\bottomrule
\end{longtable}

\hypertarget{modulok-nyomonkovetese}{%
\subsection{Modulok nyomonkövetése}\label{modulok-nyomonkovetese}}

Minden trimeszter megkezdésekor a tanulásszervező tanárok meghirdetik a
kötelező, a kötelezően választható és a választható modulokat, azaz
rögzítik, hogy

\begin{itemize}
\item
  ki a modul vezetője, és melyik pedagógus munkakörben alkalmazott
  tanulásszervező felelős (elszámoltatható a
  %\href{https://www.pmsz.hu/hirek-aktualitasok/havi-mustra/havi-mustra-a-felelosseg-hozzarendelesi-matrixrol}
  {\emph{RACI}}
  menedzsment
  rendszer értelmezésében) a modulért;
\item
  mi a modul célja, keretei és várható eredményei;
\item
  hol, mikor és milyen rendszerességű foglalkozások lesznek;
\item
  és mi a részvétel feltétele (előzetes tudás, kor, maximum létszám),
  azon belül, hogy teljes folyamatra kell-e
  elköteleződni, vagy esetileg is lehet a modult látogatni.
\end{itemize}

Ezután eldől, hogy ki mikor melyik modulon vesz részt. Ennek rendszerét
a tanárok alakítják ki: beoszthatják a gyerekeket, ahogy ők ezt jónak
látják, vagy épp hagyatkozhatnak a gyerekek választására. A lényeg, hogy
alakuljon ki a rendszer a trimeszter megkezdése előtt.

\hypertarget{nyomonkovethetoseg}{%
\paragraph{Nyomonkövethetőség}\label{nyomonkovethetoseg}}

Az őszi első trimeszterben a tanulóközösség ismerkedéssel kezd. Ebben a
trimeszterben október 1.\ a modulrendszer felállításának határideje. A
második trimeszter esetén január 1., tavasszal pedig április 1.\ a határidő.
Az ezektől a határidőktől kezdve kialakuló rendszerben mindennap lehet
tudni, hogy ki, hol, kivel, melyik modulok keretében tanul. Azaz
kialakul a gyerekek órarendje, ami nagyon hasonló a megszokott
órarendekhez: \emph{mikor melyik foglalkozáson és hol vagyok}.

A Budapest School iskolában annyi a különbség, hogy a
személyreszabhatóság miatt az egy osztályba, tanulóközösségbe járó
gyerekek órarendje akár nagy mértékben is eltérhet egymástól. Tehát itt
nem egy osztálynak vagy tanulóközösségnek van órarendje, hanem a
gyerekeknek van saját órarendjük. A fenntartó felelőssége kialakítani azt
a számítógépes rendszert, ami a gyerekek órarendjét a gyerekek, szülők,
tanárok, tanulóközösségek és a teljes Budapest School iskola szintjén
átláthatóvá és nyomonkövethető teszi.

\hypertarget{inkrementalis-fejlesztes}{%
\subsection{Inkrementális fejlesztés}\label{inkrementalis-fejlesztes}}

A moduláris rendszer nagy szabadságot enged a tanároknak abban, hogy a
mindennapokat olyan tevékenység köré szervezzék, ami szerintük a
gyerekek tanulását a legjobban szolgálja. Ez a szabadság
bizonytalanságot is adhat: ha bármilyen modult szervezhetünk, akkor
milyen modult szervezzünk? A BPS modell a tanároknak azt javasolja, hogy
induljanak ki egy számukra ismert, stabil rendszerből, és azt fejlesszék
lépésről lépésre.

Sokaknak a legbiztosabb alap a jól ismert rendszer: a NAT tantárgyaiból
létrehozott modulok, amik követik a kiadott kerettantervek tematikus
egységeit. A moduláris rendszer ezt is lehetővé teszi. Van, amikor innen
indulva, lassan, trimeszterenként változtatva érhetjük el a legjobb
eredményt: ezután össze lehet vonni tantárgyakat, és egy modulba
szervezni például a magyar nyelv és irodalmat és a történelmet egy
trimeszterre, vagy az összes természettudományi tantárgyat egy
kísérletezős modulba. Lehet tömbösítve vagy epochálisan szervezni a
mindennapi tanulást.

A modulrendszer le tudja fedni a szakkörök, iskola utáni foglalkozások,
nyári táborok rendszerét is. Egy tanulóközösség a szokásos tantárgyi
modulok mellé meghirdethet más iskolákban szakkörnek nevezett modulokat:
robotika, néptánc, fociedzés. A modulrendszer sajátja, hogy ezeket a
szakköröket ugyanúgy tudja kezelni, mint a történelem érettségire
felkészítő fakultációt.

A modulrendszer lehetővé teszi a projektpedagógiával való szabad
kísérletezést is. Lehet olyan modulrendszert alkotni, ahol minden páros
héten projekteken dolgoznak a gyerekek, a páratlan heteken pedig
klasszikus tantárgyi struktúrák mentén szervezett modulokban haladnak az
akadémiai tudás elsajátításával.

\hypertarget{biztonsagos-felfedezes}{%
\subsection{Biztonságos felfedezés}\label{biztonsagos-felfedezes}}

A tanárokat bátorítjuk egy olyan saját, rugalmas struktúra
kialakítására, amely jól működik, és biztonságot nyújt mind számukra,
mind pedig a gyerekek és a szülők számára. A Budapest School tanulásmonitoring rendszere miatt mindig tudjuk, hogy egy gyerek egy-egy
tantárgy tanulási eredményeivel hogyan haladt. Ez biztonsági hálót ad a
tanároknak: mindig tudjuk, hogy a gyerekek milyen irányban haladnak,
lemaradtak-e valamiből, előre szaladtak-e valami másból. Egyben
folyamatos visszajelzést ad a modulstruktúráról. Ezért mondhatjuk, hogy
nyugodtan kereshetjük a jobb struktúrát, a tökéleteset sose tudjuk
elérni (,,better, never the best'').

\hypertarget{kiemelt-modultipusok}{%
\subsection{Kiemelt modultípusok}\label{kiemelt-modultipusok}}

\hypertarget{elmenynapok}{%
\subsubsection{Élménynapok}\label{elmenynapok}}

Egy különleges modultípust, az élménynapokat a BPS modell külön is
kiemel. Ezek azok a napok, amikor a gyerekek előre tervezetten és
rendszeresen a tanárokkal együtt \emph{kimennek az iskolából}. Ezt a
napot nevezzük \emph{élménynapnak}. Az, hogy a hét melyik napján és hogy
hány hetente szerveznek a tanulásszervezők élménynapot, az ő
döntésük. Javasolt a pénteket választani, de telephelyenként eltérhet,
hogy mikor szervezik az élménynapot.

Ezeken a napokon nagyon változatos programok szervezhetők, a külső
moduloktól, a nem formális (tanórán és iskolán kívül szervezett) és az
informális (nem szervezett, spontán tevékenység során megvalósuló)
tanulási formákig minden. Az az egyetlenegy megkötés, hogy legalább egy
héttel előtte értesíteni kell a szülőket és a fenntartót a tervezett
programról. Hiszen azt tudni és dokumentálni kell, hogy hol vannak a
gyerekek. A terv nem azt jelenti, hogy mindig nagyon struktúrált
foglalkozást kell szervezni. Az is terv, hogy csak kint vagyunk a Duna-parton, és ott játszunk együtt.

Átszervezhető-e egy élménynap másik napra? (Mert például egy múzeum csak
egy másik nap van nyitva?) Igen, ha erről legalább egy héttel előtte
minden érintett tudomást szerez. Lehet-e egy héten két napon is külső
tanulás? Igen, ha ez a gyerekek tanulását és fejlődését segíti, és a
tanulásszervezők biztonságban meg tudják szervezni ezeket a napokat úgy,
hogy a meghirdetett és folyamatban lévő modulok vezetői hozzájárulásukat
adják.

\hypertarget{tavtanulo-modulok}{%
\subsubsection{Távtanuló modulok}\label{tavtanulo-modulok}}

Amikor már kialakul a gyerekekben az önálló olvasás,
információ-fel\-dol\-go\-zás képessége, akkor képesek a távtanulásra
(e-learning), azaz, hogy online tananyagokat önirányítottan, önállóan
feldolgozzanak. Ettől a kortól kezdve (körülbelül 5.\ évfolyam)
a tanulásszervezők meghirdethetnek online modulokat. Az első időszakban
ajánlatos valamiféle kevert élményt adni a gyerekeknek: önállóan egy
online felületen tanulnak, de együtt vannak, tanári felügyelet mellett,
az iskolában. Később természetesen nincs szükség arra, hogy egy térben
és időben legyenek a gyerekek a tanárral. Ilyenkor a modulnak lehet,
hogy csak egy kis része kerül a napirendbe, amikor együtt átbeszélik az
élményeket a gyerekek. A
3.\ és 4.\ tanulási szakaszban (\ref{harmadik-szakasz-1216-ev}.~fejezet, \pageref{harmadik-szakasz-1216-ev}.~oldal)
lévő gyerekeknek már meghirdethető teljesen online modul is.

\begin{quote}
\textbf{Ki van távol? A gyerek vagy a tanár?} Az iskola egész napos
iskola, és így 9 és 16 óra között a gyerekek csak külön megállapodás vagy
hiányzás esetén vannak távol az iskolától. A távtanulásnak ilyenkor is
van értelme: különböző gyerekek különböző témákról tanulhatnak például
a %\href{https://portal.nkp.hu/}
{\emph{Nemzeti Köznevelési Portál}}
segítségével, amíg csak egy általános, felügyeletet ellátó tanár van
jelen a teremben. Ilyenkor érdekes módon a (szak)tanárnak nem kell az
iskolában lennie.
\end{quote}
