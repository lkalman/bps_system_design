\hypertarget{tankonyvek-kivalasztasa}{%
\section{Tankönyvek kiválasztása}\label{tankonyvek-kivalasztasa}}

A szaktanárok minden esetben maguk választják a modulhoz szükséges
tankönyveket, szoftvereket, weboldalakat és egyéb eszközöket úgy, hogy

\begin{itemize}
\tightlist
\item
  az megfelelő legyen annak a csoportnak, arra a célra, amit el akar
  érni;
\item
  minden esetben legyen mindenki számára elérhető (az esetek többségében
  értsd: ingyenes) megoldás;
\item
  a szaktanárokat bátorítjuk arra, hogy új dolgokat próbáljanak ki,
  és tapasztalataikat az iskola többi tanárával osszák meg.
\end{itemize}

Mivel az iskolában a NAT-ban és a miniszter által közreadott kerettantervben
meghatározott tanulási eredményeket kell elérni, az ehhez szükséges
ismeretek megszerzéséhez a Budapest School az Oktatási Hivatal általi
jegyzékben államilag támogatott, OFI által fejlesztett tankönyveket
veszi alapul. A Budapest School tanárcsapatainak lehetőségük van arra,
hogy ettől eltérő, a mindenkori tankönyvjegyzékben szereplő tankönyvvel
segítsék a tanulási eredmények elérését. És arra is lehetőségük van, hogy
egyáltalán ne használjanak tankönyvet, mert sokszor az internet elegendő
információt tartalmaz.

A Budapest School programjának alapja, hogy a gyerekek egyéni céljaira
szabott tanulási terveket készít. Ennek előfeltétele, hogy a könyvek
használata is ehhez kapcsolódó módon, rugalmasan történjen, minden
esetben az adott tanulási modul igényeihez szabva. Ennek érdekében a
program pedagógusai folyamatosam állítják össze a gyerekek eltérő
céljaihoz és képességszintjeihez igazodó differenciált tevékenységek és
feladatsorok rendszerét.

A Budapest School-iskolákban minden (modul)csoport csak olyan\break
tankönyvet választhat, amelyik minden család számára elérhető. Ha valamelyik család
nem tudja a könyvet magának megvásárolni, akkor a csoport többi tagja
megvásárolja neki.
