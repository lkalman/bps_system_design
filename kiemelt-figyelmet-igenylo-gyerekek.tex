\hypertarget{a-kiemelt-figyelmet-igenylo-gyerekek}{%
\section{A kiemelt figyelmet igénylő
gyerekek}\label{a-kiemelt-figyelmet-igenylo-gyerekek}}

A kiemelt figyelmet igénylő gyerekek támogatásához elsődlegesen a
tanulásszervezőknek, a mentoroknak és a szaktanároknak kell
differenciáltan, sokszínűen, türelemmel és személyre szabottan
foglalkozni.

\begin{itemize}
\tightlist
\item
  A mentoroknak ismerniük kell a gyerek sajátosságait. A szülőnek és a
  mentornak őszintén, egymást segítve kell a gyerek egyéni támogatására
  felkészülniük. Ehhez sokszor külső diagnózisra, szakember bevonására és
  a felnőttek közötti nehéz beszélgetésekre van szükség.
\item
  A mentoroknak meg kell osztaniuk a gyerekek sajátos igényeit a többi
  tanárral, a szaktanárokkal, hogy ők is fel tudjanak készülni a
  gyerekek személyre szabott támogatására.
\item
  A tanulásszervezésnek, a moduloknak, a foglalkozásoknak, azaz a
  gyerekek teljes tanulási élményének differenciáltnak, egyéniesítettnek
  kell lennie, hogy az eltérő képességekkel bíró gyerekek is tudjanak
  együtt tanulni. Ahogy egy nagycsaládban is figyelünk a különböző
  gyerekek eltérő igényeire, képességszintjeire.
\item
  A szemléltetésnek, a tevékenységeknek sokoldalúaknak kell lenniük,
  sokféle feladatot, speciális eszközöket kell használni. A gyerekek
  élménye változatos kell, hogy legyen.
\item
  Külső szakemberekkel kell konzultálni, és a diagnózisban szereplő
  javaslatokat be kell építeni a mindennapok tervezésébe. Ebben a
  szülőnek és a tanároknak szorosan együtt kell működniük.
\item
  Az egyéni haladási ütem biztosítására egyéni fejlesztési és tanulási
  tervet kell készíteni.
\item
  A tanároknak együtt kell működniük a gyermek/tanuló fejlesztésében
  részt vevő szakemberekkel.
\end{itemize}

A fenti listából a legfontosabb elem: a tanár---gyerek---szülő hármas
mellé be kell általában hívni külső, a gyerek igényeit jól értő
szakembert támogatónak. A Budapest School-iskola csak akkor tud
segíteni, ha minden fél tudatosan áll hozzá a kiemelt figyelem
szükségletéhez.
