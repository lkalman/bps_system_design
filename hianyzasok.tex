\hypertarget{hianyzasok-mulasztasok-igazolasok-kesesek}{%
\section{Hiányzások, mulasztások, igazolások,
késések}\label{hianyzasok-mulasztasok-igazolasok-kesesek}}

A Budapest School feladata, hogy olyan környezetet biztosítson a
gyerekeknek, amiben boldogak, felszabadultak, magabiztosak és hatékonyak
tudnak lenni. A Budapest School-családok maguk és önszántukból választják
ezt az iskolát, sok időt és energiát áldoznak arra, hogy az iskolában
tudjanak tanulni. \emph{Ezért az iskola feltételezi, hogy a gyerekek
önszántukból akarnak az
iskolában tanulni}.
\newpage
Sok oka lehet annak, hogy egy gyerek mégsincs az iskolában. Például

\begin{itemize}
\tightlist
\item
  betegnek, fáradtnak érezheti magát, fizikailag vagy lelkileg kimerült,
  vagy lehet valamilyen fertőző betegsége;
\item
  családjával tölt értékes, minőségi időt, mert a fejlődését ez szolgálja
  a leginkább;
\item
  előre nem tervezett esemény miatt nem tud az iskolába menni;
\item
  utazik, felfedez, külső helyszínre szervezett tanulási programokon
  vesz részt;
\item
  egy projektjébe úgy belemerül, hogy érdemesnek találja fókuszáltan, nem az
  iskolában végezni a munkát.
\end{itemize}

A fenti példák is két jól elkülöníthető kategóriába sorolhatók. A
\emph{nem tervezett hiányzástól} jól elkülöníthetők azok az esetek,
amikor a gyerek, bár nem az iskolában tartózkodik, mégis szervezett,
strukturált módon biztosítva van a fejlődése, tanulása. Ezt az esetet a
„távmunka'' mintájára \emph{„távtanulásnak''} hívja az iskola,
mert ezt az időt is tanulásra szánjuk.

Alapelv, hogy \emph{a mentornak, gyereknek és szülőnek meg kell
állapodniuk a távtanulásról}. Minden félnek tudnia kell róla, meg kell
előre tervezniük, nem lehet esetleges. Mindenképpen meg kell
különböztetni a \emph{nem tervezett hiányzástól}.

\hypertarget{tervezett-tavtanulas}{%
\paragraph{Tervezett távtanulás}\label{tervezett-tavtanulas}}

Ez azt jelenti, hogy előre eltervezett módon, valamilyen program miatt nincs a gyerek az
iskolában. Ilyenkor a mentor és a gyerek megtervezi a tanulás célját,
várható eredményeit. A terv létrejöttéért a gyerek és a szülő felelős,
és minden félnek el kell fogadnia a tervet. Tehát a mentornak hozzá kell
járulnia. Ha a mentor nem járul hozzá, akkor addig nem kezdhető meg a
távtanulás, amig megállapodás nem születik. Ha mégis, akkor azt nem
tervezett hiányzásnak kell tekinteni.

Ha a tervezett távtanulás elérte a 20 napot vagy 160 órát, akkor a
mentor mellett egy másik mentorszerepben dolgozó tanulásszervezőnek is\break
meg kell ismernie és el kell fogadnia a tervet. 40 nap felett három
mentornak kell együtt elfogadnia a tervet, melyek egyike egy másik
tanulóközösség mentora.

\hypertarget{nem-tervezett-hianyzas}{%
\paragraph{Nem tervezett hiányzás}\label{nem-tervezett-hianyzas}}

Ilyenkor a gyerek és a mentortanár nem tud előre felkészülni az iskolán kívüli
tanulásra, mert a hiányzás előző nap vagy aznap derül ki, vagy más okból
a megállapodás nem jön létre. A szülő feladata, hogy még ebben az
esetben is erről reggel 9 óra előtt értesítse a mentortanárt.
Mikroiskolánként eltérhet a preferált kommunikációs eszköz, ezért a
tanulásszervezők feladata meghatározni, milyen értesítési
formát kérnek.

Egy tanévben 15 munkanap vagy 120 óra, de alkalmanként csak 5 munkanap
nem megtervezett hiányzást igazolhat a szülő (rögzített és dokumentált
módon). Orvos által igazolt betegség, hatósági intézkedés és egyéb
alapos indok esetén a 20/2012. (VIII.~31.) EMMI-rendelet 51.~§~(2)
értelmében igazoltnak kell tekinteni a hiányzást.

\hypertarget{igazolatlan-hianyzas}{%
\paragraph{Igazolatlan hiányzás}\label{igazolatlan-hianyzas}}

Ez az az eset, amikor a szülő vagy a mentortanár nem tudott a hiányzásról,
nem volt előre megtervezve, vagy a 15 napos, 120 órás keret kimerült.
Ebben az esetben az iskola szigorúbban jár el, mint a legtöbb más
iskola. Ilyenkor a 20/2012. (VIII.~31.) EMMI-rendelet 51.~§~(3) pontja
értelmében minden esetben az iskola értesíti a szülőt, és 10 igazolatlan
óra után figyelmezteti, hogy a következő igazolatlan után \emph{„az
iskola a gyermekjóléti szolgálat közreműködését igénybe véve megkeresi a
tanuló szülőjét"}. Az iskola megközelítése egyszerű: mivel partneri
viszonyban van a tanár, a gyerek és a szülő, ezért az az alapértelmezés,
hogy vagy előre meg lehetett volna beszélni a hiányzásokat, amely
esetben tervezett távtanulásról beszélnénk, vagy betegség miatt kellett
túllépni a 15 napot. Elég tág keretet enged az iskola. Abban az esetben
azonban, amikor a gyerek vagy a szülő nem tartja be a kereteket, nem él
a partneri viszonnyal, akkor ott valami baj van. Gyorsan kell reagálni.

\hypertarget{hogyan-biztositja-a-rendszer-a-visszaelesek-kikuszoboleset}{%
\subsubsection{Hogyan biztosítja a rendszer a visszaélések
kiküszöbölését?}\label{hogyan-biztositja-a-rendszer-a-visszaelesek-kikuszoboleset}}

A Budapest Schoolba járó gyerek szülei és tanárai egyetlen igazolatlan
óra hiányzás után értesítést kapnak arról, hogy a gyerek nem jelent meg
az iskolában. Tehát a gyerek nem tud a szülei tudta nélkül távolmaradni.

A szülő 15 napon keresztül „igazolhat'' nem tervezett hiányzást, hogy
ne kelljen minden náthánál a körzeti orvosi rendszert terhelni, ahogy
azt a Házi Gyermekorvosok Egyesülete javasolja. Miért feltételezzük,
hogy nem él ezzel vissza a szülő és a gyerek? Mert az iskolában maradás
feltétele a tanulás és a folyamatosan újra felállított tanulási célok
követése. Az ezzel való visszaélés az iskola céljaival ellentétes, és
legkésőbb a soron következő trimeszter tanulási szerződésekor a
felszínre kerül.

A távtanulást pedig nem tekinti az iskola hiányzásnak, mert a tanulás
folytatólagos, dokumentált, megtervezett. A portfóliók bővülését pedig
folyamatosan monitorozza az iskola. Előfordulhat, hogy a mentortanár és a gyerek tévesen
méri fel a helyzetet, és hogy tanulásnak, fejlődésnek
látnak valamit, ami nem az. Ezért került a rendszerbe a „külső
megfigyelő'' kitétel, hogy 20 nap után új tanárt kell bevonni a döntésbe.
Ha a gyerek tanulási veszélybe kerül, akkor az a tanulási eredmények
elmaradásából fél éven belül felfedezhető.

\hypertarget{kesesek-kezelese}{%
\subsection{Késések kezelése}\label{kesesek-kezelese}}

A Budapest School-tanulóközösségek maguk állítják fel a napirenddel\break
kapcsolatos kereteket: mikor kezdenek, meddig tartanak a strukturált
foglalkozások, mikor vannak a szünetek, és hogyan kezdődik újra a nap
folyamán a fókuszált munka. A kereteket a tanulásszervező tanárok
feladata kialakítani és a trimeszterek megkezdése előtt kihirdetni.

Fontos, megbeszélendő részlet, hogy hogyan kezeli a közösség a
késéseket: mikortól lehet érkezni, mikor kezd a közösség annyira
dolgozni, hogy zavaró, ha valaki a belépésével megzavarja a folyamatot.
Megállapodást köt a közösség, hogy hogyan kívánja kezelni a késéseket,
mi segíti a csapatot leginkább a céljai elérésében.

Az iskola nem regisztrálja a késéseket, mert az iskola nem tudhatja,
hogy egy-egy késés elfogadható-e a közösségnek, vagy nem. Egy színdarab
főpróbájáról 5 percet késni mást jelent a közösség számára, mint arról
az óráról, ahol mindenki egyedül füllhallgatóval böngészi egy online
tananyag számára legrelevánsabb fejezetét.

Ha egy csoportot megzavar valakinek az ismételt késése, akkor
konfliktus alakul ki a csoport és a késő vagy a tanár és a késő
között. Ezt a típusú konfliktust (is) a konfliktusko kezeléséről szóló
pont
(\ref{konfliktusok-feszultsegek-kezelese}., \apageref{konfliktusok-feszultsegek-kezelese}.~oldal)
szerint kell feloldani.
