\hypertarget{tanarok-kivalasztasa-tanulasa-fejlodese-es-ertekelese}{%
\section{Tanárok kiválasztása, tanulása, fejlődése
és értékelése}\label{tanarok-kivalasztasa-tanulasa-fejlodese-es-ertekelese}}

Az iskola legmeghatározóbb összetevői a tanárok. Ezért a Budapest School
külön figyelmet fordít arra, hogy ki lehet tanár az iskolákban, és
hogyan segítjük az ő fejlődésüket.

\paragraph{Alapelveink}
        
\begin{enumerate}
\def\labelenumi{\arabic{enumi}.}
\tightlist
\item
  Minden tanárnak tanulnia kell. Amit ma tudunk, az nem biztos, hogy
  elég arra, hogy a holnap iskoláját működtessük. És az is biztos, hogy
  még sokkal hatékonyabban lehetne segíteni a gyerekek tanulását, mint
  amilyenek a ma ismert módszereink.
\item
  A tanároknak csapatban kell dolgozniuk, mert összetett
  (interdiszciplináris) tanulást csak vegyes összetételű (diverz)
  csapatok tudnak támogatni.
\item
  A szakképesítés nem szükséges és nem elégséges feltétele annak, hogy a
  Budapest Schoolban valaki jól teljesítő tanár legyen.
\end{enumerate}

\hypertarget{felvetel}{%
\paragraph{Felvétel}\label{felvetel}}

A Budapest School tanulásszervező tanárainak felvétele egy legalább
háromlépcsős folyamat, ahol vizsgálni kell a tanár egyéniségét
(attitűdjét), felnőtt---felnőtt kapcsolatokban a viselkedésmódját (társas
kompetenciáit), és minden jelöltnek próbafoglalkozást kell tartania,
amit az erre kijelölt Budapest School-tanárok megfigyelnek. A felvételi
folyamatot a fenntartó felügyeli és irányítja.

\hypertarget{sajat-cel}{%
\paragraph{Saját cél}\label{sajat-cel}}

Minden tanárnak van saját, egyéni fejlődési célja: \emph{mitől tudok én
jobb tanár lenni, jobban támogatni a gyerekek tanulását, segíteni a
munkatársaimat és partnerként dolgozni a szülőkkel?}

\hypertarget{tanarok-mentora}{%
\paragraph{Tanárok mentora}\label{tanarok-mentora}}

A gyerekekhez hasonlóan minden tanárnak van mentora, aki segíti a
saját céljainak kialakításában, és folyamatosan támogatja ezek elérésében.

\hypertarget{tanarok-ertekelese}{%
\paragraph{Tanárok értékelése}\label{tanarok-ertekelese}}

Minden tanárt évente legalább kétszer értékelnek a munkatársai. Ez az a
folyamat, amit 360 fokos értékelésnek hívnak az üzleti szférában. A
visszajelzések feldolgozása után a saját célokat frissíteni kell.

Minden tanárt értékelnek a szülők is (kifejezetten a mentorált gyerekek
szülei) és a gyerekek is legalább évente kétszer.
