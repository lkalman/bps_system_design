\vspace*{.5ex}
\hypertarget{szervezet-architekturaja}{%
\section{A szervezet architektúrája}\label{szervezet-architekturaja}}

Az iskolák hagyományosan hierarchikus vezetési és munkaszervezési
elveket követnek. Főnök---beosztott viszonyok, felülről jövő döntések és
autoritás határozza meg a mindennapi folyamatokat. Az állandóság és a
biztos siker érdekében ezek az iskolák a centralizált hatalmat
alkalmazzák: a főnök dönt, a beosztottak végrehajtanak. A Budapest
School a mai világra reflektálva egy dinamikusabb, gyors változásokat
támogató modellt dolgozott ki az egyes tanulóközösségek hatékony és
mégis decentralizált működtetésére.

Olyan kihívásokra reflektál ez a működési mód amelyek nagyobb\break
komplexitást, transzparenciát és szélesebb körű kommunikációs
lehetőségeket kívánnak meg. Így rövidülhet a döntéshozatal, a gazdasági,
a környezetet és a tanulás körülményeit érintő kérdésekben pedig
növekszik az érintettek felhatalmazása és bevonódása. A Budapest School
egy olyan modell működtetését határozza meg, amely önfenntartó,
környezettudatos és etikus. Támogatja a munkavállalók kreativitását és
lelkesedését, és egy agilis, gyorsan változó és fejlődő szervezet
épülését segíti.

\vspace*{.5ex}
\hypertarget{a-budapest-school-szervezeti-modellje-szociokracia}{%
\subsection{A Budapest School szervezeti modellje,
szociokrácia}\label{a-budapest-school-szervezeti-modellje-szociokracia}}

A szociokrácia egy olyan új „szociális technológiát'' biztosít a
szervezetek irányítására és működtetésére, amelyet a hagyományos,
hierarchikus szervezetektől eltérő alapvető szabályok határoznak meg.


A célirányos és változásra fogékony
szervezetek számára kidolgozott öngazdálkodó gyakorlatról van szó.

A megközelítés ---~az ,,agilis'' és a ,,lean'' megoldásokhoz hasonlóan~--- „just in
time'', vagyis épp a kellő pillanatban reagál a felmerülő változásokra és
lehetőségekre. Mindezt pedig a vállalat minden szintjén önállóan teszi.

Az iskolai szervezet alapelemei a következők:

\begin{itemize}
\item
  egy keretrendszer, ami lefekteti a „játékszabályokat'', és újraosztja a
  hatalmat,
\item
  egy módszer a szervezet újrastrukturálásához és az emberek szerepeinek
  és önállóságának meghatározásához, és
\item
  egy egyedülálló döntéshozatali folyamat a hagyományos döntési
  folyamatok felfrissítéséhez.
\end{itemize}

\hypertarget{dinamikusan-valtozo-szerepek}{%
\subsection{Dinamikusan változó
szerepek}\label{dinamikusan-valtozo-szerepek}}

A klasszikus iskolában minden alkalmazott rendelkezik egy munkaköri
leírással, mely felsorolja feladatait és hatáskörét. Ez általában a
legkevésbé sincs összhangban a nap mint nap végzett feladatokkal. A
Budapest School iskolában a dolgozók gyakran számos szereppel
rendelkeznek, és ezeket akár teljesen különböző
munkacsapatokon belül is betölthetik. A szerepek folyamatos alakuláson mennek
keresztül, és a csapat résztvevői frissítik őket. A szerepkörök tehát
nincsenek szorosan egy emberhez kötve: mindig az látja majd el, aki
ért hozzá, van szabad kapacitása és elvállalja. Ezzel pedig sok
személyes konfliktus kerülhető el.

Ahogy például \emph{a fociban is
pontosan tudod, hogy a labdát a csatárnak kell passzolnod}.
Nem azért, mert jóban vagytok, hanem mert ő van a legjobb helyzetben,
hogy gólt lőjön a csapatnak. Még ha nem is vagy vele jóban, nem kedveled
vagy ki nem állhatod, akkor is neki fogsz passzolni, mert a játék
stratégiája szerint így kell tenned. Ugyanígy az iskolánkban az egyes
szerepek rendelkeznek hatalommal, nem pedig a személyek. Ez azt is
jelenti, hogy a szerepek és a hatalom állandóan változhatnak anélkül,
hogy a játékszabályokat megsértenénk. Nincs állandó főnök---beosztott
viszony.

\hypertarget{szetosztott-hatalom}{%
\subsection{Szétosztott hatalom}\label{szetosztott-hatalom}}

A hagyományos iskolai struktúrában az igazgató és a vezetőség tagjai nem
delegálnak hatalmat. Minden döntés átmegy az ő kezeiken. A
Budapest School-iskolában a hatalmat ténylegesen elosztjuk, így a
hierarchia helyett egymással kapcsolatban lévő és kommunikáló, de
önállóan döntő csapatok (úgynevezett körök) a döntéshozók. A körök
közötti kapcsolat jelentősen meg tudja növelni az iskola alkalmazkodóképességét.

\hypertarget{allando-tyuklepesek}{%
\subsection{Állandó tyúklépések}\label{allando-tyuklepesek}}

Budapest School-iskolában a szervezeti struktúrát minden hónapban
felülvizsgálhatják: megvizsgálják, hogy az adott szerepek milyen
feladatokkal és döntésekkel járnak. A változások tömérdek apró lépésben
történnek, szünet nélkül, folyamatosan. Így a csapat kihasználhatja a
lehetőségeket, hogy tanuljon az esetleges hibákból, fejlessze önmagát, és
tökéletesítse a folyamatokat. Ahogy Alexis Gonzales-Black, a Zappos
munkatársa mondja (ez a cég talán a legelsők között csatlakozott a
,,holakrácia'' nevű rendszernek a valóságba való
átültetéséhez), „a holakrácia nem fogja megoldani helyetted a
problémáidat; viszont jó eszköz arra, hogy te saját magad megoldhasd őket.''

\hypertarget{mindenki-szamara-atlathato-szerepek}{%
\subsection{Mindenki számára átlátható
szerepek}\label{mindenki-szamara-atlathato-szerepek}}

„Mi mindig így csináljuk'' --- hangzik el számos iskolában a válasz, hogy
ez vagy az miért éppen így történik. Gyakran senki sem tudja, hogy az
adott szabály miért létezik, mi célt szolgál, ki döntött róla, vagy ki
tudná megváltoztatni. Ez pedig a hatalom szétosztását lehetetlenné
teszi. A Budapest School iskolában
\emph{a hatalom nem a csoport élén álló vezetők kezében van,}
hanem az expliciten definiált folyamatokhoz kötődik. Ezek a
„játékszabályok'' mindenki számára elérhetők és ismertek, legyen szó
akár régi motorosokról, akár újoncokról.

A Budapest School modellel tehát nem káoszt és hatalmi játszmákat
generálunk, hanem
\emph{eltöröljük a hagyományos hierarchiából fakadó lassú folyamatokat,}
az átláthatatlan felelősségi köröket, és szabad utat engedünk a
kreativitásnak, az önállóságnak. Azáltal, hogy felhatalmazza az
embereket arra, hogy értelmes döntéseket hozhassanak, és részt vehessenek
a változásban, a holakrácia felszabadítja a szervezet kihasználatlan
erőforrásait.
