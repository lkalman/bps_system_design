\hypertarget{a-tanulasi-szerzodes}{%
\section{A tanulási szerződés}\label{a-tanulasi-szerzodes}}

A tanulási szerződés a gyerek, a mentor és a szülő közötti, korábban
említett megállapodás, ami rögzíti

\begin{enumerate}
\def\labelenumi{\arabic{enumi}.}
\tightlist
\item
  a gyerek, a mentor (iskola) és a szülő igényeit, elvárásait; ezek
  lehetnek: \emph{„szeretném, ha a gyerekem naponta olvasna''} típusú
  folyamatra vonatkozó kérések, vagy erősebb \emph{„változtatnod kell a
  viselkedéseden, ha a közösségben akarsz maradni''} típusú igények, határok
  megfogalmazása;
\item
  a gyerek céljait a következő trimeszterre, vagy a tanév végéig;
\item
  a gyerek, mentorok (iskola) és szülő vállalásait, amivel támogatják a
  cél elérését és a felek igényének teljesülését.
\end{enumerate}

A tanulási szerződésre jellemző:

\begin{itemize}
\tightlist
\item
  A kitűzött célokat minél specifikusabban és mérhetőbben kell
  megfogalmazni. Javasolt az
  %\href{https://en.wikipedia.org/wiki/OKR}
  {\emph{OKR}}
  (Objectives and Key
  Results, azaz Cél és Kulcseredmények) vagy a
  % \href{https://en.wikipedia.org/wiki/SMART_criteria}
  {\emph{SMART}} (Specific,
  Measurable, Achievable, Relevant, Time-bound, azaz Specifikus,
  Mérhető, Elérhető, Releváns és Időhöz kötött) technika alkalmazása,
  hogy minél specifikusabb, teljesíthetőbb, tervezhetőbb és könnyen
  mérhető célokat tűzzenek ki.
\item
  A kitűzött célokban való megállapodást követően megállapodást\break
  kell
  kötni arról is, hogy ki és mit tesz azért, hogy a gyerek a célokat
  elérje.
\item
  A mentor a teljes tanulóközösséget (a többi tanárt, a közösséget)
  képviseli a megállapodás során.
\end{itemize}

A tanulási szerződést néha hívjuk \emph{megállapodásnak} is. A
megállapodás és szerződés szavakat az iskola szinonimának tekinti. A
\emph{learning contract} az önirányított tanulást hangsúlyozó
felnőttképzéssel foglalkozó irodalomban bevett szakkifejezés már a 80-as
évektől {\autocite{Knowles1977}}.

Egy másik szakterületen, a pszichoterápiás munkában a terápiás
szerződések megkötésekor a közös munka kereteinek kialakítását és
fenntarthatóságát hangsúlyozzák {\autocite{Perczel2009}}. Erre is
utalunk a tanulási szerződés elnevezéssel. És van, amikor a \emph{hármas
szerződés} kifejezést használjuk, hangsúlyozva, hogy mind a három
szereplőnek, gyereknek, tanárnak és szülőnek elfogadhatónak kell
tartania a szerződés tartalmát.
