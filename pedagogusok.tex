\hypertarget{pedagogusokra-vonatkozo-eloirasok}{%
\section{Pedagógusokra vonatkozó
előírások}\label{pedagogusokra-vonatkozo-eloirasok}}

Az iskolában többféle tanárszerepben lehet valaki pedagógus munkakörben.

\begin{itemize}
\tightlist
\item
  A gyerekek mentora (\ref{mentor}.~fejezet, \pageref{mentor}.~oldal),
  az a felnőtt, aki segíti célja elérésében, figyel jóllétére és
  kitüntetett figyelmet ad a gyerekek számára.
\item
  A
  szaktanárok (\ref{szaktanarok}.~fejezet, \pageref{szaktanarok}.~oldal)
  azok a tanárok, akik valami specifikus képesség, tudás elsajátításában
  és tevékenységben kísérik a gyerekekek, így például a tanulási
  eredmények elérésében.
\item
  \emph{Mentortanárnak} hívja az iskola azokat a pedagógusokat, akik
  mentorok és szaktanárok is.
\item
  Az iskola mentora és szaktanárai közül kerülnek ki a
  tanulóközösségeket vezető tanárcsapatok tagjai, a
  tanulásszervezők (\ref{tanulasszervezo}.~fejezet, \pageref{tanulasszervezo}.~oldal).
\end{itemize}

\hypertarget{biztonsagi-eloirasok}{%
\subsection{Biztonsági előírások}\label{biztonsagi-eloirasok}}

Ahhoz, hogy az iskola törekedjen a gyermekek számára biztonságos
környezet kialakítására minden pedagógus munkakörben alkalmazott tanár,
valamint minden, a gyerekek körül és az épületekben általában feladatot
ellátó személy esetén meg kell bizonyosodni, hogy

\begin{itemize}
\tightlist
\item
  cselekvőképes,
\item
  büntetlen előéletű és nem áll a tevékenység folytatását kizáró
  foglalkozástól eltiltás hatálya és
\item
  referenciát adó személy vagy munkáltató igazolta az alkalmazott
  képességeit és rátermettségét a feladatra.
\end{itemize}

\hypertarget{elfogadott-vegzettsegek-es-szakkepzettsegek}{%
\subsection{Elfogadott végzettségek és
szakképzettségek}\label{elfogadott-vegzettsegek-es-szakkepzettsegek}}

Fontos alapelv, hogy a tanárok személyisége, tudása, képességei és
kompetenciái határozzák meg a gyerekek élményét. Ezért az és csak az
lehet tanár, aki képes segíteni a gyerekeket a tanulásban. A végzettség
a tanárok hatékonyságának egyik indikátora. A különböző tanári
szerepekhez különböző képességekre van szükség és így különböző
végzettségek lehetnek a képesség indikátorai.

\begin{longtable}[]{@{}ll@{}}
\toprule
\begin{minipage}[b]{0.18\columnwidth}\raggedright
Pedagógus-munkakör\strut
\end{minipage} & \begin{minipage}[b]{0.76\columnwidth}\raggedright
Az alkalmazáshoz szükséges feltétetel\strut
\end{minipage}\tabularnewline
\midrule
\endhead
\begin{minipage}[t]{0.18\columnwidth}\raggedright
Mentor\strut
\end{minipage} & \begin{minipage}[t]{0.76\columnwidth}\raggedright
A köznevelési jogszabályok által pedagógus munkakörben alkalmazható + 30
órás mentor képzés\strut
\end{minipage}\tabularnewline
\begin{minipage}[t]{0.18\columnwidth}\raggedright
Szaktanár\strut
\end{minipage} & \begin{minipage}[t]{0.76\columnwidth}\raggedright
Nkt. előírása alapján\strut
\end{minipage}\tabularnewline
\begin{minipage}[t]{0.18\columnwidth}\raggedright
Mentortanár\strut
\end{minipage} & \begin{minipage}[t]{0.76\columnwidth}\raggedright
Mentor + szaktanár\strut
\end{minipage}\tabularnewline
\begin{minipage}[t]{0.18\columnwidth}\raggedright
Tanulásszervező tanár\strut
\end{minipage} & \begin{minipage}[t]{0.76\columnwidth}\raggedright
A köznevelési jogszabályok által pedagógus munkakörben alkalmazható + 30
órás mentor képzés\strut
\end{minipage}\tabularnewline
\bottomrule
\end{longtable}

\hypertarget{intezmenyvezeto-es--helyettes}{%
\subsection{Intézményvezető és
-helyettes}\label{intezmenyvezeto-es--helyettes}}

Az intézmény vezetését intézményvezető látja el, amely szerepkör
ellátása elsősorban a tanügyigazgatási rendszer követelményeinek való
megfelelés miatt szükséges. A Budapest Schoolban az intézményvezető nem
köteles órát, foglalkozást tartani.

Az iskola belső működéséből és szervezeti kultúrájából következik, hogy
iskolában intézményvezető-helyettes megbízása nem szükséges. Az
intézményvezető munkája a gyerekek létszámával nem növekszik, feladatai
elvégzésében segíthetik más tanárok, megbízott szakértők, és a fenntartó
is.

Amennyiben az intézményvezető akadályoztatva van, vagy munkaköre
megürült, a fenntartó jogosult olyan személy ideiglenes megbízására, aki
pedagógus végzettséggel és legalább 2 éves, igazolt vezetői
tapasztalattal rendelkezik. A fenntartónak ebben az esetben azonnal ki
kell nevezni az ideiglenes intézményvezetőt.

\hypertarget{kotott-es-kotetlen-munkaido-szabalyozasa}{%
\subsection{Kötött és kötetlen munkaidő
szabályozása}\label{kotott-es-kotetlen-munkaido-szabalyozasa}}

A Budapest School iskolákban alkalmazott tanárok munkaviszonyban vagy
megbízási jogviszonyban állnak. A munkaviszonyban álló tanárok rendes
vagy kötetlen munkaidőben dolgoznak.

Az iskola megközelítése, hogy a tanárok a NAT, a saját és közösségi
célok által kialakított eredmények elérésére vállalnak kötelezettséget.
Ezért az Nkt. és kapcsolódó jogszabályok által előírt munkaidő-beosztási
és nyilvántartási szabályoktól a jelen modell eltérést enged. Az
eltérésekben az iskolának és a tanároknak meg kell egyezniük, és a
részleteknek a munka- illetve megbízási szerződésekben meg kell
jelenniük.

A tanulóközösség tanárcsapatának a feladata a szükséges óraszámok,
beosztások kialakítása és a megfelelő szaktanárok megtalálása.

\hypertarget{tanar---gyerek-arany}{%
\subsection{Tanár - gyerek arány}\label{tanar---gyerek-arany}}

Egy mentor egyik legnagyobb ajándéka a mentoráltja számára, a minőségi
figyelme. Odafigyel rá, meghallgatja, egyénileg támogatja, együttérez
vele, kapcsolatait és tudását átadja számára. Hogy ennek a kapcsolatnak
a minőségét ne veszélyeztessük, a BPS modell megköti, hogy egy tanár
maximum hány mentorálttal dolgozhat. Az első négy évfolyamon maximum 15,
a felső négy évfolyamon 20 és 8-12. évfolyamon 25 gyerek tartozhat
maximum egy mentorhoz.

Tanulásszervező tanárcsapat minimális mérete a tanulóközösség közösség
méretétől függ. Egy egy közösségben 20 gyerekenként legalább egy
tanulásszervezőnek kell dolgoznia.

Szaktanárok esetén ilyen arányszámot nehéz megadni. Egy kevert (blended)
tanulási környezetben egy szaktanár akár több száz tanulóval tud
egyszerre foglalkozni egy foglalkozás keretében. Egy kémiai kisérletező
modul során nem ajánlott 15 embernél több a laborban.
