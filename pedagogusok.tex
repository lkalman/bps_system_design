\hypertarget{pedagogusokra-vonatkozo-eloirasok}{%
\section{Pedagógusokra vonatkozó
előírások}\label{pedagogusokra-vonatkozo-eloirasok}}

Az iskolában többféle tanárszerepben lehet valaki pedagógus-mun\-ka\-kör\-ben.

\begin{itemize}
\item
  A gyerekek mentora (\ref{mentor}.~fejezet, \pageref{mentor}.~oldal)
  az a felnőtt, aki segíti céljaik elérésében, figyel jóllétükre, és
  kitüntetett figyelmet ad nekik.
\item
  A
  szaktanárok (\ref{szaktanarok}.~fejezet, \pageref{szaktanarok}.~oldal)
  azok a tanárok, akik valamilyen specifikus képesség, tudás elsajátításában, így például a tanulási
  eredmények elérésében
  és tevékenységükben kísérik a gyerekeket.
\item
  \emph{Mentortanárnak} hívja az iskola azokat a pedagógusokat, akik
  mentorok és szaktanárok is.
\item
  Az iskola mentora és szaktanárai közül kerülnek ki a
  tanulóközösségeket vezető tanárcsapatok tagjai, a
  tanulásszervezők (\ref{tanulasszervezo}.~fejezet, \pageref{tanulasszervezo}.~oldal).
\end{itemize}

\hypertarget{biztonsagi-eloirasok}{%
\subsection{Biztonsági előírások}\label{biztonsagi-eloirasok}}

Ahhoz, hogy az iskola törekedjen a gyermekek számára biztonságos
környezet kialakítására, minden pedagógus-munkakörben alkalmazott tanár,
valamint minden a gyerekek körül és az épületekben általában feladatot
ellátó személy esetén meg kell bizonyosodni arról, hogy

\begin{itemize}
\tightlist
\item
  cselekvőképes,
\item
  büntetlen előéletű, és nem áll a tevékenység folytatását kizáró
  foglalkozástól eltiltás hatálya alatt, és
\item
  referenciát adó személy vagy munkáltató igazolta a
  képességeit és a feladatra való rátermettségét.
\end{itemize}

\hypertarget{elfogadott-vegzettsegek-es-szakkepzettsegek}{%
\subsection{Elfogadott végzettségek és
szakképzettségek}\label{elfogadott-vegzettsegek-es-szakkepzettsegek}}

Fontos alapelv, hogy a tanárok személyisége, tudása, képességei és
kompetenciái határozzák meg a gyerekek élményét. Ezért az és csak az
lehet tanár, aki képes segíteni a gyerekeket a tanulásban. A végzettség
a tanárok hatékonyságának egyik indikátora. A különböző tanári
szerepekhez különböző képességekre van szükség, és így különböző
végzettségek lehetnek a képesség indikátorai.

\begin{longtable}[]{@{}>{\begin{minipage}{.2\textwidth}\raggedright\hangindent
.5em\strut}l<{\strut\end{minipage}}>{\begin{minipage}{.75\textwidth}\hangindent
.5em\strut}l<{\strut\end{minipage}}@{}}
\bfseries Pedagógus-munkakör & 
\bfseries Az alkalmazáshoz szükséges feltétetel\tabularnewline
\hline
Mentor & 
A köznevelési jogszabályok által pedagógus munkakörben alkalmazható + 30
órás mentor képzés\tabularnewline
Szaktanár &
Nkt. előírása alapján\tabularnewline
Mentortanár &
Mentor + szaktanár\tabularnewline
Tanulásszervező tanár &
A köznevelési jogszabályok által pedagógus munkakörben alkalmazható + 30
órás mentor képzés\tabularnewline
\hline
\end{longtable}



%% \begin{longtable}[]{@{}ll@{}}
%% \toprule
%% \begin{minipage}[b]{0.18\columnwidth}\raggedright
%% Pedagógus-munkakör\strut
%% \end{minipage} & \begin{minipage}[b]{0.76\columnwidth}\raggedright
%% Az alkalmazáshoz szükséges feltétetel\strut
%% \end{minipage}\tabularnewline
%% \midrule
%% \endhead
%% \begin{minipage}[t]{0.18\columnwidth}\raggedright
%% Mentor\strut
%% \end{minipage} & \begin{minipage}[t]{0.76\columnwidth}\raggedright
%% A köznevelési jogszabályok által pedagógus munkakörben alkalmazható + 30
%% órás mentor képzés\strut
%% \end{minipage}\tabularnewline
%% \begin{minipage}[t]{0.18\columnwidth}\raggedright
%% Szaktanár\strut
%% \end{minipage} & \begin{minipage}[t]{0.76\columnwidth}\raggedright
%% Nkt. előírása alapján\strut
%% \end{minipage}\tabularnewline
%% \begin{minipage}[t]{0.18\columnwidth}\raggedright
%% Mentortanár\strut
%% \end{minipage} & \begin{minipage}[t]{0.76\columnwidth}\raggedright
%% Mentor + szaktanár\strut
%% \end{minipage}\tabularnewline
%% \begin{minipage}[t]{0.18\columnwidth}\raggedright
%% Tanulásszervező tanár\strut
%% \end{minipage} & \begin{minipage}[t]{0.76\columnwidth}\raggedright
%% A köznevelési jogszabályok által pedagógus munkakörben alkalmazható + 30
%% órás mentor képzés\strut
%% \end{minipage}\tabularnewline
%% \bottomrule
%% \end{longtable}

\hypertarget{intezmenyvezeto-es--helyettes}{%
\subsection{Intézményvezető és
-helyettes}\label{intezmenyvezeto-es--helyettes}}

Az intézmény vezetését intézményvezető látja el, amely szerepkör
ellátása elsősorban a tanügyigazgatási rendszer követelményeinek való
megfelelés miatt szükséges. A Budapest Schoolban az intézményvezető nem
köteles órát, foglalkozást tartani.

Az iskola belső működéséből és szervezeti kultúrájából következik, hogy
iskolában intézményvezető-helyettes megbízása nem szükséges. Az
intézményvezető munkája a gyerekek létszámával nem növekszik, feladatainak
elvégzésében segíthetik más tanárok, megbízott szakértők és a fenntartó
is.

Amennyiben az intézményvezető akadályoztatva van, vagy munkaköre
megürült, a fenntartó jogosult olyan személy ideiglenes megbízására, aki
pedagógus végzettséggel és legalább 2 éves, igazolt vezetői
tapasztalattal rendelkezik. A fenntartónak ebben az esetben azonnal ki
kell neveznie az ideiglenes intézményvezetőt.

\hypertarget{kotott-es-kotetlen-munkaido-szabalyozasa}{%
\subsection{A kötött és a kötetlen munkaidő
szabályozása}\label{kotott-es-kotetlen-munkaido-szabalyozasa}}

A Budapest School-iskolákban alkalmazott tanárok munkaviszonyban\break
vagy
megbízási jogviszonyban állnak. A munkaviszonyban álló tanárok rendes
vagy kötetlen munkaidőben dolgoznak.

Az iskola megközelítése az, hogy a tanárok a NAT, a saját maguk és közösségi
célok által kialakított eredmények elérésére vállalnak kötelezettséget.
Ezért az Nkt. és a kapcsolódó jogszabályok által előírt munkaidő-beosztási
és -nyilvántartási szabályoktól a jelen modell eltérést enged. Az
eltérésekben az iskolának és a tanároknak meg kell egyezniük, és a
részleteknek a munka-, illetve megbízási szerződésekben meg kell
jelenniük.

A tanulóközösség tanárcsapatának a feladata a szükséges óraszámok,
beosztások kialakítása és a megfelelő szaktanárok megtalálása.

\hypertarget{tanar---gyerek-arany}{%
\subsection{Tanár---gyerek arány}\label{tanar---gyerek-arany}}

A mentor egyik legnagyobb ajándéka a mentoráltja számára a minőségi
figyelme. Odafigyel rá, meghallgatja, egyénileg támogatja, együttérez
vele, kapcsolatait és tudását átadja számára. Hogy ennek a kapcsolatnak
a minőségét ne veszélyeztessük, a BPS modell megköti, hogy egy tanár
maximum hány mentorálttal dolgozhat. Az első négy évfolyamon maximum 15,
a felső négy évfolyamon 20, a 8--12. évfolyamon pedig 25 gyerek tartozhat
egy mentorhoz.

A tanulásszervező tanárcsapat minimális mérete a tanulóközösség
méretétől függ. Egy-egy közösségben 20 gyerekenként legalább egy
tanulásszervezőnek kell dolgoznia.

Szaktanárok esetén ilyen arányszámot nehéz megadni. Egy kevert\break
(blended)
tanulási környezetben egyetlen szaktanár akár több száz tanulóval tud
egyszerre foglalkozni foglalkozás keretében. De például kémiai kisérletező
modulok során nem ajánlott, hogy 15 embernél több legyen jelen a laborban.
