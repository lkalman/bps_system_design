\hypertarget{ut-az-eselyegyenloseg-fele}{%
\section{Út az esélyegyenlőség felé}\label{ut-az-eselyegyenloseg-fele}}

A Budapest School lehetőséget kíván biztosítani arra, hogy a a
társadalom minél szélesebb rétegeiből kerülhessenek be gyerekek, hogy a
családok társadalmi, gazdasági státuszától függetlenül, kizárólag saját
fejlődési útjuk kiteljesedése jegyében válhassanak a közösség részévé.

Tudjuk, hogy az esélyek sosem egyenlőek, tenni viszont lehet azért, hogy
minél kiegyenlítettebbek legyenek. A következő területeken a családok
kiválasztásakor és az iskolába járó gyerekekkel való partneri
kapcsolatban azon dolgozunk, hogy senki se érezhesse magát hátrányosan
megkülönböztetve testi, szellemi, kulturális, szociális, nemi vagy
hitbéli egyediségei miatt.

\hypertarget{a-tarsadalmi-statuszban}{%
\paragraph{A társadalmi státuszban}\label{a-tarsadalmi-statuszban}}

A Budapest Schoolba járó családok legalább 30\%-a kevesebb hozzájárulást
fizet, mint amennyi az iskola működésének teljes bekerülési költsége. Az
ő számuk, az általuk fizetendő hozzájárulás mértéke a támogató
családoktól függ. Minél érzékenyebb, minél nagyvonalúbb egy közösség
azok irányába, akik nem tehetik meg, hogy a gyerekük tanulásáért
kifizessék az ahhoz szükséges költségeket, annál nagyobb mértékű a
társadalmi státuszbéli esélyegyenlőség egy adott Budapest School
tanulóközösségben.

\hypertarget{a-nemek-aranyaban}{%
\paragraph{A nemek arányában}\label{a-nemek-aranyaban}}

A Budapest School tanulóközösségeiben a nemek arányának
kiegyenlítésére törekszünk. Folyamatosan monitorozzuk a fiúk és lányok arányát az egyes
közösségekben, és ha az elcsúszik valamely irányba, akkor a
felvételi során a kiegyenlítés irányába hozunk döntéseket.

\hypertarget{a-kulturalis-es-vallasi-egyedisegekben}{%
\paragraph{A kulturális és vallási
egyediségekben}\label{a-kulturalis-es-vallasi-egyedisegekben}}

A BPS-iskolában meghatározó, észrevehető a családok értékrendje, mert a
családok részei a közösségnek. A családok pedig jöhetnek azonos, de
jöhetnek diverz kulturális és vallási környezetből is. Az ő szempontjaik
tiszteletben tartása mindaddig kiemelten fontos, amíg az nem
veszélyezteti a közösségben együtt tanuló gyerekek fejlődését.

\hypertarget{a-fejlodes-sebessegeben}{%
\paragraph{A fejlődés sebességében}\label{a-fejlodes-sebessegeben}}

A Budapest School lehetőséget biztosít arra, hogy a gyerek a saját
tempójában, a saját maga által kijelölt és komfortos tanulási úton
haladjon addig, amíg ebben a mentortanárával és a szüleivel is közös
megállapodást kötnek. A fejlődés sebessége azonban nem akadályozhatja a
közösségben tanulást. A tanulásszervezők felelőssége annak
meghatározása, hogy a közösségben lassabban fejlődő gyerekek
mennyiben veszélyeztetik, vagy mennyiben segítik a tanulásszervezést a
közösség egésze szempontjából.

\hypertarget{a-tanulas-tartalmaban}{%
\paragraph{A tanulás tartalmában}\label{a-tanulas-tartalmaban}}

A BPS-iskolában minden gyereknek van saját\break
célja. Ennek hossza,
komplexitása minden esetben egyedi, függ a gyerek érdeklődésétől,
érettségétől, családi helyzetétől, mentális és fizikai állapotától. A
tanulás tartalmában mindenkinek van lehetősége arra, hogy a maga útját
járja, ha figyelembe veszi, hogy ezt a közösség részeként kell tennie.

\hypertarget{a-testi-es-szellemi-egyedisegben}{%
\paragraph{A testi és szellemi
egyediségben}\label{a-testi-es-szellemi-egyedisegben}}

A Budapest School tanulóközösségeiben a tanulásszervezők felelőssége
annak eldöntése, hogy egy adott közösség milyen mértékben tud befogadni
sérült vagy sajátos nevelési igényű gyerekeket. Az ő befogadásukkor és a
velük való kiemelt foglalkozáskor mindig arra a kérdésre kell
válaszolni, hogy tudjuk-e garantálni a gyerek fejlődését, elég
biztonságos-e a közeg a számára, és az iskolában való szerepe miként
segíti a többi gyerek fejlődését.

\hypertarget{a-donteshozasban}{%
\paragraph{A döntéshozásban}\label{a-donteshozasban}}

A Budapest School-döntéshozatal nem többségi és nem konszenzusos
megállapodások alapján történik. A döntések esélyt adnak arra, hogy
minden kellően biztonságos, elég jó javaslatot ki lehessen próbálni
akkor, ha az nem áll szemben a közösen elfogadott célokkal, és nem sérti
bármely egyén érdekeit olyan mértékben, ami sérti a biztonságérzetét.
