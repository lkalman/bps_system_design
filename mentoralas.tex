\hypertarget{mentoralas}{%
\section{Mentorálás}\label{mentoralas}}

Minden gyereknek van egy \emph{mentora}, aki a saját céljainak
megfogalmazásában és a fejlődése követésében segíti. A mentor segít az
általa mentorált gyereknek, hogy a tantárgyi fejlesztési célok és a
saját magának megfogalmazott saját célok között megtalálja az
egyensúlyt, és segít megalkotni a gyerek \emph{saját tanulási tervét}.

Minden gyerek rendszeresen találkozik a mentorával, hogy átbeszéljék, hol
is tart a gyerek a céljai elérésében, milyen lehetőségei, döntései
és tervei vannak. A mentor feladata segíteni a gyereket
felismerni céljait, lehetőségeit és megtalálni a saját válaszait. A
kapcsolat lényege, hogy intim, közvetlen, őszinte, gyerekfókuszú és
rendszeres.

\hypertarget{mentorido}{%
\subsection{Mentoridő}\label{mentorido}}

Minden gyerek egy órát hetente a mentorával tölt, amikor a mentor a
gyerek számára egyéni, minőségi figyelmet ad. A mentor ilyenkor a
gyereket segíti a saját céljainak megfogalmazásában és hogy legyen
lehetősége reflektálni a saját fejlődésére. Ez az óra a gyerekek és a
mentorok számára kötelező foglalkozás.

\hypertarget{mentornaplo}{%
\subsection{Mentornapló}\label{mentornaplo}}

Minden gyereknek van egy mentornaplója, amibe mentorként minden
mentortalálkozó esetén érdemes jegyzeteket készíteni arról, hogy mit
beszélt meg a mentorált és a mentor. Ezekhez sorvezetőként a következő
kérdéseket javasoljuk:

\begin{itemize}
\tightlist
\item
  Mi \emph{van} most? Hogy van a mentorált, mi foglalkoztatja
  mostanában. Mi az, amiről mindenképpen kell ma beszélni?
\item
  Mi \emph{volt} az elmúlt időszakban? Mivel haladt, mit tanult, miyen
  problémái voltak a mentoráltnak? Mi ment jól, mi volt nehézség? Milyen
  tanulságok vannak?
\item
  Mi \emph{lehetne} az új időszakra a cél, miket lehetne megcsinálni,
  kipróbálni, változtatni, folytatni?
\item
  Mi \emph{lesz} ebből a terv? Mi az a 2-3 dolog, amit ezért biztosan
  meg fogsz tenni? Milyen érzések vannak benned ezzel kapcsolatban?
  Kire, mire számíthatsz ebben? Milyen nehézségekre számítasz? Mik azok
  az erőforrásaid, amik már most rendelkezésedre állnak ahhoz, hogy
  ezeken a nehézségeken átlendülj? Hogyan tudok neked én segíteni?
\end{itemize}

Érdemes feljegyezni, hogy milyen kérdések merültek fel, és a mentorált
milyen válaszokat adott magának.

\hypertarget{grow-modell}{%
\subsection{GROW modell}\label{grow-modell}}

Minden mentornak meg kell találnia azokat a kérdéseket, amik ott és
akkor segítik a mentorált fejlődését. Alapjában azt javasoljuk, hogy
minden mentor a GROW modellt szabja magára. A GROW modell egy nagyon
egyszerű és nagyszerű coaching modell. A szó maga azt jelenti:
,,növekedés'', a betűk a következőket jelentik benne:

\begin{itemize}
\tightlist
\item
  \emph{G}oal: \textsc{cél}
\item
  current \emph{R}eality: \textsc{jelenlegi helyzet}
\item
  \emph{O}ptions, \emph{o}bstacles: \textsc{lehetőségek, akadályok}
\item
  \emph{W}ill, \emph{W}ay forward: \textsc{szándék, előre vezető út}
\end{itemize}

A GROW coaching modell használatát a szakirodalom egy utazás\break
megtervezéséhez hasonlítja. Ahhoz, hogy egyáltalán elérj valahová,
tudnod kell, hová akarsz menni (cél), és azt is, hogy hol vagy jelenleg
(jelenlegi helyzet). Aztán az úticél eléréséhez ismerned kell, hogy
hogyan juthatsz oda, mik lehetnek az akadályok, végül pedig el kell
határoznod magad, hogy el is indulsz és legyőzöd ezeket.

\paragraph{G: Célkitűzés.} A mentor segít a gyereknek abban, hogy reális célt tűzzön
ki magának. A célkitűzésben segíthet az, ha a cél SMART cél, tehát
legyen elérhető, reális, időkerethez kötött, mérhető és konkrét.

\paragraph{R: Jelenlegi helyzet meghatározása.} Kritikus pont, hiszen sokszor a
gyerek úgy akar célt kitűzni, hogy nem gondolja át alaposan jelenlegi
helyzetét és azt, így nem rendelkezik a megfelelő mennyiségű
információval ahhoz, hogy reálisan lássa lehetőségeit.

\paragraph{O: Lehetőségek meghatározása.} Ha sikerül az előző lépésben ismertetett helyzetmeghatározás, a
mentor és a gyerek ötletbörzét tarthat a lehetőségekről. A lényeg
megtalálni a lehető legtöbb választási lehetőséget, hogy aztán könnyebb
legyen dönteni, hogy merre induljon a gyerek. Fontos, hogy a mentor
segítse kliensét elszakadni gondolati korlátaitól, segítsen neki
,,messzebbre látni''. Fontos felmérni az akadályokat is.

\paragraph{W: Szándék, elhatározás.} A mentor ebben a szakaszban segít a kliensnek
biztosítani azt, hogy valóban el is induljon a célja felé. A munka
könnyebbik része volt a terep felmérése és a lehetőségek
áttekintése.
Most jön az oroszlánrésze, amikor a kliensnek tennie is kell, méghozzá
nap mint nap azért, hogy el is érje a célját. A mentor feladata a
támogatás, motiválás, inspiráció és a számonkérés is.

\hypertarget{mentor-vagy-coach}{%
\subsection{Mentor vagy coach}\label{mentor-vagy-coach}}

A BPS-mentortanár néha a coach-a, néha mentora a gyereknek. A szakirodalmi
különbség a két fogalom között az, hogy a coach nem ismeri a kliensének
szakterületét, nem ad tanácsot, nem ért jobban a kliensénél a
területhez, ezért tényleg csak abban tud segíteni, hogy a kliens
megtalálja a saját válaszait. Ezzel szemben a mentor az, aki ,,csinálta
már'' (have done it), tapasztalai alapján tanácsokkal tudja ellátni a
mentoráltat. A BPS-mentortanár ebben az értelemben egy tapasztalt,
integráns felnőtt, aki tud segíteni a gyereknek érettebb, önálóbb,
boldogabb felnőtté vállni. Szóval néha tanácsokat ad. A BPS-mentor
ugyanakkor tud coach is lenni. Nem kell ahhoz nagyon érteni az
emeltszíntű biológiához, hogy segíteni tudjunk egy gyereket abban, hogy
ha a céljai úgy kívánják, akkor hetente kétszer üljön le tanulni.
