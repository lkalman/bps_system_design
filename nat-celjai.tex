\hypertarget{nat-celjainak-tamogatasa}{%
\section{NAT céljainak támogatása}\label{nat-celjainak-tamogatasa}}

A Nemzeti alaptantervben szereplő fejlesztési célok elérését és a
kulcskompetenciák fejlődését több minden támogatja:

\begin{itemize}
\item
  Egyrészt a tantárgyak 100\%-ban lefedik a NAT fejlesztési céljait,
  kulcskompetenciáit, műveltségi területeit és tananyagtartalmát.
\item
  Másrészt az iskola életében, folyamatában való részvétel sok esetben
  már önmagában
  biztosítja a kulcskompetenciák fejlődését és a NAT fejlesztési
  céljainak teljesülését.
\end{itemize}

A NAT fejlesztési céljainak elérését nemcsak a tantárgyak, hanem az
iskola struktúrája is támogatja.

\begin{longtable}[]{@{}ll@{}}
\toprule
\textbf{A NAT fejlesztési céljai} & \textbf{Struktúra}\tabularnewline
\midrule
\endhead
Az erkölcsi nevelés & közösség\tabularnewline
Nemzeti öntudat, hazafias nevelés & projektek\tabularnewline
Állampolgárságra, demokráciára nevelés & közösség\tabularnewline
Az önismeret és a társas kultúra fejlesztése & saját tanulási út,
közösség\tabularnewline
A családi életre nevelés & saját tanulási út, közösség\tabularnewline
A testi és lelki egészségre nevelés & közösség\tabularnewline
Felelősségvállalás másokért, önkéntesség & közösség,
projektek\tabularnewline
Fenntarthatóság, környezettudatosság & projektek\tabularnewline
Pályaorientáció & saját tanulási út\tabularnewline
Gazdasági és pénzügyi nevelés & projektek\tabularnewline
Médiatudatosságra nevelés & projektek\tabularnewline
A tanulás tanítása & saját tanulási út, mentorság\tabularnewline
\bottomrule
\end{longtable}

A \emph{saját tanulási} út fogalma például önmagában segíti a tanulás
tanulását, hiszen az a gyerek, aki képes önmagának saját célt állítani
(mentori segítséggel), azt elérni, és a folyamatra való reflektálás
során képességeit javítani, az fejleszti a tanulási képességét.

Egy másik példa: a Budapest School iskoláiban a \emph{közösség}
maga hozza a működéséhez szükséges szabályokat, folyamatosan alakítja és
fejleszti saját működését a tagok aktív részvételével. Ez az aktív
állampolgárságra, a demokráciára való nevelésnek a Nemzeti Alaptantervben
előírt céljait is támogatja.

A NAT kulcskompetenciáinak fejlesztését támogatják a tantárgyak és az
iskola felépítése is az alábbiak szerint:

\begin{longtable}[]{@{}ll@{}} % \label{screwedUpTable1}
\toprule
\begin{minipage}[b]{0.56\columnwidth}\raggedright
\textbf{NAT kulcskompetenciái}\strut
\end{minipage} & \begin{minipage}[b]{0.38\columnwidth}\raggedright
\textbf{Struktúra}\strut
\end{minipage}\tabularnewline
\midrule
\endhead
\begin{minipage}[t]{0.56\columnwidth}\raggedright
Tanulás kompetenciái\strut
\end{minipage} & \begin{minipage}[t]{0.38\columnwidth}\raggedright
saját tanulási út, mentorság\strut
\end{minipage}\tabularnewline
\begin{minipage}[t]{0.56\columnwidth}\raggedright
Kommunikációs kompetenciák (anyanyelvi és idegen nyelvi)\strut
\end{minipage} & \begin{minipage}[t]{0.38\columnwidth}\raggedright
tanulási szerződés, portfólió\strut
\end{minipage}\tabularnewline
\begin{minipage}[t]{0.56\columnwidth}\raggedright
Digitális kompetenciák\strut
\end{minipage} & \begin{minipage}[t]{0.38\columnwidth}\raggedright
digitális portfóliókezelés\strut
\end{minipage}\tabularnewline
\begin{minipage}[t]{0.56\columnwidth}\raggedright
Matematikai, gondolkodási kompetenciák\strut
\end{minipage} & \begin{minipage}[t]{0.38\columnwidth}\raggedright
\strut
\end{minipage}\tabularnewline
\begin{minipage}[t]{0.56\columnwidth}\raggedright
Személyes és társas kapcsolati kompetenciák\strut
\end{minipage} & \begin{minipage}[t]{0.38\columnwidth}\raggedright
közösség fókusz\strut
\end{minipage}\tabularnewline
\begin{minipage}[t]{0.56\columnwidth}\raggedright
Kreativitás, a kreatív alkotás, önkifejezés és kulturális tudatosság
kompetenciái\strut
\end{minipage} & \begin{minipage}[t]{0.38\columnwidth}\raggedright
interdiszciplináris modulok, KULT fejlesztési cél\strut
\end{minipage}\tabularnewline
\begin{minipage}[t]{0.56\columnwidth}\raggedright
Munkavállalói, innovációs és vállalkozói kompetenciák\strut
\end{minipage} & \begin{minipage}[t]{0.38\columnwidth}\raggedright
saját tanulási út, önálló tanulás, közösség, projektek\strut
\end{minipage}\tabularnewline
\bottomrule
\end{longtable}

\hypertarget{a-tantargykozi-tudas--es-kepessegteruletek-fejlesztesenek-feladata}{%
\subsection{A tantárgyközi tudás- és képességterületek fejlesztésének\\
feladata}\label{a-tantargykozi-tudas--es-kepessegteruletek-fejlesztesenek-feladata}}

A Budapest School iskola elvégzi a NAT kulcskompetenciáinak
fejlesztését, támogatja a NAT által meghatározott fejlesztési területek
céljait, és ellátja a műveltségi területekhez rendelt fejlesztési
feladatokat. A foglalkozások többsége nem tantárgyak alapján
szerveződik, így nálunk a tantárgyközi tudás az alapértelmezett.
