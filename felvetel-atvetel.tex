\hypertarget{a-felvetel-es-az-atvetel}{%
\section{A felvétel és az átvétel}\label{a-felvetel-es-az-atvetel}}

Aki a BPS iskolához csatlakozik, az rögtön a BPS egyik
tanulóközösségéhez csatlakozik. A tanulóközösségéhez bármikor lehet
csatlakozni, ha és amikor a csatlakozó család ezt szeretné, és ha ettől
a tanulóközösség minden tagjának valamiért jobb lesz, vagy nem változik
(de rosszabb nem lehet). Az iskola nem azért fogad be valakit, mert 
kénytelen, hanem mert a tanulóközösség ezt szeretné. A családok nem azért
csatlakoznak, mert valamit kell találni a gyereknek, hanem mert
szeretnének a Budapest School egyik tanulóközösségéhez tartozni.

A 12 évfolyamos egységes iskolába bármikor lehet csatlakozni, az iskola
normálisnak tartja, hogy a közösség tagjai változnak. Ezért az iskolában
egy iskolát most kezdő 6 éves felvétele, egy 8 éves, az előző iskoláját
nem kedvelő felvételi kérelme, egy 9 éves külföldről hazaköltöző év
közbeni csatlakozása, egy 12 éves „gimnáziumba'' jelentkezése és egy 16
éves más városból érkező között az iskola számára a \emph{felvételi és
átvételi folyamatot} tekintve nincs különbség.

Ahhoz, hogy egy család csatlakozzon a tanulóközösséghez, kizárólag egy tanulóközösség tanulásszervezőinek a hozzájárulása
szükséges.

Egy család jelentkezése után legalább három dolognak kell történnie.

\begin{itemize}
\item
  A családnak meg kell ismernie a Budapest School alapelveit,
  működését, jellegzetességeit. Az iskolának meg kell mutatnia önmagát.
  A családnak meg kell értenie, és meg kell fogalmaznia, hogy miért
  akarnak csatlakozni a közösséghez.
\item
  A tanulásszervezőknek meg kell ismerniük a családot, megnézni, hogy
  „működik-e a kémia'', tudják-e vállalni a gyerek tanulásának
  támogatását.
\item
  A gyereknek időt kell eltöltenie a tanulóközösségben, a mindennapokhoz
  minél inkább hasonló körülmények között, hogy mindenki meg tudja
  tapasztalni, érezni, hogy milyen lenne együtt és egymástól tanulni.
\end{itemize}

\hypertarget{szempontok-a-donteshez}{%
\subsection{Szempontok a döntéshez}\label{szempontok-a-donteshez}}

A tanulóközösség legyen minél inkább diverz és kiegyensúlyozott: kevert
korosztályú, kevert nemi, kevert szociális státuszú, kevert érdeklődésű,
kevert személyiségjegyű csoport, úgy, hogy legyen egy erős, mindenkit
megtartó szociális háló. A tanulóközösségek közösségét egyenként kell
kiegyensúlyozni.

Így az is előfordulhat, hogy egy gyerek egy tanulóközösségben nem talál
helyet magának, de az iskola egy másik tanulóközösségében igen. Mert a
közösségek különbözőek. A legegyszerűbb példa: van, ahol több lányt
szeretnénk, mint ma, és van, ahol több fiút, és van, ahol ez most nem
szempont.

\hypertarget{nincs-felveteli-vizsga}{%
\subsection{Nincs felvételi vizsga}\label{nincs-felveteli-vizsga}}

Az iskola nem követel meg sem írásbeli (központi), sem szóbeli felvételit,
és nem is az előző iskolák osztályzatai alapján dönt. Az egyetlen
szempont az,
hogy jobban tud-e működni egy tanulóközösség egy gyerek (és család)
csatlakozásával. A döntést a tanulóközösség tanárai, a fenntartó (vagy
delegáltja) és a család hozzák meg.

\hypertarget{szakitas-tavozas-elengedes}{%
\subsection{Szakítás, távozás,
elengedés}\label{szakitas-tavozas-elengedes}}

Működésünk része, hogy konfliktusok, kényelmetlenségek, változó
körülmények között, aki egyszer csatlakozott, az egyszer távozhat is a
közösségből.

\begin{itemize}
\item
  Az iskola, a tanárok és a család alapelvei, értékei közötti
  különbségek okozhatnak annyi és olyan konfliktust, amit már nem tudnak
  a felek feloldani.
\item
  Van, hogy egy gyerek nem találja meg a helyét, vagy épp valamiért
  elkezd a közösségben „nem boldog'' pozícióba kerülni. Vagy épp a
  közösség többi tagjának lesz kényelmetlen az együttlét.
\item
  A családok élete, vágyai, motivációjuk, körülményei változhatnak
  úgy,
  hogy épp más közösségben jobb helyet találnának.
\end{itemize}

Bármi legyen is az ok, a távozás, szakítás feszültséggel teli szituáció.
Ezért is fontos, hogy minden fél betartsa a közösségi lét szabályairól
szóló fejezetben (\ref{a-kozossegi-let-szabalyai}.~fejezet, \pageref{a-kozossegi-let-szabalyai}.~oldal) leírt
konfliktuskezelési szokásokat
(\pageref{konfliktusok-feszultsegek-kezelese}. oldal).

A csatlakozáskor a családok szerződésben vállalják a jelen program és az
iskola egyéb szabályozóinak betartását --- ennek ismételt vagy súlyos
megszegése esetén az iskola jogosult a tanulói jogviszonyt megszüntetni.
Ez a döntés a fenntartó jogkörébe tartozik.
