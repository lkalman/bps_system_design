\hypertarget{sarga-narancs-piros-riasztas}{%
\section{Sárga, narancs, piros
riasztás}\label{sarga-narancs-piros-riasztas}}

A BPS-tanulóközösségek bonyolult rendszerként működnek. A tanulóközösség
tagjai, a gyerekek, a tanárok, a szülők és a környezet kölcsönhatásban vannak
egymással: az egyikünk tettei kihatnak a többiekre. Az, hogy egy gyerek
hogyan viselkedik a csoportban, úgyanúgy függ a személyiségétől,
a képességeitől és a csoport, a többi gyerek és a tanárok működésétől is.
Vannak, olyan nehézségek, konfliktusok, feszültségek, amikor úgy érzi
valamelyik csapattag, hogy tenni kell valamit, mert a helyzet sokáig nem
tartható. Ilyenkor \emph{riasztást} küld az érintetetteknek.

Mikre is gondolunk?

\begin{itemize}
\tightlist
\item
  Abony zavarja a többieket a foglalkozáson, mert eltereli a figyelmet.
\item
  Fáni állandóan cukkolja, zaklatja a társait. Bántó, sértő
  megjegyzéseket tesz.
\item
  Orku kirántja társa kezéből az ollót, nem kéri el, és nem tud várni.
\item
  Barka nem képes a többiekkel haladni a matekórán. Nem zavar senkit, és
  nagyon édes, amikor bámul kifelé az ablakon. Látjuk rajta, hogy egyre
  szomorúbb a szeme, és már olyanokat mond, hogy ,,én nem vagyok jó
  matekból''.
\item
  Kinga jól érzi magát a BPS-ben, de szülei régóta elégedtelenek, és néha
  kiabálnak a tanárokkal, agresszív e-maileket küldenek.
\end{itemize}

A példák jól mutatják, hogy a nehézségeknek mindig több oka van, eltérő
hátterűek lehetnek.

\begin{itemize}
\tightlist
\item
  Amikor Abony számára érdekesebb a foglalkozás, vagy kevesebb gyerek
  van a foglalkozáson, akkor Abony nagyon el tud mélyülni. Abonynak
  egyébként a foglalkozás előtti sport sokat segít.
\item
  Fáni apukája épp elköltözött. Régebben Fáni mindig a
  konfliktus-elsimító volt. Valószínűleg ez sem segít most, mert
  hiányzik neki a törődés. Fáni legjobb barátja elment az iskolából, mert
  külföldre költöztek, és az sem jó, hogy Fáni kedvenc tanára elment egy
  nagy céghez dolgozni.
\item
  Orku nem bírja, ha várnia kell. Ha több olló lenne az iskolában, akkor
  ez a konfliktus nem lenne. És egyébként Orku tudna az
  indulatkezelésén javítani.
\item
  Barka Hollandiából költözött haza egy éve. Felmerült, hogy valami alapvető
  részképesség-rendellenessége van, amit nem vizsgáltak sose. Nem segít,
  hogy hollandul tanulta a matekot.
\item
  Kinga ugyan saját céljai szerint halad a tanulásban, de a szülei nem
  ismerik még eléggé a BPS módszereit, vagy kevéssé tudnak azonosulni
  azokkal. Úgy érzik, hogy nem teljesülnek az elvárásaik, vagy hogy más
  iskolákban jobban tanulnak a gyerekek. Félnek, hogy\break
  gyermekük lemarad
  a tanulásban, és erről már többször beszéltek a tanárokkal, de eddig
  nem kaptak számukra megnyugtató választ.
\end{itemize}

Abony, Fáni, Orku, Barka és Kinga példája mutatja, hogy bár ők a
főszereplők a történetükben, a problémákat sok dolog együttállása okozza.
Ezen gyerekek helyzete a csoportban \emph{intervenciót} kíván. Valamit
tenni kell, mert se nekik, se a társainak, se a tanároknak nem tesz jót,
hogy mindig helyzetben vannak. Olyan tüneteket látunk, amik az ő
fejlődésükre vagy a közösségre negatív hatással vannak. Tehát ezekkel
foglalkozni kell, több figyelmet kell fordítani rájuk.

A legrosszabb, ami történhet, hogy különböző okokból nem lépünk, nem
változtatunk, nem próbálunk ki intervenciókat, és a helyzet csak
romlik és romlik. Már a többiek érzelmi, fizikai biztonsága vagy
tanulása is veszélyben van, a tanárok is vesztenek motivációjukból a
tehetetlenség és bűntudat miatt, és a szülők is csak az elveszettséget
érzik.
\emph{Alapelv: A problémák maguktól nem oldódnak meg.}

A jó példák sora viszont hosszú: több esetben sikerült külső szakértő
bevonásával, a napi működés egyszerű változtatásával eredményt elérni,
és előfordult, hogy egy hosszabb terápia segített. Néha kiderül, hogy
gyógyszer segíthet neurológiai problémákon. A napirend megváltoztatása vagy
a csoportok összetételének megváltoztatása is segített már más
helyzetekben. Előfordult, hogy átmeneti időszakra egy adott gyerekhez
dedikált „shadow teacher'' (plusz tanár) kellett a nehéz helyzetek
megoldásához.

Van, amikor arra az álláspontra jutunk, hogy itt és most a tanárok, az
iskola nem tud segíteni. Ilyenkor a családokon az új környezet eddig
mindig segített.

Ezért határoztuk el a Budapest School szintjén, hogy kialakítunk egy
folyamatot, aminek segítségével már az elején fel tudjuk ismerni a
feszültséget, lépéseket tehetünk, hogy a helyzet javuljon, és azt is fel
tudjuk ismerni, hogy mikor jutunk el oda, hogy már nem tudunk érdemben
együttműködni. Ezt a folyamatot hívjuk \emph{piros-sárga-narancs riasztásnak}.

\hypertarget{folyamat-roviden}{%
\subsection{A folyamat röviden}\label{folyamat-roviden}}

A lényeg, hogy tegyünk valamit a szituáció javítása érdekében. Amikor a
tanárok úgy érzik, hogy ,,valami van'', akkor minden szituációt a
következő kategóriákba sorolnak be:

\begin{enumerate}
\def\labelenumi{\arabic{enumi}.}
\tightlist
\item
  Sárga: oda kell figyelnünk, legyen utánkövetés, esetleg ajánlott
  diagnózisok.
\item
  Narancssárga: Előfordultak olyan esetek, amik zavarták a közösséget.
\item
  Piros: többször előfordult, hogy a közösséget kár érte. Intervenciót
  kell alkalmazni és kiértékelni. Ezt határidőhöz kötjük, hogy a
  többieket ért
  kár korlátozott legyen.
\end{enumerate}

Fontos, hogy a riasztásról a szülő is, és 12 év felett a gyerek is
tudjon. Az intervenciókat közösen kell kialakítaniuk. A gyerek
mentortanárának a feladata a folyamatot végigvinni, az érintetetteket
(,,stake holder'') bevonni.

\hypertarget{nem-lesz-cimkezes-onbeteljesito-joslat}{%
\subsubsection{Nincs címkézés, önbeteljesítő
jóslat}\label{nem-lesz-cimkezes-onbeteljesito-joslat}}

Mindent meg kell tennünk azért, hogy mindenki értse: a jelenlegi
helyzet, állapot okozta a riasztást, nem pedig a gyerek. A gyerek „furcsa''
viselkedése az ő képességeinek, vágyainak és a környezet képességeinek
és vágyainak az eltéréséből adódik. Nem a gyerek személyisége okozta, nem
megváltoztathatlan adottságokról beszélünk, hanem magunkról. Pont azért\break
cselekszünk, mert hiszünk a változásban.

\hypertarget{tehat-most-kirugjak-a-gyerekem}{%
\subsubsection{Tehát most kirúgják a
gyerekem?}\label{tehat-most-kirugjak-a-gyerekem}}

Amikor nem tudjuk feloldani a konfliktust, a gyerek és a közösség
disz\-kon\-fort-érzését, akkor a végső esetben el kell szakadnunk
egymástól, szakítunk. Nem tudunk együtt dolgozni, és ez hosszú távon
valószínűleg mindenkinek így lesz jobb. Ne tekintsd ezt a gyereked
jellemzésének. Az, hogy valaki egy iskolát elhagy, még ma is súlyos
bélyeg. Pedig lehet, hogy csak jobb helyet keresel neki.
(Régen az váltott csak iskolát, aki problémás volt. A kultúrának része
volt,
hogy a jó gyerek mindig ugyanabba az iskolába jár. Abban az időben
mindenkinek egész életében egyetlen munkahelye volt.)

Ha nem tetszik az úszóedző, akkor újat keresel, ha nem fejlődik eleget
egy felnőtt egy munkahelyen, akkor vált. Hát mi is így gondolunk a
gyerekeinkre: az a dolgunk, hogy olyan helyen legyenek, ahol boldogak,
hatékonyak és egészségesek tudnak ma lenni. Ha ez nem az a csoport, ahol
most van, akkor váltani érdemes.

\hypertarget{mikor-vannak-riasztasok}{%
\subsection{Mikor vannak riasztások?}\label{mikor-vannak-riasztasok}}

\begin{enumerate}
\def\labelenumi{\arabic{enumi}.}
\tightlist
\item
  Amikor a közösség érzelmi, fizikai biztonsága vagy fejlődése veszélybe
  kerül. Ezt az iskola bármelyik tanára megállapíthatja.
\item
  Amikor a tanárok figyelmét, munkáját egy gyerekhez köthető
  esetek aránytalanul elviszik. Szintén a tanároktól jön a jelzés.
\item
  Amikor egy gyerek az iskolán kívül veszélybe kerül. Tanároktól, a gyerektől vagy a
  szülők valamelyikétől jön a jelzés.
\item
  Amikor egy gyerek viselkedése önveszélyes. Tanároktól vagy a szülőtől jön a
  jelzés.
\item
  Amikor a gyerek nem érzi jól magát, nem fejlődik úgy, ahogy szeretne. Szülőktől
  vagy a gyerektől jön a jelzés.
\end{enumerate}

\hypertarget{fokozatok}{%
\subsubsection{Fokozatok}\label{fokozatok}}

Háromféle esetet különböztetünk meg.

\begin{enumerate}
\def\labelenumi{\arabic{enumi}.}
\tightlist
\item
  Amikor \emph{sárga}, elsőfokú jelzést küldesz a szülőnek, akkor a
  gyerek viselkedése nyugtalanít, vagy éppen néhányszor zavar, és az
  a megérzésed, hogy a gyereket külön vizsgálatra, fejlesztére kéne
  vinni, mert nem tudjuk neki megadni, amire szüksége van. Vagy
  valamilyen más intervencióra van szükség (korábban feküdjön
  le,kevesebb videójátékot játszon). Általában ilyen az, amikor a gyerek
  kiesik a foglalkozásokból, láthatólag nehezen kezeli a frusztrációját,
  vagy épp a mozgáskoordinációján látszik, hogy valami egyénire van
  szüksége.
\item
  Narancssárga riasztásnak hívjuk, amikor már néhány esetben olyan is
  előfordult, ami a közösség többi tagjára vagy magára a gyerekre hosszú távon
  veszélyesnek tűnik.
\item
  Súlyos esetben a piros, harmadfokú jelzést küldöd, amikor a gyerek
  viselkedése már saját magára vagy többiek érzelmi és fizikai biztonságára
  ártalmas, és (vagy) a csoport munkáját annyira megnehezíti, hogy azt te
  már nem tudod vállalni.
\end{enumerate}

\begin{longtable}[]{*{4}{>{\begin{minipage}[t]{.23\textwidth}\strut\raggedright}l<{\strut\end{minipage}}}}
\toprule
&\textbf{Sárga}&\textbf{Narancs}&\textbf{Piros}\tabularnewline
\midrule
\endhead
Fő üzenet
  &Intervenciót javaslunk
  &Intervencióban
   megállapodunk
  &A helyzet elfogadhatatlan, ha nincs változás, akkor ki kell lépni.  Az intervenció a maradás feltéte
\tabularnewline  
\midrule  
Megállapodás
  &A tanár informálja a szülőt
  &A tanár és a szülő megállapodást köt
  &A tanár és a szülő megállapodást köt
\tabularnewline
\midrule
Follow up
  &Egy hónap múlva emailben ír a szülő az internencióról
  &Maximum négy hetente konzultáció a szülőkkel
  &Maximum két heti konzultáció. Javasolt heti fejlődési jelentés a gyerek viselkedéséről (a tanár írja) és az intervencióról (a szülő írja)
\tabularnewline
\midrule
Intervenciók általában
  &Egyéni kivizsgálás, fejlesztés, otthoni feladatok
  &Egyéni kivizsgálás, fejlesztés, otthoni feladatok
  &Shadow teacher, terápia, szakszolgálat, félnap
\tabularnewline
\midrule
Nyomonköveté-\break
sért felelős
  &Mentortanár
  &Mentortanár
  &Operátor
\tabularnewline
\midrule
Informáltak
  &Tanárcsapat
  &Tanárcsapat
  &Tanárcsapat $+$ központ
\tabularnewline
\bottomrule
\end{longtable}




%% \begin{longtable}[]{@{}llll@{}}
%% \toprule
%% \begin{minipage}[b]{0.09\columnwidth}\raggedright
%% \strut
%% \end{minipage} & \begin{minipage}[b]{0.19\columnwidth}\raggedright
%% Sárga\strut
%% \end{minipage} & \begin{minipage}[b]{0.18\columnwidth}\raggedright
%% Narancs\strut
%% \end{minipage} & \begin{minipage}[b]{0.43\columnwidth}\raggedright
%% Piros\strut
%% \end{minipage}\tabularnewline
%% \midrule
%% \endhead
%% \begin{minipage}[t]{0.09\columnwidth}\raggedright
%% Fő üzenet\strut
%% \end{minipage} & \begin{minipage}[t]{0.19\columnwidth}\raggedright
%% Intervenciót javaslunk\strut
%% \end{minipage} & \begin{minipage}[t]{0.18\columnwidth}\raggedright
%% Intervencióban megállapodunk\strut
%% \end{minipage} & \begin{minipage}[t]{0.43\columnwidth}\raggedright
%% Helyzet elfogadhatatlan, ha nincs változás, akkor ki kell lépni.
%% Az intervenció a maradás feltéte\strut
%% \end{minipage}\tabularnewline
%% \begin{minipage}[t]{0.09\columnwidth}\raggedright
%% Megállapodás\strut
%% \end{minipage} & \begin{minipage}[t]{0.19\columnwidth}\raggedright
%% Tanár informálja a szülőt\strut
%% \end{minipage} & \begin{minipage}[t]{0.18\columnwidth}\raggedright
%% Tanár és a szülő megállapodást köt\strut
%% \end{minipage} & \begin{minipage}[t]{0.43\columnwidth}\raggedright
%% Tanár és a szülő megállapodást köt\strut
%% \end{minipage}\tabularnewline
%% \begin{minipage}[t]{0.09\columnwidth}\raggedright
%% Follow up\strut
%% \end{minipage} & \begin{minipage}[t]{0.19\columnwidth}\raggedright
%% 1 hónap múlva emailben ír a szülő az internencióról\strut
%% \end{minipage} & \begin{minipage}[t]{0.18\columnwidth}\raggedright
%% Max 4 hetente konzultáció a szülőkkel.\strut
%% \end{minipage} & \begin{minipage}[t]{0.43\columnwidth}\raggedright
%% Max 2 heti konzultáció. Javasolt heti progress a gyerek viselkedéséről
%% (tanár írja), az intervencióról (szülő írja)\strut
%% \end{minipage}\tabularnewline
%% \begin{minipage}[t]{0.09\columnwidth}\raggedright
%% Intervenciók általában\strut
%% \end{minipage} & \begin{minipage}[t]{0.19\columnwidth}\raggedright
%% Egyéni kivizsgálás, fejlesztés, otthoni feladatok\strut
%% \end{minipage} & \begin{minipage}[t]{0.18\columnwidth}\raggedright
%% Egyéni kivizsgálás, fejlesztés, otthoni feladatok\strut
%% \end{minipage} & \begin{minipage}[t]{0.43\columnwidth}\raggedright
%% Shadow teacher, terápia, szakszolgálat, félnap\strut
%% \end{minipage}\tabularnewline
%% \begin{minipage}[t]{0.09\columnwidth}\raggedright
%% Felelős nyomonkövetésre\strut
%% \end{minipage} & \begin{minipage}[t]{0.19\columnwidth}\raggedright
%% Mentor tanár\strut
%% \end{minipage} & \begin{minipage}[t]{0.18\columnwidth}\raggedright
%% Mentor tanár\strut
%% \end{minipage} & \begin{minipage}[t]{0.43\columnwidth}\raggedright
%% Operátor\strut
%% \end{minipage}\tabularnewline
%% \begin{minipage}[t]{0.09\columnwidth}\raggedright
%% Informáltak\strut
%% \end{minipage} & \begin{minipage}[t]{0.19\columnwidth}\raggedright
%% Tanárcsapat\strut
%% \end{minipage} & \begin{minipage}[t]{0.18\columnwidth}\raggedright
%% Tanárcsapat\strut
%% \end{minipage} & \begin{minipage}[t]{0.43\columnwidth}\raggedright
%% Tanárcsapat + központ (a.k.a. Gábor)\strut
%% \end{minipage}\tabularnewline
%% \bottomrule
%% \end{longtable}

\hypertarget{folyamat}{%
\subsection{A folyamat}\label{folyamat}}

Minden tanulóközösségnek van egy Sárga-Narancs-Piros Riasztási Felelőse,
aki azért felel, hogy a tanárcsapat kéthavonta átbeszélje, hogy van-e
gyerek sárga, narancs vagy piros szinten. Ennek megtörténtét a felelősnek
dokumentálnia kell. Az is rendben van, ha 34 másodperc alatt megbeszélik, hogy
nincs semmi különös. És az is rendben van, ha vita alakul ki egy gyerek
miatt.

\hypertarget{ki-es-hogyan-jelez}{%
\subsubsection{Ki és hogyan jelez?}\label{ki-es-hogyan-jelez}}

A riasztásról a tanárcsapat, a szülő vagy tizenkét év felett a gyerek
dönthet. Tanárcsapat esetén
mindekinek hozzá kell járulnia a riasztáshoz (\ref{donteshozas}.~fejezet, \apageref{donteshozas}.~oldal). Azaz bárki mondhatja, hogy szerinte \emph{,,a Fáninál
naracssárga a szitu, meg kell állapodnunk a tanárokkal és szülőkkel
közös intervencióban''}.

\hypertarget{jelzesek-kuldese---szemelyes-es-irasban}{%
\subsubsection{Jelzések küldése személyesen és
írásban}\label{jelzesek-kuldese---szemelyes-es-irasban}}

Jelzés (,,signal'') küldésekor a legjobb protokoll, amit gyakorlatunk alapján
kialakítottunk, a következő:

\begin{enumerate}
\def\labelenumi{\arabic{enumi}.}
\tightlist
\item
  Leírod a dolgokat, hűvös fejjel. Még talán meg is beszéled másokkal.
\item
  Személyes találkozót kérsz a szülőtől.
\item
  A találkozón hűvösen, tárgyilagosan elmondod a dolgot. Azonnal belevágsz, a
  találkozó elején, hogy ne menjen el az idő fecsegéssel.
\item
  A találkozó közben a szülő észrevételeit jegyzeteled. Ezt az eredeti
  dokumentumba beteszed.
\item
  A találkozó után 6 órán belül elküldöd az eredeti dokumentumot.
\end{enumerate}

Miért így? Mert jobb előre összeszedni a mondandódat. Mert a nehéz
szituációk átbeszélése közben fontos részletek kimaradhatnak, és a másik
fél sokszor másra emlékszik.

\hypertarget{megallapodas}{%
\subsubsection{Megállapodás}\label{megallapodas}}

Narancssárga és piros szituációk esetén megállapodást kell kötni a
szülőkkel intervenciókról. A megállapodásnak ki kell terjednie a
vállalásokra, vagyis arra, hogy ki mit tesz. És határidőre: mikor és hogyan
követik nyomon a felek a fejlődést, illetve piros riasztás esetében meddig tudnak a
hatásra várni.

\hypertarget{retrospektiv}{%
\subsubsection{Retrospektív}\label{retrospektiv}}

Minden piros jelzéses eset feloldása után egy
% \href{https://www.pagerduty.com/blog/postmortems-vs-retrospectives/}
\emph{,,post
mortem retrospektívet''} kell tartania a tanároknak a fenntartó, a BPS Lab
egyik képviselőjével arról, hogy az esetből mit tudunk tanulni.
