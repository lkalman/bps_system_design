Öt éve gyűltünk össze először baráti körben, hogy post-itekkel,
filctollakkal, különféle írt és rajzolt jegyzetekkel elkezdjük
megtervezni a gyerekeink tanulási élményét. Egy olyan iskolarendszer
gondolata fogalmazódott meg akkor bennünk, amelyben tanárcsapatok
által vezetett mikroiskolák kapcsolódnak egymáshoz egy rugalmasan
alakítható szervezetben. Nem is iskolát, hanem egy tanulási
környezetet akartunk létrehozni, ami a gyerekek, a tanárok és a szülők
partnerségére épül. Ekkor kezdtük először megfogalmazni magunkban a
Budapest School modell alapjait, amin azóta is dolgozunk.

 
Az első évben még nem összegeztük az elképzeléseinket, mert azt
gondoltuk, fontosabb, hogy a gyerekekkel töltsük az időt. A szülők
számára azonban fontos volt, hogy a működésünk alapjait írásban is
dokumentáljuk. Hálásak vagyunk azért, hogy erre ráébresztettek
minket. A \textit{Konfliktuskezelés\/}
alapjait Fellegi Borcsával írtuk, és sokat
segített Iványi Klára is a nehéz kérdéseivel, amikor az első évben még
óvodás csoportként keresgéltük az utat.

 
A 2016-17-es tanévben közvetlenül az egyetem elvégzése után érkezett
hozzánk Csóti Orsi,  aki  intézményfejlesztő szakirányon végzett. A
friss diplomások energiájával kezdett bele  Kovács Ágival
egybegyűjteni a pedagógiai módszereinkkel kapcsolatos írásainkat, és
segítettek ezeket összevetni a Lauder, a Waldorf, a Rogers és más
iskolák programjaival. Bundschuh Hanna, aki szintén kezdő tanárként
került hozzánk.  Ő összeszerkesztette azokat a válaszainkat, amiket az
első években tartott szülői tájékoztatókon a szülők kérdéseire adtunk.
Később is sokat
segített, amikor ezeket elkezdtük beépíteni a pedagógiai programunkba.

 
Ebben az évben indult el az újlipótvárosi \emph{P10} iskolánk, és egy igen
kísérletező szellemű tanár csapat a gyakorlatban is lerakta az alapjait annak,
amit ma Budapest School modellnek nevezünk.

Kéri Juli, Kökényesi Ági, Csóti Heni és Galambos Attila az első
iskoláskorú csoportunk tanáraiként sok újdonságot hoztak be. A P10
kísérletezéseiből alakult az a gondolat, hogy váltogatható a
projektezés és az egy-egy tantárgyhoz kapcsolódó tudás, képesség
elsajátítása. És az is, hogy nekünk olyan rendszert kell építenünk, ahol a
tanárok rugalmasan alakíthatják az órarendet az épp aktuális
igényeknek megfelelően. Mivel magántanulók voltak a gyerekek, a P10
megmutatta, hogyan lehet megtanulni mindazt, ami a vizsgákhoz
szükséges, és még annál is többet.

 
2018 márciusában kezdtünk el azon dolgozni, hogy államilag elismert,
akkreditált iskolává válhassunk. Akkor még alternatív kerettantervnek
nevezték azt, amit ma már egyedi megoldások alkalmazásának
lehetőségeként definiál a jogszabály. Az volt a tervünk, hogy egy
rövid szövegben összefogaljuk a Budapest School modellt, és erre
kérünk engedélyt. Folyamatosan azon dolgoztunk, hogy minél inkább
megfeleljünk a folyamat közben megismert jogszabályoknak, és közben
megtartsuk egyediségeinket. Legalább egy évig formáltuk a szöveget,
éles vitákat folytattunk egyes részein, hetente szétszedtük, majd újra
egyberaktuk. Többek közt Bunyevácz Anasztázia, Fülöp Gabi, Herman
Dóra, Martinecz Viki, Milassin Rita, Papp Bori, Pilz Fanni, Porteleki
Sára végig aktívan segítették ezt a műhelymunkát.

 
Az elmúlt két évben főleg az EMMI Köznevelési Tartalomfejlesztési
Főosztályának és a Budapesti Kormányhivatal Köznevelési Osztályának
voltunk ügyfelei. Munkatársaik, kirendelt szakértőik erős és kemény
kereteket szabtak, és segítettek megérteni, hogy mit szabad, és mit nem.
A hivatalokkal való interakciók mindig a biztonságosabb
irányba vitték a rendszerünket, és inspirálták a kreativitásunkat.

