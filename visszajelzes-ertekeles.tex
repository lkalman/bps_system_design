\hypertarget{visszajelzes-ertekeles}{%
\section{Visszajelzés, értékelés}\label{visszajelzes-ertekeles}}

Ahhoz, hogy hatékony legyen a tanulás, fejlődés, fontos, hogy a
gyerekek, tanárok és szülők is tudják, hogy

\begin{enumerate}
\def\labelenumi{\arabic{enumi}.}
\tightlist
\item
  hol tart most a gyerek, mit tud most,
\item
  hova akar vagy kell eljutni, azaz, mi a célja,
\item
  mi kell ahhoz, hogy elérje a célját.
\end{enumerate}

Ezek mellett mindenkinek hinnie kell abban, hogy odafigyeléssel,
gyakorlással a gyerek meg tud tanulni egy konkrét dolgot. Fontos, hogy
magas legyen a gyerekek énhatékonysága, erős legyen az önbizalmuk, és
nem szabad félniük a hibázástól, a nem-tudástól, mert a tanulás első
lépése, hogy elfogadjuk, hogy valamit nem tudunk. Azaz fontos, hogy
fejlődésfókuszú legyen a gondolkodásuk (growth mindset) {\autocite{Dweck2006}}, azaz

\begin{enumerate}
\def\labelenumi{\arabic{enumi}.}
\setcounter{enumi}{3}
\tightlist
\item
  hinniük kell, hogy el tudják érni a céljukat.
\end{enumerate}

A visszajelzés, értékelés akkor jó és hasznos, azaz hatékony, ha ebben
a négy dologban segít. Mai tudásunk szerint ehhez:

\begin{itemize}
\tightlist
\item
  Rendszeresen visszajelzést kell kapniuk és adniuk.
\item
  A tanulási céloknak és visszajelzéseknek minél specifikusabbaknak kell
  lenniük azaz például ne a 8. osztályos \emph{matematikatudást}
  értékeljük, hanem hogy mennyire képes valaki \emph{fagráfokat
  használni feladatmegoldások során}. (Ez a konkrét példa a matematika
  tantárgy egyik tanulási eredménye.)
\item
  A \emph{„hol tartok most''} diagnózisnak mindig cselekvésre,
  viselkedésre kell vonatkoznia. Ne az legyen a visszajelzés, hogy
  \emph{„ügyes vagy egyenletekből''}, hanem hogy {„gyorsan és pontosan
  oldottad meg a 4 egyenletet''}. A legjobb, amikor a visszajelzés
  konkrét megfigyelésen alapul, és tudni, hogy mikor, hol történt az
  eset: \emph{„amikor társaiddal Minecraftban házat építettél, akkor
  pontosan kiszámoltad a ház területét''.}
\item
  Ha a cél nem a mások legyőzése, akkor a visszajelzés se tartalmazzon
  olyan állítást, ami másokkal való összehasonlítást tartalmaz (így kerüljük a \emph{tehetség}
  szót is, aminek bevett definíciója szerint az átlagnál jobb képesség).
  A mások szintjéhez hasonlított szint felmérése akkor (és csak akkor) fontos, amikor a
  cél egy versenyszituációban jó eredményt elérni.
\item
  A gyerek legyen részese a visszajelzésnek, ő is értékelje saját
  munkáját. Így fejlődik a
  % \href{https://educationendowmentfoundation.org.uk/public/files/Publications/Metacognition/EEF_Metacognition_and_self-regulated_learning.pdf}
  {\emph{metakogníciója}} {\autocite{eef:18}}.
  Értse, tudja, hogy miért és mire vonatkozik a visszajelzés.
\item
  A visszajelzésnek transzparensen hatással kell lennie a
  tanulásszervezésre. Legyen része a folyamatnak, és a gyerek, a tanár és
  a szülő is értse, hogy a visszajelzés alapján mit és hogyan csinálunk
  másképp.
\end{itemize}

\hypertarget{tobbszintu-visszajelzes}{%
\subsection{Többszintű visszajelzés}\label{tobbszintu-visszajelzes}}

A Budapest School-iskolákban a gyerekek többféle visszajelzést kapnak.

\begin{enumerate}
\def\labelenumi{\arabic{enumi}.}
\tightlist
\item
  Minden modul elvégzése után a modul céljai, témája, fókusza alapján a
  gyerekek visszajelzést kapnak a tanulásukról, alkotásaikról,
  eredményeikről, fejlődésükről és viselkedésükről.
\item
  Trimeszterenként a gyerekek visszajelzést kapnak arról, hogy
  általában hogyan haladtak a tanulási célok felé.
\item
  Ennek része, hogy a tantárgyi tanulási eredmények alapján hogyan
  haladt a gyerek a tantárgyakhoz tartozó követelmények teljesítésében.
  Ennek részletes leírását az
  ,,évfolyamok, osztályzatok, bizonyítvány'' fejezet (\ref{evfolyam-osztalyzatok-bizonyitvany}.~fejezet, \pageref{evfolyam-osztalyzatok-bizonyitvany}.~oldal) tárgyalja.
\item
  Szintén legalább trimeszterenként a gyerekek arról is kapnak
  visszajelzést, hogyan működnek a közösségben.
\end{enumerate}

\hypertarget{az-ertekelo-tablazatok}{%
\subsection{Az értékelő táblázatok}\label{az-ertekelo-tablazatok}}

A Budapest School visszajelzéseinek sokkal részletesebbeknek kell
lenniük, mint azt a tantárgyi érdemjegyek és osztályzatok lehetővé
teszik, ezért az érdemjegyek helyett a iskola értékelő táblázatokat
(angolul rubric) alkalmaz. Az értékelő táblázatban szerepelnek az
értékelés szempontjai és a szempontonkénti szintek, rövid leírásokkal.
Ezek alapján a gyerekek maguk is láthatják, hogy hol tartanak, hogyan
javíthatnak még a munkájukon. A táblázatok formája minden visszajelzés
esetén (értsd: modulonként, célonként) változtatható.

\hypertarget{ertekeles-nyelvezete}{%
\subsection{Az értékelés nyelvezete}\label{ertekeles-nyelvezete}}

Fontos alapelv, hogy minél inkább a folyamatot, cselekvést jutalmazzuk,
értékeljük. Tehát arra fókuszáljunk, hogy \emph{mit csinált} a gyerek,
és ne arra, hogy \emph{milyen} a gyerek. Azokra a viselkedésmintákra
adjunk megerősítő visszajelzést, amelyeket látni szeretnénk a gyerekben
később. Lehetőleg kerüljük a statikus jellemzőkre, személyiségjegyekre
vonatkozó értékelést. Így nem azt mondjuk, hogy \emph{„mindig jó vagy
matekból''}, hanem, hogy \emph{„kitartóan és odafigyelve oldottad meg a
feladatot''}.

Álljon itt néhány példa cselekvésre vonatkozó visszajelzésre.

\begin{itemize}
\tightlist
\item
  Fantasztikus, ma egy nagy kihívást választottál!
\item
  Bátran vállaltad a rizikót!
\item
  Nagyon jó! Tényleg sokat próbálgattad.
\item
  Kitartóan csináltad, erre nagyon büszke vagyok!
\item
  De jó, valami újat próbáltál ki ma!
\item
  Köszönöm, hogy ma valakinek segítettél.
\item
  Nagyon nagy öröm látni a haladásodat!
\item
  Ne feledd, mindannyian tudunk a hibáinkból tanulni. Örüljünk annak,
  hogy ma valamit jobban tudunk, mint előtte.
\item
  Hú, egy nehéz feladatot oldottál meg!
\item
  Szép munka! Kipróbáltál egy másik módszert.
\end{itemize}

Az iskolában történő folyamatos visszajelzés a mindennapok része.
