\hypertarget{pedagogiai-es-pszichologiai-hatter}{%
\section{Pedagógiai és pszichológiai
háttér}\label{pedagogiai-es-pszichologiai-hatter}}

Az iskolák működésének tanulmányozása mellett a Budapest School komoly
tudományos-elméleti háttérre alapozta koncepcióját.

Ezek közül is oktatási programjának központjában
%\href{http://tantrend.hu/hir/fejlodeskozpontu-szemlelet-az-iskolaban}
{Carol
Dweck fejlődésközpontú szemlélete} {\autocite{gerencser:18}}
áll. Emellett nagy hangsúlyt
fektetünk az alábbi elméletek gyakorlati alkalmazására is:

Reformpedagógiai irányzatok elméletei, különös tekintettel:

\begin{itemize}
\tightlist
\item
  Montessori-pedagógia (Maria Montessori),
\item
  kritikai pedagógia (Paulo Freire)
\item
  élménypedagógia (John Dewey)
\item
  felfedeztető tanulás (Jerome Bruner)
\item
  projektmódszer (William Kilpatrick)
\item
  kooperatív tanulás (Spencer Kagan)
\end{itemize}

Pszichológia és szociálpszichológiai kutatások eredményei:

\begin{itemize}
\tightlist
\item
  kognitív interakcionista tanuláselmélet (Jean Piaget)
\item
  személyközpontú pszichológia (Carl Rogers)
\item
  kommunikáció és konfliktuskezelés (Thomas Gordon)
\item
  erőszakmentes kommunikáció (Marshall Rosenberg)
\item
  pozitív pszichológia eredményei, különös tekintettel: flow-elmélet,
  kreativitás-kutatások (Csíkszentmihályi Mihály)
\item
  érzelmi és társas intelligencia (Peter Salovey, John D. Mayer, Daniel
  Goleman)
\item
  motivációkutatások; ezeket jól foglalja össze Daniel H. Pink műve
  % Motiváció 3.0 műve % LK: ???
  {\autocite{Pink2011}}
\item
  hősiesség pszichológiai alapjai (Phil Zimbardo)
\item
  fejlődésfókuszú szemlélet (Carol Dweck)
\end{itemize}
