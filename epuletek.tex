\hypertarget{epuletekre-vonatkozo-eloirasok}{%
\section{Épületekre vonatkozó
előírások}\label{epuletekre-vonatkozo-eloirasok}}

\hypertarget{telephelyek}{%
\subsection{Telephelyek}\label{telephelyek}}

A Budapest School több telephellyel rendelkezik. Minden telephely egy
intézményhez tartozik, az egységes iskola működéséhez nincs szükség
tagintézmények létrehozására (azaz a BPS nem hoz létre tagintézményeket,
kizárólag telephelyeket), tekintettel arra, hogy az iskolák szervezeti
irányítása a BPS modell és azt leképező szoftverrendszer segítségével
megoldható anélkül, hogy az egyes helyszíneket telephelyek helyett
tagintézményként lenne szükséges működtetni.

A telephelyek kialakíthatók nem nevelési, oktatási ingatlanokban.

\hypertarget{helyisegek}{%
\subsection{Helyiségek}\label{helyisegek}}

A telephelyeken található helyiségeket az iskola a 20/2012.(VIII.31.)
EMMI rendelet 2.számú mellékletében meghatározott felhatalmazás alapján
alakítja ki: \emph{Az eltérő pedagógiai elveket tartalmazó nevelési
program az eszköz- és felszerelési jegyzéktől eltérően határozhatja meg
a nevelőmunka eszköz és felszerelési feltételeit.}

BPS modellben részletezett strukturális, szervezeti és tanulásszervezési
módok miatt az iskola az iskolában az alábbi helyiségek megléte
szükséges és elégséges:

\begin{longtable}[]{@{}lll@{}}
\toprule
\begin{minipage}[b]{0.13\columnwidth}\raggedright
\textbf{helyiség megnevezése}\strut
\end{minipage} & \begin{minipage}[b]{0.26\columnwidth}\raggedright
\textbf{mennyiségi mutató}\strut
\end{minipage} & \begin{minipage}[b]{0.51\columnwidth}\raggedright
\textbf{megjegyzés}\strut
\end{minipage}\tabularnewline
\midrule
\endhead
\begin{minipage}[t]{0.13\columnwidth}\raggedright
tanterem (1)\strut
\end{minipage} & \begin{minipage}[t]{0.26\columnwidth}\raggedright
a telephelyen egyidőben jelenlévő gyerekek mind elférjenek\strut
\end{minipage} & \begin{minipage}[t]{0.51\columnwidth}\raggedright
min 15nm - max 120nm, jogszabályok által szabályozott gyerek / nm\strut
\end{minipage}\tabularnewline
\begin{minipage}[t]{0.13\columnwidth}\raggedright
laboratóriumok és szertárak (2)\strut
\end{minipage} & \begin{minipage}[t]{0.26\columnwidth}\raggedright
iskolánként egy\strut
\end{minipage} & \begin{minipage}[t]{0.51\columnwidth}\raggedright
laboratóriumi tevékenység végzésére alkalmas létesítmény üzemeltetőjével
kötött megállapodással is teljesíthető\strut
\end{minipage}\tabularnewline
\begin{minipage}[t]{0.13\columnwidth}\raggedright
tornaterem\strut
\end{minipage} & \begin{minipage}[t]{0.26\columnwidth}\raggedright
iskolánként egy\strut
\end{minipage} & \begin{minipage}[t]{0.51\columnwidth}\raggedright
kiváltható szerződéssel\strut
\end{minipage}\tabularnewline
\begin{minipage}[t]{0.13\columnwidth}\raggedright
tornaszoba\strut
\end{minipage} & \begin{minipage}[t]{0.26\columnwidth}\raggedright
székhelyen, telephelyeken egy\strut
\end{minipage} & \begin{minipage}[t]{0.51\columnwidth}\raggedright
csak ha ha a székhelynek v.telephelynek nincs saját tornaterme és akkor
is kiváltható szerződéssel\strut
\end{minipage}\tabularnewline
\begin{minipage}[t]{0.13\columnwidth}\raggedright
sportudvar\strut
\end{minipage} & \begin{minipage}[t]{0.26\columnwidth}\raggedright
székhelyen, telephelyeken egy\strut
\end{minipage} & \begin{minipage}[t]{0.51\columnwidth}\raggedright
kiváltható szerződéssel, vagy helyettesíthető alkalmas szabad
területtel\strut
\end{minipage}\tabularnewline
\begin{minipage}[t]{0.13\columnwidth}\raggedright
sportszertár (4)\strut
\end{minipage} & \begin{minipage}[t]{0.26\columnwidth}\raggedright
székhelyen, telephelyeken egy\strut
\end{minipage} & \begin{minipage}[t]{0.51\columnwidth}\raggedright
tornateremhez kapcsolódóan (kiváltható szerződéssel)\strut
\end{minipage}\tabularnewline
\begin{minipage}[t]{0.13\columnwidth}\raggedright
általános szertár (4)\strut
\end{minipage} & \begin{minipage}[t]{0.26\columnwidth}\raggedright
székhelyen, telephelyeken egy\strut
\end{minipage} & \begin{minipage}[t]{0.51\columnwidth}\raggedright
tornateremhez kapcsolódóan (kiváltható szerződéssel)\strut
\end{minipage}\tabularnewline
\begin{minipage}[t]{0.13\columnwidth}\raggedright
iskolatitkári iroda (5)\strut
\end{minipage} & \begin{minipage}[t]{0.26\columnwidth}\raggedright
iskolánként egy\strut
\end{minipage} & \begin{minipage}[t]{0.51\columnwidth}\raggedright
teljesen külön épületben, irodában is kialakítható\strut
\end{minipage}\tabularnewline
\begin{minipage}[t]{0.13\columnwidth}\raggedright
nevelőtestületi szoba\strut
\end{minipage} & \begin{minipage}[t]{0.26\columnwidth}\raggedright
iskolánként (telephelyenként) egy\strut
\end{minipage} & \begin{minipage}[t]{0.51\columnwidth}\raggedright
\strut
\end{minipage}\tabularnewline
\begin{minipage}[t]{0.13\columnwidth}\raggedright
könyvtár\strut
\end{minipage} & \begin{minipage}[t]{0.26\columnwidth}\raggedright
iskolánként egy\strut
\end{minipage} & \begin{minipage}[t]{0.51\columnwidth}\raggedright
nyilvános könyvtár elláthatja a funkcót, megállapodás alapján\strut
\end{minipage}\tabularnewline
\begin{minipage}[t]{0.13\columnwidth}\raggedright
orvosi szoba\strut
\end{minipage} & \begin{minipage}[t]{0.26\columnwidth}\raggedright
iskolánként egy\strut
\end{minipage} & \begin{minipage}[t]{0.51\columnwidth}\raggedright
amennyiben külső egészségügyi intézményben a gyerekek ellátása
megoldható, nem kötelező; lehet ideiglenesen kialakított\strut
\end{minipage}\tabularnewline
\begin{minipage}[t]{0.13\columnwidth}\raggedright
ebédlő (6)\strut
\end{minipage} & \begin{minipage}[t]{0.26\columnwidth}\raggedright
székhelyen, telephelyeken egy\strut
\end{minipage} & \begin{minipage}[t]{0.51\columnwidth}\raggedright
gyerek- és felnőttétkező közös helyiségben; tanteremmel közös helységben
is kialakítható, de egyidőben csak egy funkció\strut
\end{minipage}\tabularnewline
\begin{minipage}[t]{0.13\columnwidth}\raggedright
főzőkonyha\strut
\end{minipage} & \begin{minipage}[t]{0.26\columnwidth}\raggedright
székhelyen, telephelyeken egy\strut
\end{minipage} & \begin{minipage}[t]{0.51\columnwidth}\raggedright
ha helyben főznek\strut
\end{minipage}\tabularnewline
\begin{minipage}[t]{0.13\columnwidth}\raggedright
melegítőkonyha (7)\strut
\end{minipage} & \begin{minipage}[t]{0.26\columnwidth}\raggedright
székhelyen, telephelyeken egy\strut
\end{minipage} & \begin{minipage}[t]{0.51\columnwidth}\raggedright
ha nem helyben főznek, de helyben étkeznek\strut
\end{minipage}\tabularnewline
\begin{minipage}[t]{0.13\columnwidth}\raggedright
tálaló-mosogató\strut
\end{minipage} & \begin{minipage}[t]{0.26\columnwidth}\raggedright
székhelyen, telephelyeken egy\strut
\end{minipage} & \begin{minipage}[t]{0.51\columnwidth}\raggedright
ha nem helyben főznek, de helyben étkeznek; melegítőkonyhával közös
helyiségben is kialakítható\strut
\end{minipage}\tabularnewline
\begin{minipage}[t]{0.13\columnwidth}\raggedright
szárazáru raktár\strut
\end{minipage} & \begin{minipage}[t]{0.26\columnwidth}\raggedright
székhelyen, telephelyeken egy\strut
\end{minipage} & \begin{minipage}[t]{0.51\columnwidth}\raggedright
ha helyben főznek\strut
\end{minipage}\tabularnewline
\begin{minipage}[t]{0.13\columnwidth}\raggedright
földesáru raktár\strut
\end{minipage} & \begin{minipage}[t]{0.26\columnwidth}\raggedright
székhelyen, telephelyeken egy\strut
\end{minipage} & \begin{minipage}[t]{0.51\columnwidth}\raggedright
ha helyben főznek\strut
\end{minipage}\tabularnewline
\begin{minipage}[t]{0.13\columnwidth}\raggedright
éléskamra\strut
\end{minipage} & \begin{minipage}[t]{0.26\columnwidth}\raggedright
székhelyen, telephelyeken egy\strut
\end{minipage} & \begin{minipage}[t]{0.51\columnwidth}\raggedright
ha helyben főznek\strut
\end{minipage}\tabularnewline
\begin{minipage}[t]{0.13\columnwidth}\raggedright
élelmiszerhulladék-tároló (8)\strut
\end{minipage} & \begin{minipage}[t]{0.26\columnwidth}\raggedright
székhelyen, telephelyeken egy\strut
\end{minipage} & \begin{minipage}[t]{0.51\columnwidth}\raggedright
helyben étkeznek; nem kell külön helységben, ha biztonságosan
eltávolítható a hulladék\strut
\end{minipage}\tabularnewline
\begin{minipage}[t]{0.13\columnwidth}\raggedright
személyzeti WC\strut
\end{minipage} & \begin{minipage}[t]{0.26\columnwidth}\raggedright
székhelyen, telephelyeken egy\strut
\end{minipage} & \begin{minipage}[t]{0.51\columnwidth}\raggedright
ha fülkés, különálló WC kerül kialakításra, akkor a gyerek WC-t és
személyzeti WC-t nem kell megkülönbözteti\strut
\end{minipage}\tabularnewline
\begin{minipage}[t]{0.13\columnwidth}\raggedright
gyerek WC\strut
\end{minipage} & \begin{minipage}[t]{0.26\columnwidth}\raggedright
székhelyen, telephelyeken a gyereklétszám figyelembevételével\strut
\end{minipage} & \begin{minipage}[t]{0.51\columnwidth}\raggedright
ha fülkés, különálló WC kerül kialakításra, akkor nem kell elkülöníteni
a fiú és lány WC-t\strut
\end{minipage}\tabularnewline
\begin{minipage}[t]{0.13\columnwidth}\raggedright
mozgássérült WC\strut
\end{minipage} & \begin{minipage}[t]{0.26\columnwidth}\raggedright
a mozgássérült gyereklétszám figyelembevételével\strut
\end{minipage} & \begin{minipage}[t]{0.51\columnwidth}\raggedright
\strut
\end{minipage}\tabularnewline
\bottomrule
\end{longtable}

\begin{enumerate}
\def\labelenumi{\arabic{enumi}.}
\tightlist
\item
  Tanterem esetén az iskola azért nem osztályonként, hanem gyereklétszám
  arányában határozza meg a szükséges tantermek számát, mert
  osztálybontások esetén egy osztályra két tanteremnek kell jutnia,
  összevont foglalkozások esetén pedig fele annyi tanteremre van
  szükség. Ahhoz, hogy a tanárok meg tudják választani a megfelelő
  tanulásszervezési módszert, csoportbontást, ahhoz rugalmas terekre van
  szükség, és nem egyelő méretű tantermekre.
\item
  Amennyiben a 9-12. évfolyam tanulási eredményeinek eléréséhez
  szükséges laboratóriumi tevékenységek az iskolában található
  eszközökkel és termekben nem végezhetők el, az iskola és a fenntartó
  is kötelességet vállal arra, hogy a tanulási eredmény eléréséhez
  szükséges tárgyi feltételeket megállapodás, szerződés alapján
  biztosítsa.
\item
  A székhely és a telephelyek általánostól eltérően nagyobb száma és
  kisebb mérete más típusú intézmény vezetést tesz szükségessé, melynek
  része az egyes feladatellátási helyek rendszeres felkeresése. Az
  intézményvezető tanórát, foglalkozást nem, illetve minimális számban
  tart, így nem szükséges folyamatos jelenléte az adott székhelyen,
  telephelyen.
\item
  Az egyértelműség okán célszerű rögzíteni, hogy a sportszertár és
  általános szertár osztja a tornaterem sorsát.
\item
  A székhely és a telephelyek általánostól eltérően nagyobb száma és
  kisebb mérete miatt az adminisztráció működtetése célszerűen és
  hatékonyan megoldható a székhelyen kívül.
\item
  Technikailag megoldható hordozható hideg-meleg kézmosó alkalmazásával
  (népegészségügyi szerv által megerősített megoldás).
\item
  A melegítőkonyha funkciója az ebédlő részeként is kialakítható, ezen
  termek funkciója nem zárja ki az egy helyiségben történő kialakítást -
  különös tekintettel az Intézmény feladatellátási helyeinek
  általánostól eltérően nagyobb számára és kisebb méretére.
\item
  852/2004/EK rendelet alapján az élelmiszer-hulladékot, a nem ehető
  melléktermékeket és egyéb hulladékot a felgyülemlésük elkerülése
  érdekében a lehető leggyorsabban el kell távolítani azokból a
  helyiségekből, amelyekben élelmiszer található. A rendelet nem írja
  elő, hogy ezt külön helységben kell tárolni.
\end{enumerate}

A 20/2012. (VIII. 31.) EMMI rendelettől néhány terem esetén oly módon
kívánunk eltérni, hogy az adott terem ne legyen kötelező a
telephelyeken:

\begin{itemize}
\item
  \textbf{Aula}: Kevés gyereket befogadó épületben a legnagyobb tanterem
  általában elegendő a közösségi programokhoz. Nagyobb épületekben pedig
  a rendelet is kiválthatónak határozza meg.
\item
  \textbf{Csoportterem}: A tantermek csoportteremként is funkcionálnak.
  Tekintettel arra, hogy az iskolában az osztályok gyerekszáma tipikus
  esetben 10 fő alatti, és az egyes foglalkozásokra létrejött
  csoportokban sok esetben több osztály tanulói is részt vesznek, a
  tanulásra kialakított terek megkülönböztetése (tan-, csoport-,
  szakterem) nem szükséges.
\item
  \textbf{Egyéb raktár}: Amennyiben nem helyben étkeznek, a tanulói
  létszám feladatellátási helyenként alacsony mértéke okán nem szükséges
  egyéb helyiség kialakítása, közelebbről meg nem határozott tárgyak
  számára.
\item
  \textbf{Felnőtt étkező}: Gyermekvédelmi szempontból nem megoldható,
  hogy a gyerekek tanár jelenléte nélkül tartózkodjanak külön
  helyiségben az étkezés idejére. Ezért a felnőttek együtt étkeznek a
  gyerekkel.
\item
  \textbf{Intézményvezetői iroda}: A székhely és a telephelyek
  általánostól eltérően nagyobb száma és kisebb mérete más típusú
  intézmény vezetést tesz szükségessé, melynek része az egyes
  feladatellátási helyek rendszeres felkeresése. Az intézményvezető
  tanórát, foglalkozást nem, illetve minimális számban tart, így nem
  szükséges folyamatos jelenléte az adott székhelyen, telephelyen. Az
  iskolatitkári iroda olyan módon kerül kialakításra, hogy az az
  intézményvezető számára is használható.
\item
  \textbf{Iskolapszichológusi szoba}: Az iskolapszichológusi
  tevékenységek megoldhatók a tantermekben, különös tekintettel arra,
  hogy egész napos iskolaként működik az iskola.
\item
  \textbf{Logopédiai foglalkoztató, egyéni fejlesztő szoba}: A
  logopédiai foglalkozások megoldhatók a tantermekben, különös
  tekintettel arra, hogy egész napos iskolaként működik az iskola.
\item
  \textbf{Porta}: Funkciója, a gyerekek biztonságának biztosítása,
  szabályzatokkal, beléptető rendszerekkel megoldható.
\item
  \textbf{Szaktanterem a hozzá tartozó szertárral}: A szaktantermek
  funkciói a legtöbb esetben a tanteremben is megvalósíthatóak. Így
  számítástechnikai terem, társadalomtudományi szaktanterem, művészeti
  nevelés szaktanterem, technikai szaktanterem és gyakorló tanterem
  eszközigénye mobil eszközökkel is megoldható, úgy mint pl. laptop,
  rajztábla, satu, elektromos zongora. A természettudományi szaktanterem
  funkcióit pedig a szerződött laboratóriumokban elérhető a gyerekek
  számára. Amennyiben az adott tantárgyhoz, tantárgycsoporthoz tartozó
  tanulási eredmény az Intézményben található eszközökkel és termekben
  nem érhető el, az Intézmény és a fenntartó is kötelességet vállal
  arra, hogy a tanulási eredmény eléréséhez szükséges tárgyi
  feltételeket megállapodás, szerződés alapján biztosítsa.
\item
  \textbf{Személyzeti öltöző} és \textbf{személyzeti mosdó-zuhanyzó}: Az
  intézmény a tanárokon és óraadókon kívül állandó egyéb alkalmazottat
  egy feladatellátási helyszínen nem foglalkoztat. Az egyéb
  alkalmazottak (pl. karbantartó, takarító, kisegítők) ideiglenes, a
  feladatuk ellátásához szükséges ideig tartózkodnak az intézmény
  területén, ezért ilyen terem kialakítása szükségtelen.
\item
  \textbf{Teakonyha}: Tálaló-mosogató, ebédlő, és melegítőkonyha mellett
  külön teakonyha terem létesítése nem célszerű, az említett három
  ellátja a teakonyha funkcióját Intézményünkben, tekintettel az egy
  feladatellátási helyen egyszerre jelen lévő tanárok alacsony számára.
\item
  \textbf{Technikai alkalmazotti mosdó-zuhanyzó, WC helyiség}: Az
  intézmény a tanárokon és óraadókon kívül állandó egyéb alkalmazottat
  egy feladatellátási helyszínen nem foglalkoztat. Az egyéb
  alkalmazottak (pl. karbantartó, takarító, kisegítők) ideiglenes, a
  feladatuk ellátásához szükséges ideig tartózkodnak az intézmény
  területén, a tanári WC-t használata számukra elégséges.
\end{itemize}

\hypertarget{helyisegek-butorzata-es-egyeb-berendezesi-targyai}{%
\subsection{Helyiségek bútorzata és egyéb berendezési
tárgyai}\label{helyisegek-butorzata-es-egyeb-berendezesi-targyai}}

A Budapest School telephelyein az alábbi feltételek megléte szükséges és
elégséges:

\begin{itemize}
\tightlist
\item
  fenntartónak stabil szélessávú internethozzáférést kell biztosítania
  minden telephelyen;
\item
  minden tanteremben minden órán elérhetőnek kell lennie legalább egy
  internethez kötött számítógépnek vagy tabletnek;
\item
  tanulói asztalok, székek a gyerekek számának megfelelően;
\item
  eszköztároló szekrény;
\item
  tábla vagy flipchart legalább három különböző színű tollal;
\item
  szeméttároló.
\end{itemize}

Amennyiben a tanároknak többletigénye merül fel, akkor a szervezeti és
működési szabályzatban lefektett módon tudják az iskola és a fenntartó
segítségét kérni.
