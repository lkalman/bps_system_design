\hypertarget{az-iskola-kormanyzasa}{%
\section{Az iskola kormányzása}\label{az-iskola-kormanyzasa}}

Az iskola szervezetét a tagjai együtt alakítják. A szervezet élő,
folyamatosan változik, a következő alapelvek mentén:

\begin{itemize}
\item
  Gyorsan és folyamatosan tanuló, agilis szervezetként az iskola
  mindennap jobban támogatja a gyerekek tanulását, mint az előző napon.
\item
  Minden résztvevőnek a stabilitás, biztonság, kiszámíthatóság iránti igénye
  pont annyira fontos, mint a változás, a javulás, a fejlődés igénye.
\item
  A gyerekeket leginkább ismerő, a hozzájuk legközelebb álló tanárok
  (a gyerekekkel és a szülőkkel erős kapcsolatban) minél több helyzetben
  hoznak döntéseket.
\item
  Állandó az együttműködés, a partneri kapcsolat, a kölcsönös
  felhatalmazás
  nem csak a tanár---gyerek kapcsolatban,
  hanem a tanárok és adminisztrátorok között is.
\end{itemize}

Azt szeretnénk, hogy gyerekeink kreatív, környezetüket aktívan alakító,
mély partneri kapcsolatokban élő, problémamegoldó, csapatjátékos,
folyamatosan tanuló, a világot változtatni tudó felnőttekké váljanak.
Ehhez olyan iskolát építünk, ahol a tanárok kreatívak, környezetüket
aktívan alakítják, mély partneri kapcsolatban élnek, problémákat oldanak
meg, csapatban dolgoznak, és folyamatosan tanulnak.

Az iskolát mint szervezetet a \emph{szociokrácia}
szervezeti modellje alapján működtetjük, mert így tudjuk leginkább elérni,
hogy az együttműködésünk egyszerre legyen
örömteli és hatékony. Ez egy
olyan, az üzleti világban is kipróbált dinamikus döntéshozatali
rendszer, amely egyszerre segíti a harmonikus és örömteli közösségi
együttműködést, és jelent garanciát arra, hogy az együttműködés hatékony
lesz az egyes csapatokon belül.

\hypertarget{donteshozas}{%
\subsection{Döntéshozás}\label{donteshozas}}

Az iskola minden csapata (a tanulócsoportokat vezető tanulásszervezők,
a gyerekek egy csapata stb.) a \emph{hozzájáruláson alapuló döntési
mechanizmust} (consent based decision making) használja ahhoz, hogy a
szervezet gyorsan tudjon döntést hozni, \emph{és} minden tagja
hallathassa a hangját.

A csapat valamennyi tagja tehet javaslatokat, ha nem hatékony működést,
feszültséget, problémát észlel, és van rá megoldása. A javaslat
értelmezése után az érintettek mindegyikét meg kell hallgatni, hogy
\emph{elfogadhatónak} tartja-e a javaslatot, azaz hozzájárul-e a
változáshoz, mert \emph{„elég jónak és biztonságosnak találja, hogy
kipróbáljuk az új működést”} (\emph{,,is this good enough for now and
safe enough to try?''}). Fontos, hogy mindenki egyenként hallassa
a hangját. A javaslatot akkor tekintjük elfogadottnak, ha és amikor minden érintett
hozzájárult.

Mindenki kifejezheti a \emph{fenntartásait} (concern), és a csapat
feladata ezeket meghallani, és reagálni rájuk. A fenntartás azonban még
nem jelenti a javaslat elutasítását, csak fontos információt ad a döntés
végrehajtásához.

A javaslatot a csapat nem fogadja el, ha valamelyik csapattag
\emph{ellenzi} azt (objection). Az ellenzés egy én-üzenet, valami
ilyesmi: „ha a csapat ezt a döntést meghozná, akkor mélyen sérülne a
csapathoz való elköteleződésem, mert az én igényemet, ami~\ldots{}, nem
elégíti ki a javaslat. Nekem szükségem van~\ldots{}, ezért inkább
javaslom, hogy~\ldots{}''. Fontos, hogy az ellenvetést megfogalmazó
mondja el a saját igényeit, szükségleteit és tegyen új javaslatot, vagy
kérjen segítséget, hogy milyen új javaslatot tehetne. Ellenvetés esetén
a csapat együtt dolgozik azon, hogy új javaslatot találjon, ami az
ellenvetést feloldja, és az eredeti javaslat célja felé viszi a
szervezetet.

\hypertarget{a-hozzajarulas-nem-konszenzus}{%
\paragraph{A hozzájárulás nem
konszenzus}\label{a-hozzajarulas-nem-konszenzus}}

A konszenzusalapú döntések esetében mindenkinek egyet kell értenie abban,
hogy a döntés a legjobb, leghelyesebb, leghelyénvalóbb. A Budapest
School iskolában azt a kérdést tesszük fel inkább, hogy van-e valakinek
ellenvetése és a javaslat kellően biz\-ton\-sá\-gos-e ahhoz, hogy kipróbáljuk.
Nem azt a kérdést tesszük fel, hogy mindenki ezt a döntést hozta volna-e,
és hogy mindenki egyetért-e a döntéssel, hanem azt, hogy mindenki tudja-e
támogatni a csapat egy másik tagját, és hogy nincs-e olyan ismert kockázat,
ami az egyén vagy a szervezet szempontjából nem vállalható.

\hypertarget{a-hozzajarulas-nem-szavazas}{%
\paragraph{A hozzájárulás nem
szavazás}\label{a-hozzajarulas-nem-szavazas}}

A Budapest School-iskolában nem a többség dönt, és nem az számít, hogy
hányan akarnak egy döntés mellé állni. Mindenki hozhat döntést, amit
elfogad a csapat minden tagja, azaz egyetlenegy ellenvetés sincs.

\hypertarget{az-ellenvetes-nem-veto}{%
\paragraph{Az ellenvetés nem vétó}\label{az-ellenvetes-nem-veto}}

A vétójog a döntés megakadályozását jelenti. Amikor valaki
megvétóz egy döntést, akkor azzal a folyamat általában megakad. Az
iskola működésében használt ellenvetés egy beszélgetés megindítását
jelenti: „ezt én így nem tudom támogatni, helyette ezt javaslom inkább''.

\hypertarget{a-van-egy-jobb-otletem-nem-ellenvetes}{%
\paragraph{A „van egy jobb ötletem'' nem
ellenvetés}\label{a-van-egy-jobb-otletem-nem-ellenvetes}}

A szervezetnek nem az a feladata, hogy a legjobb döntéseket hozza, hanem
hogy amikor szükséges, akkor javítson a működésén. Ezért minden
döntéskor mindenkinek azt kell mérlegelnie először, hogy elfogadható-e
neki, hogy azt a bizonyos javaslatot kipróbálja a csapat. Attól, hogy
valaki jobb, más javaslatot is tud, attól még először az eredeti
javaslatot érdemes kipróbálni és tesztelni.

\hypertarget{minden-javaslat-csak-egy-hipotezis}{%
\paragraph{Minden javaslat csak egy
hipotézis}\label{minden-javaslat-csak-egy-hipotezis}}

Amikor valamit változtatunk a szervezet működésén, akkor egy kísérletbe
vágunk bele: kipróbáljuk, hogy az új működés tényleg jobb-e, megoldja-e
a problémát, feszültséget, ki\-e\-lé\-gí\-ti-e az igényeket. A döntés
támogatásakor ezt a próbálkozást támogatjuk.

\hypertarget{nem-dontunk-mindenrol-egyutt}{%
\paragraph{Nem döntünk mindenről
együtt}\label{nem-dontunk-mindenrol-egyutt}}

A Budapest School-iskolában csak az iskola és a tanulóközösség működését
megváltoztató kormányzási kérdésekről döntünk együtt.~A Budapest School
minden tagja szerepeiből kifolyólag fel van hatalmazva arra, hogy a
mindennapi döntéseit maga meghozhassa, ezért nem kell mindent
megbeszélnünk. A cél, hogy olyan szerepeket és rendszereket alakítsunk
ki, hogy a mindennapi döntéseket mindenki maga meg tudja hozni.

\hypertarget{csapatok-az-iskola-szervezeti-egysegei}{%
\subsection{Csapatok --- az iskola szervezeti
egységei}\label{csapatok-az-iskola-szervezeti-egysegei}}

A Budapest School csapatai (a szociokrácia terminológiájában a
\emph{körök}) önálló csoportok egy jól meghatározott céllal, felruházott
felelősséggel, döntési körrel. A csapatok maguk határozzák meg a saját
működésüket (policy making), és végzik el a saját feladatukat. Az
iskolában azok döntenek együtt, akik együtt dolgoznak, egy csapatban
(,,those who associate together govern together''). És fontos, hogy akik
együtt dolgoznak, jól legyenek egymással.

Azt is tudjuk, hogy akik egy munkát elvégeznek, azok a munka szakértői,
ezért ők tudnak arról a legjobban dönteni, hogy hogyan érdemes a
munkájukat szervezni, alakítani. Nincs főnök, külső szakértő, aki
megmondja egy csapatnak, mit és hogyan csináljanak addig, amíg a rájuk
felhatalmazott kereteken belül maradnak. Az természetes, hogy minden
segítséget, támogatást, információt megkapnak, amire szükségük van. De a
kormányzás az ő kezükben van.

\hypertarget{tanulokozossegek-tanulasszervezo-csapata}{%
\paragraph{Tanulóközösségek tanulásszervező
csapata}\label{tanulokozossegek-tanulasszervezo-csapata}}

A Budapest School szervezet állandó csapatai az egy-egy tanulóközösséget
vezető tanulásszervezők csapata, ami egy \emph{szociokratikus kör}. A
tanulóközösség gyerekeinek (családjainak) és tanárainak életét
meghatározó döntéseket maguk hozzák\break
meg. Így például a napirend, a
csoportbontások, a szülői értekezletek tematikája a saját döntéseik
alapján alakul ki. Fontos, hogy a csapattagok maguk tudják meghatározni,
kivel tudnak és akarnak együtt dolgozni, mikor és mit akarnak csinálni.

\hypertarget{csapatok-kapcsolodasa}{%
\paragraph{Csapatok kapcsolódása}\label{csapatok-kapcsolodasa}}

Egy-egy ember több csapatnak is tagja lehet. Egyrészt munkacsoportok
alakulhatnak egy-egy feladat elvégzésére, és a Budapest Schoolban egy
ember több részfeladatot is ellát. Másrészt a csapatokat kifejezetten
úgy alakítja a közösség, hogy legyenek köztük kapcsolódások, olyan
tagok, akik összekötik a csapatokat.

Vannak olyan csapatok, melyek elsődleges célja összekötni a kisebb
csapatokat. Például minden tanulóközösség tanárcsapata delegál egy
képviselőt az iskola közös naptárát létrehozó munkacsoportba.

\hypertarget{csapatok-vezetoi}{%
\paragraph{Csapatok vezetői}\label{csapatok-vezetoi}}

Minden csapatnak van egy \emph{vezetője} (a szociokrácia\break
terminológiájában a
\emph{circle leader}). A vezető feladata, hogy mindenki ismerje a csapat
célját, a választ a \emph{„miért létezünk?''} kérdésre, és hogy a csapat
működjön: tiszták legyenek a szerepek, és megtörténjen az, amiben
a csapat megállapodott, működjenek és fejlődjenek a folyamatok.

A Budapest School rendszerében a csapat vezetője nem az az ember, aki megmondja,
ki mit csináljon; nem osztja, ellenőrzi vagy felügyeli a feladatokat,
nem rúg ki, és nem vesz fel embereket, hanem szolgálja a csapatot
(servant leadership) azzal, hogy segíti a megállapodásokat betartani:
facilitál, moderál, szintetizál, kísér, kérdez. A csapatvezető
megválasztásához, mint minden szerep megválasztásához, a csapat minden
tagjának hozzájárulása szükséges.

\hypertarget{mi-van-amikor-egy-csapat-nem-tud-dontest-hozni-egyuttmukodni}{%
\subsection{Mi van, amikor egy csapat nem tud döntést hozni,
együttműködni}\label{mi-van-amikor-egy-csapat-nem-tud-dontest-hozni-egyuttmukodni}}

Amikor a csapattagok úgy érzik, hogy nem tudnak mindenki számára
elfogadható döntéseket hozni, nem haladnak, vagy megjelentek a játszmák,
és ezért már nem tudják a csapat célját szolgálni, akkor konfliktus,
feszültség alakul ki, aminek feloldásához segítséget hívhatnak be a
szervezet töb-\break
bi részétől.

A csapat folyamatos harmóniájáért folyamatosan dolgozni kell, ahogy az
egészségünk megőrzésének és a problémák megelőzésének is napi rutinná kell
válnia. Ezért a Budapest School csapatainak erősen ajánlott a
rendszeres visszajelzés, visszatekintés (retrospektív) és a team
coaching.
