\hypertarget{kulonbozo-tanari-szerepek}{%
\section{Különböző tanári szerepek}\label{kulonbozo-tanari-szerepek}}

A Budapest Schoolban a gyerekek azokat a felnőtteket tekintik
tanáruknak, akik minőségi időt töltenek velük, és segítik, támogatják
vagy vezetik őket a tanulásukban. Több szerepre bontjuk a tanár
fogalmát: a gyerek legjobban egy felnőtthöz kapcsolódik, a
\emph{mentorához}, aki rá különösen figyel. A foglalkozásokon
megjelenhetnek további tanárok, a \emph{szaktanárok}, akik egy adott
foglalkozást, szakkört, órát, kurzust vezetnek. Szervezetileg minden
tanulóközösségnek van egy állandó \emph{tanárcsapata}, a
\emph{tanulásszervezők}, akik a mentorokból és a szaktanárokból állnak
össze. A tanulóközösség mindennapjait a tanulásszervezők határozzák meg.

A tanulásszervezők, mentorok legalább egy tanévre elköteleződnek,
szemben a szaktanárokkal, akik lehet, hogy csak egy pár hetes projekt
keretében vesznek részt a munkában. A tanulásszervezők általában
mentorok is, de nem minden esetben. Egy tanulásszervező lehet több
tanulóközösségben is ebben a szerepben, és így mentor is lehet több
tanulóközösségben.

\hypertarget{mentor}{%
\subsection{A mentor}\label{mentor}}

Minden gyereknek van egy \emph{mentora}, aki a saját céljainak
megfogalmazásában és a fejlődése követésében segíti. Minden mentorhoz
több gyerek tartozik, de nem több, mint 12. A mentor együtt dolgozik a
tanulóközösség többi tanulásszervezőjével, a szülőkkel és az általa
mentorált gyerekekkel. A mentor segít az általa mentorált gyereknek,
hogy a tantárgyi fejlesztési célok és a saját magának megfogalmazott
saját célok között megtalálja az egyensúlyt, és segít megalkotni a
gyerek \emph{saját tanulási tervét}. A mentor a kapocs a Budapest School, a szülő és a gyerek között.

\begin{itemize}
\tightlist
\item
  Képviseli a Budapest Schoolt, a tanulóközösséget a szülő felé.
\item
    Ismeri a Budapest Schoolt, a lehetőségeket, a tanulásszervezés
    folyamatait.
\item
  Együtt tanul a többi Budapest School-mentorral, együtt dolgozik a
  tanártársaival.
\item
  Ismeri, segíti, képviseli a gyereket.
\item
  Az első trimeszter alatt a mentor megismeri mentoráltjának
  személyiségjegyeit, képességeit, érdeklődését, motivációit. Megismeri
  a mentorált családot.
\item
  Trimeszterenként a mentorált gyerekkel és szülőkkel egyetértésben
  kialakítja, majd folyamatosan monitorozza a mentorált gyerek
  haladását.
\item
  Tudja, hol és merre tart a mentoráltja, ismeri a képességeit,
  körülményeit, szándékait, vágyait.
\item
  Minimum kéthetente találkozik mentoráltjával. Követi, tudja, hogy a
  gyerek hogy érzi magát az iskolában, a családban, a mindennapokban.
\item
  Segít a saját célok elérésében, felügyeli a haladást.
\item
  Megerősíti a mentoráltjai pszichológiai biztonságérzetét.
\item
  Mentorként nyomon követi, monitorozza a mentoráltjai fejlődését, és
  szükség esetén továbblendíti, inspirálja őket. Visszajelzéseket ad a
  mentoráltjainak. A mentorált gyerekkel
  együtt rendszeresen reflektál 
  a tanulási céljaikra és haladásukra.
\item
  Segít abban, hogy az elért célok a portfólióba kerüljenek, biztosítja,
  hogy a mentorált gyerek portfóliója friss legyen.
\item
  Összeveti a portfólió tartalmát a tantárgyak fejlesztési céljaival, és
  jelzi, ha egy-egy tantárgyból hiány mutatkozik.
\item
  Segíti a mentorált választásait.
\item
  Együtt dolgozik, gondolkozik a szülőkkel, képviseli igényüket a
  közösség előtt.
\item
  Megállapodik a családdal a kapcsolattartás szabályaiban.
\item
  Bevezeti a családot a Budapest School rendszerébe.
\item
  Erős partneri kapcsolatot épít ki a szülőkkel.
\item
  Rendszeresen információt oszt meg.
\item
  Elérhető.
\item
  Asszertíven kommunikál.
\item
  Konkrét, specifikus, mérhető megállapodásokat köt.
\item
  Segít a gyerekekkel közös célokat állítani.
\item
  Amikor a gyereknek külső fejlesztésre, mentorra, tanárra, trénerre van
  szüksége, akkor a családot segíti a megfelelő segítő felkeresésében, a
  külsőssel kapcsolatot tart, és konzultál a mentoráltja haladásáról.
\item
  A szülő számára a mentor az elsődleges kapcsolattartó a különféle
  iskolai ügyekkel kapcsolatban.
\end{itemize}

A mentor egyszerre felelős a mentorált gyerek előrehaladásának
segítéséért, és közös felelőssége van a mentortársakkal, hogy az
iskolában a lehető legtöbbet tanuljanak a gyerekek. A mentor
folyamatosan figyelemmel követi az egyéni tanulási tervben
megfogalmazottakat, és ezzel kapcsolatos visszajelzést ad a mentoráltnak
és a szülőnek.

\hypertarget{szaktanarok}{%
\subsection{Szaktanárok}\label{szaktanarok}}

A szaktanárok egy-egy foglalkozás, vagy foglalkozássorozat, modul, azaz
tanulási egység megszervezéséért, lebonyolításáért felelnek. A
mentorok fő fókusza, hogy a gyerekek jól vannak-e. A szaktanárok fő
fókusza, hogy mit tanulnak a gyerekek. Ők segítik a gyerekeket egy
tanulási cél felé való haladásban akár egyetlen alkalommal, vagy éppen
egy egész trimeszteren át tartó tanulási, alkotási folyamatban.

Ők általában az adott tudományos, művészeti, nyelvi vagy bármilyen más
terület szakértői.

\begin{itemize}
\tightlist
\item
  Izgalmas, érdekes foglalkozásokat tartanak, amire felkészülnek, és amiben a
  gyerekeket flow-ban {\autocite{Csikszentmihalyi1991}} tudják tartani.
\item
  Amikor a gyerekek velük vannak, akkor folyamatosan dolgoznak,
  figyelnek, fókuszálnak, koncentrálnak, tanulnak.
\item
  Kedvesen és határozottan vezetik a csoportot, figyelnek arra, hogy
  mindenkit bevonjanak.
\item
  Változatos, gazdag módszertani eszköztárukból mindig a foglalkozáshoz
  megfelelő módszert tudják elővenni.
\item
  A foglalkozásaik fejlesztik a gyerekek kritikai gondolkozását,
  kreativitását, kommunikációját, úgy általában a 21. századi
  kompetenciákat {\autocite{Trilling2009}}.
\end{itemize}

\vspace*{.5ex}
\hypertarget{tanulasszervezo}{%
\subsection{Tanulásszervező}\label{tanulasszervezo}}

Egy tanulóközösséget vezető tanárok állandó tanári csapatát 1--12
tanulásszervező alkotja, akik egyedileg meghatározott szerepek mentén a
tanulóközösség mindennapjainak működtetéséért felelnek. A
tanulásszervezők tarthatnak foglalkozásokat, sőt kívánatos is, hogy
dolgozzanak a gyerekekkel, ne csak szervezzék az életüket. Ők rendelik
meg a külső szaktanároktól a munkát, ilyen értelemben a tanulási utak
projektmenedzserei.

\begin{itemize}
\tightlist
\item
  Kiszámítható, átlátható rendszert építenek, ahol a szülők és a gyerekek is
  biztonságban, informálva, bevonva érzik magukat.
\item
  A tanulási célokkal rezonáló foglalkozásokat, modulokat hirdetnek meg,
  szerveznek le.
\item
  A szülők és a gyerekek is érzik, értik, hogyan „történik'' a tanulás.
  Biztosítják számukra az átláthatóságot.
\item
  A szaktanároknak megadnak minden szükséges információt, kontextust, hogy
  hatékonyan tudják végezni a munkájukat.
\item
  Megadják a résztvevőknek az informált választás lehetőségét.
\end{itemize}

\hypertarget{a-kozos-szerepek}{%
\subsection{A közös szerepek}\label{a-kozos-szerepek}}

Minden BPS-es tanulásszervező, mentor és szaktanár egyszerre
\emph{csapattag}, \emph{BPS-tag}, \emph{EQ-nindzsa},
\emph{változásmenedzser (Change Agent)} és \emph{SNI-szakértő}.

\hypertarget{csapattag}{%
\paragraph{Csapattag}\label{csapattag}}

\begin{itemize}
\tightlist
\item
  Rendszeresen jelen van a tanári csapat megbeszélésein.
\item
  El lehet érni telefonon vagy online, a csapattal megállapodott
  kereteken belül.
\item
  Feladatokat vállal magára, és azokat megbízhatóan, határidőre
  végrehajtja.
\item
  Kooperatív és támogató a közös munkák, ötletelések, megbeszélések
  alatt, képviseli a saját nézőpontját, gondolatait, érzéseit, miközben
  a csapat és a többiek igényeire is figyel.
\item
  Kifejezi támogatását, ellenérveit és javaslatait a jobb megoldás
  érdekében.
\item
  A visszajelzést keresi, a kritikát jól fogadja, és megfontolja,
  átgondolja a lehetséges változtatásokat.
\item
  A kollégák fejlődését segíti rendszeres visszajelzésekkel.
\end{itemize}

\hypertarget{bps-tag}{%
\paragraph{BPS-tag}\label{bps-tag}}

\begin{itemize}
\tightlist
\item
  Rendszeresen jelen van a közös tanári eseményeken.
\item
  Részt vesz a tanulóközösségének és a Budapest Schoolnak a
  fejlesztésében.
\item
  Proaktívan alakít ki rendszereket, folyamatokat, és a legjobb
  gyakorlatokat megosztja BPS-szinten.
\item
  Közös témákban aktív, hozzászól, alakítja a véleményével és tudásával
  a BPS rendszerét.
\end{itemize}

\hypertarget{az-erzelmi-intelligencia-eq-nindzsa}{%
\paragraph{Az érzelmi intelligencia (EQ-nindzsa)}\label{az-erzelmi-intelligencia-eq-nindzsa}}

\begin{itemize}
\tightlist
\item
  A gyerekek pszichológiai biztonságérzetét megerősíti.
\item
  Olyan visszajelzéseket ad, amelyek az erőfeszítésekre, a befektetett
  energiára, munkára és a jövőbeni fejlődésre fókuszálnak (growth
  mindset), pozitív megerősítést alkalmaz, épít a gyerek erősségeire, és
  egyértelműen megfogalmazza, mit tehetne másként.
\item
  Odaforduló, barátságos, nyitott.
\item
  Munkája során gondoskodik a gyermek személyiségének, énképének,
  énhatékonyságának fejlődéséről.
\item
  Figyelembe veszi a gyerekek egyéni képességeit, adottságait,
  fejlődésének ütemét, szociokulturális helyzetét.
\item
  Előmozdítja a gyerek erkölcsi fejlődését.
\item
  Segít a gyerekeknek jobban működni a csoportban.
\end{itemize}

\hypertarget{sni-szakerto}{%
\paragraph{SNI-szakértő}\label{sni-szakerto}}

\begin{itemize}
\tightlist
\item
  Jól tudja kezelni az egyéni, különleges bánásmódot, speciális nevelést
  igénylő gyerekeket a csoportban.
\item
  A különleges bánásmódot igénylő gyermekek családjával, a nevelést,
  oktatást segítő más szakemberekkel együttműködik, hogy a\break
  gyerek
  támogatása minden esetben megtörténjen.
\item
  Akinek külső segítségre van szüksége, azoknak a szüleivel ezt
  proaktívan egyezteti, és menedzseli a folyamatot.
\end{itemize}

\hypertarget{change-agent}{%
\paragraph{Change agent}\label{change-agent}}

\begin{itemize}
\tightlist
\item
  A nehézségeken könnyen továbblendül. Megtalálja a kiégés ellenszerét,
  tudatosan keresi a feltöltődés alkalmait és lehetőségeit.
\item
  Nemcsak a problémákra, hanem a megoldásokra keresésére is odafigyel.
  Keresi, hogy mire lehet hatása, mit tud előremozdítani, és meg is
  teszi a szükséges lépéseket.
\item
  Felismeri, elismeri és megünnepli az előrelépéseket, a közös
  sikereket.
\end{itemize}
