\hypertarget{a-tanulni-tanulas-het-pillere}{%
\section{A tanulni tanulás hét
pillére}\label{a-tanulni-tanulas-het-pillere}}

A tanulás egyénenként változó, és ha vegyes korosztályokban is, mégis\break
egy közösségben valósul meg. \emph{A tanulni tanulás hét pillére\/} minden
korosztályban meghatározza a Budapest School működését.

\hypertarget{tanulni-tanulunk-fejlodeskozpontuak-vagyunk}{%
\subsection{Tanulni tanulunk --- fejlődésközpontúak
vagyunk}\label{tanulni-tanulunk-fejlodeskozpontuak-vagyunk}}

A BPS iskolában mindig ott, akkor és annak kell történnie a gyerekekkel,
ami őket a fejlődésükben a leginkább támogatja. Minden, ami az iskolában
történik, újra és újra erre az alapkérdésre kell, hogy visszatérjen. Az
segíti a leginkább a gyerekek fejlődését, amit most csinálunk, vagy
változtatnunk kell rajta? A Budapest School tehát rugalmas és
integratív, a gyerekek fejlődéséhez igazodó.

A gyerekek tanulása
% \href{https://pszichoforyou.hu/carol-s-dweck-minden-hiba-ujabb-lehetoseg-tanulasra/}
\emph{fejlődésközpontú
szemléletben} {\autocite{szabo:17}} (angolul: growth mindset) történik. Erőfeszítéseik
segítségével képességeik fejleszthetők, megváltoztathatóak, ha kellő
támogatást kapnak. Számukra inspiráló a kihívás, és a hibázás kevésbé
töri le lelkesedésüket. A gyerekek és a tanárok számára ezáltal nagyobb
fontossággal bír a tanulásba fektetett erőfeszítés és a fejlődés, mint
az aktuális teljesítmény.

A hibázás inkább válik a gyakorlás és az új megismerésének jelzőjévé.
Ennek feszültségmentes kezelése kulcsfontosságú abban, hogy a gyerekek
merjék feszegetni a saját határaikat, hogy magabiztosan dolgozzanak
azon, hogy képességeiket, ismereteiket vagy gyakorlataikat folyamatosan
fejlesszék. A hibázásból való tanulás fő célja, hogy mindig új hibákat
ismerjenek meg, és a korábbiakra minél jobb megoldásokat találjanak a
gyerekek.

\hypertarget{tanulni-tanulunk-sajat-celokat-allitunk}{%
\subsection{Tanulni tanulunk --- saját célokat
állítunk}\label{tanulni-tanulunk-sajat-celokat-allitunk}}

A gyerekek saját erősségeiket fejlesztve saját célokat állítanak, közben
céljaikat folyamatosan igazítják a világ adta lehetőségekhez és
szükségletekhez.

A tanulás egésze egy olyan folyamatként írható le, amely különböző
állomásokra, rövid célokra bontható. A tanulási célok állításának
folyamata, annak minősége az évek alatt folyamatosan változik, egyre
tudatosabbá, pontosabbá, komplexebbé válik. Ennek feltétele, hogy már az
első évektől el kell kezdeni az életkornak és személyes
lehetőségeknek megfelelő célállítás gyakorlását.

A gyerekek jellemzője a kíváncsiság, az igény a felfedezésre,
tapasztalásra. A BPS-gyerekek számára \emph{a tanulás egy önvezérelt aktív
folyamat}, melynek megtartása és folyamatos fejlesztése a Budapest
School tanárainak legfőbb feladata. Nem csak azokra a képességekre
fókuszálunk, amelyekre ma szükségük van, mert így tudásuk veszítene a
világ változásával a korszerűségéből. Az iskola abban segíti őket, hogy
megtaníthassák maguknak azokat a képességeket, amelyekre épp az adott
élethelyzetükben szükségük lesz. A tanulás így élményszerűvé válik,
ismeretszerző jellege csökken, és nő az önálló felfedezés lehetősége.

\hypertarget{tanulni-tanulunk-mindig-es-mindenhol}{%
\subsection{Tanulni tanulunk --- mindig és
mindenhol}\label{tanulni-tanulunk-mindig-es-mindenhol}}

A tanulásra a Budapest Schoolban mindig van lehetőség. A tanulási
környezet pontos kialakítása a gyerekek igényeitől, fejlettségi
szintjétől és\break
korosztályától is függ. A tanulási rend meghatározásáért a
BPS tanulásszervezői felelnek.

A tanulás során jut idő egyéni és csoportos, gyakorló, ismeretszerző és
alkotó foglalkozásokra is. A tanulási egységek között van idő
fellélegezni és felkészülni az újabb modulokra. Van, amikor az is
megoldható, hogy a gyerekek és tanárok újratervezzék az időrendjüket, ha
egy tanulási egység nagyon magával ragadja a gyerekeket, és nagyon benne
maradnának abban a tevékenységben.

A tanulás az iskolában nem ér véget. A tanulás szeretetének
kialakulásával folyamatossá válik az ezzel való foglalkozás, így a
Budapest School tanulásnak tekinti az otthon vagy a szünetekben
tanulással eltöltött időt is, ahol néha hatékonyabb módon tud egy gyerek
gyakorolni, kutatni, alkotni, mint az iskolában, amikor társaival van
egy közösségben és ezáltal számos más inger is éri.

Az iskolában történő tanulással egyenrangúnak tekintjük az otthon\break
tanulást, az iskolától független iskola utáni programokat, a (nyári)
táborokat, a családi utazásokat, a vállalatoknál töltött gyakornoki
időt, az egyéni tanulást és projekteket. A tanulás bárhol és bármikor
történhet, amíg az eléri a célját. Célunk, hogy a gyerekek mindenhol és
mindig tanuljanak.

\hypertarget{tanulni-tanulunk-egyutt-egymastol}{%
\subsection{Tanulni tanulunk --- együtt,
egymástól}\label{tanulni-tanulunk-egyutt-egymastol}}

A tanulás egyénileg és csoportokban is történhet. A csoportok
megszervezése mindig azon múlik, hogy az adott tanulási célt mi
szolgálja a legjobban. Ennek megfelelően a gyerekek nem állandó, hanem a
tanulási célokhoz, az érdeklődéshez, a képességi szintekhez alkalmazkodó
rugalmas csoportokban tanulnak. A tanulás ezáltal kevert korcsoportokban
is történhet, akár egy nagy családban. Együtt, egymástól tanulnak a
gyerekek, egymást segítik a fejlődésben. Az egymásnak adott
visszajelzések, megerősítések révén folyamatosan alakul ki a tanulás
tisztelete és a képességek fejlesztésébe vetett hit, az együttműködésre
és az alkalmazkodásra való képesség.

A közösségben tanulás módja nagyban függhet attól, hogy egy gyerek
mennyire zárkózott, mennyire tud és akar önállóan tanulni. A csoportos
munkák során alapelv, hogy a zárkózott gyerekek is lehetőséget kapjanak,
hogy csöndesen vagy kisebb csoportban végezhessék a munkájukat,
mondhassák el ötleteiket. Az egyéni tanulásban minden gyereknek
lehetőséget kell adni arra, és segíteni kell abban, hogy önállóan,
fókuszáltan tudjon tanulni.

\hypertarget{tanulni-tanulunk-alkotunk-es-felfedezunk}{%
\subsection{Tanulni tanulunk --- alkotunk és
felfedezünk}\label{tanulni-tanulunk-alkotunk-es-felfedezunk}}

A tanulás három rétege, az ismeretszerzés, a gondolkodás fejlesztése és
az alkotás egyszerre jelenik meg a Budapest School mindennapjaiban. Az
alkotó munka rugalmas időkereteket, változó csoportbontásokat, és a
projektmódszerek sokszínű alkalmazását igényli. A tanulás ilyenkor
sokszor inkább alkotássá válik, az ismeret pedig termékké változik.

A gyerekek önmaguk és a világ számára releváns kérdésekkel foglalkoznak,
amihez külső szakértőket is bevonnak, ha szükséges. A tanulás tehát
célokhoz, nem pedig tárgyakhoz kötött. A tanulás tartalmát igazítjuk a
tanulás céljához, ezáltal az egyes tudományterületek, művészetek, vagy
épp mesterségek gyakran keverednek egymással egy-egy modulon belül.
Szintén a célhoz igazított tanuláshoz kötődik a kutató-felfedező
attitűd, ami az ismeretlen felfedezésére, a megválaszolhatatlan
megválaszolására irányul.

\vspace*{.5ex}
\hypertarget{tanulni-tanulunk-egymasra-figyelunk}{%
\subsection{Tanulni tanulunk --- egymásra
figyelünk}\label{tanulni-tanulunk-egymasra-figyelunk}}

A gyerekek tanulását családi hátterük változása, egyéni problémák,
számos mindennapi esemény befolyásolhatja. Ezek figyelembevétele a
mindennapokban, a \emph{mentortanárral} való bizalmi viszonynak
köszönhetően válik lehetségessé. Ennek a kapcsolatnak az alapjait ezért
a partnerség, az értő figyelem adja.

A Budapest School emellett kiemelt figyelmet fordít arra, hogy a sajátos
nevelési igényű tanulók is lehetőséget kapjanak a csoportban való
munkára, amennyiben az a közösség számára is hasznos. Tanulásukat, ha
szükséges, külső szakember segíti. A Budapest School a hátrányos
helyzetű gyerekek számára is biztosítani kívánja az elfogadó, fejlesztő
környezetet. Az egyenlő bánásmód megvalósulása érdekében olyan
differenciált tanulási környezetet alakít ki, ami biztosítja a minél
nagyobb mértékű inkluzivitást.

\vspace*{.5ex}
\hypertarget{tanulni-tanulunk-tanulva-tanitunk}{%
\subsection{Tanulni tanulunk --- tanulva
tanítunk}\label{tanulni-tanulunk-tanulva-tanitunk}}

A Budapest School \emph{tanulásszervezői} partnerként, a tanulás
folyamán segítő társként vannak jelen a gyerekek életében. A tanulás
tanórák helyett pontos tanulási célokat tartalmazó tanulási modulokból
épül fel, melyek során a tanár az adott cél eléréséhez szükséges
eszközöket, tanulási segédleteket biztosítja.

A tanár akkor és annyira segíti a gyerekeket a saját céljuk elérésében,
amennyire az épp szükséges, és folyamatosan tekintettel van arra, amit a
gyerek saját fejlődési üteme megkíván. Ehhez tudatosan kell kezelnie
nemcsak a gyerekeket érintő fejlesztési lehetőségeket, hanem azt is
pontosan látnia kell, hogy egy adott tanulási cél elérésének milyen
készségszintű vagy gyakorlatias alapfeltételei vannak. Ezért a gyerekek
tanulási célját támogatandó segítenie kell abban, hogy a gyakorló idő,
az alkotó idő és az új ismeret megszerzésének ideje folyamatos
egyensúlyban legyen. A tanár a gyerekekkel együtt fejlődik, saját
tanulása a segítői szerepben folytonos.
