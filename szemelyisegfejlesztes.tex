\hypertarget{az-egeszseges-es-boldog-gyerek}{%
\section{Az egészséges és boldog
gyerek}\label{az-egeszseges-es-boldog-gyerek}}

A Budapest School-gyerekek boldogak, egészségesek, hasznosak
közösségüknek. Képesek önmaguknak célokat állítani, azokat elérni.
Képesek már kisgyerekkortól sajátjukként megélni a tanulást, és ahhoz
kapcsolódóan célokat elérni, és fokozatosan tanulják meg azt, hogy
egyénileg és csoportosan is tudnak nagyszabású projekteket véghezvinni.
Tesznek a saját egészségükért, jövőjükért, társaikért, kapcsolódnak
önmagukhoz és társaikhoz.

A gyerekek személyiségfejődését két szinten támogatja az iskola. Az első
szintet a mindennapi működés adja, mert az iskola működése önmagában
személyiség- és egészségfejlesztő hatással bír. A második szintet a
\emph{harmónia} kiemelt fejlesztési irányelv biztosítja, ami holisztikus
megközelítésével támogatja a gyerekek fizikai, lelki jóllétét és
kapcsolódásukat a környezethez.

\hypertarget{mukodesbol-adodo-fejlesztesek}{%
\subsection{A működésből adódó
fejlesztések}\label{mukodesbol-adodo-fejlesztesek}}

Az iskola alapműködése, hogy a gyerekek csoportban, közösségben élnek,
tanulnak, dolgoznak, ezért „természetes'', hogy fejlődik az
\emph{empátiájuk, kooperációs, kollaborációs képességük és érzelmi
intelligenciájuk}. Alapelvünk: minél közelebb áll az iskola működése
a jövő hétköznapjaihoz, a családhoz és a munkahelyhez, annál jobban
támogatja a boldog
családi életre, a sikeres munkahelyre való felkészülést már önmagában az iskolában
való aktív részvétel is. Ehhez hasonlóan a támogató,
funkcionális, boldog családban felnőtt gyerekek nagyobb valószínűséggel
lesznek maguk is egészségesebbek és boldogabbak.

A \emph{fejlődésfókuszú gondolkodásmód} kialakítását kulcstényezőnek
gondoljuk a gyerekeink hosszú távú boldogulásához. Ezért az iskolában a saját célok
által irányított tanulási környezettől kezdve a jutalmazás, értékelés,
visszajelzés módjáig minden azt a célt szolgálja, hogy a
gyerekek képesek legyenek pozitívan gondolkodni magukról, ami az
ép és egészséges embernek talán egyik legfontosabb jellemzője.

A \emph{teljes körű iskolai egészségfejlesztést} az alábbi négy
egészségfejlesztési feladat rendszeres végzése adja:

\begin{itemize}
\item
  egészséges táplálkozás megvalósítása (elsősorban megfelelő, magas
  minőségű, lehetőleg helyi alapanyagokból);
\item
  mindennapi testmozgás minden gyereknek (változatos foglalkozásokkal,
  koncentráltan az egészségjavító elemekre, módszerekre, pl. tartásjavító
  torna, tánc, jóga);
\item
  a gyerekek érett személyiséggé válásának elősegítése személyközpontú
  pedagógiai módszerekkel és a művészetek személyiségfejlesztő
  hatékonyságú alkalmazásával (ének, tánc, rajz, mesemondás, népi
  játékok, stb.);
\item
  a környezeti, médiatudatossági, fogyasztóvédelmi, balesetvédelmi\break
  egészségfejlesztési modulok, modulrészletek hatékony (azaz
  ,,bensővé váló'') oktatása.
\end{itemize}

\hypertarget{egeszsegugyi-felmeres-szervezese-es-hatasa-a-gyerekek-eletere}{%
\subsection{Egészségügyi felmérések szervezése és ezek hatása a gyerekek
életére}\label{egeszsegugyi-felmeres-szervezese-es-hatasa-a-gyerekek-eletere}}

A Budapest School-gyerekek megelőző jelleggel rendszeresen iskolaorvosi,
védőnői és fogorvosi felülvizsgálaton vesznek részt. Az orvosi, védőnői
és fogorvosi vizsgálatot a fenntartó a mindenkori jogszabályokban
meghatározott rendszerességgel szervezi meg. Jelenleg ezeket külső
helyszínen, megbízott orvossal, fogorvossal és védőnővel szervezi meg a
fenntartó.

Minden tanulóközösség tanulásszervező csapata a tanév megkezdése előtt
kijelöli az egészségnapokat: amikor a védőnői, orvosi, fogorvosi és
egyéb fizikai és mentális felülvizsgálatokat megszervezi. Húsz főig egy,
afölött pedig két napot kell megjelölni és a fenntartóval egyeztetni.
Egyeztetni azért kell, hogy a különböző tanulóközösségek között ne
legyen időpontütközés. Ha a gyerek az egészségnapon hiányzik az
iskolából, akkor a szülő feladata a felülvizsgálatot megszerveznie.

Ezeken a napokon a gyerekek és a tanárok iskolaidőben elutaznak a
rendelőkbe, felkeresik az orvost, fogorvost, védőnőt. Mivel a gyerekek
kivizsgálása feltehetőleg egyesével történik, az éppen nem soron levőknek
sokat kell várniuk. Ezért a tanárok erre az időre
mikromodulokat terveznek az egészség témájában.  Ilyenkor a gyerekek olyan tanulási eredményeket
érhetnek el, mint a \emph{„Tisztában van az egészség megőrzésének
jelentőségével, és tudja, hogy maga is felelős ezért''} (5. évfolyam 1.
félév).

A felülvizsgálatok eredményeit a gyerekek a mentorokkal megbeszélik,
és ha szükséges, akkor az eredmények alapján fejlődési célokat
fogalmaznak meg. A vizsgálatok eredményeit a gyerekek a portfóliójukban
ugyanúgy megőrzik, mint például egy tudáspróba eredményét.
