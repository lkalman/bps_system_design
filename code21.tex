\hypertarget{bps-code-egy-regi-iskolatipus-a-jovo-generaciojanak-ujragondolt-technikum}{%
\section{\texorpdfstring{\texttt{BPS\ Code}, egy régi iskolatípus a jövő
generációjának -- újragondolt
technikum}{BPS Code, egy régi iskolatípus a jövő generációjának -- újragondolt technikum}}\label{bps-code-egy-regi-iskolatipus-a-jovo-generaciojanak-ujragondolt-technikum}}

A \texttt{BPS\ Code} versenyképes, a mai kor szakmai igényeinek
megfelelő és a jövő kihívásaira válaszokat adó technikumi program. Az
iskolában a diákok örömmel tanulnak, egészségesen fejlődnek, és már
tanulmányaik során kapcsolódnak a társadalomhoz, hogy a közösség hasznos
tagjává válhassanak. Olyan iskola, amely egyszerre készít fel arra, hogy
az onnan kikerülő diákok szakemberként megállhassák a helyüket, és arra,
hogy önálló és felelős döntéseket hozhassanak felnőtt életük és pályájuk
megkezdésének szakaszában. Ehhez a következő kiemelt fejlesztési
területekre fókuszál az iskola:

\begin{itemize}
\item
  \textbf{Kreatív innováció.} A technikum első évétől megjelenik
  termékek és szolgáltatások tervezésének formájában. A tanulók valódi,
  aktuális problémákra válaszokat kereső projekteken dolgoznak, melyek
  kialakításában ők is aktívan szerepet vállalnak.
\item
  \textbf{Szaktudás}. A tanulók a programozás, a digitális kézművesség
  és alkotás egyes szakterületein magas képzettséget szereznek,
  kompetenciáik kiterjednek a területen használt nyelvezetre, a
  legkorszerűbb eszközök és módszerek alkalmazására. Ezzel a piacon
  komoly versenyelőnyük lesz.
\item
  \textbf{Agilis, flexibilis struktúra.} A tanulásszervezés modern
  eszközei támogatják a személyre szabott, az egyéni igényeket
  kiszolgálni és a lehetőségeket fejleszteni tudó tanulást. A diákok
  tanulmányaik során felkészülnek arra, hogy szaktudásuk a világ
  változásaihoz adaptálódjon.
\item
  \textbf{\href{https://en.wikipedia.org/wiki/T-shaped_skills}{T-alakú
  személyiségjegyeket}} fejlesztünk: A diákok képesek elmélyülten
  haladni a szakterületükön, miközben más szakterületen működő
  emberekkel is könnyedén kooperálnak. Így válnak képessé komplex
  problémák megoldására.
\end{itemize}

\hypertarget{alkotunk-es-kodolunk-iskolat-epitunk}{%
\subsection{Alkotunk és kódolunk -- iskolát
építünk}\label{alkotunk-es-kodolunk-iskolat-epitunk}}

A \texttt{BPS\ Code} techinum a BPS modell alapján működik. Ez egy
modern, nyílt iskolaplatform, amely többféle specifikus,
téma(szakma)orientált tanulási műhely (Budapest School terminológiában:
tanulóközösség) létrehozását támogatja.

A BPS modell szerint működik más intézmény is, általános iskola,
gimnázium és a \texttt{BPS\ Code} technikum is. A technikumba járó
gyerekek és a programozás iránt érdeklődő gimnazisták tanulási élménye
között lehet, hogy nincs is különbség, még ha különböző intézménybe is
járnak.

\begin{itemize}
\tightlist
\item
  Egy hatosztályos „erős'' gimnázium világversenyt nyerő robotika
  szakköre remekül tud együtt dolgozni egy technikumban duális képzésen
  programozást tanuló csoporttal az önvezető Golf autók átalakításán,
  hogy segítsék az idős emberek boltba jutását.
\item
  Egy digitális médiát technikumban tanuló gyerek együtt tanul emelt
  színtű biológiát az „elit'' gimnáziumba járókkal, ha éppen úgy dönt,
  hogy pszichológiát akar egyetemen tanulni.
\end{itemize}

\hypertarget{bps-alapelvek-szerint-mukodo-technikum}{%
\subsection{BPS alapelvek szerint működő
technikum}\label{bps-alapelvek-szerint-mukodo-technikum}}

A BPS modell a
tanulás hét pillére (\ref{a-tanulni-tanulas-het-pillere}.~fejezet, \pageref{a-tanulni-tanulas-het-pillere}.~oldal)
támaszkodik. Ezek a \texttt{BPS\ Code} esetében a következőket jelentik.

\begin{enumerate}
\def\labelenumi{\arabic{enumi}.}
\tightlist
\item
  Az iskola rugalmas és integratív, a tanulók fejlődéséhez igazodó
  személyreszabott tanulási környezetet nyújt.
\item
  A tanulás önvezérelt és aktív folyamat.
\item
  A tanulás az iskolában kezdődik és a világban folytatódik.
\item
  A tanulás mércéje épp annyira egymás segítése, mint az egyéni
  eredmények elérése.
\item
  A tanulás projektek mentén halad.
\item
  Egymásra figyelünk, együtt tanulunk közösség vagyunk.
\item
  A tanárok a tanulás szervezik, segítik, a tanulók partnerei.
\end{enumerate}

\hypertarget{uj-iskolamodell-tanulasi-tesztkornyezet}{%
\subsection{Új iskolamodell -- tanulási
tesztkörnyezet}\label{uj-iskolamodell-tanulasi-tesztkornyezet}}

Néhány strukturális részletben egyedi megoldásokat tesztel az iskola. A
ma már 340 gyerek tanulását segítő, 2015 óta fejlesztett Budapest School
Modell alapján működik a technikum. Ezek elemeit fejlesztjük tovább
most, és alkalmazzuk a Szakképzés 4.0 koncepció startégiája alapján.
Ennek a tanulási tesztkörnyezetnek a létrehozását az ITM támogatja, az
eredményeket folyamatosan monitorozza, ezzel lehetőséget biztosítva
arra, hogy a tapasztalatok akár 1-2 éven belül alkalmazhatóak legyenek
más iskolákban is.

Az iskola több helyszínen működik. A gyerekek az iskolában kisebb
közösségekbe tartoznak, a
tanulóközösségekbe (\ref{a-budapest-school-tanulokozossegei}.~fejezet, \pageref{a-budapest-school-tanulokozossegei}.~oldal).
Ezek a közösségek több évfolyamot átölelő, 6-60 fős tanulási
közösségekként működnek. A tanulóközösségek se jogi, se szervezeti
szempontból nem önálló iskolák. A tanulóközösség nem más, mint az iskola
egyik szervezeti egysége, ami önálló identitással, sajátos
szubkultúrával rendelkezhet.

A kisközösségükön belül a gyereket többfajta csoportbontásban tanulnak
és különböző közösségek tagjai is összeállhatnak egy-egy foglalkozásra,
kurzusra, de identitásban, ,,egymáshoz-tartozás érzésben'' a
tanulóközösség az elsődleges közösségük. A közösség tagjai együtt és
egymástól tanulnak.

Az egyes tanulóközösségeket
tanulásszervező tanárok (tanulásszervezők) (\ref{tanulasszervezo}.~fejezet, \pageref{tanulasszervezo}.~oldal)
csapata vezeti. Minden gyereknek van egy kitüntetett tanára, a
mentora (\ref{mentor}.~fejezet, \pageref{mentor}.~oldal), aki egyéni
figyelmével a fejlődésben segíti. Minden gyerek a mentortanára
segítségével és a szülők aktív részvételével trimeszterenként
meghatározza a
saját tanulási céljait (\ref{sajat-tanulasi-celok}.~fejezet, \pageref{sajat-tanulasi-celok}.~oldal).

A tanulásszervezők
modulokat (\ref{tanulasi-tanitasi-egysegek-a-modulok}.~fejezet, \pageref{tanulasi-tanitasi-egysegek-a-modulok}.~oldal)
hirdetnek ezen célokból, és a tantárgyak tanulási eredmények alapján. A
modulok reflektálnak a mai világ alapvető kérdéseire, integrálják a
tudományterületeket és művészeti ágakat, azaz a tantárgyakat, és egyenlő
lehetőséget adnak a tudásszerzésre, az önálló gondolkodásra és az
alkotásra a gyerekek mindennapjaiban.

A modulok végeztével a gyerekek eredményei bekerülnek saját
portfóliójukba (\ref{portfolio}.~fejezet, \pageref{portfolio}.~oldal),
melyek tartalmazhatnak önálló vagy csoportos alkotásokat, tudáspróbákat,
vizsgafeladatokat, egymás felé történő visszajelzéseket, a fejlődést jól
mérő dokumentációkat vagy bármit, amire a gyerek és tanárai büszkék vagy
amit fontosnak tartanak. Erre a portfólióra épül a Budapest School
visszajelző és értékelő rendszere (\ref{visszajelzes-ertekeles}.~fejezet, \pageref{visszajelzes-ertekeles}.~oldal).

A gyerekek mindennapjait meghatározó modulok több műveltségi területet,
többféle kompetenciát, több tantárgy anyagát is lefedhetik, és egy
tantárgy anyagát több modul is érintheti. Ezért is mondhatjuk, hogy a
BPS iskolákban a tantárgyközi tevékenységek vannak előtérben. Az iskola
szándéka, hogy a gyerekek folyamatosan fejlődjenek a világ tudományos
megismerésében (STEM), a saját és mások kulturális közegéhez való
kapcsolódásban (KULT), valamint a testi-lelki egyensúlyuk fenntartásában
(Harmónia), vagyis a
kiemelt tantárgyközi fejlesztési területekben (\ref{kiemelt-fejlesztesi-teruletek}.~fejezet, \pageref{kiemelt-fejlesztesi-teruletek}.~oldal).

Az iskola szerint az a tanárok döntése, hogy a gyerekek matematika vagy
digitális kultúra órán foglalkoznak algoritmusokkal, vagy algoritmusok
órán foglalkoznak matematikával. A iskola annyit határoz meg, hogy a
9--12. évfolyamszinten matematika tantárgyhoz kapcsolódóan 135 különböző
tanulási eredményt kell elérni, és algoritmusokkal kapcsolatban pedig 11
különböző tanulási eredményt több különböző tantárgyból.

Tehát a tantárgyak a tanulás tartalmi elemeinek forrása és keretei: a
tanulandó dolgok halmazaként működik. Az, hogy milyen csoportosításban
történik a tanulás, az a szaktanárokra van bízva. A gyerekek lehet, hogy
csak félévente, az elszámolás időszakában találkoznak a tantárgyak
taxonómiájával. Ebben az időszakban veti össze minden gyerek és mentor,
hogy amit tanultak, alkottak és amiben fejlődtek, az hogyan viszonyul a
társadalom és a törvények elvárásaival, a Nemzeti alaptantervvel.

Az iskola a NAT tantárgyak és a \emph{szoftverfejlesztés és -tesztelés
képzési kimeneti követelményeinek} témaköreit, tartalmát és
követelményeit \emph{tanulási eredmények} halmazaként adja meg. A
gyerekek feladata az iskolában, hogy tanulási eredményeket érjenek el és
így sajátítsák el a tantárgyak által szabott követelményeket. Tanulási
eredményeket modulok elvégzésével (is) lehet elérni, tehát a modulok
elsődleges feladata, hogy a tanulási eredményekhez vezető utat mutassák.

Az iskolában egyszerre jelennek meg a NAT tantárgyi elvárásai, a
közoktatást szabályozó törvények szándékai, szoftverfejlesztés és
-tesztelés képzési kimeneti követelményei, a gyerekek saját céljai és a
mai világra való integrált reflexió.

\hypertarget{a-tanulas-rendszerszemleletu-megkozelitese}{%
\subsubsection{A tanulás rendszerszemléletű
megközelítése}\label{a-tanulas-rendszerszemleletu-megkozelitese}}

Az oktatás tartalmának előzetes szabályozása helyett a BPS Modell --- és
így a \texttt{Code21} technikum is --- a tanulás módjára helyezi a
hangsúlyt. Az iskola alapelve, hogy integratív módon folyamatosan
keresse és fejlessze a pedagógiai, pszichológiai és szervezetfejlesztési
módszereket, amelyek korszerű módon tudják segíteni a tanulás tanulását,
az egyéni és csoportos fejlődést, a konfliktusok feloldását.

A tanulás tartalmát tekintve a \texttt{BPS\ Code} a NAT tartalmára és a
szoftverfejlesztés és -tesztelés képzési kimeneti követelményeire
támaszkodik. A Budapest School Modell pedig a tanulás rendszerét, annak
folyamatát szabályozza.

\hypertarget{elteresek-a-megszokott-iskolaktol}{%
\subsection{Eltérések a megszokott
iskoláktól}\label{elteresek-a-megszokott-iskolaktol}}

\hypertarget{mentor-tanulokozosseg-kozosseg}{%
\subsubsection{Mentor, tanulóközösség
közösség}\label{mentor-tanulokozosseg-kozosseg}}

Az iskola alapegysége a
\href{/tanulasi-elmeny/tanulo-kozosseg.md}{tanuló közösség, ami egy 6-60
fős kevert korosztályú közösség}. Ők folyamatosan az igények és a
feladatok mentén újjáalakuló csoportbontásokban tanulnak. Vagyis a
csoportok, a feladatok és projektek, valamint a tanulási célok, és nem
évfolyamok szerint szerveződnek. A közösséget
3-5 fős tanárcsapat (\ref{a-tanulokozossegeket-tanarok-vezetik}.~fejezet, \pageref{a-tanulokozossegeket-tanarok-vezetik}.~oldal)
vezeti, akik a specifikus feladatokhoz szaktanárokat hívnak be. Minden
gyereknek van
mentora (\ref{mentor}.~fejezet, \pageref{mentor}.~oldal), aki a
tanulási céljai meghatározásában, érzelmi és társas biztonságában,
valamint a tanulási útja követésében segíti mentoráltjait.

\begin{longtable}[]{@{}ll@{}}
\toprule
\begin{minipage}[b]{0.33\columnwidth}\raggedright
Iskola, amibe mi jártunk\strut
\end{minipage} & \begin{minipage}[b]{0.61\columnwidth}\raggedright
Iskola, ahol a gyerekeink tanulnak\strut
\end{minipage}\tabularnewline
\midrule
\endhead
\begin{minipage}[t]{0.33\columnwidth}\raggedright
Osztályfőnök\strut
\end{minipage} & \begin{minipage}[t]{0.61\columnwidth}\raggedright
Mentortanár\strut
\end{minipage}\tabularnewline
\begin{minipage}[t]{0.33\columnwidth}\raggedright
Életkor tekintetében homogén, korcsoport alapján szervezett
osztályok.\strut
\end{minipage} & \begin{minipage}[t]{0.61\columnwidth}\raggedright
Életkor tekintetében heterogén, a tanulás célja és az alapító okiratban
lefektetett kritériumok alapján szerveződött csoportok.\strut
\end{minipage}\tabularnewline
\begin{minipage}[t]{0.33\columnwidth}\raggedright
A tanulók osztályok szerint általában együtt tanulnak.\strut
\end{minipage} & \begin{minipage}[t]{0.61\columnwidth}\raggedright
A tanulók differenciáltan szervezett csoportbontásban tanulnak.\strut
\end{minipage}\tabularnewline
\begin{minipage}[t]{0.33\columnwidth}\raggedright
Az ismeretek tantárgyak szerint szegmentáltak.\strut
\end{minipage} & \begin{minipage}[t]{0.61\columnwidth}\raggedright
Az ismeretek komplex projektek köré szervezett, interdiszciplináris
modulok.\strut
\end{minipage}\tabularnewline
\bottomrule
\end{longtable}

\hypertarget{modularis-tanulas}{%
\subsubsection{Moduláris tanulás}\label{modularis-tanulas}}

A gyerekek tanulnak, gyakorolnak, projekteken dolgoznak, a NAT tanulási
eredményei, illetve a saját maguk, vagy a tanáraik által meghatározott
tanulási eredmények elérése érdekében. Tanulásuk -- a tantárgyak
tanulási eredményeit is figyelembe vevő -- tanulási modulok során
történik. Ezek egyes esetekben a tantárgyak alkotóelemeire vagy
összevont, tantárgyközi tartalmakra (\ref{tanulasi-tanitasi-egysegek-a-modulok}.~fejezet, \pageref{tanulasi-tanitasi-egysegek-a-modulok}.~oldal)
épülnek. Például: angolul Javascriptet tanulni egyszerre angol nyelvi és
programozásóra is.

A tantárgyfelosztást és a rögzített óraszámokat felváltja az
interdiszciplináris tanulási modulok, alakítható tanulási utak,
negyedévenkénti -- NAT által meghatározott tanulási eredmények alapján
történő -- tudásmérés.

\begin{longtable}[]{@{}ll@{}}
\toprule
\begin{minipage}[b]{0.47\columnwidth}\raggedright
Iskola, amibe mi jártunk\strut
\end{minipage} & \begin{minipage}[b]{0.47\columnwidth}\raggedright
Iskola, ahol a gyerekeink tanulnak\strut
\end{minipage}\tabularnewline
\midrule
\endhead
\begin{minipage}[t]{0.47\columnwidth}\raggedright
Egész tanévre szóló tantárgyfelosztás és órarend .\strut
\end{minipage} & \begin{minipage}[t]{0.47\columnwidth}\raggedright
Negyedévre szóló órarend, amiben interdiszciplináris modulok
vannak.\strut
\end{minipage}\tabularnewline
\begin{minipage}[t]{0.47\columnwidth}\raggedright
A tanítási-tanulási folyamatszervezés szaktanár-szaktantárgy
fókuszú.\strut
\end{minipage} & \begin{minipage}[t]{0.47\columnwidth}\raggedright
A tanítási-tanulási folyamatszervezés tanár-diákközösség fókuszú.\strut
\end{minipage}\tabularnewline
\bottomrule
\end{longtable}

\hypertarget{onallo-tanulas}{%
\subsubsection{Önálló tanulás}\label{onallo-tanulas}}

Az iskola külön értéknek tekinti, ha a gyerek
önállóan (\ref{az-onallo-tanulas}.~fejezet, \pageref{az-onallo-tanulas}.~oldal),
hiteles források felhasználásával, például videókból tanulja meg a
természettudományos alapismereteket. A tanári szerep ekkor nagyobbrészt
a tanulás facilitálására, az új lehetőségek megmutatására, a folyamatos
kihívásban tartásra irányul, és kevésbé a tudásátadásra.

\begin{longtable}[]{@{}ll@{}}
\toprule
\begin{minipage}[b]{0.33\columnwidth}\raggedright
Iskola, amibe mi jártunk\strut
\end{minipage} & \begin{minipage}[b]{0.61\columnwidth}\raggedright
Iskola, ahol a gyerekeink tanulnak\strut
\end{minipage}\tabularnewline
\midrule
\endhead
\begin{minipage}[t]{0.33\columnwidth}\raggedright
Egységes órarend, kínálatvezérelt, választható szakkörök,
fakultációk.\strut
\end{minipage} & \begin{minipage}[t]{0.61\columnwidth}\raggedright
Igény- és szükségletközpontú egyéni órarend a modulokból - megadott
kritériumok szerint összeállítva.\strut
\end{minipage}\tabularnewline
\begin{minipage}[t]{0.33\columnwidth}\raggedright
A haladás tempója egységes.\strut
\end{minipage} & \begin{minipage}[t]{0.61\columnwidth}\raggedright
Az egyéni haladási utak képesség- és érdeklődés függvényében jelentősen
eltérhetnek egymástól.\strut
\end{minipage}\tabularnewline
\begin{minipage}[t]{0.33\columnwidth}\raggedright
A tananyagban való haladást osztálynaplóban rögzíti a tanár.\strut
\end{minipage} & \begin{minipage}[t]{0.61\columnwidth}\raggedright
A tananyagban való egyéni haladást a tanulók és tanárok a
portfóliójukban rögzítik.\strut
\end{minipage}\tabularnewline
\begin{minipage}[t]{0.33\columnwidth}\raggedright
A minősítést az értékelések alapján adott érdemjegy jelenti.\strut
\end{minipage} & \begin{minipage}[t]{0.61\columnwidth}\raggedright
A minősítés alapját a pedagógiai és tematikus szakaszonként adott
szöveges értékelések képezik, amelyek érdemjegyre válthatók.\strut
\end{minipage}\tabularnewline
\begin{minipage}[t]{0.33\columnwidth}\raggedright
A tanuló továbbhaladásának feltétele az elégséges érdemjegy.\strut
\end{minipage} & \begin{minipage}[t]{0.61\columnwidth}\raggedright
A tanuló továbbhaladásának feltétele a tantárgyi követelmények legalább
40\%-os teljesítettsége.\strut
\end{minipage}\tabularnewline
\begin{minipage}[t]{0.33\columnwidth}\raggedright
Formális tanulási alkalomnak a tanár által vezetett foglakozások
tekinthetők.\strut
\end{minipage} & \begin{minipage}[t]{0.61\columnwidth}\raggedright
Formális tanulásnak tekinthető minden tanár által vezetett vagy csak
kontrollált, önálló, kis csoportos, a tanulók által vezérelt tanulás
is.\strut
\end{minipage}\tabularnewline
\bottomrule
\end{longtable}

\hypertarget{szakismeretek-mar-a-technikum-elso-evetol-rugalmas-tanulasi-struktura}{%
\subsubsection{Szakismeretek már a technikum első évétől -- rugalmas
tanulási
struktúra}\label{szakismeretek-mar-a-technikum-elso-evetol-rugalmas-tanulasi-struktura}}

Ha a diákokat egy téma nagyon érdekli, akkor az iskola feladata kísérni
és támogatni őket a tanulásban. Az alapozó + szakmai 2+3 év helyett,
negyedévenkénti újratervezés van. Már az iskola elején megjelenik az
elmélyült szakmai tanulás. A tanárok a diákokkal együtt alakítják a
struktúrát, annak megfelelően ami a leginkább segíti a tanulók
folyamatos fejlődését. Ha a tanulást az 1 hét projektidőszak 1 hét
„rendes iskolai'' hét beosztás segíti az adott csoportban, akkor azt
követik. Ha ugyanez a csoport negyedévvel később a délelőtt
programozunk, délután haladunk a NAT követelményekkel beosztást tartja
jobbnak, akkor változtatnak a tanulás struktúráján.

\hypertarget{pedagogusdiploma}{%
\subsubsection{Pedagógusdiploma}\label{pedagogusdiploma}}

Az iskola legfontosabb segítő szereplője a mentortanár, aki személyes
odafigyeléssel segíti mentoráltjait, és akikkel együtt szervezik a
tanulási környezetet. Tőlük az egyetlen előzetes elvárás, a tinédzserek
támogatásában szerzett 3 éves tapasztalat. (\ref{pedagogusokra-vonatkozo-eloirasok}.~fejezet, \pageref{pedagogusokra-vonatkozo-eloirasok}.~oldal)
Minden további tanárral kapcsolatos legfontosabb elvárás, hogy az adott
területeken, melyekhez kapcsolódóan modulokat hirdet meg, megfelelően
tudja támogatni a tanulókat abban, hogy a kötelező tanulási eredményeket
és a közösen kialakítottakat elérhessék.

\begin{longtable}[]{@{}ll@{}}
\toprule
\begin{minipage}[b]{0.15\columnwidth}\raggedright
Iskola, amibe mi jártunk\strut
\end{minipage} & \begin{minipage}[b]{0.79\columnwidth}\raggedright
Iskola, ahol a gyerekeink tanulnak\strut
\end{minipage}\tabularnewline
\midrule
\endhead
\begin{minipage}[t]{0.15\columnwidth}\raggedright
Csak pedagógus diplomával lehet tanítani.\strut
\end{minipage} & \begin{minipage}[t]{0.79\columnwidth}\raggedright
Az alapító okiratban meghatározott kompetenciákkal lehet tanítani.
Bizonyos területeken elvárt a pedagógusdiploma, ám van, ahol elsősorban
a szaktudás, illetve a feladatban szerzett képzettség és tapasztalat az
elvárt.\strut
\end{minipage}\tabularnewline
\bottomrule
\end{longtable}

\hypertarget{vezetesi-modell}{%
\subsubsection{Vezetési modell}\label{vezetesi-modell}}

A klasszikus igazgató-típusú vezetői modell kevésbé reflektál a XXI.
századi agilis szakképzés koncepciójára. Az igazgatótól a hatályos
jogszabályok nem várják el, hogy vezetői tapasztalata legyen, csupán a
közoktatás-vezetői szakvizsga meglétét. Pedagógiai tapasztalattal
rendelkező, de elsősorban menedzsertípusú, tapasztalt vezető működése
alatt biztosítható a \texttt{Code21} Technikum kitűzött céljainak
elérése.

Az egyes -- jövőbeni -- telephelyekre tekintettel a
tanulóközösségi-modellre,
szükségtelen (\ref{intezmenyvezeto-es-helyettes}.~fejezet, \pageref{intezmenyvezeto-es-helyettes}.~oldal)
intézményvezető-helyettes, tagintézmény-vezető kijelölése.

\begin{longtable}[]{@{}ll@{}}
\toprule
\begin{minipage}[b]{0.43\columnwidth}\raggedright
Iskola, amibe mi jártunk\strut
\end{minipage} & \begin{minipage}[b]{0.51\columnwidth}\raggedright
Iskola, ahol a gyerekeink tanulnak\strut
\end{minipage}\tabularnewline
\midrule
\endhead
\begin{minipage}[t]{0.43\columnwidth}\raggedright
Az igazgatói poszt betöltésének feltétele a közoktatás-vezetői
szakvizsga.\strut
\end{minipage} & \begin{minipage}[t]{0.51\columnwidth}\raggedright
Az igazgatói poszt betöltésének feltétele a pedagógiai, oktatási
területen szerzett legalább 5 éves és legalább 3 éves vezetői
tapasztalat.\strut
\end{minipage}\tabularnewline
\begin{minipage}[t]{0.43\columnwidth}\raggedright
Az intézményvezetői-döntéshozatali folyamatok az igazgató, illetve
helyettesei feladat- és felelősségkörében vannak.\strut
\end{minipage} & \begin{minipage}[t]{0.51\columnwidth}\raggedright
Az intézményvezetői-döntéshozatali folyamatok az igazgató, illetve a
megbízott pedagógusok feladat- és felelősségkörében vannak.\strut
\end{minipage}\tabularnewline
\bottomrule
\end{longtable}

\hypertarget{epuletek}{%
\subsubsection{Épületek}\label{epuletek}}

Kisebb irodaépületben, átalakított üzlethelyiségben, esetleg volt ipari
épületben is \href{/jogszabalyok/epuletek.md}{elképzelhető} a tanulás. A
tanulás terei a mai kor igényeihez mérten rugalmasan alakíthatóak, és
nem egy helyben történnek. Az iskola inkább bázis, ahonnan a tanulás
szerveződik, ennek megfelelően nem feltétlenül kell rendelkezzen minden
eszközzel a tanuláshoz. Ezek külső helyszíneken is elérhetőek lehetnek.

\begin{longtable}[]{@{}ll@{}}
\toprule
\begin{minipage}[b]{0.33\columnwidth}\raggedright
Iskola, amibe mi jártunk\strut
\end{minipage} & \begin{minipage}[b]{0.61\columnwidth}\raggedright
Iskola, ahol a gyerekeink tanulnak\strut
\end{minipage}\tabularnewline
\midrule
\endhead
\begin{minipage}[t]{0.33\columnwidth}\raggedright
A nagyméretű intézményekre optimalizált oktatási, nevelési funkciójú
épületre előírt szabvány szerinti épületkialakítás.\strut
\end{minipage} & \begin{minipage}[t]{0.61\columnwidth}\raggedright
A --- közegészségügyi és tűzvédelmi szempontokból biztonságos --- kisebb
ingatlanra vagy ingatlanegyüttesre optimalizált, rugalmas, tanulást
támogató terekkel rendelkező épület vagy épületegyüttes, ami eltérhet a
szabványtól.\strut
\end{minipage}\tabularnewline
\bottomrule
\end{longtable}
