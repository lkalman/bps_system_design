\hypertarget{nemzetisegek-megismerese}{%
\section{Nemzetiségek megismerése}\label{nemzetisegek-megismerese}}

Az iskola környezetében élő nemzetiségek kultúrájának megismerését\break
fontosnak tartja az iskola. Ezért minden második tanévben a településen
működő nemzetiségi önkormányzatokkal felveszi a kapcsolatot, és velük
együttműködve legalább 4 órás modulokat alakít ki. A modulok célja, hogy
a modul résztvevői megismerjék a nemzetiségekről azt, amit az
önkormányzatok fontosnak tartanak megmutatni.

Minden második évben legalább három önkormányzattal három különböző
modult kínál fel az iskola választásra. Amelyik településen ennél több
nemzetiségi önkormányzat működik, ott a tanév megkezdése előtt,
augusztus folyamán véletlenszerűen sorsoljuk ki az adott évi 3
megismerendő nemzetiséget.
