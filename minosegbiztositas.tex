\hypertarget{minosegfejlesztes-folyamatszabalyozas}{%
\section{Minőségfejlesztés,
folyamatszabályozás}\label{minosegfejlesztes-folyamatszabalyozas}}

Honnan tudjuk, hogy jól működik az iskola? Minden gyerek azt tanulja,
amit szeretne, és amire neki a leginkább szüksége van? Megkérdezzük az
érintetteket, figyeljük az adatokat, és azok alapján fejlesztünk.

A Budapest School-iskolában a gyerekek tanulását egy komplex rendszer
egyik elemének tekintjük {\autocite{Barabasi2007}}. Iskoláink a
világ felé nyitott hálózatot alkotnak: az egy közösségben lévő gyerekek,
a családok, a velük foglalkozó tanárok, a helyi környezet, a Budapest
School további közösségei, az ország és a nemzet állapota, valamint a
globális társadalmi folyamatok is befolyásolják, hogy mi történik egy
iskolában.

Hogy egy gyerek épp mit és mennyit tanul, az nemcsak az iskola programjától
vagy a NAT-tól, hanem számos más tényezőtől is függ, melyek
befolyásolják a fejlődést és a közösség egymás fejlődésére gyakorolt
hatását: többek közt függ a gyerekek múltjától, aktuális hangulatától és
vágyaitól, a tanárok személyiségétől, a családtól, a csoportdinamikától
és a társadalomban történő változásoktól is.

Ezért a gyermekeink fejlődését, tanulását és boldogságának alakulását
egy \emph{komplex rendszer} működésével modellezhetjük.

A tanulási folyamataink minőségfejlesztésénél a következő szempontokat
vesszük figyelembe:

\begin{enumerate}
\def\labelenumi{\arabic{enumi}.}
\item
  A történéseket, eseményeket, (rész)eredményeket folyamatosan kell
  monitorozni.
\item
  A visszajelzéseket a rendszer minden tagjától folyamatosan gyűjteni
  kell: a gyerekektől, tanároktól, szülőktől és az adminisztrátoroktól.
\item
  Anomália esetén a helyzetfelismerés, az eltérések okának felkutatása a
  cél.
\item
  A feltárt hibák alapján a rendszert folyamatosan kell javítani.
\end{enumerate}

A minőségfejlesztés célja az iskola mint tanulórendszer folyamatos
fejlesztése. A monitorozás folyamatos, így az iskola hamar felismeri az
anomáliákat, és kivizsgálás után, ha szükséges, meg tudja azokat
szüntetni, tanulni is tud belőlük, és javítja a rendszert.

A fenntartó folyamatosan monitorozza a tanulóközösségeket és
visszajelzéseket ad, aminek alapján a tanulóközösség javítja a saját
működési folyamatait. A BPS modellben leírt működést a fenntartó mérhető
és megfigyelhető indikátorokra fordítja le, és kidolgozza, üzemelteti a
metrikák 2-3 hónapnyi rendszerességű nyomon követésére alkalmas
rendszerét.

A fenntartónak meg kell figyelnie legalább a következő metrikákat:

\begin{enumerate}
\def\labelenumi{\arabic{enumi}.}
\tightlist
\item
  A szülő, a gyerek és a tanár közötti saját célokat megfogalmazó hármas
  megállapodások időben megszülettek, nincs olyan gyerek, akinek nincs
  elfogadott saját tanulási célja. Indikátor: elkészült szerződések
  száma.
\item
  A modulok végén a portfóliók bővülnek, és azok tartalma a
  tantárgyakhoz kapcsolódik. Indikátor: a portfólió elemeinek száma és
  kapcsolhatósága.
\item
  A
  tanulási eredményekre (\ref{tantargyak}.~fejezet, \pageref{tantargyak}.~oldal)
  vonatkozó megkötések időben teljesülnek. Indikátor: elért tanulási
  eredmények száma gyerekekre bontva.
\item
  A szülők biztonságban érzik gyereküket, és eleget tudnak
  arról,\break
  hogy
  mit tanulnak. Indikátor: kérdőíves vizsgálat alapján.
\item
  A tanárok hatékonynak tartják a munkájukat. Indikátor: kérdőíves
  felmérés alapján.
\item
  A gyerekek úgy érzik, folyamatosan tanulnak, támogatva vannak, vannak
  kihívásaik. Indikátor: kérdőíves felmérés alapján.
\end{enumerate}
