\hypertarget{tantargyak-tartalma-a-tanulasi-eredmenyek}{%
\section{Tantárgyak tartalma, a tanulási
eredmények}\label{tantargyak-tartalma-a-tanulasi-eredmenyek}}

\hypertarget{allampolgari-ismeretek}{%
\subsection{Állampolgári ismeretek}\label{allampolgari-ismeretek}}

\hypertarget{evfolyamon}{%
\subsubsection{12. évfolyamon}\label{evfolyamon}}

\begin{itemize}
\item
  Megérti a család szerepét, alapvető feladatait az egyén és a nemzet
  szempontjából egyaránt.
\item
  Értékeli a nemzeti identitás jelentőségét az egyén és a közösség
  szempontjából is.
\item
  Ismeri a választások alapelveit és a törvényhozás folyamatát.
\item
  Megismeri a demokratikus jogállam működésének alapvető sajátosságait.
\item
  Érti és vallja a haza védelmének, a nemzetért történő tenni akarás
  fontosságát.
\item
  A mindennapi életének megszervezésében alkalmazza a jogi és
  gazdasági-pénzügyi ismereteit.
\item
  Saját pénzügyeiben tudatos döntéseket hoz.
\item
  Felismeri az életpálya-tervezés és a munkavállalás egyéni és
  társadalmi jelentőségét.
\item
  Ismeri a munka világát érintő alapvető jogi szabályozást, a
  munkaerőpiac jellemzőit, tájékozódik a foglalkoztatás és a
  szakmaszerkezet változásairól.
\item
  Értelmezi a család mint a társadalom alapvető intézményének szerepét
  és jellemzőit.
\item
  Társaival megbeszéli a párválasztás, a családtervezés fontos
  szakaszait, szempontjait és a gyermekvállalás demográfiai
  jelentőségét: tájékozódás, minták, orientáló példák, átgondolt
  tervezés, felelősség.
\item
  Felismeri, hogy a családtagok milyen szerepet töltenek be a
  szocializáció folyamatában.
\item
  Értelmezi a családi szocializációnak az ember életútját befolyásoló
  jelentőségét.
\item
  Felismeri az alapvető emberi jogok egyetemes és társadalmi
  jelentőségét.
\item
  Bemutatja magyarország alaptörvényének legfontosabb részeit:
  alapvetés; az állam; szabadság és felelősség.
\item
  Érti a társadalmi normák és az egyéni cselekedetek, akaratok, célok
  egyeztetésének, összehangolásának követelményét. elméleti és
  tapasztalati úton ismereteket szerez a társadalmi
  felelősségvállalásról, a segítségre szorulók támogatásának
  lehetőségeiről.
\item
  Megérti a honvédelem szerepét az ország biztonságának fenntartásában,
  megismeri a haza védelmének legfontosabb feladatcsoportjait és
  területeit, az egyén kötelezettségeit.
\item
  Felismeri és értelmezi az igazságosság, az esélyegyenlőség
  biztosításának jelentőségét és követelményeit.
\item
  Értelmezi a választójog feltételeit és a választások alapelveit.
\item
  Értelmezi a törvényalkotás folyamatát.
\item
  Megérti a nemzeti érzület sajátosságait és a hazafiság fontosságát,
  lehetséges megnyilvánulási formáit.
\item
  Véleményt alkot a nemzetállamok és a globalizáció összefüggéseiről.
\item
  Felismeri a világ magyarsága mint nemzeti közösség összetartozásának
  jelentőségét.
\item
  Érti és felismeri a honvédelem mint nemzeti ügy jelentőségét.
\item
  Felismeri és értékeli a helyi, regionális és országos közgyűjtemények
  nemzeti kulturális örökség megőrzésében betöltött szerepét.
\item
  Azonosítja a mindennapi ügyintézés alapintézményeit.
\item
  Életkori sajátosságainak megfelelően jártasságot szerez a jog
  területének mindennapi életben való alkalmazásában.
\item
  Tájékozott a munkavállalással kapcsolatos szabályokban.
\item
  Megtervezi egy fiktív család költségvetését.
\item
  Saját pénzügyi döntéseit körültekintően, megalapozottan hozza meg.
\item
  Megismeri a megalapozott, körültekintő hitelfelvétel szempontjait,
  illetve feltételeit.
\item
  Azonosítja az állam gazdasági szerepvállalásának elemeit.
\item
  Felismeri és megérti a közteherviselés nemzetgazdasági, társadalmi és
  morális jelentőségét.
\item
  Életvitelébe beépülnek a tudatos fogyasztás elemei, életmódjában
  figyelmet fordít a környezeti terhelés csökkentésére, érvényesíti a
  fogyasztóvédelmi szempontokat.
\item
  Értelmezi a vállalkozás indítását befolyásoló tényezőket.
\item
  Felismeri a véleménynyilvánítás, érvelés, a párbeszéd és a vita
  társadalmi hasznosságát.
\item
  Képes arra, hogy feladatait akár önálló, akár társas tanulás révén
  végezze el, célorientáltan képes az együttműködésre.
\item
  Önállóan vagy társaival együttműködve javaslatokat fogalmaz meg.
\item
  Tiszteletben tartja a másik ember értékvilágát, gondolatait és
  véleményét, ha szükséges, kritikusan viszonyul emberi cselekedetekhez,
  magatartásformákhoz.
\item
  Megismerkedik a tudatos médiafogyasztói magatartással és a közösségi
  média használatával.
\item
  A tanulási tevékenységek szakaszaiban használja az infokommunikációs
  eszközöket, lehetőségeket, tisztában van azok szerepével, innovációs
  potenciáljával és veszélyeivel is.
\end{itemize}

\hypertarget{digitalis-kultura}{%
\subsection{Digitális kultúra}\label{digitalis-kultura}}

\hypertarget{evfolyamon-1}{%
\subsubsection{9. évfolyamon}\label{evfolyamon-1}}

\begin{itemize}
\item
  Elmélyülten dolgozik digitális környezetben, önellenőrzést végez.
\item
  Megvizsgálja és értékeli az általa vagy társai által alkalmazott,
  létrehozott, megvalósított eljárásokat.
\item
  Társaival együttműködve online és offline környezetben egyaránt megold
  különböző feladatokat, ötleteit, véleményét megfogalmazza, részt vesz
  a közös álláspont kialakításában.
\item
  Kiválasztja az általa ismert informatikai eszközök és alkalmazások
  közül azokat, melyek az adott probléma megoldásához szükségesek.
\item
  Eredményétől függően módosítja a problémamegoldás folyamatában az
  adott, egyszerű tevékenységsorokat.
\item
  A rendelkezésére álló eszközökkel, forrásokból meggyőződik a talált
  vagy kapott információk helyességéről.
\item
  Közvetlen otthoni vagy iskolai környezetéből megnevez néhány
  informatikai eszközt, felsorolja fontosabb jellemzőit.
\item
  Megfogalmazza, néhány példával alátámasztja, hogyan könnyíti meg a
  felhasználó munkáját az adott eszköz alkalmazása.
\item
  Egyszerű feladatokat old meg informatikai eszközökkel. esetenként
  tanítói segítséggel összetett funkciókat is alkalmaz.
\item
  Önállóan vagy tanítói segítséggel választ más tantárgyak tanulásának
  támogatásához applikációkat, digitális tananyagot, oktatójátékot,
  képességfejlesztő digitális alkalmazást.
\item
  Kezdetben tanítói segítséggel, majd önállóan használ néhány,
  életkorának megfelelő alkalmazást, elsősorban információgyűjtés,
  gyakorlás, egyéni érdeklődésének kielégítése céljából.
\item
  A feladathoz, problémához digitális eszközt, illetve alkalmazást,
  applikációt, felhasználói felületet választ; felsorol néhány érvet
  választásával kapcsolatosan.
\item
  Adott szempontok alapján megfigyel néhány, grafikai alkalmazással
  készített produktumot; személyes véleményét megfogalmazza.
\item
  Grafikai alkalmazással egyszerű, közvetlenül hasznosuló rajzot,
  grafikát, dokumentumot hoz létre.
\item
  Egy rajzos dokumentumot adott szempontok alapján értékel, módosít.
\item
  Állításokat fogalmaz meg grafikonokról, infografikákról,
  táblázatokról; a kapott információkat felhasználja napi tevékenysége
  során.
\item
  Információkat keres, a talált adatokat felhasználja digitális
  produktumok létrehozására.
\item
  Értelmezi a problémát, a megoldási lehetőségeket eljátssza,
  megfogalmazza, egyszerű eszközök segítségével megvalósítja.
\item
  Információt keres az interneten más tantárgyak tanulása során, és
  felhasználja azt.
\item
  Egyszerű prezentációt, ábrát, egyéb segédletet készít.
\item
  Felismer, eljátszik, végrehajt néhány hétköznapi tevékenysége során
  tapasztalt, elemi lépésekből álló, adott sorrendben végrehajtandó
  cselekvést.
\item
  Egy adott, mindennapi életből vett algoritmust elemi lépésekre bont,
  értelmezi a lépések sorrendjét, megfogalmazza az algoritmus várható
  kimenetelét.
\item
  Feladat, probléma megoldásához többféle algoritmust próbál ki.
\item
  A valódi vagy szimulált programozható eszköz mozgását értékeli, hiba
  esetén módosítja a kódsorozatot a kívánt eredmény eléréséig.
  tapasztalatait megfogalmazza, megvitatja társaival.
\item
  Adott feltételeknek megfelelő kódsorozatot tervez és hajtat végre,
  történeteket, meserészleteket jelenít meg padlórobottal vagy más
  eszközzel.
\item
  Alkalmaz néhány megadott algoritmust tevékenység, játék során, és
  néhány egyszerű esetben módosítja azokat.
\item
  Információkat keres az interneten, egyszerű eljárásokkal meggyőződik
  néhány, az interneten talált információ igazságértékéről.
\item
  Kiválasztja a számára releváns információt, felismeri a hamis
  információt.
\item
  Tisztában van a személyes adat fogalmával, törekszik megőrzésére,
  ismer néhány példát az e-világ veszélyeivel kapcsolatban.
\item
  Ismeri és használja a kapcsolattartás formáit és a kommunikáció
  lehetőségeit a digitális környezetben.
\item
  Ismeri a mobileszközök alkalmazásának előnyeit, korlátait, etikai
  vonatkozásait.
\item
  Közvetlen tapasztalatokat szerez a digitális eszközök használatával
  kapcsolatban.
\item
  Képes feladat, probléma megoldásához megfelelő applikáció, digitális
  tananyag, oktatójáték, képességfejlesztő digitális alkalmazás
  kiválasztására.
\item
  Ismer néhány, kisiskolások részére készített portált,
  információforrást, digitálistananyag-lelőhelyet.
\item
  Önállóan használja a digitális eszközöket, az online kommunikáció
  eszközeit, tisztában van az ezzel járó veszélyekkel.
\item
  Elsajátítja a digitális írástudás eszközeit, azokkal feladatokat old
  meg.
\item
  Megismeri a felmerülő problémák megoldásának módjait, beleértve az
  adott feladat megoldásához szükséges algoritmus értelmezését,
  alkotását és számítógépes program készítését és kódolását a
  blokkprogramozás eszközeivel.
\item
  Digitális tudáselemek felhasználásával, társaival együttműködve
  különböző problémákat old meg.
\item
  Megismeri a digitális társadalom elvárásait, lehetőségeit és
  veszélyeit.
\item
  Célszerűen választ a feladat megoldásához használható informatikai
  eszközök közül.
\item
  Az informatikai eszközöket önállóan használja, a tipikus felhasználói
  hibákat elkerüli, és elhárítja az egyszerűbb felhasználói szintű
  hibákat.
\item
  Értelmezi az informatikai eszközöket működtető szoftverek
  hibajelzéseit, és azokról beszámol.
\item
  Önállóan használja az operációs rendszer felhasználói felületét.
\item
  Önállóan kezeli az operációs rendszer mappáit, fájljait és a
  felhőszolgáltatásokat.
\item
  Használja a digitális hálózatok alapszolgáltatásait.
\item
  Tapasztalatokkal rendelkezik a digitális jelek minőségével,
  kódolásával, tömörítésével, továbbításával kapcsolatos problémák
  kezeléséről.
\item
  Egy adott feladat kapcsán önállóan hoz létre szöveges vagy multimédiás
  dokumentumokat.
\item
  Ismeri és tudatosan alkalmazza a szöveges és multimédiás dokumentum
  készítése során a szöveg formázására, tipográfiájára vonatkozó
  alapelveket.
\item
  A tartalomnak megfelelően alakítja ki a szöveges vagy a multimédiás
  dokumentum szerkezetét, illeszti be, helyezi el és formázza meg a
  szükséges objektumokat.
\item
  Ismeri és kritikusan használja a nyelvi eszközöket (például
  helyesírás-ellenőrzés, elválasztás).
\item
  A szöveges dokumentumokat többféle elrendezésben jeleníti meg papíron,
  tisztában van a nyomtatás környezetre gyakorolt hatásaival.
\item
  Ismeri a prezentációkészítés alapszabályait, és azokat alkalmazza.
\item
  Etikus módon használja fel az információforrásokat, tisztában van a
  hivatkozás szabályaival.
\item
  Digitális eszközökkel önállóan rögzít és tárol képet, hangot és
  videót.
\item
  Digitális képeken képkorrekciót hajt végre.
\item
  Ismeri egy bittérképes rajzolóprogram használatát, azzal ábrát készít.
\item
  Bemutató-készítő vagy szövegszerkesztő programban rajzeszközökkel
  ábrát készít.
\item
  Érti, hogyan történik az egyszerű algoritmusok végrehajtása a
  digitális eszközökön.
\item
  Megkülönbözteti, kezeli és használja az elemi adatokat.
\item
  Értelmezi az algoritmus végrehajtásához szükséges adatok és az
  eredmények kapcsolatát.
\item
  Egyszerű algoritmusokat elemez és készít.
\item
  Ismeri a kódolás eszközeit.
\item
  Adatokat kezel a programozás eszközeivel.
\item
  Ismeri és használja a programozási környezet alapvető eszközeit.
\item
  Ismeri és használja a blokkprogramozás alapvető építőelemeit.
\item
  A probléma megoldásához vezérlési szerkezetet (szekvencia, elágazás és
  ciklus) alkalmaz a tanult blokkprogramozási nyelven.
\item
  Az adatokat táblázatos formába rendezi és formázza.
\item
  Cellahivatkozásokat, matematikai tudásának megfelelő képleteket,
  egyszerű statisztikai függvényeket használ táblázatkezelő programban.
\item
  Az adatok szemléltetéséhez diagramot készít.
\item
  Problémákat old meg táblázatkezelő program segítségével.
\item
  Tapasztalatokkal rendelkezik hétköznapi jelenségek számítógépes
  szimulációjáról.
\item
  Vizsgálni tudja a szabályozó eszközök hatásait a tantárgyi
  alkalmazásokban.
\item
  Ismeri az információkeresés technikáját, stratégiáját és több keresési
  szempont egyidejű érvényesítésének lehetőségét.
\item
  Önállóan keres információt, a találatokat hatékonyan szűri.
\item
  Az internetes adatbázis-kezelő rendszerek keresési űrlapját helyesen
  tölti ki.
\item
  Ismeri, használja az elektronikus kommunikáció lehetőségeit, a családi
  és az iskolai környezetének elektronikus szolgáltatásait.
\item
  Ismeri és betartja az elektronikus kommunikációs szabályokat.
\item
  Mozgásokat vezérel szimulált vagy valós környezetben.
\item
  Adatokat gyűjt szenzorok segítségével.
\item
  Tapasztalatokkal rendelkezik az eseményvezérlésről.
\item
  Ismeri a térinformatika és a 3d megjelenítés lehetőségeit.
\item
  Tapasztalatokkal rendelkezik az iskolai oktatáshoz kapcsolódó
  mobileszközökre fejlesztett alkalmazások használatában.
\item
  Tisztában van a hálózatokat és a személyes információkat érintő
  fenyegetésekkel, alkalmazza az adatok védelmét biztosító
  lehetőségeket.
\item
  Védekezik az internetes zaklatás különböző formái ellen, szükség
  esetén segítséget kér.
\item
  Ismeri a digitális környezet, az e-világ etikai problémáit.
\item
  Ismeri az információs technológia fejlődésének gazdasági, környezeti,
  kulturális hatásait.
\item
  Ismeri az információs társadalom múltját, jelenét és várható jövőjét.
\item
  Online gyakorolja az állampolgári jogokat és kötelességeket.
\item
  Ismeri az informatikai eszközök és a működtető szoftvereik célszerű
  választásának alapelveit, használja a digitális hálózatok
  alapszolgáltatásait, az online kommunikáció eszközeit, tisztában van
  az ezzel járó veszélyekkel, ezzel összefüggésben ismeri a
  segítségnyújtási, segítségkérési lehetőségeket.
\item
  Gyakorlatot szerez dokumentumok létrehozását segítő eszközök
  használatában.
\item
  Megismeri az adatkezelés alapfogalmait, képes a nagyobb adatmennyiség
  tárolását, hatékony feldolgozását biztosító eszközök és módszerek
  alapszintű használatára, érti a működésüket.
\item
  Megismeri az algoritmikus probléma megoldásához szükséges módszereket
  és eszközöket, megoldásukhoz egy magas szintű formális programozási
  nyelv fejlesztői környezetét önállóan használja.
\item
  Hatékonyan keres információt; az ikt-tudáselemek felhasználásával
  társaival együttműködve problémákat old meg.
\item
  Ismeri az e-világ elvárásait, lehetőségeit és veszélyeit.
\item
  Ismeri és tudja használni a célszerűen választott informatikai
  eszközöket és a működtető szoftvereit, ismeri a felhasználási
  lehetőségeket.
\item
  Ismeri a digitális eszközök és a számítógépek fő egységeit, ezek
  fejlődésének főbb állomásait, tendenciáit.
\item
  Tudatosan alakítja informatikai környezetét, ismeri az ergonomikus
  informatikai környezet jellemzőit, figyelembe veszi a digitális
  eszközök egészségkárosító hatásait, óvja maga és környezete
  egészségét.
\item
  Önállóan használja az informatikai eszközöket, elkerüli a tipikus
  felhasználói hibákat, elhárítja az egyszerűbb felhasználói hibákat.
\item
  Céljainak megfelelően használja a mobileszközök és a számítógépek
  operációs rendszereit.
\item
  Igénybe veszi az operációs rendszer és a számítógépes hálózat
  alapszolgáltatásait.
\item
  Követi a technológiai változásokat a digitális információforrások
  használatával.
\item
  Használja az operációs rendszer segédprogramjait, és elvégzi a
  munkakörnyezet beállításait.
\item
  Tisztában van a digitális kártevők elleni védekezés lehetőségeivel.
\item
  Használja az állományok tömörítését és a tömörített állományok
  kibontását.
\item
  Ismeri egy adott feladat megoldásához szükséges digitális eszközök és
  szoftverek kiválasztásának szempontjait.
\item
  Speciális dokumentumokat hoz létre, alakít át és formáz meg.
\item
  Tapasztalatokkal rendelkezik a formanyomtatványok, a sablonok, az
  előre definiált stílusok használatáról.
\item
  Gyakorlatot szerez a fotó-, hang-, videó-, multimédia-szerkesztő, a
  bemutató-készítő eszközök használatában.
\item
  Alkalmazza az információkeresés során gyűjtött multimédiás
  alapelemeket új dokumentumok készítéséhez.
\item
  Dokumentumokat szerkeszt és helyez el tartalomkezelő rendszerben.
\item
  Ismeri a html formátumú dokumentumok szerkezeti elemeit, érti a css
  használatának alapelveit; több lapból álló webhelyet készít.
\item
  Létrehozza az adott probléma megoldásához szükséges rasztergrafikus
  ábrákat.
\item
  Létrehoz vektorgrafikus ábrákat.
\item
  Digitálisan rögzít képet, hangot és videót, azokat manipulálja.
\item
  Tisztában van a raszter-, a vektorgrafikus ábrák tárolási és
  szerkesztési módszereivel
\item
  Érti az egyszerű problémák megoldásához szükséges tevékenységek
  lépéseit és kapcsolatukat.
\item
  Ismeri a következő elemi adattípusok közötti különbségeket: egész,
  valós szám, karakter, szöveg, logikai.
\item
  Ismeri az elemi és összetett adattípusok közötti különbségeket.
\item
  Érti egy algoritmus-leíró eszköz alapvető építőelemeit, érti a
  típusalgoritmusok felhasználásának lehetőségeit.
\item
  Példákban, feladatok megoldásában használja egy formális programozási
  nyelv fejlesztői környezetének alapszolgáltatásait.
\item
  Szekvencia, elágazás és ciklus segítségével algoritmust hoz létre, és
  azt egy magas szintű formális programozási nyelven kódolja.
\item
  A feladat megoldásának helyességét teszteli.
\item
  Adatokat táblázatba rendez.
\item
  Táblázatkezelővel adatelemzést és számításokat végez.
\item
  A problémamegoldás során függvényeket célszerűen használ.
\item
  Nagy adathalmazokat tud kezelni.
\item
  Az adatokat diagramon szemlélteti.
\item
  Ismeri az adatbázis-kezelés alapfogalmait.
\item
  Az adatbázisban interaktív módon keres, rendez és szűr.
\item
  A feladatmegoldás során az adatbázisba adatokat visz be, módosít és
  töröl, űrlapokat használ, jelentéseket nyomtat.
\item
  Strukturáltan tárolt nagy adathalmazokat kezel, azokból egyedi és
  összesített adatokat nyer ki.
\item
  Tapasztalatokkal rendelkezik hétköznapi jelenségek számítógépes
  szimulációjáról.
\item
  Hétköznapi, oktatáshoz készült szimulációs programokat használ.
\item
  Tapasztalatokat szerez a kezdőértékek változtatásának hatásairól a
  szimulációs programokban.
\item
  Ismeri és alkalmazza az információkeresési stratégiákat és
  technikákat, a találati listát a problémának megfelelően szűri,
  ellenőrzi annak hitelességét.
\item
  Etikus módon használja fel az információforrásokat, tisztában van a
  hivatkozás szabályaival.
\item
  Használja a két- vagy többrésztvevős kommunikációs lehetőségeket és
  alkalmazásokat.
\item
  Ismeri és alkalmazza a fogyatékkal élők közötti kommunikáció eszközeit
  és formáit.
\item
  Az online kommunikáció során alkalmazza a kialakult viselkedési
  kultúrát és szokásokat, a szerepelvárásokat.
\item
  Ismeri és használja a mobiltechnológiát, kezeli a mobileszközök
  operációs rendszereit és használ mobilalkalmazásokat.
\item
  Céljainak megfelelő alkalmazást választ, az alkalmazás funkcióira,
  kezelőfelületére vonatkozó igényeit megfogalmazza.
\item
  Az applikációkat önállóan telepíti.
\item
  Az iskolai oktatáshoz kapcsolódó mobileszközökre fejlesztett
  alkalmazások használata során együttműködik társaival.
\item
  Tisztában van az e-világ -- e-szolgáltatások, e-ügyintézés,
  e-kereskedelem, e-állampolgárság, it-gazdaság, környezet, kultúra,
  információvédelem -- biztonsági és jogi kérdéseivel.
\item
  Tisztában van a digitális személyazonosság és az információhitelesség
  fogalmával.
\item
  A gyakorlatban alkalmazza az adatok védelmét biztosító lehetőségeket.
\end{itemize}

\hypertarget{elso-elo-idegen-nyelv}{%
\subsection{Első élő idegen nyelv}\label{elso-elo-idegen-nyelv}}

\hypertarget{evfolyamon-2}{%
\subsubsection{9-12. évfolyamon}\label{evfolyamon-2}}

\begin{itemize}
\item
  Szóban és írásban is megold változatos kihívásokat igénylő, többnyire
  valós kommunikációs helyzeteket leképező feladatokat az élő idegen
  nyelven.
\item
  Szóban és írásban létrehoz szövegeket különböző szövegtípusokban.
\item
  Értelmez nyelvi szintjének megfelelő hallott és írott célnyelvi
  szövegeket kevésbé ismert témákban és szövegtípusokban is.
\item
  A tanult nyelvi elemek és kommunikációs stratégiák segítségével
  írásbeli és szóbeli interakciót folytat és tartalmakat közvetít idegen
  nyelven.
\item
  Kommunikációs szándékának megfelelően alkalmazza a nyelvi funkciókat
  és megszerzett szociolingvisztikai, pragmatikai és interkulturális
  jártasságát.
\item
  Nyelvtudását képes fejleszteni tanórán kívüli eszközökkel,
  lehetőségekkel és helyzetekben is, valamint a tanultakat és
  gimnáziumban a második idegen nyelv tanulásában is alkalmazza.
\item
  Felkészül az aktív nyelvtanulás eszközeivel az egész életen át történő
  tanulásra.
\item
  Használ hagyományos és digitális alapú nyelvtanulási forrásokat és
  eszközöket.
\item
  Alkalmazza nyelvtudását kommunikációra, közvetítésre, szórakozásra,
  ismeretszerzésre hagyományos és digitális csatornákon.
\item
  Törekszik a célnyelvi normához illeszkedő kiejtés, beszédtempó és
  intonáció megközelítésére.
\item
  Beazonosítja nyelvtanulási céljait és egyéni különbségeinek tudatában,
  ezeknek megfelelően fejleszti nyelvtudását.
\item
  Első idegen nyelvéből sikeresen érettségit tesz a céljainak megfelelő
  szinten.
\item
  Visszaad tankönyvi vagy más tanult szöveget, elbeszélést, nagyrészt
  folyamatos és érthető történetmeséléssel, a cselekményt logikusan
  összefűzve.
\item
  Összefüggően, érthetően és nagyrészt folyékonyan beszél az ajánlott
  tématartományokhoz tartozó és az érettségi témákban a tanult nyelvi
  eszközökkel, felkészülést követően.
\item
  Beszámol saját élményen, tapasztalaton alapuló vagy elképzelt
  eseményről a cselekmény, a körülmények, az érzések és gondolatok
  ismert nyelvi eszközökkel történő rövid jellemzésével.
\item
  Ajánlott tématartományhoz kapcsolódó képi hatás kapcsán saját
  gondolatait, véleményét és érzéseit is kifejti az ismert nyelvi
  eszközökkel.
\item
  Összefoglalja ismert témában nyomtatott vagy digitális alapú ifjúsági
  tartalmak lényegét röviden és érthetően.
\item
  Közép- és emelt szintű nyelvi érettségi szóbeli feladatokat old meg.
\item
  Összefüggő, folyékony előadásmódú szóbeli prezentációt tart önállóan,
  felkészülést követően, az érettségi témakörök közül szabadon
  választott témában, ikt-eszközökkel támogatva mondanivalóját.
\item
  Kreatív, változatos műfajú szövegeket alkot szóban, kooperatív
  munkaformákban.
\item
  Beszámol akár az érdeklődési körén túlmutató környezeti eseményről a
  cselekmény, a körülmények, az érzések és gondolatok ismert nyelvi
  eszközökkel történő összetettebb, részletes és világos jellemzésével.
\item
  Összefüggően, világosan és nagyrészt folyékonyan beszél az ajánlott
  tématartományhoz tartozó és az idevágó érettségi témákban, akár
  elvontabb tartalmakra is kitérve.
\item
  Alkalmazza a célnyelvi normához illeszkedő, természeteshez közelítő
  kiejtést, beszédtempót és intonációt.
\item
  Írásban röviden indokolja érzéseit, gondolatait, véleményét már
  elvontabb témákban.
\item
  Leír összetettebb cselekvéssort, történetet, személyes élményeket,
  elvontabb témákban.
\item
  Információt vagy véleményt közlő és kérő, összefüggő feljegyzéseket,
  üzeneteket ír.
\item
  Alkalmazza a formális és informális regiszterhez köthető
  sajátosságokat.
\item
  Használ szövegkohéziós és figyelemvezető eszközöket.
\item
  Megold változatos írásbeli, feladatokat szövegszinten.
\item
  Papíralapú vagy ikt-eszközökkel segített írott projektmunkát készít
  önállóan vagy kooperatív munkaformában.
\item
  Összefüggő szövegeket ír önállóan, akár elvontabb témákban.
\item
  A szövegek létrehozásához nyomtatott vagy digitális segédeszközt,
  szótárt használ.
\item
  Beszámol saját élményen, tapasztalaton alapuló, akár az érdeklődési
  körén túlmutató vagy elképzelt személyes eseményről a cselekmény, a
  körülmények, az érzések és gondolatok ismert nyelvi eszközökkel
  történő összetettebb, részletes és világos jellemzésével.
\item
  Beszámol akár az érdeklődési körén túlmutató közügyekkel,
  szórakozással kapcsolatos eseményről a cselekmény, a körülmények, az
  érzések és gondolatok ismert nyelvi eszközökkel történő összetettebb,
  részletes és világos jellemzésével.
\item
  A megfelelő szövegtípusok jellegzetességeit követi.
\item
  Értelmezi a szintjének megfelelő célnyelvi, komplexebb tanári
  magyarázatokat a nyelvórákon.
\item
  Megérti a célnyelvi, életkorának és érdeklődésének megfelelő hazai és
  nemzetközi hírek, események lényegét.
\item
  Kikövetkezteti a szövegben megjelenő elvontabb nyelvi elemek
  jelentését az ajánlott témakörökhöz kapcsolódó témákban.
\item
  Értelmezi a szövegben megjelenő összefüggéseket.
\item
  Megérti, értelmezi és összefoglalja az összetettebb, a
  tématartományhoz kapcsolódó összefüggő hangzó szöveget, és értelmezi a
  szövegben megjelenő összefüggéseket.
\item
  Megérti és értelmezi az összetettebb, az ajánlott témakörökhöz
  kapcsolódó összefüggő szövegeket, és értelmezi a szövegben megjelenő
  összefüggéseket.
\item
  Megérti az ismeretlen nyelvi elemeket is tartalmazó hangzó szöveg
  lényegi tartalmát.
\item
  Megérti a hangzó szövegben megjelenő összetettebb részinformációkat.
\item
  Megérti az elvontabb tartalmú hangzószövegek lényegét, valamint a
  beszélők véleményét is.
\item
  Alkalmazza a hangzó szövegből nyert információt feladatok megoldása
  során.
\item
  Célzottan keresi az érdeklődésének megfelelő autentikus szövegeket
  tanórán kívül is, ismeretszerzésre és szórakozásra.
\item
  A tanult nyelvi elemek segítségével megérti a hangzó szöveg lényegét
  számára kevésbé ismert témákban és szituációkban is.
\item
  A tanult nyelvi elemek segítségével megérti a hangzó szöveg lényegét
  akár anyanyelvi beszélők köznyelvi kommunikációjában a számára kevésbé
  ismert témákban és szituációkban is.
\item
  Megérti és értelmezi a legtöbb televíziós hírműsort.
\item
  Megért szokványos tempóban folyó autentikus szórakoztató és
  ismeretterjesztő tartalmakat, változatos csatornákon.
\item
  Elolvas és értelmez nyelvi szintjének megfelelő irodalmi szövegeket.
\item
  Megérti és értelmezi a lényeget az ajánlott tématartományokhoz
  kapcsolódó összefüggő, akár autentikus írott szövegekben.
\item
  Megérti és értelmezi az összefüggéseket és a részleteket az ajánlott
  tématartományokhoz kapcsolódó összefüggő, akár autentikus írott
  szövegekben.
\item
  Értelmezi a számára ismerős, elvontabb tartalmú szövegekben megjelenő
  ismeretlen nyelvi elemeket.
\item
  A szövegkörnyezet alapján kikövetkezteti a szövegben előforduló
  ismeretlen szavak jelentését.
\item
  Megérti az ismeretlen nyelvi elemeket is tartalmazó írott szöveg
  tartalmát.
\item
  Megérti és értelmezi az írott szövegben megjelenő összetettebb
  részinformációkat.
\item
  Kiszűr konkrét információkat nyelvi szintjének megfelelő szövegből, és
  azokat összekapcsolja egyéb ismereteivel.
\item
  Alkalmazza az írott szövegből nyert információt feladatok megoldása
  során.
\item
  Keresi az érdeklődésének megfelelő, célnyelvi, autentikus szövegeket
  szórakozásra és ismeretszerzésre tanórán kívül is.
\item
  Egyre változatosabb, hosszabb, összetettebb és elvontabb szövegeket,
  tartalmakat értelmez és használ.
\item
  Részt vesz a változatos szóbeli interakciót és kognitív kihívást
  igénylő nyelvórai tevékenységekben.
\item
  Szóban ad át nyelvi szintjének megfelelő célnyelvi tartalmakat valós
  nyelvi interakciót leképező szituációkban.
\item
  A társalgásba aktívan, kezdeményezően és egyre magabiztosabban
  bekapcsolódik az érdeklődési körébe tartozó témák esetén vagy az
  ajánlott tématartományokon belül.
\item
  Társalgást kezdeményez, a megértést fenntartja, törekszik mások
  bevonására, és szükség esetén lezárja azt az egyes tématartományokon
  belül, akár anyanyelvű beszélgetőtárs esetében is.
\item
  A társalgást hatékonyan és udvariasan fenntartja, törekszik mások
  bevonására, és szükség esetén lezárja azt, akár ismeretlen
  beszélgetőtárs esetében is.
\item
  Előkészület nélkül részt tud venni személyes jellegű, vagy érdeklődési
  körének megfelelő ismert témáról folytatott társalgásban,
\item
  Érzelmeit, véleményét változatos nyelvi eszközökkel szóban
  megfogalmazza és arról interakciót folytat.
\item
  A mindennapi élet különböző területein, a kommunikációs helyzetek
  széles körében tesz fel releváns kérdéseket információszerzés
  céljából, és válaszol megfelelő módon a hozzá intézett célnyelvi
  kérdésekre.
\item
  Aktívan, kezdeményezően és magabiztosan vesz részt a változatos
  szóbeli interakciót és kognitív kihívást igénylő nyelvórai
  tevékenységekben.
\item
  Társaival a kooperatív munkaformákban és a projektfeladatok megoldása
  során is törekszik a célnyelvi kommunikációra.
\item
  Egyre szélesebb körű témákban, nyelvi kommunikációt igénylő
  helyzetekben reagál megfelelő módon, felhasználva általános és nyelvi
  háttértudását, ismereteit, alkalmazkodva a társadalmi normákhoz.
\item
  Váratlan, előre nem kiszámítható eseményekre, jelenségekre és
  történésekre jellemzően célnyelvi eszközökkel is reagál tanórai
  szituációkban.
\item
  Szóban és írásban, valós nyelvi interakciók során jó
  nyelvhelyességgel, megfelelő szókinccsel, a természeteshez közelítő
  szinten vesz részt az egyes tématartományokban és az idetartozó
  érettségi témákban.
\item
  Informális és életkorának megfelelő formális írásos üzeneteket ír,
  digitális felületen is.
\item
  Véleményét írásban, tanult nyelvi eszközökkel megfogalmazza és arról
  írásban interakciót folytat.
\item
  Véleményét írásban változatos nyelvi eszközökkel megfogalmazza és
  arról interakciót folytat.
\item
  Írásban átad nyelvi szintjének megfelelő célnyelvi tartalmakat valós
  nyelvi interakciók során.
\item
  Írásban és szóban, valós nyelvi interakciók során jó
  nyelvhelyességgel, megfelelő szókinccsel, a természeteshez közelítő
  szinten vesz részt az egyes tématartományokban és az idetartozó
  érettségi témákban.
\item
  Összetett információkat ad át és cserél.
\item
  Egyénileg vagy kooperáció során létrehozott projektmunkával
  kapcsolatos kiselőadást tart önállóan, összefüggő és folyékony
  előadásmóddal, digitális eszközök segítségével, felkészülést követően.
\item
  Használ célnyelvi tartalmakat tudásmegosztásra.
\item
  Ismer más tantárgyi tartalmakat, részinformációkat célnyelven,
\item
  Összefoglal és lejegyzetel, írásban közvetít rövid olvasott vagy
  hallott szövegeket.
\item
  Környezeti témákban a kommunikációs helyzetek széles körében
  hatékonyan ad át és cserél információt.
\item
  Írott szöveget igénylő projektmunkát készít olvasóközönségnek.
\item
  Írásban közvetít célnyelvi tartalmakat valós nyelvi interakciót
  leképező szituációkban.
\item
  Tanult kifejezések alkalmazásával és az alapvető nyelvi szokások
  követésével további alapvető érzéseket fejez ki (pl. aggódást,
  félelmet, kételyt).
\item
  Tanult kifejezések alkalmazásával és az alapvető nyelvi szokások
  követésével kifejez érdeklődést és érdektelenséget, szemrehányást,
  reklamálást.
\item
  Tanult kifejezések alkalmazásával és az alapvető nyelvi szokások
  követésével kifejez kötelezettséget, szándékot, kívánságot,
  engedélykérést, feltételezést.
\item
  Tanult kifejezések alkalmazásával és az alapvető nyelvi szokások
  követésével kifejez ítéletet, kritikát, tanácsadást.
\item
  Tanult kifejezések alkalmazásával és az alapvető nyelvi szokások
  követésével kifejez segítségkérést, ajánlást és ezekre történő
  reagálást.
\item
  Tanult kifejezések alkalmazásával és az alapvető nyelvi szokások
  követésével kifejez ok-okozat viszony vagy cél meghatározását.
\item
  Tanult kifejezések alkalmazásával és az alapvető nyelvi szokások
  követésével kifejez emlékezést és nem emlékezést.
\item
  Összekapcsolja a mondatokat megfelelő kötőszavakkal, így követhető
  leírást ad, vagy nem kronológiai sorrendben lévő eseményeket is
  elbeszél.
\item
  A kohéziós eszközök szélesebb körét alkalmazza szóbeli vagy írásbeli
  megnyilatkozásainak érthetőbb, koherensebb szöveggé szervezéséhez.
\item
  Több különálló elemet összekapcsol összefüggő lineáris szempontsorrá.
\item
  Képes rendszerezni kommunikációját: jelzi szándékát, kezdeményez,
  összefoglal és lezár.
\item
  Használ kiemelést, hangsúlyozást, helyesbítést.
\item
  Körülírással közvetíti a jelentéstartalmat, ha a megfelelő szót nem
  ismeri.
\item
  Ismert témákban a szövegösszefüggés alapján kikövetkezteti az
  ismeretlen szavak jelentését, megérti az ismeretlen szavakat is
  tartalmazó mondat jelentését.
\item
  Félreértéshez vezető hibáit kijavítja, ha beszédpartnere jelzi a
  problémát; a kommunikáció megszakadása esetén más stratégiát
  alkalmazva újrakezdi a mondandóját.
\item
  A társalgás vagy eszmecsere menetének fenntartásához alkalmazza a
  rendelkezésére álló nyelvi és stratégiai eszközöket.
\item
  Nem értés esetén képes a tartalom tisztázására.
\item
  Mondanivalóját kifejti kevésbé ismerős helyzetekben is nyelvi eszközök
  széles körének használatával.
\item
  A tanult nyelvi elemeket adaptálni tudja kevésbé begyakorolt
  helyzetekhez is.
\item
  Szóbeli és írásbeli közlései során változatos nyelvi struktúrákat
  használ.
\item
  A tanult nyelvi funkciókat és nyelvi eszköztárát életkorának megfelelő
  élethelyzetekben megfelelően alkalmazza.
\item
  Szociokulturális ismeretei (például célnyelvi társadalmi szokások,
  testbeszéd) már lehetővé teszik azt, hogy társasági szempontból is
  megfelelő kommunikációt folytasson.
\item
  Szükség esetén eltér az előre elgondoltaktól, és mondandóját a
  beszédpartnerekhez, hallgatósághoz igazítja.
\item
  Az ismert nyelvi elemeket vizsgahelyzetben is használja.
\item
  Megértést nehezítő hibáit önállóan javítani tudja.
\item
  Nyelvtanulási céljai érdekében alkalmazza a tanórán kívüli
  nyelvtanulási lehetőségeket.
\item
  Célzottan keresi az érdeklődésének megfelelő autentikus szövegeket
  tanórán kívül is, ismeretszerzésre és szórakozásra.
\item
  Felhasználja a célnyelvű, legfőbb hazai és nemzetközi híreket
  ismeretszerzésre és szórakozásra.
\item
  Használ célnyelvi elemeket más tudásterületen megcélzott tartalmakból.
\item
  Használ célnyelvi tartalmakat ismeretszerzésre.
\item
  Használ ismeretterjesztő anyagokat nyelvtudása fejlesztésére.
\item
  Hibáit az esetek többségében önállóan is képes javítani.
\item
  Hibáiból levont következtetéseire többnyire épít nyelvtudása
  fejlesztése érdekében.
\item
  Egy összetettebb nyelvi feladat, projekt végéig tartó célokat tűz ki
  magának.
\item
  Megfogalmaz hosszú távú nyelvtanulási célokat saját maga számára.
\item
  Nyelvtanulási céljai érdekében tudatosabban foglalkozik a célnyelvvel.
\item
  Céljai eléréséhez megtalálja és használja a megfelelő eszközöket,
  módokat.
\item
  Céljai eléréséhez társaival párban és csoportban is együttműködik.
\item
  Beazonosít nyelvtanulási célokat és ismeri az ezekhez tartozó
  nyelvtanulási és nyelvhasználati stratégiákat.
\item
  Használja a nyelvtanulási és nyelvhasználati stratégiákat nyelvtudása
  fenntartására és fejlesztésére.
\item
  Hatékonyan alkalmazza a tanult nyelvtanulási és nyelvhasználati
  stratégiákat.
\item
  Céljai eléréséhez önszabályozóan is dolgozik.
\item
  Az első idegen nyelvből sikeres érettségit tesz legalább középszinten.
\item
  Nyelvi haladását fel tudja mérni.
\item
  Használ önértékelési módokat nyelvtudása felmérésére.
\item
  Egyre tudatosabban használja az ön-, tanári, vagy társai értékelését
  nyelvtudása fenntartására és fejlesztésére.
\item
  Használja az ön-, tanári, vagy társai értékelését nyelvtudása
  fenntartására és fejlesztésére.
\item
  Hiányosságait, hibáit felismeri, azokat egyre hatékonyabban
  kompenzálja, javítja a tanult stratégiák felhasználásával.
\item
  Nyelvtanulási céljai érdekében él a valós nyelvhasználati
  lehetőségekkel.
\item
  Használja a célnyelvet életkorának és nyelvi szintjének megfelelő
  aktuális témákban és a hozzájuk tartozó szituációkban.
\item
  Az ismert nyelvi elemeket vizsgahelyzetben is használja.
\item
  Beszéd- és írásprodukcióját tudatosan megtervezi, hiányosságait
  igyekszik kompenzálni.
\item
  Nyelvi produkciójában és recepciójában önállóságot mutat, és egyre
  kevesebb korlát akadályozza.
\item
  Törekszik releváns digitális tartalmak használatára beszédkészségének,
  szókincsének és kiejtésének továbbfejlesztése céljából.
\item
  Digitális eszközöket és felületeket is magabiztosan használ
  nyelvtudása fejlesztésére.
\item
  Digitális eszközöket és felületeket is használ a célnyelven
  ismeretszerzésre és szórakozásra.
\item
  Alkalmazza a célnyelvi kultúráról megszerzett ismereteit informális
  kommunikációjában,
\item
  Ismeri a célnyelvi országok történelmének és jelenének legfontosabb
  vonásait,
\item
  Tájékozott a célnyelvi országok jellemzőiben és kulturális
  sajátosságaiban.
\item
  Tájékozott, és alkalmazni is tudja a célnyelvi országokra jellemző
  alapvető érintkezési és udvariassági szokásokat.
\item
  Ismeri és keresi a főbb hasonlóságokat és különbségeket saját
  anyanyelvi és a célnyelvi közösség szokásai, értékei, attitűdjei és
  meggyőződései között.
\item
  Átadja célnyelven a magyar értékeket.
\item
  Ismeri a célnyelvi és saját hazájának kultúrája közötti hasonlóságokat
  és különbségeket.
\item
  Interkulturális tudatosságára építve felismeri a célnyelvi és saját
  hazájának kultúrája közötti hasonlóságokat és különbségeket, és a
  magyar értékek átadására képessé válik.
\item
  Környezetének kulturális értékeit célnyelven közvetíti.
\item
  Kikövetkezteti a célnyelvi kultúrákhoz kapcsolódó egyszerű, ismeretlen
  nyelvi elemeket.
\item
  A célnyelvi kultúrákhoz kapcsolódó tanult nyelvi elemeket magabiztosan
  használja.
\item
  Interkulturális ismeretei segítségével társasági szempontból is
  megfelelő kommunikációt folytat írásban és szóban.
\item
  Felismeri a legfőbb hasonlóságokat és különbségeket az ismert nyelvi
  változatok között.
\item
  Megfogalmaz főbb hasonlóságokat és különbségeket az ismert nyelvi
  változatok között.
\item
  Alkalmazza a nyelvi változatokról megszerzett ismereteit informális
  kommunikációjában.
\item
  Megérti a legfőbb nyelvi dialektusok egyes elemeit is tartalmazó
  szóbeli közléseket.
\item
  Digitális eszközökön és csatornákon keresztül is alkot szöveget szóban
  és írásban.
\item
  Digitális eszközökön és csatornákon keresztül is megérti az ismert
  témához kapcsolódó írott vagy hallott szövegeket.
\item
  Digitális eszközökön és csatornákon keresztül is alkalmazza az ismert
  témához kapcsolódó írott vagy hallott szövegeket.
\item
  Digitális eszközökön és csatornákon keresztül is folytat célnyelvi
  interakciót az ismert nyelvi eszközök segítségével.
\item
  Digitális eszközökön és csatornákon keresztül is folytat a
  természeteshez közelítő célnyelvi interakciót az ismert nyelvi
  eszközök segítségével.
\item
  Digitális eszközökön és csatornákon keresztül is megfelelő nyelvi
  eszközökkel alkot szöveget szóban és írásban.
\item
  Alkalmazza az életkorának és érdeklődésének megfelelő digitális
  műfajok főbb jellemzőit.
\end{itemize}

\hypertarget{komplex-termeszettudomany}{%
\subsection{Komplex természettudomány}\label{komplex-termeszettudomany}}

\hypertarget{evfolyamon-3}{%
\subsubsection{9. évfolyamon}\label{evfolyamon-3}}

\begin{itemize}
\tightlist
\item
  Ismeri a tudományos kutatás alapszabályait és azokat alkalmazza
\item
  Önálló tudományos kutatást tervez meg és végez el
\item
  Önálló kutatása összeállításakor tudományos modelleket használ
\item
  Tudományos kutatások során elvégzi a megszerzett adatok feldolgozását
  és értelmezését
\item
  Érti a tudomány szerepét és szükségszerűségét a társadalmi folyamatok
  alakításában
\item
  Hiteles források felhasználásával egy tudományos probléma kritikus
  elemzését adja a megszerzett információk alapján
\item
  Elméleti és gyakorlati eszközöket választ és alkalmaz egy adott
  tudományos probléma ismertetéséhez
\item
  Tudományos kutatási eredményeket érthetően mutat be digitális eszközök
  segítségével
\item
  Önállóan és kiscsoportban biztonságosan végez természettduományos
  kísérleteket
\item
  Felismeri a saját és társai által végzett tudományos kísérletek etikai
  és társadalmi vonatkozásait
\item
  Ismeri és alkalmazza az energia felhasználásának és átalakulásának
  elméleti és gyakorlati lehetőségeit (energiaáramláson alapuló
  ökoszisztémák, a föld saját energiaforrásai stb.)
\item
  Felismeri és kísérletei során alkalmazza azt a tudást, hogy az anyag
  atomi természete határozza meg a fizikai és kémiai tulajdonságokat és
  azok kölcsönhatásából eredő módosulását. (Példák kísérletekre: kémiai
  reakciók, molekuláris biológiai, anyagok körforgása a különböző
  ökoszisztémákban)
\item
  Felismeri és kísérletei során alkalmazza azt a tudást, hogy a
  természet ismert rendszerei előrejelezhető módon változnak (evolúció,
  klímaváltozás, földtörténeti korok, Föld felszíni változása,
  tektonikus mozgások, ember környezeti hatásai stb.)
\item
  Felismeri és kísérletei során alkalmazza azt a tudást, hogy a tárgyak
  mozgása előrejelezhető (erők, égitestek mozgása, molekuláris mozgások,
  hangok, fények mozgása)
\item
  Felismeri és kísérletei során alkalmazza azt a tudást, hogy a világ
  megismerésének egyik alapja a szerveződési szintek és típusok
  megértése (periódusos rendszer, sjetszintű szerveződések, állat és
  növényvilág rendszertana, a világegyetem szervező elvei) -
\item
\item
  Ismeri a föld népességgének aktuális kihívásait beleértve azok
  társadalmi és egészségügyi kockázatait (betegség-megelőzés, járványok,
  élelmezés)
\item
  Ismeri a hulladékgazdálkodás aktuális kihívásait
\item
  Ismeri az energia fogalmát és az egyéni és társadalmi
  energiafelhasználás különböző lehetőségeit
\item
  Megkülönbözteti egymástól a természetes és mesterséges anyagokat és
  felismeri, hogy miként állapítható egy anyag összetétele
\item
  Ismeri a városi és falusi életmód és életterek közötti különbségeket,
  azok környezetre gyakorolt hatását
\item
  Használja a regionalitás fogalmát és érti annak szerepét a gazdasági
  folyamatok alakulásában
\item
  Ismeri a levegő és a víz fizikai és kémiai jellemzőit, felismeri ezek
  élettani hatásait
\item
  Tudja, hogy milyen vízkészletekkel rendelkezik a Föld és azok
  felhasználásának milyen hatása van a környezeti, ipari, turisztikai és
  szállítmányozási folyamatokra
\item
  Ismeri a növények tápanyagigényét és fejlődésük alapjait
\item
  Ismeri a vadon élő állatközösségeket fenyegető veszélyeket és azt a
  kihívást, amit ez az emberekre gyakorol
\item
  Ismeri a különböző kultúrák eltérő gazdasági termelési szokásait
  (növénytermesztés, állattartás)
\item
  Ismeri a talaj természetét és annak megművelésének különböző formáit
\item
  Ismeri a föld légkörét befolyásoló globális folyamatokat (tengerek és
  szelek áramlása, klímaváltozás, üvegházhatású gázok)
\item
  Ismeri az emberi táplálkozás során hasznos ételeket, megkülönbözteti a
  gyógyító és a káros anyagokat, önállóan tud egészséges ételt készíteni
\item
  Ismeri az emberi agy alapvető működési szabályait és az azt
  befolyásoló tényezőket
\item
  Tisztában van az Univerzum létrejöttének ma ismert elméletével,
  valamint a Naprendszer kialakulásának folyamatával
\item
  Ismeri a tanulás és az emberi kommunikáció biológiai alapjait
\item
  Ismeri az emberi szervezet egészségét alapvetően befolyásoló
  tényezőket, a stressz, az öröklött hajlamok és genetikai
  tulajdonságok, valamint a környezeti hatások szerepét
\item
  Ismeri az emberi szexualitás kulturális, társadalmi és biológiai
  alapjait. Önálló véleménye van a nemi szerepek fontosságáról, érti a
  nemi identitás komplex jellegét.
\item
  Ismeri a hálózatok és a hálózatkutatás szerepét modern világunkban, az
  életközösségek, a sejtszintű gondolkodás és az információs
  technológiák területén.
\item
  Ismeri a genetikai információ átadásának alapvető szabályait.
\end{itemize}

\hypertarget{magyar-nyelv-es-irodalom}{%
\subsection{Magyar nyelv és irodalom}\label{magyar-nyelv-es-irodalom}}

\hypertarget{evfolyamon-4}{%
\subsubsection{9-12. évfolyamon}\label{evfolyamon-4}}

\begin{itemize}
\item
  Az anyanyelvről szerzett ismereteit alkalmazva képes a
  kommunikációjában a megfelelő nyelvváltozat kiválasztására,
  használatára.
\item
  Felismeri a kommunikáció zavarait, kezelésükre stratégiát dolgoz ki.
\item
  Felismeri és elemzi a tömegkommunikáció befolyásoló eszközeit, azok
  céljait és hatásait.
\item
  Reflektál saját kommunikációjára, szükség esetén változtat azon.
\item
  Ismeri az anyanyelvét, annak szerkezeti felépítését, nyelvhasználata
  tudatos és helyes.
\item
  Ismeri a magyar nyelv hangtanát, alaktanát, szófajtanát, mondattanát,
  ismeri és alkalmazza a tanult elemzési eljárásokat.
\item
  Felismeri és megnevezi a magyar és a tanult idegen nyelv közötti
  hasonlóságokat és eltéréseket.
\item
  Ismeri a szöveg fogalmát, jellemzőit, szerkezeti sajátosságait,
  valamint a különféle szövegtípusokat és megjelenésmódokat.
\item
  Felismeri és alkalmazza a szövegösszetartó grammatikai és jelentésbeli
  elemeket, szövegépítése arányos és koherens.
\item
  Ismeri a stílus fogalmát, a stíluselemeket, a stílushatást, a
  stíluskorszakokat, stílusrétegeket, ismereteit a szöveg befogadása és
  alkotása során alkalmazza.
\item
  Szövegelemzéskor felismeri az alakzatokat és a szóképeket, értelmezi
  azok hatását, szerepét, megnevezi típusaikat.
\item
  Ismeri a nyelvhasználatban előforduló különféle nyelvváltozatokat
  (nyelvjárások, csoportnyelvek, rétegnyelvek), összehasonlítja azok
  főbb jellemzőit.
\item
  Alkalmazza az általa tanult nyelvi, nyelvtani, helyesírási,
  nyelvhelyességi ismereteket.
\item
  A retorikai ismereteit a gyakorlatban is alkalmazza.
\item
  Ismeri és érti a nyelvrokonság fogalmát, annak kritériumait.
\item
  Ismeri a magyar nyelv eredetének hipotéziseit, és azok tudományosan
  megalapozott bizonyítékait.
\item
  Érti, hogy nyelvünk a történelemben folyamatosan változik, ismeri a
  magyar nyelvtörténet nagy korszakait, kiemelkedő jelentőségű
  nyelvemlékeit.
\item
  Ismeri a magyar nyelv helyesírási, nyelvhelyességi szabályait.
\item
  Tud helyesen írni, szükség esetén nyomtatott és digitális helyesírási
  segédleteket használ.
\item
  Etikusan és kritikusan használja a hagyományos, papír alapú, illetve a
  világhálón található és egyéb digitális adatbázisokat.
\item
  Elolvassa a kötelező olvasmányokat, és saját örömére is olvas.
\item
  Felismeri és elkülöníti a műnemeket, illetve a műnemekhez tartozó
  műfajokat, megnevezi azok poétikai és retorikai jellemzőit.
\item
  Megérti, elemzi az irodalmi mű jelentésszerkezetének szintjeit.
\item
  Értelmezésében felhasználja irodalmi és művészeti, történelmi,
  művelődéstörténeti ismereteit.
\item
  Összekapcsolja az irodalmi művek szerkezeti felépítését, nyelvi
  sajátosságait azok tartalmával és értékszerkezetével.
\item
  Az irodalmi mű értelmezése során figyelembe veszi a mű
  keletkezéstörténeti hátterét, a műhöz kapcsolható filozófiai,
  eszmetörténeti szempontokat is.
\item
  Összekapcsolja az irodalmi művek szövegének lehetséges értelmezéseit
  azok társadalmi-történelmi szerepével, jelentőségével.
\item
  Összekapcsol irodalmi műveket különböző szempontok alapján (motívumok,
  történelmi, erkölcsi kérdésfelvetések, művek és parafrázisaik).
\item
  Összehasonlít egy adott irodalmi művet annak adaptációival (film,
  festmény, zenemű, animáció, stb.), összehasonlításkor figyelembe veszi
  az adott művészeti ágak jellemző tulajdonságait.
\item
  Epikai és drámai művekben önállóan értelmezi a cselekményszálak, a
  szerkezet, az időszerkezet (lineáris, nem lineáris), a helyszínek és a
  jellemek összefüggéseit.
\item
  Epikai és drámai művekben rendszerbe foglalja a szereplők viszonyait,
  valamint összekapcsolja azok motivációját és cselekedeteit.
\item
  Epikai művekben értelmezi a különböző elbeszélésmódok szerepét
  (tudatábrázolás, egyenes és függő beszéd, mindentudó és korlátozott
  elbeszélő stb.).
\item
  A drámai mű értelmezésében alkalmazza az általa tanult drámaelméleti
  és drámatörténeti fogalmakat (pl. analitikus és abszurd dráma, epikus
  színház, elidegenedés).
\item
  A líra mű értelmezésében alkalmazza az általa tanult líraelméleti és
  líratörténeti fogalmakat (pl. lírai én, beszédhelyzetek, beszédmódok,
  ars poetica, szereplíra).
\item
  A tantárgyhoz kapcsolódó fogalmakkal bemutatja a lírai mű hangulati és
  hangnemi sajátosságait, hivatkozik a mű verstani felépítésére.
\item
  Szükség esetén a mű értelmezéséhez felhasználja történeti ismereteit.
\item
  A mű értelmezésében összekapcsolja a szöveg poétikai tulajdonságait a
  mű nemzeti hagyományban betöltött szerepével.
\item
  Tájékozottságot szerez régiója magyar irodalmáról.
\item
  Tanulmányai során ismereteket szerez a kulturális intézmények (múzeum,
  könyvtár, színház) és a nyomtatott, illetve digitális formában
  megjelenő kulturális folyóiratok, adatbázisok működéséről.
\item
  Különböző megjelenésű, típusú, műfajú, korú és összetettségű
  szövegeket olvas, értelmez.
\item
  A különböző olvasási típusokat és a szövegfeldolgozási stratégiákat a
  szöveg típusának és az olvasás céljának megfelelően választja ki és
  kapcsolja össze.
\item
  A megismert szöveg tartalmi és nyelvi minőségéről érvekkel
  alátámasztott véleményt alkot.
\item
  Hosszabb terjedelmű szöveg alapján többszintű vázlatot vagy részletes
  gondolattérképet készít.
\item
  Azonosítja a szöveg szerkezeti elemeit, és figyelembe veszi azok
  funkcióit a szöveg értelmezésekor.
\item
  Egymással összefüggésben értelmezi a szöveg tartalmi elemeit és a
  hozzá kapcsolódó illusztrációkat, ábrákat.
\item
  Különböző típusú és célú szövegeket hallás alapján értelmez és
  megfelelő stratégia alkalmazásával értékel és összehasonlít.
\item
  Összefüggő szóbeli szöveg (előadás, megbeszélés, vita) alapján
  önállóan vázlatot készít.
\item
  Felismeri és értelmezésében figyelembe veszi a hallott és az írott
  szövegek közötti funkcionális és stiláris különbségeket.
\item
  Folyamatos és nem folyamatos, hagyományos és digitális szövegeket
  olvas és értelmez maga által választott releváns szempontok alapján.
\item
  Feladatai megoldásához önálló kutatómunkát végez nyomtatott és
  digitális forrásokban, ezek eredményeit szintetizálja.
\item
  Felismeri és értelmezi a szövegben a kétértelműséget és a félrevezető
  információt, valamint elemzi és értelmezi a szerző szándékát.
\item
  Megtalálja a közös és eltérő jellemzőket a hagyományos és a digitális
  technikával előállított, tárolt szövegek között, és véleményt formál
  azok sajátosságairól.
\item
  Törekszik arra, hogy a különböző típusú, stílusú és regiszterű
  szövegekben megismert, számára új kifejezéseket beépítse szókincsébe,
  azokat adekvát módon használja.
\item
  Önállóan értelmezi az ismeretlen kifejezéseket a szövegkörnyezet vagy
  digitális, illetve nyomtatott segédeszközök használatával.
\item
  Ismeri a tanult tantárgyak, tudományágak szakszókincsét, azokat a
  beszédhelyzetnek megfelelően használja.
\item
  Megadott szempontrendszer alapján szóbeli feleletet készít.
\item
  Képes eltérő műfajú szóbeli szövegek alkotására: felelet, kiselőadás,
  hozzászólás, felszólalás.
\item
  Rendelkezik korának megfelelő retorikai ismeretekkel.
\item
  Felismeri és megnevezi a szóbeli előadásmód hatáskeltő eszközeit,
  hatékonyan alkalmazza azokat.
\item
  Írásbeli és szóbeli nyelvhasználata, stílusa az adott kommunikációs
  helyzetnek megfelelő. írásképe tagolt, beszéde érthető, artikulált.
\item
  A tanult szövegtípusoknak megfelelő tartalommal és szerkezettel
  önállóan alkot különféle írásbeli szövegeket.
\item
  Az írásbeli szövegalkotáskor alkalmazza a tanult szerkesztési,
  stilisztikai ismereteket és a helyesírási szabályokat.
\item
  Érvelő esszét alkot megadott szempontok vagy szövegrészletek alapján.
\item
  Ismeri, érti és etikusan alkalmazza a hagyományos, digitális és
  multimédiás szemléltetést.
\item
  Különböző, a munka világában is használt hivatalos szövegeket alkot
  hagyományos és digitális felületeken (pl. kérvény, beadvány,
  nyilatkozat, egyszerű szerződés, meghatalmazás, önéletrajz, motivációs
  levél).
\item
  Megadott vagy önállóan kiválasztott szempontok alapján az irodalmi
  művekről elemző esszét ír.
\item
  A kötelező olvasmányokat elolvassa, és saját örömére is olvas.
\item
  Tudatosan keresi a történeti és esztétikai értékekkel rendelkező
  olvasmányokat, műalkotásokat.
\item
  Olvasmányai kiválasztásakor figyelembe veszi az alkotások kulturális
  regiszterét.
\item
  Társai érdeklődését figyelembe véve ajánl olvasmányokat.
\item
  Választott olvasmányaira is vonatkoztatja a tanórán megismert
  kontextusteremtő eljárások tanulságait.
\item
  Önismeretét irodalmi művek révén fejleszti.
\item
  Részt vesz irodalmi mű kreatív feldolgozásában, bemutatásában (pl.
  animáció, dramaturgia, átirat).
\item
  A környező világ jelenségeiről, szövegekről, műalkotásokról véleményt
  alkot, és azt érvekkel támasztja alá.
\item
  Megnyilvánulásaiban, a vitákban alkalmazza az érvelés alapvető
  szabályait.
\item
  Vitahelyzetben figyelembe veszi mások álláspontját, a lehetséges
  ellenérveket is.
\item
  Feladatai megoldásához önálló kutatómunkát végez nyomtatott és
  digitális forrásokban, a források tartalmát mérlegelő módon gondolja
  végig.
\item
  A feladatokat komplex szempontoknak megfelelően oldja meg, azokat
  kiegészíti saját szempontjaival.
\item
  A kommunikációs helyzetnek és a célnak megfelelően tudatosan
  alkalmazza a beszélt és írott nyelvet, reflektál saját és társai
  nyelvhasználatára.
\end{itemize}

\hypertarget{matematika}{%
\subsection{Matematika}\label{matematika}}

\hypertarget{evfolyamon-5}{%
\subsubsection{9-12. évfolyamon}\label{evfolyamon-5}}

\begin{itemize}
\item
  Ismeretei segítségével, a megfelelő modell alkalmazásával megold
  hétköznapi és matematikai problémákat, a megoldást ellenőrzi és
  értelmezi.
\item
  Megérti a környezetében jelen lévő logikai, mennyiségi, függvényszerű,
  térbeli és statisztikai kapcsolatokat.
\item
  Sejtéseket fogalmaz meg és logikus lépésekkel igazolja azokat.
\item
  Adatokat gyűjt, rendez, ábrázol, értelmez.
\item
  A matematikai szakkifejezéseket és jelöléseket helyesen használja
  írásban és szóban egyaránt.
\item
  Megérti a hallott és olvasott matematikai tartalmú szövegeket.
\item
  Felismeri a matematika különböző területei közötti kapcsolatokat.
\item
  A matematika tanulása során digitális eszközöket és különböző
  információforrásokat használ.
\item
  A matematikát más tantárgyakhoz kapcsolódó témákban is használja.
\item
  Matematikai ismereteit alkalmazza a pénzügyi tudatosság területét
  érintő feladatok megoldásában.
\item
  Adott halmazt diszjunkt részhalmazaira bont, osztályoz.
\item
  Matematikai vagy hétköznapi nyelven megfogalmazott szövegből a
  matematikai tartalmú információkat kigyűjti, rendszerezi.
\item
  Felismeri a matematika különböző területei közötti kapcsolatot.
\item
  Látja a halmazműveletek és a logikai műveletek közötti kapcsolatokat.
\item
  Halmazokat különböző módokon megad.
\item
  Halmazokkal műveleteket végez, azokat ábrázolja és értelmezi.
\item
  Véges halmazok elemszámát meghatározza.
\item
  Alkalmazza a logikai szita elvét.
\item
  Adott állításról eldönti, hogy igaz vagy hamis.
\item
  Alkalmazza a tagadás műveletét egyszerű feladatokban.
\item
  Ismeri és alkalmazza az „és``, a (megengedő és kizáró) „vagy'' logikai
  jelentését.
\item
  Megfogalmazza adott állítás megfordítását.
\item
  Megállapítja egyszerű „ha~\dots, akkor~\dots'' és „akkor és csak akkor''
  típusú állítások logikai értékét.
\item
  Helyesen használja a „minden'' és „van olyan'' kifejezéseket.
\item
  Tud egyszerű állításokat indokolni és tételeket bizonyítani.
\item
  Megold sorba rendezési és kiválasztási feladatokat.
\item
  Konkrét szituációkat szemléltet és egyszerű feladatokat megold gráfok
  segítségével.
\item
  Adott problémához megoldási stratégiát, algoritmust választ, készít.
\item
  A problémának megfelelő matematikai modellt választ, alkot.
\item
  A kiválasztott modellben megoldja a problémát.
\item
  A modellben kapott megoldását az eredeti problémába
  visszahelyettesítve értelmezi, ellenőrzi és az észszerűségi
  szempontokat figyelembe véve adja meg válaszát.
\item
  Geometriai szerkesztési feladatoknál vizsgálja és megállapítja a
  szerkeszthetőség feltételeit.
\item
  Ismeri és alkalmazza a következő egyenletmegoldási módszereket:
  mérlegelv, grafikus megoldás, szorzattá alakítás.
\item
  Megold elsőfokú egyismeretlenes egyenleteket és egyenlőtlenségeket,
  elsőfokú kétismeretlenes egyenletrendszereket.
\item
  Megold másodfokú egyismeretlenes egyenleteket és egyenlőtlenségeket;
  ismeri és alkalmazza a diszkriminánst, a megoldóképletet és a
  gyöktényezős alakot.
\item
  Megold egyszerű, a megfelelő definíció alkalmazását igénylő
  exponenciális egyenleteket, egyenlőtlenségeket.
\item
  Egyenletek megoldását behelyettesítéssel, értékkészlet-vizsgálattal
  ellenőrzi.
\item
  Ismeri a mérés alapelvét, alkalmazza konkrét alap- és származtatott
  mennyiségek esetén.
\item
  Ismeri a hosszúság, terület, térfogat, űrtartalom, idő mértékegységeit
  és az átváltási szabályokat. származtatott mértékegységeket átvált.
\item
  Sík- és térgeometriai feladatoknál a problémának megfelelő
  mértékegységben adja meg válaszát.
\item
  Ismeri és alkalmazza az oszthatóság alapvető fogalmait.
\item
  Összetett számokat felbont prímszámok szorzatára.
\item
  Meghatározza két természetes szám legnagyobb közös osztóját és
  legkisebb közös többszörösét, és alkalmazza ezeket egyszerű gyakorlati
  feladatokban.
\item
  Ismeri és alkalmazza az oszthatósági szabályokat.
\item
  Érti a helyi értékes írásmódot 10-es és más alapú számrendszerekben.
\item
  Ismeri a számhalmazok épülésének matematikai vonatkozásait a
  természetes számoktól a valós számokig.
\item
  A kommutativitás, asszociativitás, disztributivitás műveleti
  azonosságokat helyesen alkalmazza különböző számolási helyzetekben.
\item
  Racionális számokat tizedes tört és közönséges tört alakban is felír.
\item
  Ismer példákat irracionális számokra.
\item
  Ismeri a valós számok és a számegyenes kapcsolatát.
\item
  Ismeri és alkalmazza az abszolút érték, az ellentett és a reciprok
  fogalmát.
\item
  A számolással kapott eredményeket nagyságrendileg megbecsüli, és így
  ellenőrzi az eredményt.
\item
  Valós számok közelítő alakjaival számol, és megfelelően kerekít.
\item
  Ismeri és alkalmazza a négyzetgyök fogalmát és azonosságait.
\item
  Ismeri és alkalmazza az n-edik gyök fogalmát.
\item
  Ismeri és alkalmazza a normálalak fogalmát.
\item
  Ismeri és alkalmazza az egész kitevőjű hatvány fogalmát és a
  hatványozás azonosságait.
\item
  Ismeri és alkalmazza a racionális kitevőjű hatvány fogalmát és a
  hatványozás azonosságait.
\item
  Ismeri és alkalmazza a logaritmus fogalmát.
\item
  Műveleteket végez algebrai kifejezésekkel.
\item
  Ismer és alkalmaz egyszerű algebrai azonosságokat.
\item
  Átalakít algebrai kifejezéseket összevonás, szorzattá alakítás,
  nevezetes azonosságok alkalmazásával.
\item
  Ismeri és alkalmazza az egyenes és a fordított arányosságot.
\item
  Ismeri és alkalmazza a százalékalap, -érték, -láb, -pont fogalmát.
\item
  Ismeri és használja a pont, egyenes, sík (térelemek) és szög fogalmát.
\item
  Ismeri és feladatmegoldásban alkalmazza a térelemek kölcsönös
  helyzetét, távolságát és hajlásszögét.
\item
  Ismeri és alkalmazza a nevezetes szögpárok tulajdonságait.
\item
  Ismeri az alapszerkesztéseket, és ezeket végre tudja hajtani
  hagyományos vagy digitális eszközzel.
\item
  Ismeri és alkalmazza a háromszögek oldalai, szögei, oldalai és szögei
  közötti kapcsolatokat; a speciális háromszögek tulajdonságait.
\item
  Ismeri és alkalmazza a háromszög nevezetes vonalaira, pontjaira és
  köreire vonatkozó fogalmakat és tételeket.
\item
  Ismeri és alkalmazza a pitagorasz-tételt és megfordítását.
\item
  Kiszámítja háromszögek területét.
\item
  Ismeri és alkalmazza speciális négyszögek tulajdonságait, területüket
  kiszámítja.
\item
  Ismeri és alkalmazza a szabályos sokszög fogalmát; kiszámítja a konvex
  sokszög belső és külső szögeinek összegét.
\item
  Átdarabolással kiszámítja sokszögek területét.
\item
  Ki tudja számolni a kör és részeinek kerületét, területét.
\item
  Ismeri a kör érintőjének fogalmát, kapcsolatát az érintési pontba
  húzott sugárral.
\item
  Ismeri és alkalmazza a thalész-tételt és megfordítását.
\item
  Ismer példákat geometriai transzformációkra.
\item
  Ismeri és alkalmazza a síkbeli egybevágósági transzformációkat és
  tulajdonságaikat; alakzatok egybevágóságát.
\item
  Ismeri és alkalmazza a középpontos hasonlósági transzformációt, a
  hasonlósági transzformációt és az alakzatok hasonlóságát.
\item
  Ismeri és alkalmazza a hasonló síkidomok kerületének és területének
  arányára vonatkozó tételeket.
\item
  Megszerkeszti egy alakzat tengelyes, illetve középpontos tükörképét,
  pont körüli elforgatottját, párhuzamos eltoltját hagyományosan és
  digitális eszközzel.
\item
  Ismeri a vektorokkal kapcsolatos alapvető fogalmakat.
\item
  Ismer és alkalmaz egyszerű vektorműveleteket.
\item
  Alkalmazza a vektorokat feladatok megoldásában.
\item
  Ismeri hegyesszögek szögfüggvényeinek definícióját a derékszögű
  háromszögben.
\item
  Ismeri tompaszögek szögfüggvényeinek származtatását a hegyesszögek
  szögfüggvényei alapján.
\item
  Ismeri a hegyes- és tompaszögek szögfüggvényeinek összefüggéseit.
\item
  Alkalmazza a szögfüggvényeket egyszerű geometriai számítási
  feladatokban.
\item
  A szögfüggvény értékének ismeretében meghatározza a szöget.
\item
  Ismeri és alkalmazza a szinusz- és a koszinusztételt.
\item
  Ismeri és alkalmazza a hasáb, a henger, a gúla, a kúp, a gömb, a
  csonkagúla, a csonkakúp (speciális testek) tulajdonságait.
\item
  Lerajzolja a kocka, téglatest, egyenes hasáb, egyenes körhenger,
  egyenes gúla, forgáskúp hálóját.
\item
  Kiszámítja a speciális testek felszínét és térfogatát egyszerű
  esetekben.
\item
  Ismeri és alkalmazza a hasonló testek felszínének és térfogatának
  arányára vonatkozó tételeket.
\item
  Megad pontot és vektort koordinátáival a derékszögű
  koordináta-rendszerben.
\item
  Koordináta-rendszerben ábrázol adott feltételeknek megfelelő
  ponthalmazokat.
\item
  Koordináták alapján számításokat végez szakaszokkal, vektorokkal.
\item
  Ismeri és alkalmazza az egyenes egyenletét.
\item
  Egyenesek egyenletéből következtet az egyenesek kölcsönös helyzetére.
\item
  Kiszámítja egyenesek metszéspontjainak koordinátáit az egyenesek
  egyenletének ismeretében.
\item
  Megadja és alkalmazza a kör egyenletét a kör sugarának és a középpont
  koordinátáinak ismeretében.
\item
  Megad hétköznapi életben előforduló hozzárendeléseket.
\item
  Adott képlet alapján helyettesítési értékeket számol, és azokat
  táblázatba rendezi.
\item
  Táblázattal megadott függvény összetartozó értékeit ábrázolja
  koordináta-rendszerben.
\item
  Képlettel adott függvényt hagyományosan és digitális eszközzel
  ábrázol.
\item
  Adott értékkészletbeli elemhez megtalálja az értelmezési tartomány
  azon elemeit, amelyekhez a függvény az adott értéket rendeli.
\item
  A grafikonról megállapítja függvények alapvető tulajdonságait.
\item
  Számtani és mértani sorozatokat adott szabály alapján felír, folytat.
\item
  A számtani, mértani sorozat n-edik tagját felírja az első tag és a
  különbség (differencia)/hányados (kvóciens) ismeretében.
\item
  A számtani, mértani sorozatok első n tagjának összegét kiszámolja.
\item
  Mértani sorozatokra vonatkozó ismereteit használja gazdasági,
  pénzügyi, természettudományi és társadalomtudományi problémák
  megoldásában.
\item
  Adott cél érdekében tudatos adatgyűjtést és rendszerezést végez.
\item
  Hagyományos és digitális forrásból származó adatsokaság alapvető
  statisztikai jellemzőit meghatározza, értelmezi és értékeli.
\item
  Adatsokaságból adott szempont szerint oszlop- és kördiagramot készít
  hagyományos és digitális eszközzel.
\item
  Ismeri és alkalmazza a sodrófa (box-plot) diagramot adathalmazok
  jellemzésére, összehasonlítására.
\item
  Felismer grafikus manipulációkat diagramok esetén.
\item
  Tapasztalatai alapján véletlen jelenségek jövőbeni kimenetelére
  észszerűen tippel.
\item
  Ismeri és alkalmazza a klasszikus valószínűségi modellt és a
  laplace-képletet.
\item
  Véletlen kísérletek adatait rendszerezi, relatív gyakoriságokat
  számol, nagy elemszám esetén számítógépet alkalmaz.
\item
  Konkrét valószínűségi kísérletek esetében az esemény, eseménytér,
  elemi esemény, relatív gyakoriság, valószínűség, egymást kizáró
  események, független események fogalmát megkülönbözteti és alkalmazza.
\item
  Ismeri és egyszerű esetekben alkalmazza a valószínűség geometriai
  modelljét.
\item
  Meghatározza a valószínűséget visszatevéses, illetve visszatevés
  nélküli mintavétel esetén.
\item
  A megfelelő matematikai tankönyveket, feladatgyűjteményeket,
  internetes tartalmakat értőn olvassa, a matematikai tartalmat
  rendszerezetten kigyűjti és megérti.
\item
  A matematikai fogalmakat és jelöléseket megfelelően használja.
\item
  Önállóan kommunikál matematika tartalmú feladatokkal kapcsolatban.
\item
  Matematika feladatok megoldását szakszerűen prezentálja írásban és
  szóban a szükséges alapfogalmak, azonosságok, definíciók és tételek
  segítségével.
\item
  Szöveg alapján táblázatot, grafikont készít, ábrát, kapcsolatokat
  szemléltető gráfot rajzol, és ezeket kombinálva prezentációt készít és
  mutat be.
\item
  Ismer a tananyaghoz kapcsolódó matematikatörténeti vonatkozásokat.
\item
  Számológép segítségével alapműveletekkel felírható számolási
  eredményt; négyzetgyököt; átlagot; szögfüggvények értékét, illetve
  abból szöget; logaritmust; faktoriálist; binomiális együtthatót;
  szórást meghatároz.
\item
  Digitális környezetben matematikai alkalmazásokkal dolgozik.
\item
  Megfelelő informatikai alkalmazás segítségével szöveget szerkeszt,
  táblázatkezelő programmal diagramokat készít.
\item
  Ismereteit digitális forrásokból kiegészíti, számítógép segítségével
  elemzi és bemutatja.
\item
  Prezentációhoz informatív diákat készít, ezeket logikusan és
  következetesen egymás után fűzi és bemutatja.
\item
  Kísérletezéshez, sejtés megfogalmazásához, egyenlet grafikus
  megoldásához és ellenőrzéshez dinamikus geometriai, grafikus és
  táblázatkezelő szoftvereket használ.
\item
  Szerkesztési feladatok euklideszi módon történő megoldásához dinamikus
  geometriai szoftvert használ.
\end{itemize}

\hypertarget{masodik-idegen-nyelv}{%
\subsection{Második idegen nyelv}\label{masodik-idegen-nyelv}}

\hypertarget{evfolyamon-6}{%
\subsubsection{9-12. évfolyamon}\label{evfolyamon-6}}

\begin{itemize}
\item
  Megismerkedik az idegen nyelvvel, a nyelvtanulással és örömmel vesz
  részt az órákon.
\item
  Bekapcsolódik a szóbeliséget, írást, szövegértést vagy interakciót
  igénylő alapvető és korának megfelelő játékos, élményalapú élő idegen
  nyelvi tevékenységekbe.
\item
  Szóban visszaad szavakat, esetleg rövid, nagyon egyszerű szövegeket
  hoz létre.
\item
  Lemásol, leír szavakat és rövid, nagyon egyszerű szövegeket.
\item
  Követi a szintjének megfelelő, vizuális vagy nonverbális eszközökkel
  támogatott, ismert célnyelvi óravezetést, utasításokat.
\item
  Felismeri és használja a legegyszerűbb, mindennapi nyelvi funkciókat.
\item
  Elmondja magáról a legalapvetőbb információkat.
\item
  Ismeri az adott célnyelvi kultúrákhoz tartozó országok fontosabb
  jellemzőit és a hozzájuk tartozó alapvető nyelvi elemeket.
\item
  Törekszik a tanult nyelvi elemek megfelelő kiejtésére.
\item
  Célnyelvi tanulmányain keresztül nyitottabbá, a világ felé
  érdeklődőbbé válik.
\item
  Megismétli az élőszóban elhangzó egyszerű szavakat, kifejezéseket
  játékos, mozgást igénylő, kreatív nyelvórai tevékenységek során.
\item
  Lebetűzi a nevét.
\item
  Lebetűzi a tanult szavakat társaival közösen játékos tevékenységek
  kapcsán, szükség esetén segítséggel.
\item
  Célnyelven megoszt egyedül vagy társaival együttműködésben
  megszerzett, alapvető információkat szóban, akár vizuális elemekkel
  támogatva.
\item
  Felismeri az anyanyelvén, illetve a tanult idegen nyelven történő
  írásmód és betűkészlet közötti különbségeket.
\item
  Ismeri az adott nyelv ábécéjét.
\item
  Lemásol tanult szavakat játékos, alkotó nyelvórai tevékenységek során.
\item
  Megold játékos írásbeli feladatokat a szavak, szószerkezetek, rövid
  mondatok szintjén.
\item
  Részt vesz kooperatív munkaformában végzett kreatív tevékenységekben,
  projektmunkában szavak, szószerkezetek, rövid mondatok leírásával,
  esetleg képi kiegészítéssel.
\item
  Írásban megnevezi az ajánlott tématartományokban megjelölt,
  begyakorolt elemeket.
\item
  Megérti az élőszóban elhangzó, ismert témákhoz kapcsolódó, verbális,
  vizuális vagy nonverbális eszközökkel segített rövid kijelentéseket,
  kérdéseket.
\item
  Beazonosítja az életkorának megfelelő szituációkhoz kapcsolódó, rövid,
  egyszerű szövegben a tanult nyelvi elemeket.
\item
  Kiszűri a lényeget az ismert nyelvi elemeket tartalmazó, nagyon rövid,
  egyszerű hangzó szövegből.
\item
  Azonosítja a célzott információt a nyelvi szintjének és életkorának
  megfelelő rövid hangzó szövegben.
\item
  Támaszkodik az életkorának és nyelvi szintjének megfelelő hangzó
  szövegre az órai alkotó jellegű nyelvi, mozgásos nyelvi és játékos
  nyelvi tevékenységek során.
\item
  Felismeri az anyanyelv és az idegen nyelv hangkészletét.
\item
  Értelmezi azokat az idegen nyelven szóban elhangzó nyelvórai
  szituációkat, melyeket anyanyelvén már ismer.
\item
  Felismeri az anyanyelve és a célnyelv közötti legalapvetőbb
  kiejtésbeli különbségeket.
\item
  Figyel a célnyelvre jellemző hangok kiejtésére.
\item
  Megkülönbözteti az anyanyelvi és a célnyelvi írott szövegben a betű-
  és jelkészlet közti különbségeket.
\item
  Beazonosítja a célzott információt az életkorának megfelelő
  szituációkhoz kapcsolódó, rövid, egyszerű, a nyelvtanításhoz készült,
  illetve eredeti szövegben.
\item
  Csendes olvasás keretében feldolgozva megért ismert szavakat
  tartalmazó, pár szóból vagy mondatból álló, akár illusztrációval
  támogatott szöveget.
\item
  Megérti a nyelvi szintjének megfelelő, akár vizuális eszközökkel is
  támogatott írott utasításokat és kérdéseket, és ezekre megfelelő
  válaszreakciókat ad.
\item
  Kiemeli az ismert nyelvi elemeket tartalmazó, egyszerű, írott, pár
  mondatos szöveg fő mondanivalóját.
\item
  Támaszkodik az életkorának és nyelvi szintjének megfelelő írott
  szövegre az órai játékos alkotó, mozgásos vagy nyelvi fejlesztő
  tevékenységek során, kooperatív munkaformákban.
\item
  Megtapasztalja a közös célnyelvi olvasás élményét.
\item
  Aktívan bekapcsolódik a közös meseolvasásba, a mese tartalmát követi.
\item
  A tanórán begyakorolt, nagyon egyszerű, egyértelmű kommunikációs
  helyzetekben a megtanult, állandósult beszédfordulatok alkalmazásával
  kérdez vagy reagál, mondanivalóját segítséggel vagy nonverbális
  eszközökkel kifejezi.
\item
  Törekszik arra, hogy a célnyelvet eszközként alkalmazza
  információszerzésre.
\item
  Rövid, néhány mondatból álló párbeszédet folytat, felkészülést
  követően.
\item
  A tanórán bekapcsolódik a már ismert, szóbeli interakciót igénylő
  nyelvi tevékenységekbe, a begyakorolt nyelvi elemeket tanári
  segítséggel a tevékenység céljainak megfelelően alkalmazza.
\item
  Érzéseit egy-két szóval vagy begyakorolt állandósult nyelvi fordulatok
  segítségével kifejezi, főként rákérdezés alapján, nonverbális
  eszközökkel kísérve a célnyelvi megnyilatkozást.
\item
  Elsajátítja a tanult szavak és állandósult szókapcsolatok célnyelvi
  normához közelítő kiejtését tanári minta követése által, vagy
  autentikus hangzó anyag, digitális technológia segítségével.
\item
  Felismeri és alkalmazza a legegyszerűbb, üdvözlésre és elköszönésre
  használt mindennapi nyelvi funkciókat az életkorának és nyelvi
  szintjének megfelelő, egyszerű helyzetekben.
\item
  Felismeri és alkalmazza a legegyszerűbb, bemutatkozásra használt
  mindennapi nyelvi funkciókat az életkorának és nyelvi szintjének
  megfelelő, egyszerű helyzetekben.
\item
  Felismeri és használja a legegyszerűbb, megszólításra használt
  mindennapi nyelvi funkciókat az életkorának és nyelvi szintjének
  megfelelő, egyszerű helyzetekben.
\item
  Felismeri és használja a legegyszerűbb, a köszönet és az arra történő
  reagálás kifejezésére használt mindennapi nyelvi funkciókat az
  életkorának és nyelvi szintjének megfelelő, egyszerű helyzetekben.
\item
  Felismeri és használja a legegyszerűbb, a tudás és nem tudás
  kifejezésére használt mindennapi nyelvi funkciókat az életkorának és
  nyelvi szintjének megfelelő, egyszerű helyzetekben.
\item
  Felismeri és használja a legegyszerűbb, a nem értés, visszakérdezés és
  ismétlés, kérés kifejezésére használt mindennapi nyelvi funkciókat
  életkorának és nyelvi szintjének megfelelő, egyszerű helyzetekben.
\item
  Közöl alapvető személyes információkat magáról, egyszerű nyelvi elemek
  segítségével.
\item
  Új szavak, kifejezések tanulásakor ráismer a már korábban tanult
  szavakra, kifejezésekre.
\item
  Szavak, kifejezések tanulásakor felismeri, ha új elemmel találkozik és
  rákérdez, vagy megfelelő tanulási stratégiával törekszik a megértésre.
\item
  A célok eléréséhez társaival rövid feladatokban együttműködik.
\item
  Egy feladat megoldásának sikerességét segítséggel értékelni tudja.
\item
  Felismeri az idegen nyelvű írott, olvasott és hallott tartalmakat a
  tanórán kívül is.
\item
  Felhasznál és létrehoz rövid, nagyon egyszerű célnyelvi szövegeket
  szabadidős tevékenységek során.
\item
  Alapvető célzott információt megszerez a tanult témákban tudásának
  bővítésére.
\item
  Megismeri a főbb, az adott célnyelvi kultúrákhoz tartozó országok
  nevét, földrajzi elhelyezkedését, főbb országismereti jellemzőit.
\item
  Ismeri a főbb, célnyelvi kultúrához tartozó, ünnepekhez kapcsolódó
  alapszintű kifejezéseket, állandósult szókapcsolatokat és szokásokat.
\item
  Megérti a tanult nyelvi elemeket életkorának megfelelő digitális
  tartalmakban, digitális csatornákon olvasott vagy hallott nagyon
  egyszerű szövegekben is.
\item
  Létrehoz nagyon egyszerű írott, pár szavas szöveget szóban vagy
  írásban digitális felületen.
\item
  Szóban és írásban megold változatos kihívásokat igénylő feladatokat az
  élő idegen nyelven.
\item
  Szóban és írásban létrehoz rövid szövegeket, ismert nyelvi
  eszközökkel, a korának megfelelő szövegtípusokban.
\item
  Értelmez korának és nyelvi szintjének megfelelő hallott és írott
  célnyelvi szövegeket az ismert témákban és szövegtípusokban.
\item
  A tanult nyelvi elemek és kommunikációs stratégiák segítségével
  írásbeli és szóbeli interakciót folytat, valamint közvetít az élő
  idegen nyelven.
\item
  Kommunikációs szándékának megfelelően alkalmazza a tanult nyelvi
  funkciókat és a megszerzett szociolingvisztikai, pragmatikai és
  interkulturális jártasságát.
\item
  Nyelvtudását egyre inkább képes fejleszteni tanórán kívüli
  helyzetekben is különböző eszközökkel és lehetőségekkel.
\item
  Használ életkorának és nyelvi szintjének megfelelő hagyományos és
  digitális alapú nyelvtanulási forrásokat és eszközöket.
\item
  Alkalmazza nyelvtudását kommunikációra, közvetítésre, szórakozásra,
  ismeretszerzésre hagyományos és digitális csatornákon.
\item
  Törekszik a célnyelvi normához illeszkedő kiejtés, beszédtempó és
  intonáció megközelítésére.
\item
  Érti a nyelvtudás fontosságát, és motivációja a nyelvtanulásra tovább
  erősödik.
\item
  Aktívan részt vesz az életkorának és érdeklődésének megfelelő
  gyermek-, illetve ifjúsági irodalmi alkotások közös előadásában.
\item
  Egyre magabiztosabban kapcsolódik be történetek kreatív alakításába,
  átfogalmazásába kooperatív munkaformában.
\item
  Elmesél rövid történetet, egyszerűsített olvasmányt egyszerű nyelvi
  eszközökkel, önállóan, a cselekményt lineárisan összefűzve.
\item
  Egyszerű nyelvi eszközökkel, felkészülést követően röviden,
  összefüggően beszél az ajánlott tématartományokhoz tartozó témákban,
  élőszóban és digitális felületen.
\item
  Képet jellemez röviden, egyszerűen, ismert nyelvi fordulatok
  segítségével, segítő tanári kérdések alapján, önállóan.
\item
  Változatos, kognitív kihívást jelentő szóbeli feladatokat old meg
  önállóan vagy kooperatív munkaformában, a tanult nyelvi eszközökkel,
  szükség szerint tanári segítséggel, élőszóban és digitális felületen.
\item
  Megold játékos és változatos írásbeli feladatokat rövid szövegek
  szintjén.
\item
  Rövid, egyszerű, összefüggő szövegeket ír a tanult nyelvi szerkezetek
  felhasználásával az ismert szövegtípusokban, az ajánlott
  tématartományokban.
\item
  Rövid szövegek írását igénylő kreatív munkát hoz létre önállóan.
\item
  Rövid, összefüggő, papíralapú vagy ikt-eszközökkel segített írott
  projektmunkát készít önállóan vagy kooperatív munkaformákban.
\item
  A szövegek létrehozásához nyomtatott, illetve digitális alapú
  segédeszközt, szótárt használ.
\item
  Megérti a szintjének megfelelő, kevésbé ismert elemekből álló,
  nonverbális vagy vizuális eszközökkel támogatott célnyelvi óravezetést
  és utasításokat, kérdéseket.
\item
  Értelmezi az életkorának és nyelvi szintjének megfelelő, egyszerű,
  hangzó szövegben a tanult nyelvi elemeket.
\item
  Értelmezi az életkorának megfelelő, élőszóban vagy digitális felületen
  elhangzó szövegekben a beszélők gondolatmenetét.
\item
  Megérti a nem kizárólag ismert nyelvi elemeket tartalmazó, élőszóban
  vagy digitális felületen elhangzó rövid szöveg tartalmát.
\item
  Kiemel, kiszűr konkrét információkat a nyelvi szintjének megfelelő,
  élőszóban vagy digitális felületen elhangzó szövegből, és azokat
  összekapcsolja egyéb ismereteivel.
\item
  Alkalmazza az életkorának és nyelvi szintjének megfelelő hangzó
  szöveget a változatos nyelvórai tevékenységek és a feladatmegoldás
  során.
\item
  Értelmez életkorának megfelelő nyelvi helyzeteket hallott szöveg
  alapján.
\item
  Felismeri a főbb, életkorának megfelelő hangzószöveg-típusokat.
\item
  Hallgat az érdeklődésének megfelelő autentikus szövegeket
  elektronikus, digitális csatornákon, tanórán kívül is, szórakozásra
  vagy ismeretszerzésre.
\item
  Értelmezi az életkorának megfelelő szituációkhoz kapcsolódó, írott
  szövegekben megjelenő összetettebb információkat.
\item
  Megérti a nem kizárólag ismert nyelvi elemeket tartalmazó rövid írott
  szöveg tartalmát.
\item
  Kiemel, kiszűr konkrét információkat a nyelvi szintjének megfelelő
  szövegből, és azokat összekapcsolja más iskolai vagy iskolán kívül
  szerzett ismereteivel.
\item
  Megkülönbözteti a főbb, életkorának megfelelő írott szövegtípusokat.
\item
  Összetettebb írott instrukciókat értelmez.
\item
  Alkalmazza az életkorának és nyelvi szintjének megfelelő írott,
  nyomtatott vagy digitális alapú szöveget a változatos nyelvórai
  tevékenységek és feladatmegoldás során.
\item
  A nyomtatott vagy digitális alapú írott szöveget felhasználja
  szórakozásra és ismeretszerzésre önállóan is.
\item
  Érdeklődése erősödik a célnyelvi irodalmi alkotások iránt.
\item
  Megért és használ szavakat, szókapcsolatokat a célnyelvi, az
  életkorának és érdeklődésének megfelelő hazai és nemzetközi legfőbb
  hírekkel, eseményekkel kapcsolatban
\item
  Kommunikációt kezdeményez egyszerű hétköznapi témában, a beszélgetést
  követi, egyszerű, nyelvi eszközökkel fenntartja és lezárja.
\item
  Az életkorának megfelelő mindennapi helyzetekben a tanult nyelvi
  eszközökkel megfogalmazott kérdéseket tesz fel, és válaszol a hozzá
  intézett kérdésekre.
\item
  Véleményét, gondolatait, érzéseit egyre magabiztosabban fejezi ki a
  tanult nyelvi eszközökkel.
\item
  A tanult nyelvi elemeket többnyire megfelelően használja,
  beszédszándékainak megfelelően, egyszerű spontán helyzetekben.
\item
  Váratlan, előre nem kiszámítható eseményekre, jelenségekre és
  történésekre is reagál egyszerű célnyelvi eszközökkel, személyes vagy
  online interakciókban.
\item
  Bekapcsolódik a tanórán az interakciót igénylő nyelvi tevékenységekbe,
  abban társaival közösen részt vesz, a begyakorolt nyelvi elemeket
  tanári segítséggel a játék céljainak megfelelően alkalmazza.
\item
  Üzeneteket ír,
\item
  Véleményét írásban, egyszerű nyelvi eszközökkel megfogalmazza, és
  arról írásban interakciót folytat.
\item
  Rövid, egyszerű, ismert nyelvi eszközökből álló kiselőadást tart
  változatos feladatok kapcsán, hagyományos vagy digitális alapú
  vizuális eszközök támogatásával.
\item
  Felhasználja a célnyelvet tudásmegosztásra.
\item
  Találkozik az életkorának és nyelvi szintjének megfelelő célnyelvi
  ismeretterjesztő tartalmakkal.
\item
  Néhány szóból vagy mondatból álló jegyzetet készít írott szöveg
  alapján.
\item
  Egyszerűen megfogalmazza személyes véleményét, másoktól véleményük
  kifejtését kéri, és arra reagál, elismeri vagy cáfolja mások
  állítását, kifejezi egyetértését vagy egyet nem értését.
\item
  Kifejez tetszést, nem tetszést, akaratot, kívánságot, tudást és nem
  tudást, ígéretet, szándékot, dicséretet, kritikát.
\item
  Információt cserél, információt kér, információt ad.
\item
  Kifejez kérést, javaslatot, meghívást, kínálást és ezekre reagálást.
\item
  Kifejez alapvető érzéseket, például örömöt, sajnálkozást, bánatot,
  elégedettséget, elégedetlenséget, bosszúságot, csodálkozást, reményt.
\item
  Kifejez és érvekkel alátámasztva mutat be szükségességet, lehetőséget,
  képességet, bizonyosságot, bizonytalanságot.
\item
  Értelmez és használja az idegen nyelvű írott, olvasott és hallott
  tartalmakat a tanórán kívül is,
\item
  Felhasználja a célnyelvet ismeretszerzésre.
\item
  Használja a célnyelvet életkorának és nyelvi szintjének megfelelő
  aktuális témákban és a hozzájuk tartozó szituációkban.
\item
  Találkozik életkorának és nyelvi szintjének megfelelő célnyelvi
  szórakoztató tartalmakkal.
\item
  Összekapcsolja az ismert nyelvi elemeket egyszerű kötőszavakkal
  (például: és, de, vagy).
\item
  Egyszerű mondatokat összekapcsolva mond el egymást követő eseményekből
  álló történetet, vagy leírást ad valamilyen témáról.
\item
  A tanult nyelvi eszközökkel és nonverbális elemek segítségével
  tisztázza mondanivalójának lényegét.
\item
  Ismeretlen szavak valószínű jelentését szövegösszefüggések alapján
  kikövetkezteti az életkorának és érdeklődésének megfelelő, konkrét,
  rövid szövegekben.
\item
  Alkalmaz nyelvi funkciókat rövid társalgás megkezdéséhez,
  fenntartásához és befejezéséhez.
\item
  Nem értés esetén a meg nem értett kulcsszavak vagy fordulatok
  ismétlését vagy magyarázatát kéri, visszakérdez, betűzést kér.
\item
  Megoszt alapvető személyes információkat és szükségleteket magáról
  egyszerű nyelvi elemekkel.
\item
  Ismerős és gyakori alapvető helyzetekben, akár telefonon vagy
  digitális csatornákon is, többnyire helyesen és érthetően fejezi ki
  magát az ismert nyelvi eszközök segítségével.
\item
  Tudatosan használ alapszintű nyelvtanulási és nyelvhasználati
  stratégiákat.
\item
  Hibáit többnyire észreveszi és javítja.
\item
  Ismer szavakat, szókapcsolatokat a célnyelven a témakörre jellemző,
  életkorának és érdeklődésének megfelelő más tudásterületen megcélzott
  tartalmakból.
\item
  Egy összetettebb nyelvi feladat, projekt végéig tartó célokat tűz ki
  magának.
\item
  Céljai eléréséhez megtalálja és használja a megfelelő eszközöket.
\item
  Céljai eléréséhez társaival párban és csoportban együttműködik.
\item
  Nyelvi haladását többnyire fel tudja mérni,
\item
  Társai haladásának értékelésében segítően részt vesz.
\item
  A tanórán kívüli, akár játékos nyelvtanulási lehetőségeket felismeri,
  és törekszik azokat kihasználni.
\item
  Felhasználja a célnyelvet szórakozásra és játékos nyelvtanulásra.
\item
  Digitális eszközöket és felületeket is használ nyelvtudása
  fejlesztésére,
\item
  Értelmez egyszerű, szórakoztató kisfilmeket
\item
  Megismeri a célnyelvi országok főbb jellemzőit és kulturális
  sajátosságait.
\item
  További országismereti tudásra tesz szert.
\item
  Célnyelvi kommunikációjába beépíti a tanult interkulturális
  ismereteket.
\item
  Találkozik célnyelvi országismereti tartalmakkal.
\item
  Találkozik a célnyelvi, életkorának és érdeklődésének megfelelő hazai
  és nemzetközi legfőbb hírekkel, eseményekkel.
\item
  Megismerkedik hazánk legfőbb országismereti és történelmi eseményeivel
  célnyelven.
\item
  A célnyelvi kultúrákhoz kapcsolódó alapvető tanult nyelvi elemeket
  használja.
\item
  Idegen nyelvi kommunikációjában ismeri és használja a célnyelv főbb
  jellemzőit.
\item
  Következetesen alkalmazza a célnyelvi betű és jelkészletet
\item
  Egyénileg vagy társaival együttműködve szóban vagy írásban
  projektmunkát vagy kiselőadást készít, és ezeket digitális eszközök
  segítségével is meg tudja valósítani.
\item
  Találkozik az érdeklődésének megfelelő akár autentikus szövegekkel
  elektronikus, digitális csatornákon tanórán kívül is.
\end{itemize}

\hypertarget{penzugyi-es-vallalkozoi-ismeretek}{%
\subsection{Pénzügyi és vállalkozói
ismeretek}\label{penzugyi-es-vallalkozoi-ismeretek}}

\hypertarget{evfolyamon-7}{%
\subsubsection{10. évfolyamon}\label{evfolyamon-7}}

\begin{itemize}
\tightlist
\item
  A tanuló érti a nemzetgazdaság szereplőinek (háztartások, vállalatok,
  állam, pénzintézetek) feladatait, a köztük lévő kapcsolatrendszer
  sajátosságait.
\item
  Tudja értelmezni az állam gazdasági szerepvállalásának jelentőségét,
  ismeri főbb feladatait, azok hatásait.
\item
  Tisztában van azzal, hogy az adófizetés biztosítja részben az állami
  feladatok ellátásnak pénzügyi fedezetét.
\item
  Ismeri a mai bankrendszer felépítését, az egyes pénzpiaci szereplők
  főbb feladatait.
\item
  Képes választani az egyes banki lehetőségek közül.
\item
  Tisztában van az egyes banki ügyletek előnyeivel, hátrányaival,
  kockázataival.
\item
  A bankok kínálatából bankot, bankszámla csomagot tud választani.
\item
  Tud érvelni a családi költségvetés mellett, a tudatos, hatékony
  pénzgazdálkodás érdekében.
\item
  Önismereti tesztek, játékok segítségével képes átgondolni milyen
  foglalkozások, tevékenységek illeszkednek személyiségéhez.
\item
  Tisztában van az álláskeresés folyamatával, a munkaviszonnyal
  kapcsolatos jogaival, kötelezettségeivel.
\item
  Ismer vállalkozókat, vállalatokat, össze tudja hasonlítani az
  alkalmazotti, és a vállalkozói személyiségjegyeket.
\item
  Érti a leggyakoribb vállalkozási formák jellemzőit, előnyeit,
  hátrányait.
\item
  Tisztában van a nem nyereségérdekelt szervezetek gazdaságban betöltött
  szerepével.
\item
  Ismeri a vállalkozásalapítás, -működtetés legfontosabb lépéseit, képes
  önálló vállalkozói ötlet kidolgozására.
\item
  Meg tudja becsülni egy vállalkozás lehetséges költségeit, képes adott
  időtartamra költségkalkulációt tervezni.
\item
  Tisztában van az üzleti tervezés szükségességével, mind egy új
  vállalkozás alapításakor, mind már meglévő vállalkozás működése
  esetén.
\item
  Tájékozott az üzleti terv tartalmi elemeiről.
\item
  Megismeri a nem üzleti (társadalmi, kulturális, egyéb civil)
  kezdeményezések pénzügyi-gazdasági igényeit, lehetőségeit.
\item
  Felismeri a kezdeményezőkészség jelentőségét az állampolgári
  felelősségvállalásban.
\item
  Felismeri a sikeres vállalkozás jellemzőit, képes azonosítani az
  esetleges kudarc okait, javaslatot tud tenni a problémák megoldására.
\end{itemize}

\hypertarget{szoftverfejlesztes-es--teszteles}{%
\subsection{Szoftverfejlesztés és
-tesztelés}\label{szoftverfejlesztes-es--teszteles}}

\hypertarget{evfolyamon-8}{%
\subsubsection{9-10. évfolyamon}\label{evfolyamon-8}}

\begin{itemize}
\tightlist
\item
  Adott kapcsolási rajz alapján egyszerűbb áramköröket épít próbapanel
  segítségével vagy forrasztásos technológiával.
\item
  Ismeri az elektronikai alapfogalmakat, kapcsolódó fizikai törvényeket,
  alapvető alkatrészeket és kapcsolásokat.
\item
  A funkcionalitás biztosítása mellett törekszik az esztétikus
  kialakításra (pl. minőségi forrasztás, egyenletes alkatrész sűrűség,
  olvashatóság).
\item
  Az elektromos berendezésekre vonatkozó munka- és balesetvédelmi
  szabályokat a saját és mások testi épsége érdekében betartja és
  betartatja.
\item
  Alapvető villamos méréseket végez önállóan a megépített áramkörökön.
\item
  Ismeri az elektromos mennyiségek mérési metódusait, a mérőműszerek
  használatát.
\item
  Elvégzi a számítógépen és a mobil eszközökön az operációs rendszer
  (pl. Windows, Linux, Android, iOS), valamint az alkalmazói szoftverek
  telepítését, frissítését és alapszintű beállítását. Grafikus
  felületen, valamint parancssorban használja a Windows, és Linux
  operációs rendszerek alapszintű parancsait és szolgáltatásait (pl.
  állomány- és könyvtárkezelési műveletek, jogosultságok beállítása,
  szövegfájlokkal végzett műveletek, folyamatok kezelése).
\item
  Ismeri a számítógépen és a mobil informatikai eszközökön használt
  operációs rendszerek telepítési és frissítési módjait, alapvető
  parancsait és szolgáltatásait, valamint alapvető beállítási
  lehetőségeit.
\item
  Törekszik a felhasználói igényekhez alkalmazkodó szoftverkörnyezet
  kialakítására.
\item
  Önállóan elvégzi a kívánt szoftverek telepítését, szükség esetén
  gondoskodik az eszközön korábban tárolt adatok biztonsági mentéséről.
\item
  Elvégzi a PC perifériáinak csatlakoztatását, szükség esetén új
  alkatrészt szerel be vagy alkatrészt cserél egy számítógépben.
\item
  Ismeri az otthoni és irodai informatikai környezetet alkotó
  legáltalánosabb összetevők (PC, nyomtató, mobiltelefon, WiFi router
  stb.) szerepét, alapvető működési módjukat. Ismeri a PC és a mobil
  eszközök főbb alkatrészeit (pl. alaplap, CPU, memória) és azok
  szerepét.
\item
  Törekszik a végrehajtandó műveletek precíz és előírásoknak megfelelő
  elvégzésére.
\item
  Az informatikai berendezésekre vonatkozó munka- és balesetvédelmi
  szabályokat a saját és mások testi épsége érdekében betartja és
  betartatja.
\item
  Alapvető karbantartási feladatokat lát el az általa megismert
  informatikai és távközlési berendezéseken (pl. szellőzés és
  csatlakozások ellenőrzése, tisztítása).
\item
  Tisztában van vele, hogy miért szükséges az informatikai és távközlési
  eszközök rendszeres és eseti karbantartása. Ismeri legalapvetőbb
  karbantartási eljárásokat.
\item
  A hibamentes folyamatos működés elérése érdekében fontosnak tartja a
  megelőző karbantartások elvégzését.
\item
  Otthoni vagy irodai hálózatot alakít ki WiFi router segítségével,
  elvégzi WiFi router konfigurálását, a vezetékes- és vezeték nélküli
  eszközök (PC, mobiltelefon, set-top box stb.), csatlakoztatását és
  hálózati beállítását.
\item
  Ismeri az informatikai hálózatok felépítését, alapvető technológiáit
  (pl. Ethernet), protokolljait (pl. IP, HTTP) és szabványait (pl.
  802.11-es WiFi szabványok). Ismeri az otthoni és irodai hálózatok
  legfontosabb összetevőinek (kábelezés, WiFi router, PC, mobiltelefon
  stb.) szerepét, jellemzőit, csatlakozási módjukat és alapszintű
  hálózati beállításait.
\item
  Törekszik a felhasználói igények megismerésére, megértésére, és szem
  előtt tartja azokat a hálózat kialakításakor.
\item
  Néhány alhálózatból álló kis- és közepes vállalati hálózatot alakít ki
  forgalomirányító és kapcsoló segítségével, elvégzi az eszközök
  alapszintű hálózati beállításait (pl. forgalomirányító interfészeinek
  IP-cím beállítása, alapértelmezett átjáró beállítása).
\item
  Ismeri a kis- és közepes vállalati hálózatok legfontosabb
  összetevőinek (pl. kábelrendező szekrény, kapcsoló, forgalomirányító)
  szerepét, jellemzőit, csatlakozási módjukat és alapszintű hálózati
  beállításait.
\item
  Alkalmazza a hálózatbiztonsággal kapcsolatos legfontosabb irányelveket
  (pl. erős jelszavak használata, vírusvédelem alkalmazása, tűzfal
  használat).
\item
  Ismeri a fontosabb hálózatbiztonsági elveket, szabályokat, támadás
  típusokat, valamint a szoftveres és hardveres védekezési módszereket.
\item
  Megkeresi és elhárítja az otthoni és kisvállalati informatikai
  környezetben jelentkező hardveres és szoftveres hibákat.
\item
  Ismeri az otthoni és kisvállalati informatikai környezetben
  leggyakrabban felmerülő hibákat (pl. hibás IP-beállítás, kilazult
  csatlakozó) és azok elhárításának módjait.
\item
  Önállóan behatárolja a hibát. Egyszerűbb problémákat önállóan,
  összetettebbeket szakmai irányítással hárít el.
\item
  Internetes források és tudásbázisok segítségével követi, valamint
  feladatainak elvégzéséhez lehetőség szerint alkalmazza a legmodernebb
  információs technológiákat és trendeket (virtualizáció,
  felhőtechnológia, IoT, mesterséges intelligencia, gépi tanulás stb.).
\item
  Naprakész információkkal rendelkezik a legmodernebb információs
  technológiákkal és trendekkel kapcsolatban.
\item
  Nyitott és érdeklődő a legmodernebb információs technológiák és
  trendek iránt.
\item
  Önállóan szerez információkat a témában releváns szakmai
  platformokról.
\item
  Szabványos, reszponzív megjelenítést biztosító weblapokat hoz létre és
  formáz meg stíluslapok segítségével.
\item
  Ismeri a HTML5, a CSS3 alapvető elemeit, a stíluslapok fogalmát,
  felépítését. Érti a reszponzív megjelenítéshez használt módszereket,
  keretrendszerek előnyeit, a reszponzív webdizájn alapelveit.
\item
  A felhasználói igényeknek megfelelő funkcionalitás és design
  összhangjára törekszik.
\item
  Önállóan létrehozza és megformázza a weboldalt.
\item
  Munkája során jelentkező problémák kezelésére vagy folyamatok
  automatizálására egyszerű alkalmazásokat készít Python programozási
  nyelv segítségével.
\item
  Ismeri a Python nyelv elemeit, azok céljait (vezérlési szerkezetek,
  adatszerkezetek, változók, aritmetikai és logikai kifejezések,
  függvények, modulok, csomagok). Ismeri az algoritmus fogalmát, annak
  szerepét.
\item
  Jól átlátható kódszerkezet kialakítására törekszik.
\item
  Önállóan készít egyszerű alkalmazásokat.
\item
  Git verziókezelő rendszert, valamint fejlesztést és csoportmunkát
  támogató online eszközöket és szolgáltatásokat (pl.: GitHub, Slack,
  Trello, Microsoft Teams, Webex Teams) használ.
\item
  Ismeri a Git, valamint a csoportmunkát támogató eszközök és online
  szolgáltatások célját, működési módját, legfontosabb funkcióit.
\item
  Törekszik a feladatainak megoldásában a hatékony csoportmunkát
  támogató online eszközöket kihasználni.
\item
  A Git verziókezelőt, valamint a csoportmunkát támogató eszközöket és
  szolgáltatásokat önállóan használja.
\item
  Társaival hatékonyan együttműködve, csapatban dolgozik egy
  informatikai projekten. A projektek végrehajtása során társaival
  tudatosan és célirányosan kommunikál.
\item
  Ismeri a projektmenedzsment lépéseit (kezdeményezés, követés,
  végrehajtás, ellenőrzés, dokumentáció, zárás).
\item
  Más munkáját és a csoport belső szabályait tiszteletben tartva,
  együttműködően vesz részt a csapatmunkában.
\item
  A projektekben irányítás alatt, társaival közösen dolgozik. A
  ráosztott feladatrészt önállóan végzi el.
\item
  Munkája során hatékonyan használja az irodai szoftvereket.
\item
  Ismeri az irodai szoftverek főbb funkcióit, felhasználási területeit.
\item
  Az elkészült termékhez prezentációt készít és bemutatja, előadja azt
  munkatársainak, vezetőinek, ügyfeleinek.
\item
  Ismeri a hatékony prezentálás szabályait, a prezentációs szoftverek
  lehetőségeit.
\item
  Törekszik a tömör, lényegre törő, de szakszerű bemutató
  összeállítására.
\item
  A projektcsapat tagjaival egyeztetve, de önállóan elkészíti az
  elvégzett munka eredményét bemutató prezentációt.
\end{itemize}

\hypertarget{evfolyamon-9}{%
\subsubsection{11-13. évfolyamon}\label{evfolyamon-9}}

\begin{itemize}
\tightlist
\item
  Használja a Git verziókezelő rendszert, valamint a fejlesztést
  támogató csoportmunkaeszközöket és szolgáltatásokat (pl. GitHub,
  Slack, Trello, Microsoft Teams, Webex Teams).
\item
  Ismeri a legelterjedtebb csoportmunkaeszközöket, valamint a Git
  verziókezelőrendszer szolgáltatásait.
\item
  Igyekszik munkatársaival hatékonyan, igazi csapatjátékosként együtt
  dolgozni. Törekszik a csoporton belül megkapott feladatok precíz,
  határidőre történő elkészítésére, társai segítésére.
\item
  Szoftverfejlesztési projektekben irányítás alatt dolgozik, a rábízott
  részfeladatok megvalósításáért felelősséget vállal.
\item
  Az általa végzett szoftverfejlesztési feladatok esetében kiválasztja a
  legmegfelelőbb technikákat, eljárásokat és módszereket.
\item
  Elegendő ismerettel rendelkezik a meghatározó szoftverfejlesztési
  technológiák (programozási nyelvek, keretrendszerek, könyvtárak stb.),
  illetve módszerek erősségeiről és hátrányairól.
\item
  Nyitott az új technológiák megismerésére, tudását folyamatosan
  fejleszti.
\item
  Önállóan dönt a fejlesztés során használt technológiákról és
  eszközökről.
\item
  A megfelelő kommunikációs forma (e-mail, chat, telefon, prezentáció
  stb.) kiválasztásával munkatársaival és az ügyfelekkel hatékonyan
  kommunikál műszaki és egyéb információkról magyarul és angolul.
\item
  Ismeri a különböző kommunikációs formákra (e-mail, chat, telefon,
  prezentáció stb.) vonatkozó etikai és belső kommunikációs szabályokat.
\item
  Angol nyelvismerettel rendelkezik (KER B1 szint). Ismeri a gyakran
  használt szakmai kifejezéseket angolul.
\item
  Kommunikációjában konstruktív, együttműködő, udvarias. Feladatainak a
  felhasználói igényeknek leginkább megfelelő, minőségi megoldására
  törekszik.
\item
  Felelősségi körébe tartozó feladatokkal kapcsolatban a vállalati
  kommunikációs szabályokat betartva, önállóan kommunikál az ügyfelekkel
  és munkatársaival.
\item
  Szabványos, reszponzív megjelenítést biztosító weblapokat hoz létre és
  formáz meg stíluslapok segítségével. Kereső optimalizálási
  beállításokat alkalmaz.
\item
  Ismeri a HTML5 és a CSS3 szabvány alapvető nyelvi elemeit és eszközeit
  (strukturális és szemantikus HTML-elemek, attribútumok, listák,
  táblázatok, stílus jellemzők és függvények). Ismeri a a reszponzív
  webdizájn alapelveit és a Bootstrap keretrendszer alapvető
  szolgáltatásait.
\item
  Törekszik a weblapok igényes és a használatot megkönnyítő
  kialakítására.
\item
  Kisebb webfejlesztési projekteken önállóan, összetettebbekben
  részfeladatokat megvalósítva, irányítás mellett dolgozik.
\item
  Egyszerűbb webhelyek dinamikus viselkedését (eseménykezelés, animáció
  stb.) biztosító kódot, készít JavaScript nyelven.
\item
  Alkalmazási szinten ismeri a JavaScript alapvető nyelvi elemeit,
  valamint az aszinkron programozás és az AJAX technológia működési
  elvét. Tisztában van a legfrissebb ECMAScript változatok (ES6 vagy
  újabb) hatékonyság növelő funkcióival.
\item
  Egyszerűbb JavaScript programozási feladatokat önállóan végez el.
\item
  RESTful alkalmazás kliens oldali komponensének fejlesztését végzi
  JavaScript nyelven.
\item
  Tisztában van a REST szoftverarchitektúra elvével, alkalmazás szintjén
  ismeri az AJAX technológiát.
\item
  A tiszta kód elveinek megfelelő, megfelelő mennyiségű megjegyzéssel
  ellátott, kellőképpen tagolt, jól átlátható, kódot készít.
\item
  Ismeri a tiszta kód készítésének alapelveit.
\item
  Törekszik arra, hogy az elkészített kódja jól átlátható, és mások
  számára is értelmezhető legyen.
\item
  Adatbázis-kezelést is végző konzolos vagy grafikus felületű asztali
  alkalmazást készít magas szintű programozási nyelvet (C\#, Java)
  használva.
\item
  Ismeri a választott magas szintű programozási nyelv alapvető nyelvi
  elemeit, illetve a hozzá tartozó fejlesztési környezetet.
\item
  Törekszik a felhasználó számára minél könnyebb használatot biztosító
  felhasználói felület és működési mód kialakítására.
\item
  Kisebb asztali alkalmazás-fejlesztési projekteken önállóan,
  összetettebbekben részfeladatokat megvalósítva, irányítás mellett
  dolgozik.
\item
  Adatkezelő alkalmazásokhoz relációs adatbázist tervez és hoz létre,
  többtáblás lekérdezéseket készít.
\item
  Tisztában van a relációs adatbázis-tervezés és -kezelés alapelveivel.
  Haladó szinten ismeri a különböző típusú SQL lekérdezéseket, azok
  nyelvi elemeit és lehetőségeit.
\item
  Törekszik a redundanciamentes, világos szerkezetű, legcélravezetőbb
  kialakítású adatbázis szerkezet megvalósítására.
\item
  Kisebb projektekhez néhány táblás adatbázist önállóan tervez meg,
  nagyobb projektekben a biztosított adatbáziskörnyezetet használva
  önállóan valósít meg lekérdezéseket.
\item
  Önálló- vagy komplex szoftverrendszerek részét képző kliens oldali
  alkalmazásokat fejleszt mobil eszközökre.
\item
  Ismeri a választott mobil alkalmazás fejlesztésére alkalmas nyelvet és
  fejlesztői környezetet. Tisztában van a mobil alkalmazásfejlesztés
  alapelveivel.
\item
  Törekszik a felhasználó számára minél könnyebb használatot biztosító
  felhasználói felület és működési mód kialakítására.
\item
  Kisebb projektek mobil eszközökre optimalizált kliens oldali
  alkalmazását önállóan megvalósítja meg.
\item
  Webes környezetben futtatható kliens oldali (frontend) alkalmazást
  készít JavaScript keretrendszer (pl. React, Vue, Angular)
  segítségével.
\item
  Érti a frontend fejlesztésre szolgáló JavaScript keretrendszerek
  célját. Meg tudja nevezni a 3-4 legelterjedtebb keretrendszert.
  Alkalmazás szintjén ismeri a könyvtárak és modulok kezelését végző
  csomagkezelő rendszereket (package manager, pl. npm, yarn). Ismeri a
  választott JavaScript keretrendszer működési elvét, nyelvi és
  strukturális elemeit.
\item
  Törekszik maximálisan kihasználni a választott keretrendszer előnyeit,
  követi az ajánlott fejlesztési mintákat.
\item
  Kisebb frontend alkalmazásokat önállóan készít el, nagyobb
  projektekben irányítás mellett végzi el a kijelölt komponensek
  fejlesztését.
\item
  RESTful alkalmazás adatbázis-kezelési feladatokat is ellátó
  szerveroldali komponensének (backend) fejlesztését végzi erre alkalmas
  nyelv vagy keretrendszer segítségével (pl. Node.js, Spring, Laravel).
\item
  Érti a RESTful szoftverarchitektúra lényegét. Tisztában van legalább
  egy backend készítésére szolgáló nyelv vagy keretrendszer működési
  módjával, nyelvi és strukturális elemeivel. Alkalmazás szintjén ismeri
  az objektum-relációs leképzés technológiát (ORM).
\item
  Igyekszik backend működését leíró precíz, a frontend fejlesztők
  számára könnyen értelmezhető dokumentáció készítésére.
\item
  Kisebb backend alkalmazásokat önállóan készít el, nagyobb projektekben
  részletes specifikációt követve, irányítás mellett végzi el a kijelölt
  komponensek fejlesztését.
\item
  Objektum orientált (OOP) programozási módszertant alkalmazó asztali,
  webes és mobil alkalmazást készít.
\item
  Ismeri az objektumorientált programozás elvét, tisztában van az
  öröklődés, a polimorfizmus, a metódus/konstruktor túlterhelés
  fogalmával.
\item
  Törekszik az OOP technológia nyújtotta előnyök kihasználására,
  valamint igyekszik követni az OOP irányelveket és ajánlásokat.
\item
  Kisebb projektekben önállóan tervezi meg a szükséges osztályokat,
  nagyobb projektekben irányítás mellett, a projektben a projektcsapat
  által létrehozott osztálystruktúrát használva, illetve azt kiegészítve
  végzi a fejlesztést.
\item
  Tartalomkezelő rendszer (CMS, pl. WordPress) segítségével webhelyet
  készít, egyéni problémák megoldására saját beépülőket hoz létre.
\item
  Ismeri a tartalomkezelő-rendszerek célját és alapvető szolgáltatásait.
  Ismeri a beépülők célját és alkalmazási területeit.
\item
  Törekszik az igényes kialakítású és a felhasználók számára könnyű
  használatot biztosító webhelyek kialakításra.
\item
  Kevésbé összetett portálokat igényes vizuális megjelenést biztosító
  sablonok, valamint magas funkcionalitást biztosító beépülők
  használatával önállóan valósít meg. Összetettebb projekteken irányítás
  mellett, grafikus tervezőkkel, UX szakemberekkel és más fejlesztőkkel
  együttműködve dolgozik.
\item
  Manuális és automatizált szoftvertesztelést végezve ellenőrzi a
  szoftver hibátlan működését, dokumentálja a tesztek eredményét.
\item
  Ismeri a unit tesztelés, valamint más tesztelési, hibakeresési
  technikák alapelveit és alapvető eszközeit.
\item
  Törekszik a mindenre kiterjedő, az összes lehetséges hibát felderítő
  tesztelésre, valamint a tesztek körültekintő dokumentálására.
\item
  Saját fejlesztésként megvalósított kisebb projektekben önállóan végzi
  a tesztelést, tesztelői szerepben nagyobb projektekben irányítás
  mellett végez meghatározott tesztelési feladatokat.
\item
  Szoftverfejlesztés vagy -tesztelés során felmerülő problémákat old meg
  és hibákat hárít el webes kereséssel és internetes tudásbázisok
  használatával (pl. Stack Overflow).
\item
  Ismeri a hibakeresés szisztematikus módszereit, a problémák
  elhárításának lépéseit.
\item
  Ismeri a munkájához kapcsolódó internetes keresési módszereket és
  tudásbázisokat.
\item
  Törekszik a hibák elhárítására, megoldására, és arra, hogy azokkal
  lehetőség szerint ne okozzon újabb hibákat.
\item
  Internetes információszerzéssel önállóan old meg problémákat és hárít
  el hibákat.
\item
  Munkája során hatékonyan használja az irodai szoftvereket, műszaki
  tartalmú dokumentumokat és bemutatókat készít.
\item
  Ismeri az irodai szoftverek haladó szintű szolgáltatásait.
\item
  Precízen készíti el a műszaki tartalmú dokumentációkat,
  prezentációkat. Törekszik arra, hogy a dokumentumok könnyen
  értelmezhetők és mások által is szerkeszthetők legyenek.
\item
  Felelősséget vállal az általa készített műszaki tartalmú
  dokumentációkért.
\item
  Munkája során cél szerint alkalmazza a legmodernebb információs
  technológiákat és trendeket (virtaulizáció, felhőtechnológia, IoT,
  mesterséges intelligencia, gépi tanulás stb.).
\item
  Alapszintű alkalmazási szinten ismeri a legmodernebb információs
  technológiákat és trendeket (virtualizáció, felhőtechnológia, IoT,
  mesterséges intelligencia, gépi tanulás stb.).
\item
  Nyitott az új technológiák megismerésére, és törekszik azok hatékony,
  a felhasználói igényeknek és a költséghatékonysági elvárásoknak
  megfelelő felhasználására a szoftverfejlesztési feladatokban.
\item
  Részt vesz szoftverrendszerek ügyfeleknél történő bevezetésében, a
  működési környezetet biztosító IT-környezet telepítésében és
  beállításában.
\item
  Ismeri a számítógép és a mobil informatikai eszközök felépítését (főbb
  komponenseket, azok feladatait) és működését. Ismeri az eszközök
  operációs rendszerének és alkalmazói szoftvereinek telepítési és
  beállítási lehetőségeit.
\item
  A szoftverrendszerek bevezetése és a működési környezet kialakítása
  során törekszik az ügyfelek elvárásainak megfelelni, valamint
  tiszteletben tartja az ügyfél vállalati szabályait.
\item
  Az elvégzett eszköz- és szoftvertelepítésekért felelősséget vállal.
\item
  A szoftverfejlesztés és tesztelési munkakörnyezetének kialakításához
  beállítja a hálózati eszközöket, elvégzi a vezetékes és vezetéknélküli
  eszközök csatlakoztatását és hálózatbiztonsági beállítását. A
  fejlesztett szoftverben biztonságos, HTTPS protokollt használó webes
  kommunikációt valósít meg.
\item
  Ismeri az IPv4 és IPv6 címzési rendszerét és a legalapvetőbb hálózati
  protokollok szerepét és működési módját (IP, TCP, UDP, DHCP, HTTP,
  HTTPS, telnet, ssh, SMTP, POP3, IMAP4, DNS, TLS/SSL stb.). Ismeri a
  végponti berendezések IP-beállítási és hibaelhárítási lehetőségeit.
  Ismeri az otthoni és kisvállalati hálózatokban működő multifunkciós
  forgalomirányítók szolgáltatásait, azok beállításának módszereit.
\end{itemize}

\hypertarget{tortenelem}{%
\subsection{Történelem}\label{tortenelem}}

\hypertarget{evfolyamon-10}{%
\subsubsection{9-12. évfolyamon}\label{evfolyamon-10}}

\begin{itemize}
\item
  Megbízható ismeretekkel bír az európai, valamint az egyetemes
  történelem és mélyebb tudással rendelkezik a magyar történelem
  fontosabb eseményeiről, történelmi folyamatairól, fordulópontjairól.
\item
  Képes a múlt és jelen társadalmi, gazdasági, politikai és kulturális
  folyamatairól, jelenségeiről többszempontú, tárgyilagos érveléssel
  alátámasztott véleményt alkotni, ezekkel kapcsolatos problémákat
  megfogalmazni.
\item
  Ismeri a közös magyar nemzeti és európai, valamint az egyetemes emberi
  civilizáció kulturális örökségének, kódrendszerének lényeges elemeit.
\item
  Különbséget tud tenni történelmi tények és történelmi interpretáció,
  illetve vélemény között.
\item
  Képes következtetni történelmi események, folyamatok és jelenségek
  okaira és következményeire.
\item
  Képes a tanulási célhoz megfelelő információforrást választani, a
  források között szelektálni, azokat szakszerűen feldolgozni és
  értelmezni.
\item
  Kialakul a hiteles és tárgyilagos forráshasználat és kritika igénye.
\item
  Képes a múlt eseményeit és jelenségeit a saját történelmi
  összefüggésükben értelmezni, illetve a jelen viszonyait kapcsolatba
  hozni a múltban történtekkel.
\item
  Ismeri a demokratikus államszervezet működését, a társadalmi
  együttműködés szabályait, a piacgazdaság alapelveit; autonóm és
  felelős állampolgárként viselkedik.
\item
  Kialakul és megerősödik a történelmi múlt, illetve a társadalmi,
  politikai, gazdasági és kulturális kérdések iránti érdeklődés.
\item
  Kialakulnak a saját értékrend és történelemszemlélet alapjai.
\item
  Elmélyül a nemzeti identitás és hazaszeretet, büszke népe múltjára,
  ápolja hagyományait, és méltón emlékezik meg hazája nagyjairól.
\item
  Megerősödnek az európai civilizációs identitás alapelemei.
\item
  Megerősödik és elmélyül a társadalmi felelősség és normakövetés, az
  egyéni kezdeményezőkészség, a hazája, közösségei és embertársai iránt
  való felelősségvállalás, valamint a demokratikus elkötelezettség.
\item
  Ismeri az ókori civilizációk legfontosabb jellemzőit, valamint az
  athéni demokrácia és a római állam működését, hatásukat az európai
  civilizációra.
\item
  Felidézi a monoteista vallások kialakulását, legfontosabb
  jellemzőiket, tanításaik főbb elemeit, és bemutatja terjedésüket.
\item
  Bemutatja a keresztény vallás civilizációformáló hatását, a középkori
  egyházat, valamint a reformáció és a katolikus megújulás folyamatát és
  kulturális hatásait; érvel a vallási türelem, illetve a
  vallásszabadság mellett.
\item
  Képes felidézni a középkor gazdasági és kulturális jellemzőit,
  világképét, a kor meghatározó birodalmait és bemutatni a rendi
  társadalmat.
\item
  Ismeri a magyar nép őstörténetére és a honfoglalásra vonatkozó
  tudományos elképzeléseket és tényeket, tisztában van legfőbb vitatott
  kérdéseivel, a különböző tudományterületek kutatásainak főbb
  eredményeivel.
\item
  Értékeli az államalapítás, valamint a kereszténység felvételének
  jelentőségét.
\item
  Felidézi a középkori magyar állam történetének fordulópontjait,
  legfontosabb uralkodóink tetteit.
\item
  Ismeri a magyarság törökellenes küzdelmeit, fordulópontjait és hőseit;
  felismeri, hogy a magyar és az európai történelem alakulását
  meghatározóan befolyásolta a török megszállás.
\item
  Be tudja mutatni a kora újkor fő gazdasági és társadalmi folyamatait,
  ismeri a felvilágosodás eszméit, illetve azok kulturális és politikai
  hatását, valamint véleményt formál a francia forradalom európai
  hatásáról.
\item
  Összefüggéseiben és folyamatában fel tudja idézni, miként hatott a
  magyar történelemre a habsburg birodalomhoz való tartozás, bemutatja
  az együttműködés és konfrontáció megnyilvánulásait, a függetlenségi
  törekvéseket és értékeli a rákóczi-szabadságharc jelentőségét.
\item
  Ismeri és értékeli a magyar nemzetnek a polgári átalakulás és nemzeti
  függetlenség elérésére tett erőfeszítéseit a reformkor, az 1848/49-es
  forradalom és szabadságharc, illetve az azt követő időszakban; a kor
  kiemelkedő magyar politikusait és azok nézeteit, véleményt tud
  formálni a kiegyezésről.
\item
  Fel tudja idézni az ipari forradalom szakaszait, illetve azok
  gazdasági, társadalmi, kulturális és politikai hatásait; képes
  bemutatni a modern polgári társadalom és állam jellemzőit és a 19.
  század főbb politikai eszméit, valamint felismeri a hasonlóságot és
  különbséget azok mai formái között.
\item
  Fel tudja idézni az első világháború előzményeit, a háború jellemzőit
  és fontosabb fordulópontjait, értékeli a háborúkat lezáró békék
  tartalmát, és felismeri a háborúnak a 20. század egészére gyakorolt
  hatását.
\item
  Bemutatja az első világháború magyar vonatkozásait, a háborús vereség
  következményeit; példákat tud hozni a háborús helytállásra.
\item
  Képes felidézni azokat az okokat és körülményeket, amelyek a
  történelmi magyarország felbomlásához vezettek.
\item
  Tisztában van a trianoni békediktátum tartalmával és
  következményeivel, be tudja mutatni az ország talpra állását, a
  horthy-korszak politikai, gazdasági, társadalmi és kulturális
  viszonyait, felismeri a magyar külpolitika mozgásterének
  korlátozottságát.
\item
  Össze tudja hasonlítani a nemzetiszocialista és a kommunista
  ideológiát és diktatúrát, példák segítségével bemutatja a rendszerek
  embertelenségét és a velük szembeni ellenállás formáit.
\item
  Képes felidézni a második világháború okait, a háború jellemzőit és
  fontosabb fordulópontjait, ismeri a holokausztot és a hozzávezető
  okokat.
\item
  Bemutatja magyarország revíziós lépéseit, a háborús részvételét, az
  ország német megszállását, a magyar zsidóság tragédiáját, a szovjet
  megszállást, a polgári lakosság szenvedését, a hadifoglyok embertelen
  sorsát.
\item
  Össze tudja hasonlítani a nyugati demokratikus világ és a kommunista
  szovjet blokk politikai és társadalmi berendezkedését, képes
  jellemezni a hidegháború időszakát, bemutatni a gyarmati rendszer
  felbomlását és az európai kommunista rendszerek összeomlását.
\item
  Bemutatja a kommunista diktatúra magyarországi kiépítését, működését
  és változatait, az 1956-os forradalom és szabadságharc okait,
  eseményeit és hőseit, összefüggéseiben szemléli a rendszerváltoztatás
  folyamatát, felismerve annak történelmi jelentőségét.
\item
  Bemutatja a gyarmati rendszer felbomlásának következményeit, india,
  kína és a közel-keleti régió helyzetét és jelentőségét.
\item
  Ismeri és reálisan látja a többpólusú világ jellemzőit napjainkban,
  elhelyezi magyarországot a globális világ folyamataiban.
\item
  Bemutatja a határon túli magyarság helyzetét, a megmaradásért való
  küzdelmét trianontól napjainkig.
\item
  Ismeri a magyar cigányság történetének főbb állomásait, bemutatja
  jelenkori helyzetét.
\item
  Ismeri a magyarság, illetve a kárpát-medence népei együttélésének
  jellemzőit, példákat hoz a magyar nemzet és a közép-európai régió
  népeinek kapcsolatára, különös tekintettel a visegrádi
  együttműködésre.
\item
  Ismeri hazája államszervezetét, választási rendszerét.
\item
  Önállóan tud használni általános és történelmi, nyomtatott és
  digitális információforrásokat (tankönyv, kézikönyvek, szakkönyvek,
  lexikonok, képzőművészeti alkotások, könyvtár és egyéb adatbázisok,
  filmek, keresők).
\item
  Önállóan információkat tud gyűjteni, áttekinteni, rendszerezni és
  értelmezni különböző médiumokból és írásos vagy képi forrásokból,
  statisztikákból, diagramokból, térképekről, nyomtatott és digitális
  felületekről.
\item
  Tud forráskritikát végezni és különbséget tenni a források között
  hitelesség, típus és szövegösszefüggés alapján.
\item
  Képes azonosítani a különböző források szerzőinek a szándékát,
  bizonyítékok alapján értékeli egy forrás hitelességét.
\item
  Képes a szándékainak megfelelő információkat kiválasztani különböző
  műfajú forrásokból.
\item
  Összehasonlítja a forrásokban talált információkat saját ismereteivel,
  illetve más források információival és megmagyarázza az eltérések
  okait.
\item
  Képes kiválasztani a megfelelő forrást valamely történelmi állítás,
  vélemény alátámasztására vagy cáfolására.
\item
  Ismeri a magyar és az európai történelem tanult történelmi korszakait,
  időszakait, és képes azokat időben és térben elhelyezni.
\item
  Az egyes események, folyamatok idejét konkrét történelmi korhoz,
  időszakhoz kapcsolja vagy viszonyítja, ismeri néhány kiemelten fontos
  esemény, jelenség időpontját, kronológiát használ és készít.
\item
  Össze tudja hasonlítani megadott szempontok alapján az egyes
  történelmi korszakok, időszakok jellegzetességeit az egyetemes és a
  magyar történelem egymáshoz kapcsolódó eseményeit.
\item
  Képes azonosítani a tanult egyetemes és magyar történelmi
  személyiségek közül a kortársakat.
\item
  Felismeri, hogy a magyar történelem az európai történelem része, és
  példákat tud hozni a magyar és európai történelem kölcsönhatásaira.
\item
  Egyszerű történelmi térképvázlatot alkot hagyományos és digitális
  eljárással.
\item
  A földrajzi környezet és a történeti folyamatok összefüggéseit
  példákkal képes alátámasztani.
\item
  Képes különböző időszakok történelmi térképeinek összehasonlítására, a
  történelmi tér változásainak és a történelmi mozgások követésére
  megadott szempontok alapján a változások hátterének feltárásával.
\item
  Képes a történelmi jelenségeket általános és konkrét történelmi
  fogalmak, tartalmi és értelmező kulcsfogalmak felhasználásával
  értelmezni és értékelni.
\item
  Fel tud ismerni fontosabb történelmi fogalmakat meghatározás alapján.
\item
  Képes kiválasztani, rendezni és alkalmazni az azonos korhoz, témához
  kapcsolható fogalmakat.
\item
  Össze tudja foglalni rövid és egyszerű szaktudományos szöveg
  tartalmát.
\item
  Képes önállóan vázlatot készíteni és jegyzetelni.
\item
  Képes egy-egy korszakot átfogó módon bemutatni.
\item
  Történelmi témáról kiselőadást, digitális prezentációt alkot és mutat
  be.
\item
  Történelmi tárgyú folyamatábrákat, digitális táblázatokat, diagramokat
  készít, történelmi, gazdasági, társadalmi és politikai modelleket
  vizuálisan is meg tud jeleníteni.
\item
  Megadott szempontok alapján történelmi tárgyú szerkesztett szöveget
  (esszét) tud alkotni, amelynek során tételmondatokat fogalmaz meg,
  állításait több szempontból indokolja és következtetéseket von le.
\item
  Társaival képes megvitatni történelmi kérdéseket, amelynek során
  bizonyítékokon alapuló érvekkel megindokolja a véleményét, és
  választékosan reflektál mások véleményére, árnyalja saját
  álláspontját.
\item
  Képes felismerni, megfogalmazni és összehasonlítani különböző
  társadalmi és történelmi problémákat, értékrendeket, jelenségeket,
  folyamatokat.
\item
  A tanult ismereteket problémaközpontúan tudja rendezni.
\item
  Hipotéziseket alkot történelmi személyek, társadalmi csoportok és
  intézmények viselkedésének mozgatórugóiról.
\item
  Önálló kérdéseket fogalmaz meg történelmi folyamatok, jelenségek és
  események feltételeiről, okairól és következményeiről.
\item
  Önálló véleményt tud alkotni történelmi eseményekről, folyamatokról,
  jelenségekről és személyekről.
\item
  Képes különböző élethelyzetek, magatartásformák megfigyelése által
  következtetések levonására, erkölcsi kérdéseket is felvető történelmi
  helyzetek felismerésére és megítélésére.
\item
  A változás és a fejlődés fogalma közötti különbséget ismerve képes
  felismerni és bemutatni azokat azonos korszakon belül, vagy azokon
  átívelően.
\item
  Képes összevetni, csoportosítani és súlyozni az egyes történelmi
  folyamatok, jelenségek, események okait, következményeit, és ítéletet
  alkotni azokról, valamint a benne résztvevők szándékairól.
\item
  Összehasonlít különböző, egymáshoz hasonló történeti helyzeteket,
  folyamatokat, jelenségeket.
\item
  Képes felismerni konkrét történelmi helyzetekben, jelenségekben és
  folyamatokban valamely általános szabályszerűség érvényesülését.
\item
  Összehasonlítja és kritikusan értékeli az egyes történelmi
  folyamatokkal, eseményekkel és személyekkel kapcsolatos eltérő
  álláspontokat.
\item
  Feltevéseket fogalmaz meg, azok mellett érveket gyűjt, illetve
  mérlegeli az ellenérveket.
\item
  Felismeri, hogy a jelen társadalmi, gazdasági, politikai és kulturális
  viszonyai a múltbeli események, tényezők következményeiként alakultak
  ki.
\end{itemize}
