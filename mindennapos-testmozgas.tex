\hypertarget{mindennapos-testmozgas}{%
\section{Mindennapos testmozgás}\label{mindennapos-testmozgas}}

Az iskolában „a tanulóközösségeknek saját fókuszuk, helyszínük, stílusuk
alakulhat ki''. Ennek részeként a gyerekek mindennapos testmozgását is
tanulóközösségenként másképp, az identitásuknak és lehetőségeiknek megfelelően
alakítják ki. Az iskola csak annyit ír elő, hogy mindennap legyen
minimum 35 perc, aminek elsődleges célja a testmozgás és testnevelés.

Nincs olyan testmozgásforma, amit ez a program preferálna, vagy ami
kizárt lenne. A mindennapos testmozgás fizikai szükségleteinek
megszervezése is változatos módon történhet:

\begin{enumerate}
\tightlist
\def\labelenumi{\arabic{enumi}.}
\item
  Egyes telephelyek saját tornatermében, tornaszobájában vagy udvarán;
\item
  szerződés, megállapodás alapján elérhető közelben lévő más oktatási
  vagy sportlétesítményben (úszás, korcsolya, torna, foci, falmászás,
  sípálya);
\item
  a szabadban, épített helyszínhez nem kötődő mozgásszervezés (pl.
  kirándulás, séta, barangolás, felfedező játékok);
\item
  a feladatellátási helyeken, magyarul a tantermekben,
  olyan mozgások esetében, amelyek nem igényelnek speciálisan
  sportolásr kiképzett helyiséget (pl. jóga).
\end{enumerate}

A testmozgás minden esetben a mozgást ismerő, abban képesítéssel,
végzettséggel vagy megfelelő gyakorlattal rendelkező személy vezetésével
vagy felügyelete mellett zajlik. Az iskola feladata monitorozni a
mindennapos testnevelés megvalósulását: ki, mikor, hol, milyen
testmozgást vezetett a gyerekeknek.
