\hypertarget{az-onallo-tanulas}{%
\section{Az önálló tanulás}\label{az-onallo-tanulas}}

Budapest School
célja (\ref{emberkep}.~fejezet, \pageref{emberkep}.~oldal), hogy a
\emph{„gyerekek az iskola befejeztével önállóan és kreatívan gondolkodó,
önmagával és közösségével integránsan élő érett nagykorúvá
váljanak''}. Például az érettségi évére a gyerekek az iskolában
\emph{„már megtanulnak szakaszosan célokat állítani''} és \emph{„önállóan
készülnek az érettségire''}. Az önálló tanulás képességét már a legkisebb
kortól kezdve folyamatosan gyakorolni, fejleszteni kell. Nem az a
kérdés, hogy tud-e egy gyerek önállóan tanulni, hiszen járni és beszélni
is önállóan tanult minden gyerek. A fő kérdés, hogy hogyan tud egyre
nagyobb célokat elérni, egyre nehezebb képességeket, összetettebb tudást
megtanulni és egyre komplexebb projektekben részt venni. A BPS modell
feltételezi, hogy mindenki képes az önálló tanulásra, és mindenki tudja
fejleszteni ezt a képességét. A BPS modell
négy tanulási szakasza (\ref{emberkep}.~fejezet, \pageref{emberkep}.~oldal)
tulajdonképpen a önálló tanulási képesség négy szintjét írják le.

A modell a tanulás megközelítésével és strukturálásával is támogatja az
önálló tanulás képességének fejlesztését. Legkisebb kortól kezdve
gyakorolják a gyerekek a
saját cél állítást (\ref{sajat-tanulasi-celok}.~fejezet, \pageref{sajat-tanulasi-celok}.~oldal).
Az erős keretek és határokon belüli \emph{választás szabadságát} a
moduláris tanmenet (\ref{tanulasi-tanitasi-egysegek-a-modulok}.~fejezet, \pageref{tanulasi-tanitasi-egysegek-a-modulok}.~oldal)
biztosítja. És a közösségi kultúra része a rendszeres és
folyamatos reflexió (\ref{visszajelzes-ertekeles}.~fejezet, \pageref{visszajelzes-ertekeles}.~oldal).
Nem utolsó szempont, hogy a BPS modell teret ad a tanároknak önállóan,
kreatívan és alkotó módon hozzáállni a munkájukhoz. Az önállóság,
kreativitás és reflexió a kultúra, azaz a mindennapi működés része kell
hogy legyen.

Ezért is fontos, hogy a gyerekek egyre többet gyakorolják az
\emph{önálló tanulást, azaz azokat a célorientált tevékenységeket,
amikor nem a tanár (szülő) határozza meg hogy mikor, kivel és hogyan
tanul egy gyerek.}

\begin{quote}
A házi feladat, az otthoni tanulás, a könyvtárban tanulás, a
tanulószobai tanulás, az online tanulás, az egyéni kutatás, a
szakirodalom feldolgozás, a tanulókörös tanulás, a korrepetálás, az
iskola újság készítés, a saját projekteken dolgozás, a
fordított/tükrözött osztályterem, mind mind olyan elfoglaltságok, amikor
a gyerekek maguk irányíthatják a saját tanulási és alkotási
folyamataikat.

Az iskola szempontjábál önálló tanulásnak tekinthető az is, amikor a
gyerek önállóan beiratkozik egy nyári táborba, ahol robotikát tanul,
délutáni iskola utáni tanfolyamon vesz részt, vagy épp egy másik
intézményben készül a nemzetközi érettségire és felvételire.
\end{quote}

Az önírányított, önálló tanulási módokat a BPS modell a tanulási élmény
ugyanolyan fontos elemének tartja, mint a tanárok által vezetett
foglalkozásokat. Például ugyanolyan értékes (tanulási) eredménynek kell
tekinteni, ha egy gyerek egy matematika tanórán egy tanártól hall a
\emph{logikai szitáról}, ha a tankönyből szerzi ismereteit a
tanulószobán, vagy ha egy online tananyagból tanul erről a fogalomról,
például a
%\href{https://portal.nkp.hu/Search?keyword=logikai\%20szita}
{\emph{Nemzeti
Köznevelési Portálról}}, esetleg a
%\href{https://www.khanacademy.org/math/statistics-probability/probability-library/basic-set-ops/e/basic_set_notation}
{\emph{Khan
Academy}} oldalán, vagy ha társától tanulja meg, mit takar ez a fogalom,
és hogyan lesz számára hasznos a céljai elérésében.

\hypertarget{az-onallo-tanulas-aranya}{%
\paragraph{Az önálló tanulás aránya}\label{az-onallo-tanulas-aranya}}

A javasolt önálló, önirányított tanulás mértéke évfolyamonként növekszik
az iskolában, ahogy a gyerekek képessége egyre jobban megengedi ezt a
tanulási formát.

\begin{longtable}[]{@{}lllllllllllll@{}}
\toprule
\begin{minipage}[b]{0.33\columnwidth}\raggedright
Évfolyam\strut
\end{minipage} & \begin{minipage}[b]{0.03\columnwidth}\raggedright
1\strut
\end{minipage} & \begin{minipage}[b]{0.03\columnwidth}\raggedright
2\strut
\end{minipage} & \begin{minipage}[b]{0.03\columnwidth}\raggedright
3\strut
\end{minipage} & \begin{minipage}[b]{0.03\columnwidth}\raggedright
4\strut
\end{minipage} & \begin{minipage}[b]{0.03\columnwidth}\raggedright
5\strut
\end{minipage} & \begin{minipage}[b]{0.03\columnwidth}\raggedright
6\strut
\end{minipage} & \begin{minipage}[b]{0.03\columnwidth}\raggedright
7\strut
\end{minipage} & \begin{minipage}[b]{0.03\columnwidth}\raggedright
8\strut
\end{minipage} & \begin{minipage}[b]{0.03\columnwidth}\raggedright
9\strut
\end{minipage} & \begin{minipage}[b]{0.03\columnwidth}\raggedright
10\strut
\end{minipage} & \begin{minipage}[b]{0.03\columnwidth}\raggedright
11\strut
\end{minipage} & \begin{minipage}[b]{0.03\columnwidth}\raggedright
12\strut
\end{minipage}\tabularnewline
\midrule
\endhead
\begin{minipage}[t]{0.33\columnwidth}\raggedright
Önálló, önírányított tanulás javasolt mértéke\strut
\end{minipage} & \begin{minipage}[t]{0.03\columnwidth}\raggedright
25\%\strut
\end{minipage} & \begin{minipage}[t]{0.03\columnwidth}\raggedright
25\%\strut
\end{minipage} & \begin{minipage}[t]{0.03\columnwidth}\raggedright
30\%\strut
\end{minipage} & \begin{minipage}[t]{0.03\columnwidth}\raggedright
35\%\strut
\end{minipage} & \begin{minipage}[t]{0.03\columnwidth}\raggedright
40\%\strut
\end{minipage} & \begin{minipage}[t]{0.03\columnwidth}\raggedright
45\%\strut
\end{minipage} & \begin{minipage}[t]{0.03\columnwidth}\raggedright
50\%\strut
\end{minipage} & \begin{minipage}[t]{0.03\columnwidth}\raggedright
55\%\strut
\end{minipage} & \begin{minipage}[t]{0.03\columnwidth}\raggedright
60\%\strut
\end{minipage} & \begin{minipage}[t]{0.03\columnwidth}\raggedright
65\%\strut
\end{minipage} & \begin{minipage}[t]{0.03\columnwidth}\raggedright
70\%\strut
\end{minipage} & \begin{minipage}[t]{0.03\columnwidth}\raggedright
75\%\strut
\end{minipage}\tabularnewline
\bottomrule
\end{longtable}
