\hypertarget{pedagogiai-modszerek}{%
\section{Pedagógiai módszerek}\label{pedagogiai-modszerek}}

A Budapest School iskola tanárainak feladata, hogy mindig keressék azt a
módszert, azt a környezetet, ami az adott gyerekekkel, adott
környezetben, adott időben a leginkább működik. Nem tudjuk megmondani
előre, hogy mikor milyen módszert érdemes választani, de azt tudjuk,
hogy mi alapján keressük a megfelelő technikákat. Vannak olyan
módszerek, amelyek a saját célok lehetőségeinek kitágítását és azok
elérését nagyban támogatják. Ezek alkalmazása javasolt a csoportmunkák
és az egyéni gyakorlások alatt.

Azt is tudjuk, hogy nem baj, ha nem elsőre találtuk meg a megfelelő
módszert, mert a próbálkozások során rengeteg új információt nyerünk,
amelyek segítségével már könnyebb megtalálni a valóban megfelelő
megoldást. Az alábbiakban néhány a Budapest School számára meghatározó
fontosságú módszert emelünk ki.

\hypertarget{rugalmas-csoportbontasok}{%
\subsection{Rugalmas csoportbontások}\label{rugalmas-csoportbontasok}}

Pedagógiai elvek az osztályok, csoportbontások, foglalkozások
megszervezése tekintetében A BPS Általános Iskola és Gimnázium az Nkt.
4. § 24. pontjában, 25. § (7) bekezdésében, 26. §-ában, 31. § (1), (2)
bekezdésében valamint a 20/2012. EMMI rendelet 13. § (1) bekezdésében
foglalt felhatalmazása alapján a 20/2012. (VIII. 31.) EMMI rendelet 2.
mellékletében definiált eltérő pedagógiai elveket, valamint a 20/2012.
(VIII. 31.) EMMI rendelet 7.§ (4) bekezdése szerint definiált eltérő
pedagógiai módszereket alkalmaz. A fentiek alapján a BPS csoportbontási
és foglalkozás-szervezési elvei a következők:

\begin{itemize}
\tightlist
\item
  alsó tagozaton összevont osztály megszervezése lehetséges;
\item
  az osztályok létszámára vonatkozó minimális rendelkezéseket nem
  alkalmazza a BPS;
\item
  a tanórákat és foglalkozásokat a BPS van, hogy különböző évfolyamok,
  különböző osztályok tanulóiból álló csoportok részére szervezi meg és
  van, amikor egy évfolyamra járó gyerekek csoportjának, azaz főleg az
  osztályoknak szervez egy foglalkozást;
\item
  a BPS alkalmazza a projektoktatás módszerét, melynek során a
  témaegységek feldolgozása, a feladat megoldása a tanulók
  érdeklődésére, a tanulók és a pedagógusok közös tevékenységére,
  együttműködésére épül.
\end{itemize}

Magyarul megfogalmazva: a gyerekeket a tanároknak úgy kell csoportokra
bontaniuk, ahogy ez szerintük a leginkább támogatja a gyerekek tanulását.
Ha egy 10 éves gyerek jól tud egy 14 éves gyerektől tanulni, és mindenki
élvezi ezt, akkor egy kevert korosztályú, kooperatív differenciálás a
legcélrevezetőbb. Egy emeltszíntű matematika érettségire felkészítő
csoportban célszerű az egy tudásszinten lévő gyerekeket egy csoportba
tenni.

Tehát amennyiben a különböző évfolyamra járó gyerekek oktatása az adott
tantárgyat, foglalkozást vagy modult illetően hatékonyabb az
évfolyamonkénti csoportokban történő oktatással szemben, úgy olyan
csoportok kialakíthatók, amelyekben a legidősebb és a legfiatalabb
gyerek évfolyamának különbsége hatnál nem több.

A különböző évfolyamra járó gyerekek csoportmunkája során is figyelemmel
kell lenni a tanulók életkori sajátosságaira. A tanárnak differnciálni
kell. Leginkább abban, hogy az eltérő évfolyamszinten lévő gyerekeknek
eltérő tanulási eredmények elérése a célja.

A csoportok összeállításakor figyelembe kell venni a közösség minden
tagjának, és minden csoportjának, így az osztályok, évfolyamok és
tanulóközösségek érdekeit.

\hypertarget{projektmodszer}{%
\subsection{Projektmódszer}\label{projektmodszer}}

Projektmódszert alkalmazó modulok során fő célunk, hogy a gyerekek
aktívak és kreatívak legyenek, és ezért a tevékenységek sokszínűségét
helyezzük fókuszba. Projektmódszer alkalmazásakor is arra bíztatjuk a
tanárokat, hogy a legváltozatosabb és legadekvátabb módszereket
alkalmazzák.

\hypertarget{a-projektmunka-folyamata}{%
\subsubsection{A projektmunka
folyamata}\label{a-projektmunka-folyamata}}

\hypertarget{tema-cel}{%
\paragraph{Téma, cél}\label{tema-cel}}

Az első lépés a projekt céljának és tartalmának meghatározása. Erre
javaslatot tehet a tanár, a gyerekek vagy akár egy szülő is. Fontos,
hogy a gyerekek a projekt témáját vagy célját már önmagában értelmesnek,
relevánsnak tartsák.

\hypertarget{otletroham}{%
\paragraph{Ötletroham}\label{otletroham}}

Egy-egy téma feldolgozását csoportalakítással és öt\-let\-ro-\break
hammal kezdjük.
Ennek célja, hogy a résztvevők bevonódjanak, illetve megmutassák, hogy
nekik milyen elképzeléseik vannak az adott témáról, továbbá milyen
produktummal, eredménnyel szeretnék zárni a folyamatot. A létrehozott
produktumoknak csak a képzelet szabhat határt. Lehetnek videók,
prezentációk, fotók, rapdalok, telefonos applikációk, rajzok, tablók,
tudományos cikkek stb.

\hypertarget{kutatoi-kerdes}{%
\paragraph{Kutatói kérdés}\label{kutatoi-kerdes}}

Ezek után úgynevezett kutatói kérdéseket teszünk fel,\break
melyek
meghatározzák a vizsgálat irányát. A kérdések feldolgozása a
legváltozatosabb módokon történhet. Az egyéni munkától kezdve a frontális
instruáláson vagy a kooperatív csoportmunkán keresztül egészen a drá\-ma-
és zenefoglalkozásokig minden hasznosítható a tanár, a csoport és a téma
igényeihez mérten.

\hypertarget{elmelyult-csoportmunka}{%
\paragraph{Elmélyült csoportmunka}\label{elmelyult-csoportmunka}}

A projekt azon szakasza, amikor a tervek, kutatások alapján az
implementáción dolgozik a csapat.

\hypertarget{prezentacio}{%
\paragraph{Prezentáció}\label{prezentacio}}

A létrehozott produktumok bemutatására külön hangsúlyt kell fektetni.
Ennek több módja is lehet: prezentációk, demonstrációk, plakátok,
projektfesztiválok.

Az iskolai projektek célja egy fejlődésfókuszú tanár és gyerek számára
mindig kettős: egyrészt cél a téma feldolgozása, a produktum
létrehozása, másrészt az iskola fő célja, hogy a gyerekek, a csapatok
mindig fejlesszék alkotó, együttműködő, problémamegoldó képességüket.
Ezért a projekt folyamatára való reflektálás, visszajelzés ugyanolyan
fontos, mint maga a cél elérése.

A munka során külön figyelmet kell fordítani arra, hogy mindent
dokumentáljanak a résztvevők. Lehetőleg online felületen.

\hypertarget{a-projekt-ertekelese}{%
\subsubsection{A projekt értékelése}\label{a-projekt-ertekelese}}

A projekt során több értékelési pontot érdemes beépíteni. A
foglalkozások végén a résztvevők visszajeleznek a folyamatra, értékelik
a saját, a csoport és a tanár munkáját. A folyamat végén az egész
projektfolyamatot értékelik, szintén kitérve a saját, a csoport és a
tanár munkájára. A produktumok, az eredmény értékelése csoport- és
egyéni szinten is megtörténik.

Az értékelés a folyamatra fókuszál, és nem csak az eredményre, hogy
fejlessze a fejlődésfókuszú gondolkodásmódot.

\hypertarget{onszervezodo-tanulasi-kornyezet}{%
\subsection{Önszerveződő tanulási
környezet}\label{onszervezodo-tanulasi-kornyezet}}

A Sugata Mitra által kialakított módszertan (angolul Self Organizing
Learning Environment) lényege, hogy a tanárok arra bátorítják a
gyerekeket, hogy csoportban, az internet segítségével \emph{nagy
kérdéseket} válaszoljanak meg. A jó kérdés az, amire nem egyszerű a
válasz, sőt lehet, hogy nincs is rá egyfajta válasz. Cél, hogy a
gyerekek maguk alakítsák a folyamatot, formálják a kérdést, és
találjanak válaszokat.

\begin{itemize}
\tightlist
\item
  A tanár kialakítja a teret: körülbelül négy gyerekre jut számítógép,
  amit körbe lehet ülni.
\item
  A gyerekek maguk formálják a csoportjukat, sőt még csoportot is
  válthatnak a munka során. Mozoghatnak, kérdezhetnek, „leshetnek'' más
  csoportoktól.
\item
  Körülbelül 30--45 perc után a csoportok prezentálják a kutatásuk
  eredményét.
\end{itemize}

\hypertarget{a-jo-kerdesek}{%
\paragraph{A jó kérdések}\label{a-jo-kerdesek}}

A \emph{nagy kérdések} nyílt és nehéz kérdések, és előfordulhat, hogy
senki sem tudja még rájuk a választ. A cél, hogy mély és hosszú
beszélgetéseket generáljanak. Ezek azok a kérdések, amikre érdemes
nagyobb elméleteket állítani, amelyeket jobb csoportban megvitatni,
amelyekről érvelni lehet és kritikusan gondolkodni.

A jó kérdések több témát, területet (tantárgyat) kapcsolnak össze: a
\emph{„Mi a hangya''} kérdés például nem érint annyi különböző területet,
mint a \emph{„Mi történne a Földdel, ha minden hangya eltűnne?''} kérdés.

\hypertarget{fegyelmezes-nelkul}{%
\paragraph{Fegyelmezés nélkül}\label{fegyelmezes-nelkul}}

A tanár feladata a folyamat során meghatározni a \emph{nagy kérdést}, és
tartani a kereteket. A cél, hogy a gyerekek maguk szervezzék saját
munkájukat, így minimális beavatkozás javasolt a tanár részéről.
Kezdetben, gyakorló csoportoknál, a tanárnak sokszor kell emlékeztetnie
magát, hogy idővel kialakul a rend. \emph{„Bízz a folyamatban!''} Amikor
a tanár úgy látja, hogy nem megy a munka, akkor csak finoman emlékezteti
a csoportokat, hogy lassan jön a prezentáció ideje. Amikor valaki a
csoportjáról panaszkodik, akkor elmondhatja, hogy szabad csoportot
váltani. Ha valaki zavarja a többieket, akkor megfigyelheti, hogy a
gyerekek tudnak-e már konfliktust feloldani. Ha valaki nem vesz részt a
munkában, akkor gondolkozhat olyan kérdésen, ami az éppen demotivált
gyerekeket is bevonzza.

\hypertarget{az-onallo-tanulas}{%
\subsection{Az önálló tanulás}\label{az-onallo-tanulas}}

Budapest School
célja (\ref{emberkep}.~fejezet, \pageref{emberkep}.~oldal), hogy
\emph{„a gyerekek az iskola befejeztével önállóan és kreatívan gondolkodó,
önmagukkal és közösségükkel integránsan élő érett nagykorúakká
váljanak''}. Például az érettségi évére a gyerekek az iskolában
\emph{„már megtanulnak szakaszosan célokat állítani''} és \emph{„önállóan
készülnek az érettségire''}. Az önálló tanulás képességét már a legkisebb
kortól kezdve folyamatosan gyakorolni, fejleszteni kell. Nem az a
kérdés, hogy tud-e egy gyerek önállóan tanulni, hiszen járni és beszélni
is önállóan tanult minden gyerek. A fő kérdés, hogy hogyan tud egyre
nagyobb célokat elérni, egyre nehezebb képességeket, összetettebb tudást
megtanulni és egyre komplexebb projektekben részt venni. A BPS modell
feltételezi, hogy mindenki képes az önálló tanulásra, és mindenki tudja
fejleszteni ezt a képességét. A BPS modell
négy tanulási szakasza (\ref{emberkep}.~fejezet, \pageref{emberkep}.~oldal)
tulajdonképpen az önálló tanulási képesség négy szintjét írják le.

A modell a tanulás megközelítésével és strukturálásával is támogatja az
önálló tanulás képességének fejlesztését. Legkisebb kortól kezdve
gyakorolják a gyerekek a
saját cél állítást (\ref{sajat-tanulasi-celok}.~fejezet, \pageref{sajat-tanulasi-celok}.~oldal).
Az erős keretek és határokon belüli \emph{választás szabadságát} a
moduláris tanmenet (\ref{tanulasi-tanitasi-egysegek-a-modulok}.~fejezet, \pageref{tanulasi-tanitasi-egysegek-a-modulok}.~oldal)
biztosítja. És a közösségi kultúra része a rendszeres és
folyamatos reflexió (\ref{visszajelzes-ertekeles}.~fejezet, \pageref{visszajelzes-ertekeles}.~oldal).
Nem utolsó szempont, hogy a BPS modell teret ad a tanároknak önállóan,
kreatívan és alkotó módon hozzáállni a munkájukhoz. Az önállóság,
kreativitás és reflexió a kultúra, azaz a mindennapi működés része kell
hogy legyen.

Ezért is fontos, hogy a gyerekek egyre többet gyakorolják az
\emph{önálló tanulást, azaz azokat a célorientált tevékenységeket,
amikor nem a tanár (szülő) határozza meg hogy mikor, kivel és hogyan
tanul egy gyerek.}

\begin{quote}
A házi feladat, az otthoni tanulás, a könyvtárban tanulás, a
tanulószobai tanulás, az online tanulás, az egyéni kutatás, a
szakirodalom feldolgozás, a tanulókörös tanulás, a korrepetálás, az
iskolaújság készítése, a saját projekteken dolgozás, a
fordított/tükrözött osztályterem, mind-mind olyan elfoglaltságok, amikor
a gyerekek maguk irányíthatják a saját tanulási és alkotási
folyamataikat.

Az iskola szempontjábál önálló tanulásnak tekinthető az is, amikor a
gyerek önállóan beiratkozik egy nyári táborba, ahol robotikát tanul,
délutáni iskola utáni tanfolyamon vesz részt, vagy épp egy másik
intézményben készül a nemzetközi érettségire és felvételire.
\end{quote}

Az önírányított, önálló tanulási módokat a BPS modell a tanulási élmény
ugyanolyan fontos elemének tartja, mint a tanárok által vezetett
foglalkozásokat. Például ugyanolyan értékes (tanulási) eredménynek kell
tekinteni, ha egy gyerek egy matematika tanórán egy tanártól hall a
\emph{logikai szitáról}, ha a tankönyből szerzi ismereteit a
tanulószobán, vagy ha egy online tananyagból tanul erről a fogalomról,
például a
%\href{https://portal.nkp.hu/Search?keyword=logikai\%20szita}
{\emph{Nemzeti
Köznevelési Portálról}}, esetleg a
%\href{https://www.khanacademy.org/math/statistics-probability/probability-library/basic-set-ops/e/basic_set_notation}
{\emph{Khan
Academy}} oldalán, vagy ha társától tanulja meg, mit takar ez a fogalom,
és hogyan lesz számára hasznos a céljai elérésében.

\hypertarget{egyeni-tanulasi-idok}{%
\subsubsection{Egyéni tanulási idők}\label{egyeni-tanulasi-idok}}

Az önálló tanulás egyik megjelenése az \emph{egyéni tanulási idő (ETI)}
vagy az \emph{ügyfélszolgálat} nevű tanóraként meghirdetett és vezetett
(tehát pedagógus munkakörben alkalmazott tanár által vezetett)
foglalkozástípusok: ilyenkor a gyerekek több tantárggyal
foglalkozhatnak egyszerre, mindenki hozza a saját kérdését, feladatát és
a csoportban a gyerekek együtt vagy egymástól tanulnak. A tanár pedig
azon kívül, hogy biztosítja a kereteket és a biztonságot, igény esetén
elérhető is.

\hypertarget{megfontolt-gyakorlas}{%
\subsection{Megfontolt gyakorlás}\label{megfontolt-gyakorlas}}

Ahogy Anders Ericsson pszichológus
%\href{http://projects.ict.usc.edu/itw/gel/EricssonDeliberatePracticePR93.PDF}
1993-ban
megjelent, szerzőtársakkal írt tanulmánya {\autocite{ericsson:etal:1993}} is kimutatja, bárki tudja valamennyi készségét,
képességét fejleszteni, ha megtervezetten, megfontoltan gyakorolja. A
Budapest School azt az álláspontot képviseli, hogy gyakorlással --- a
hagyományos készségtárgyakhoz hasonlóan --- sokféle képesség és tudás
fejleszthető. Tehát odafigyeléssel és megfontolt gyakorlással egy gyerek
egyre jobban fog tudni írásbeli érettségi vizsgát tenni magyarból,
geopolitikai elemzéseket végezni, hiperbolikus függvényekkel egyenletet
megoldani, domináns csoporttagokkal együttműködni vagy
stresszhelyzetben
önmagát lenyugtatni.

A készség- és képességfejlesztés legjobb eszköze a megtervezett
gyakorlás: a fejlődés érdekében okosan gyakorlunk. A megfontolt
gyakorlás jellemzője:

\begin{itemize}
\item
  \textbf{Világos és specifikus cél} Fontos, hogy tudjuk, mit
  gyakorlunk, mit akarunk elérni. Lehetőleg a cél legyen mérhető és
  reálisan elérhető.
\item
  \textbf{Fókusz} Gyakorlás során egy dologra érdemes figyelni.
\item
  \textbf{Komfortzónán kívül kell lenni} Az edzőnek, tanárnak, trénernek
  néha érdemes a tanulót kicsit „nyomni''. Emlékeztetni, hogy mindig
  lehet kicsit többet elérni.
\item
  \textbf{Folyamatos visszajelzés} Nagyon gyakran kap a tanuló
  visszajelzést, mindig tudja, hogy mikor és miben fejlődött.
\end{itemize}
