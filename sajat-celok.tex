\hypertarget{sajat-tanulasi-celok}{%
\section{Saját tanulási célok}\label{sajat-tanulasi-celok}}

Minden gyerek megfogalmazza és háromhavonta újrafogalmazza a \emph{saját
tanulási céljait}: eredményeket, amelyeket el akar érni, képességeket,
amelyeket fejleszteni akar, szokásokat, amelyeket ki akar alakítani. A
saját célok elfogadásakor a gyerek és a mentora a szülőkkel együtt
\emph{tanulási szerződést} köt.

Csak olyan célok kerülhetnek a saját célok közé, amelyek minden
érintettnek biztonságosak. Annyi megkötés van, hogy céloknak mindenképp
ki kell terjedniük a tantárgyi tanulási eredményekre (\emph{,,mennyit és
hogyan akarok elérni a NAT-ból származó tanulási eredményekből''}) és a
tantárgyi rendszeren kívüli célokra, feladatokra is.

Háromhavonta a tanulásszervezők és a gyerekek megállnak, reflektálnak az
elmúlt időszakra, és a tapasztalatok, valamint az elért célok
ismeretében és az új célok figyelembevételével újraszervezik a
foglalkozások rendjét, tehát azt, hogy mikor és mit csinálnak majd a
gyerekek az iskolában. A mindennapi tevékenység során tapasztalt
élmények, alkotások, elvégzett feladatok és vizsgák, tehát
mindaz, ami a gyerekekkel történik, bekerül a portfóliójukba. Még az is,
amit nem terveztek meg előre.

A gyerekeket a mentoruk segíti a saját célok kitűzésében, a különböző
választásoknál, a portfólióépítésben, a reflektálásban. A tanulási célok
kitűzése az önirányított tanulás fokozatos fejlődésével és az életkor
előrehaladtával folyamatosan egyre önállóbb tevékenységgé válik.
Tanulási útjukon, céljaik kitűzésében a mentor kíséri végig a gyerekeket.

A Budapest School személyre szabott tanulásszervezésének
jellegzetessége, hogy a gyerekek a saját céljuk irányába haladnak, az
adott célhoz az adott kontextusban leghatékonyabb úton. Tehát mindenki
rendelkezik saját célokkal, még akkor is, ha egy közösség tagjainak
céljai a tantárgyi tanulási eredmények azonossága, vagy a hasonló
érdeklődés miatt akár 80\% átfedést mutatnak.

A NAT műveltségi területeiben és a tantárgyakban megfogalmazott
követelmények teljesítése is célja a tanulásnak, a tanulás fő irányítója
azonban más. Mi azt kérdezzük a gyerekektől, hogy \emph{ezen felül} mi
az ő személyes céljuk.
