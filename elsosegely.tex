\hypertarget{elsosegely-nyujtasi-alapismeretek}{%
\section{Elsősegélynyújtási
alapismeretek}\label{elsosegely-nyujtasi-alapismeretek}}

A cél, hogy a gyerekek és tanárok megtanulják aktívan úgy alakítani
környezetüket és viselkedésüket, hogy a balesetek számát minimalizálják,\break
hogy felismerjék, amikor segítségre van szükség, hogy hatékonyan
segítsenek, és tudjanak segítséget hívni. Ehhez gyakorlásra, a témával
kapcsolatos védett időre van szükség. Ezért a 2., 4., 6., 8. és
10. évfolyam csak akkor teljesíthető, ha a gyerek minden második évben
elvégez egy minimum négyórás modult, aminek célja,

\begin{itemize}
\tightlist
\item
  hogy a gyerekek sajátítsák el a legalapvetőbb és legkorszerűbb
  elsősegélynyújtási módokat, azaz tudjanak egymásnak segíteni
  baj esetén (nemcsak elméletben, hanem a gyakorlatban is);
\item
  sajátítsák el, mikor és hogyan kell mentőt, segítséget hívni;
\item
  foglalkozzanak azzal, hogy hogyan tudják környezetüket, csoportjukat,
  tanulóközösségüket biztonságosabbá tenni, és ezt dokumentálják is.
\end{itemize}

A modul szervezője próbáljon meg elsősegélynyújtási bemutatót szervezni
a gyerekeknek az Országos Mentőszolgálat, a Magyar Ifjúsági Vöröskereszt,
az Ifjúsági Elsősegélynyújtók Országos Egyesülete szakembereinek vagy más,
magyar vagy külföldi képesítést szerzett szakembernek a bevonásával.

Az elsősegélynyújtási alapismeretek elsajátításával kapcsolatos
feladatok megvalósításának elősegítése érdekében az iskola

\begin{itemize}
\tightlist
\item
  kapcsolatot épít ki az Országos Mentőszolgálattal, a Magyar Ifjúsági
  Vöröskereszttel vagy az Ifjúsági Elsősegélynyújtók Országos
  Egyesületével. Tanulóink ---~választásuk szerint~--- bekapcsolódhatnak
  az elsősegélynyújtással kapcsolatos iskolán kívüli vetélkedőkbe;
\item
  minden második évben legalább egyszer a tanároknak lehetőséget
  biztosít elsősegély-tanfolyam látogatására.
\end{itemize}
