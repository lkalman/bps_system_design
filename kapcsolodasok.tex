\hypertarget{kapcsolat-a-hazai-es-nemzetkozi-oktatasi-iranyzatokkal}{%
\section{Kapcsolat a hazai és nemzetközi oktatási
irányzatokkal}\label{kapcsolat-a-hazai-es-nemzetkozi-oktatasi-iranyzatokkal}}

A Budapest School programja a hazai és nemzetközi oktatási reformok
kontextusában és a pszichológia, a szociálpszichológia, valamint a
szervezetfejlesztés terén elvégzett kortárs kutatások tükrében válik
könnyebben értelmezhetővé.

Magyarországról több iskola története, működése is nagy hatással volt
ránk. A 90-es évektől induló alternatív iskolák világát mi a
% \href{https://www.rogersiskola.hu/}
{\emph{Rogers Személyközpontú
Általános Iskola}}, a %\href{https://www.lauder.hu/wp/}
{\emph{Lauder
Javne Iskola}}, a
% \href{https://www.kincskereso-iskola.hu/}
{\emph{Kincskereső Iskola}} és
a %\href{https://gyermekekhaza.hu/}
{\emph{Gyermekek Háza}} alapján
ismertük meg. A megújuló középiskolák modelljének mi az
% \href{https://www.akg.hu/}
{\emph{Alternatív Közgazdasági Gimnáziumot}}
és a %\href{https://poli.hu/wp/}
{\emph{Közgazdasági Politechnikumot}}
tartjuk. Ezek az iskolák a személyközpontúság, a gyerekközpontúság
hangsúlyozása mellett elkezdték a gyakorlatban alkalmazni a partnerség
alapú kommunikációt, a differenciálás, a kooperatív technikák módszerét
és egyes projektmódszertanokat.

Programunk kidolgozásában nagy szerepe volt annak, hogy ezek az iskolák
olyan szemléletmódbeli alapokat fektettek le, amelyek mára
alapelvárásként fogalmazódnak meg a szülők oldaláról az iskolákkal
szemben.

Gyakorlati tapasztalatokat a világ más részein is gyűjtöttünk. A 21.
században a Budapest School-hoz hasonló kezdeményezések sorra indulnak a
világban. Ezek egyes jegyei a Budapest School modelljével összhangban
vannak:

A %\href{https://wildflowerschools.org/}
{\emph{Wildflower School}}
tanulóközösségek hálózatát működteti kisebb üzlethelyiségekben. A
Budapest School-hoz hasonlóan célja, hogy falakat\break
romboljon a gyerekek és
a világ között: az egyéni tanulás és az intézményes tanulás, a tanár és
a tudós szerepe, valamint az iskola és környezete közötti határok
elmosása az egyik fő üzenete.

Hasonlóan az otthon tanulás és az \emph{unschooling} strukturált
formáját keresi az amerikai Texasban alapított
%\href{https://www.actonacademy.org/}
{\emph{Acton Academy}}, amely a
szokratikus módszereket (azaz, hogy megbeszéljük közösen), a valós
projektből való tanulást, és a gyakornokoskodáshoz hasonló munka közbeni
tanulást („learning on the job'') teszi a megközelítésének középpontjába.

A %\href{https://www.hightechhigh.org/}
{\emph{High Tech High}} iskoláiban
a gyerekek elsősorban projektmódszertan alapján tanulnak. A tanulási
jogokban való egyenlőség mellett az egyéni célokra szabott tanulás, a
világ alakulásához kapcsolódó tartalmi elemek, valamint az
együttműködés-alapú tanulás is megjelenik pedagógiájukban a Budapest
School által is alkalmazott jegyekből.

A %\href{https://www.school21.org.uk/}
{\emph{School21}} brit iskola
21.~századi képességek fejlesztését tűzte ki célul. Ezért a
prezentációs, előadói skillek kiemelt jelentőségűek. Az iskola
egyensúlyt akar teremteni a tudásbéli (akadémiai), a szívbéli
(személyiség és jóllét) és a kézzel fogható (problémamegoldó, alkotó)
között. A Budapest School iskoláinak hasonló módon célja, hogy a tanulás
három rétegét, a tudást, a gondolkodást és az alkotást folyamatos
harmóniában tartsa.

A %\href{https://khanlabschool.org/}
{\emph{Khan Lab School}} a
Montessori-módszert keveri az online tanulással. Kevert korosztályú
csoportokban, személyre szabott módszerekkel segítik a
képességfejlesztést és a projektalapú munkát. A BPS modell négy tanulási
szakasza tulajdonképpen megfelel a Khan Lab School
%\href{https://khanlabschool.org/independence-levels}
{\emph{függetlenségi
szintjeinek}}.
