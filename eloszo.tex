Európa fölött repültünk épp egy fapados járattal, amikor először fogalmazódott meg bennünk, hogy szeretnénk azt az élményt megadni a gyerekeinknek, amit mi is kaptunk az élettől. A szabadon alkotás és tanulás semmivel sem összehasonlítható örömét. Hogy az iskolájukban minél többet lehessenek abban az állapotban, amikor a képzeletük segítségével olyan célokat tűznek ki maguk elé, amelyek kellő mértékű kihívást jelentenek, és közben lázba hozza őket a megvalósítás reménye. Amikor ilyen céljaink vannak, akkor azokra sajátunkként tudunk tekinteni. Felnőttként és gyerekként egyaránt. Az elképzeléseinket ilyenkor átbeszéljük másokkal, hogy ötleteinket kritikusan újragondolhassuk, majd belevágunk, és amikor úgy érezzük, hogy eljutottunk valahova, akkor rátekintünk és megpróbáljuk kiértékelni az elért eredményeinket. Így írunk, mi emberek könyveket, így építünk hidakat, így kutatunk, és így hozunk létre újabb és újabb szervezeteket, melyekben közösségek működnek együtt, hogy céljaikat elérhessék. És azt szeretnénk, hogy így tanuljanak a gyerekeink is.

Budapest felé repülve arról álmodoztunk, hogy létrehozhatunk egy olyan iskolát, ami semmi másra nem figyel, pusztán a gyerekeknek arra a képességére, ahogy tanulni, fejlődni tudnak a bennük rejlő kíváncsiság és érdeklődés mentén. És közben ott lebegett előttünk az ismeretlen megismerésének lehetősége. Hogy belevághatunk valamibe együtt, amiről akkor még semmit sem tudtunk. Egy új iskola megalapításába. Így jött létre a Budapest School 2015 nyarán. Barátainkkal való beszélgetéseinkben hamar eljutottunk a mikroiskolák hálózatának gondolatáig, melyek egy tanulási környezetbe szerveződve egymást segítik, egymástól tanulnak. Az első két óvodánk alakult meg ekkor, és bár hatalmas elánnal vágtunk bele, tudtuk, hogy ez még csak a bemelegítés. Mire egy évvel később már az első iskoláskorú gyerekeket is fogadtuk új helyszíneinken, úgy éreztük, hogy soha ilyen intenzíven nem tanultunk, mint a Budapest School megalapításának első évében. És ez az érzésünk azóta is elkísér.

Tanuló szervezetként tekintünk a Budapest Schoolra több okból is. Az iskolák feladata globálisan átalakulóban van. A tantárgyi és tudásalapú tanulási tartalmak mellett egyre nagyobb szerepet kap a mai világ aktuális problémáit feszegető kérdések megválaszolása. Komplex problémák megoldásához pedig a tantárgyi gondolkodáson túlmutató, interdiszciplináris szemléletre van szükség. Ilyen komplex probléma lehet például egy 21. századi iskola alapítása és működtetése. Ahogyan nekünk, iskolaapítóknak, az iskola kereteinek megalkotása jelent komplex kihívást, úgy a benne lévő gyerekek számára pedig az, hogy ezt további tartalommal feltöltsék. Vagyis önvezérelt módon és személyre szabottan tanulhassanak. Az egyén azonban pusztán a közösség kontextusában teljesedhet ki. Miközben a világ az egyén szabadsága felé tolja az embert, egyre nő a közösségben fejlődés, az egymástól tanulás, az együtt alkotás jelentősége is. Csak együtt leszünk képesek a világ technológiai, kulturális és kommunikációs változásaihoz folyamatosan igazodni. Tanuló szervezetként tekintünk ezért önmagunkra a közösségben lét felől is. Az iskola számunkra egy olyan hely, ahol örömmel tanulhatunk másoktól és egymástól, ahol a tanár, a szülő és a gyerek együtt tesz a fejlődésért. Tanulunk magunkról, a világról és a kapcsolatainkról.

Ebben a könyvben azt mutatjuk be, amit az elmúlt években a tanulásról megtanultunk. Annak az útnak a tapasztalatait összegezzük, amely során az álomból az első csoportjaink létrejöttek, majd újabbak és újabbak alakultak. Ahogy nőtt a Budapest School, úgy kezdtük egyre jobban megérteni, mit is csinálunk és miként kell ezt a tanárok, a családok, a gyerekek, valamint a társadalom és a jogalkotó elvárásaihoz igazítanunk. 

Az elmúlt két évben azon dolgoztunk, hogy magántanulók közösségéből államilag elfogadott iskolává válhasson a Budapest School. Arra a kérdésre kerestük a választ, hogy létrehozhatunk-e egy személyre szabott, önvezérelt tanulási modellt úgy, hogy az mindenben megfeleljen a törvényi előírásoknak. Hogy a gyerekek biztonságosan felkészülhessenek az érettségire, hogy biztosítsuk az átjárhatóságot az ország valamennyi iskolájával és közben egyéni céljaik mentén is fejlődjenek. Ezért írtunk egy \emph{kerettantervet}, ami arra a kérdésre válaszol, hogy mit tanulnak a gyerekek. Amikor ezzel elkészültünk, leírtuk azt is, hogyan tanulnak, és mire a kérdés minden részlete kibomlott előttünk, elkészült a \emph{pedagógiai programunk} is. Menet közben ráébredtünk, hogy a mit és a hogyan kérdése közötti határok jóval elmósódottabbak, mint azt elsőre gondolnánk. A Budapest Schoolba járó gyerekek ugyanis éppannyira dönthetnek a tartalomról, mint amennyire formálói lehetnek a mindennapok működésének is. A komplex egyedi kerettantervek elfogadásának lehetősége 2019-ben egy törvénymódosítás hatására hirtelen megszűnt, és a jogalkotó  \emph{egyedi megoldások alkalmazására} adott felhatalmazást. Így egy újabb jogi eljárást kezdeményeztünk miközben tovább csiszoltuk a programunkat. Ennek hatására pedig még inkább megértettük, miben is nyújthat újat a Budapest School tanuló iskolaként. Ahogy a jogalkotó is megváltoztatja a törvényeket, hogy szándéka szerint jobbá tegye az állam működését, úgy egy tanuló szervezet is folyamatosan újraírja a saját szabályait igazodva a körülményekhez. Ebben a könyvben a hogyan és mit tanulnak a gyerekek kérdésekre adott válaszainkat gyúrtuk egybe.       


\bigskip
{
\raggedright Halácsy Péter és Halmos Ádám\\
\raggedright 2019. augusztus
}