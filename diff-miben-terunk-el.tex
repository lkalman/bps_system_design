\documentclass{article}
%DIF LATEXDIFF DIFFERENCE FILE
%DIF DEL build.05.11/miben-terunk-el.tex   Mon May 11 21:46:53 2020
%DIF ADD build/miben-terunk-el.tex         Mon May 11 23:21:34 2020
\usepackage{xr}
\usepackage{hyperref}
\usepackage[magyar]{babel}
\usepackage[T1]{fontenc}
\usepackage[utf8]{inputenc}
\usepackage{polyglossia}
\usepackage{url}
\usepackage{hyperref}
\hypersetup{
    colorlinks=true,
    linkcolor=black,
    filecolor=black,
    urlcolor=black,
    anchorcolor=black,
    citecolor=black,
    filecolor=black,
    pdftitle={BPS modell},
    bookmarks=true,
    pdfpagemode=FullScreen,
}

\providecommand{\tightlist}{%
  \setlength{\itemsep}{0pt}\setlength{\parskip}{0pt}}

  \usepackage{graphicx,grffile}
  \makeatletter
  \def\maxwidth{\ifdim\Gin@nat@width>\linewidth\linewidth\else\Gin@nat@width\fi}
  \def\maxheight{\ifdim\Gin@nat@height>\textheight\textheight\else\Gin@nat@height\fi}
  \makeatother
  % Scale images if necessary, so that they will not overflow the page
  % margins by default, and it is still possible to overwrite the defaults
  % using explicit options in \includegraphics[width, height, ...]{}
  \setkeys{Gin}{width=\maxwidth,height=\maxheight,keepaspectratio}

  \usepackage{longtable,booktabs}

\setdefaultlanguage{magyar}
\externaldocument{egyedi_program}
\renewcommand{\thesection}{}
\makeatletter
\renewcommand\thesection{}
\renewcommand\thesubsection{\@arabic\c@subsection}
\makeatother
%DIF PREAMBLE EXTENSION ADDED BY LATEXDIFF
%DIF UNDERLINE PREAMBLE %DIF PREAMBLE
\RequirePackage[normalem]{ulem} %DIF PREAMBLE
\RequirePackage{color}\definecolor{RED}{rgb}{1,0,0}\definecolor{BLUE}{rgb}{0,0,1} %DIF PREAMBLE
\providecommand{\DIFaddtex}[1]{{\protect\color{blue}\uwave{#1}}} %DIF PREAMBLE
\providecommand{\DIFdeltex}[1]{{\protect\color{red}\sout{#1}}}                      %DIF PREAMBLE
%DIF SAFE PREAMBLE %DIF PREAMBLE
\providecommand{\DIFaddbegin}{} %DIF PREAMBLE
\providecommand{\DIFaddend}{} %DIF PREAMBLE
\providecommand{\DIFdelbegin}{} %DIF PREAMBLE
\providecommand{\DIFdelend}{} %DIF PREAMBLE
%DIF FLOATSAFE PREAMBLE %DIF PREAMBLE
\providecommand{\DIFaddFL}[1]{\DIFadd{#1}} %DIF PREAMBLE
\providecommand{\DIFdelFL}[1]{\DIFdel{#1}} %DIF PREAMBLE
\providecommand{\DIFaddbeginFL}{} %DIF PREAMBLE
\providecommand{\DIFaddendFL}{} %DIF PREAMBLE
\providecommand{\DIFdelbeginFL}{} %DIF PREAMBLE
\providecommand{\DIFdelendFL}{} %DIF PREAMBLE
%DIF HYPERREF PREAMBLE %DIF PREAMBLE
\providecommand{\DIFadd}[1]{\texorpdfstring{\DIFaddtex{#1}}{#1}} %DIF PREAMBLE
\providecommand{\DIFdel}[1]{\texorpdfstring{\DIFdeltex{#1}}{}} %DIF PREAMBLE
\newcommand{\DIFscaledelfig}{0.5}
%DIF HIGHLIGHTGRAPHICS PREAMBLE %DIF PREAMBLE
\RequirePackage{settobox} %DIF PREAMBLE
\RequirePackage{letltxmacro} %DIF PREAMBLE
\newsavebox{\DIFdelgraphicsbox} %DIF PREAMBLE
\newlength{\DIFdelgraphicswidth} %DIF PREAMBLE
\newlength{\DIFdelgraphicsheight} %DIF PREAMBLE
% store original definition of \includegraphics %DIF PREAMBLE
\LetLtxMacro{\DIFOincludegraphics}{\includegraphics} %DIF PREAMBLE
\newcommand{\DIFaddincludegraphics}[2][]{{\color{blue}\fbox{\DIFOincludegraphics[#1]{#2}}}} %DIF PREAMBLE
\newcommand{\DIFdelincludegraphics}[2][]{% %DIF PREAMBLE
\sbox{\DIFdelgraphicsbox}{\DIFOincludegraphics[#1]{#2}}% %DIF PREAMBLE
\settoboxwidth{\DIFdelgraphicswidth}{\DIFdelgraphicsbox} %DIF PREAMBLE
\settoboxtotalheight{\DIFdelgraphicsheight}{\DIFdelgraphicsbox} %DIF PREAMBLE
\scalebox{\DIFscaledelfig}{% %DIF PREAMBLE
\parbox[b]{\DIFdelgraphicswidth}{\usebox{\DIFdelgraphicsbox}\\[-\baselineskip] \rule{\DIFdelgraphicswidth}{0em}}\llap{\resizebox{\DIFdelgraphicswidth}{\DIFdelgraphicsheight}{% %DIF PREAMBLE
\setlength{\unitlength}{\DIFdelgraphicswidth}% %DIF PREAMBLE
\begin{picture}(1,1)% %DIF PREAMBLE
\thicklines\linethickness{2pt} %DIF PREAMBLE
{\color[rgb]{1,0,0}\put(0,0){\framebox(1,1){}}}% %DIF PREAMBLE
{\color[rgb]{1,0,0}\put(0,0){\line( 1,1){1}}}% %DIF PREAMBLE
{\color[rgb]{1,0,0}\put(0,1){\line(1,-1){1}}}% %DIF PREAMBLE
\end{picture}% %DIF PREAMBLE
}\hspace*{3pt}}} %DIF PREAMBLE
} %DIF PREAMBLE
\LetLtxMacro{\DIFOaddbegin}{\DIFaddbegin} %DIF PREAMBLE
\LetLtxMacro{\DIFOaddend}{\DIFaddend} %DIF PREAMBLE
\LetLtxMacro{\DIFOdelbegin}{\DIFdelbegin} %DIF PREAMBLE
\LetLtxMacro{\DIFOdelend}{\DIFdelend} %DIF PREAMBLE
\DeclareRobustCommand{\DIFaddbegin}{\DIFOaddbegin \let\includegraphics\DIFaddincludegraphics} %DIF PREAMBLE
\DeclareRobustCommand{\DIFaddend}{\DIFOaddend \let\includegraphics\DIFOincludegraphics} %DIF PREAMBLE
\DeclareRobustCommand{\DIFdelbegin}{\DIFOdelbegin \let\includegraphics\DIFdelincludegraphics} %DIF PREAMBLE
\DeclareRobustCommand{\DIFdelend}{\DIFOaddend \let\includegraphics\DIFOincludegraphics} %DIF PREAMBLE
\LetLtxMacro{\DIFOaddbeginFL}{\DIFaddbeginFL} %DIF PREAMBLE
\LetLtxMacro{\DIFOaddendFL}{\DIFaddendFL} %DIF PREAMBLE
\LetLtxMacro{\DIFOdelbeginFL}{\DIFdelbeginFL} %DIF PREAMBLE
\LetLtxMacro{\DIFOdelendFL}{\DIFdelendFL} %DIF PREAMBLE
\DeclareRobustCommand{\DIFaddbeginFL}{\DIFOaddbeginFL \let\includegraphics\DIFaddincludegraphics} %DIF PREAMBLE
\DeclareRobustCommand{\DIFaddendFL}{\DIFOaddendFL \let\includegraphics\DIFOincludegraphics} %DIF PREAMBLE
\DeclareRobustCommand{\DIFdelbeginFL}{\DIFOdelbeginFL \let\includegraphics\DIFdelincludegraphics} %DIF PREAMBLE
\DeclareRobustCommand{\DIFdelendFL}{\DIFOaddendFL \let\includegraphics\DIFOincludegraphics} %DIF PREAMBLE
%DIF END PREAMBLE EXTENSION ADDED BY LATEXDIFF

\begin{document} 
\noindent
Emberi Erőforrások Minisztériuma \\
részére
\vspace{0.75cm}

\hypertarget{elteresek-reszletes-bemutatasa}{%
\section{Eltérések részletes
bemutatása}\label{elteresek-reszletes-bemutatasa}}

Alábbiakban részletes bemutatjuk, hogy a 2011. évi CXC. törvény 9. § (8)
értelmében mennyiben és miben tér el a Budapest School programja a
Nemzeti alaptantervben (NAT) és az oktatásért felelős miniszter által
közzétett kerettantervekben leírtaktól és más jogszabályokban,
szabályozókban foglaltaktól.

\hypertarget{neveles-oktatas-tartalmi-kerdesei}{%
\subsection{Nevelés-oktatás tartalmi
kérdései}\label{neveles-oktatas-tartalmi-kerdesei}}

Az iskola programja a Nemzeti alaptanterv tantárgyait veszi alapul,
tehát nem tér el a Nat-ban és az oktatásért felelős miniszter által
kiadott kerettantervben foglalt \emph{tantárgyi struktúrától}.

Az is belátható, hogy a helyi tanterv a Nemzeti alaptanterv és az
oktatásért felelős miniszter által közzétett kerettantervekben
meghatározott \emph{ismeretanyagot teljes mértékben tartalmazza}, hiszen
azok tanulási eredményeit száz százalékban átvette.

Óraszámokban nincs eltérés a helyi tanterv és a Nat óraszám ajánlásai
között, kivéve hogy

\begin{enumerate}
\def\labelenumi{\arabic{enumi}.}
\tightlist
\item
  1-4. évfolyamon megjelenik a kötelező \emph{Mentoridő} heti egy
  órában;
\item
  5-12. évfolyamon a \emph{Közösségi nevelés (osztályfőnöki)} helyébe
  heti egy órában a mentoridő lép;
\end{enumerate}

Minden gyerek egy órát hetente a mentorával tölt\DIFaddbegin \DIFadd{, amikor a mentor a
gyerek számára egyéni, minőségi figyelmet ad}\DIFaddend . A mentor ilyenkor a
gyereket segíti a saját céljainak megfogalmazásában és hogy legyen
\DIFdelbegin \DIFdel{a
gyereknek }\DIFdelend lehetősége reflektálni a saját fejlődésére. Ez az óra a gyerekek és a
mentorok számára kötelező foglalkozás.

\DIFdelbegin \DIFdel{A mentoridő egyéni vagy maximum 12 fős csoportos foglalkozás. }\DIFdelend Ez azt is eredményezi, hogy az iskola átlagban magasabb tanár/gyerek
aránnyal tudja csak elvégezni az egyedi pedagógia programban előírtakat
mert nem csak tanítókra és szaktanárokra van szüksége, hanem mentorokra
is.

\hypertarget{a-vezetesi-modell}{%
\subsection{A vezetési modell}\label{a-vezetesi-modell}}

Az intézmény vezetését intézményvezető látja el, amely szerepkör
ellátása elsősorban a tanügyigazgatási rendszer követelményeinek való
megfelelés miatt szükséges. A Budapest Schoolban az intézményvezető nem
köteles órát, foglalkozást tartani.

Az iskola belső működéséből és szervezeti kultúrájából következik, hogy
iskolában intézményvezető-helyettes megbízása nem szükséges. Az
intézményvezető munkája a gyerekek létszámával nem növekszik, feladatai
elvégzésében segítik más tanárok, megbízott szakértők, és a fenntartó
is.

\DIFdelbegin \DIFdel{Amennyiben az intézményvezető akadályoztatva van, vagy munkaköre
megürült, a fenntartó jogosult olyan személy ideiglenes megbízására, aki
pedagógus végzettséggel és legalább 2 éves, igazolt vezetői
tapasztalattal rendelkezik. A fenntartónak ebben az esetben azonnal ki
kell nevezni az ideiglenes intézményvezetőt, biztosítva, hogy az iskola
egy nap se lehessen kinevezett vezető nélkül.
}%DIFDELCMD < 

%DIFDELCMD < %%%
\DIFdelend \hypertarget{a-neveles--es-oktatasszervezesi-modszerek}{%
\subsection{A nevelés- és oktatásszervezési
módszerek}\label{a-neveles--es-oktatasszervezesi-modszerek}}

\hypertarget{mentor}{%
\subsubsection{Mentor}\label{mentor}}

Osztályfőnök feladatkörét az adott gyerek tekintetében a mentor látja
el.

\hypertarget{folyamatos-tervezes-megvalositas-meres-fejlesztes}{%
\subsubsection{Folyamatos tervezés, megvalósítás, mérés,
fejlesztés}\label{folyamatos-tervezes-megvalositas-meres-fejlesztes}}

A program a tanárok autonómiájára kívánja bízni, hogy a gyerekek
tanulási céljai, képességei alapján trimeszterenként alakítsák ki

\begin{enumerate}
\def\labelenumi{\arabic{enumi}.}
\tightlist
\item
  a foglalkozások rendjét és idejét;
\item
  a csoportbontások rendszerét, a konkrétan kialakításra került
  csoportok összetételét, a bennük résztvevő tanulók nevét;
\item
  az aktív tanulási alapelvek szerint szerveződő, több tantárgy,
  tanulási terület ismereteinek integrálását igénylő témákat,
  jelenségeket; feldolgozó tanórák, foglalkozások, témanapok, témahetek,
  tematikus hetek és projektek alkalmazását;ű
\item
  20/2012. (VIII. 31.) EMMI 12.§ (1-3) pontjaiban említett, a rendelet
  szerint a helyi tanterv kompentenciájába tartozó,

  \begin{enumerate}
  \def\labelenumii{\arabic{enumii}.}
  \tightlist
  \item
    a gyerekek, az egyes évfolyamok, ezen belül az egyes osztályok,
    valamint az osztályokon belüli csoportok tanítási óráit;
  \item
    különböző évfolyamok, különböző osztályok tanulóiból álló csoportok
    kialakításának rendjét;
  \item
    a kötelező, kötelezően választandó és szabadon választható
    foglalkozások rendszerét;
  \item
    azon kötelező tanítási foglalkozásokat, amelyeken egy adott osztály
    valamennyi tanulója köteles részt venni;
  \item
    azon kötelező tanítási foglalkozásokat, amelyeken a gyerekeknek a
    választásra felkínált tantárgyak közül kötelezően választva, az a
    tanárok által megadott óraszámban kell résztvenni.
  \end{enumerate}
\item
  a tantárgyi blokkosítások, és tömbösítések rendjét;
\end{enumerate}

\hypertarget{ertekeles}{%
\subsubsection{Értékelés}\label{ertekeles}}

\hypertarget{erdemjegyek-helyett-strukturalt-szoveges-visszajelzes-es-tanulasi-eredmenyek-elszamolasa}{%
\paragraph{Érdemjegyek helyett struktúrált szöveges visszajelzés és
tanulási eredmények
elszámolása}\label{erdemjegyek-helyett-strukturalt-szoveges-visszajelzes-es-tanulasi-eredmenyek-elszamolasa}}

A BPS iskolában az érdemjegyek helyett a gyerekek folyamatosan kapnak
visszajelzést teljesítményükről és előrehaladásukról:

\begin{enumerate}
\def\labelenumi{\arabic{enumi}.}
\item
  Tanulási egységek, modulok, foglalkozásák, szakaszok, feladatok,
  projektek végén értékelő a tanárok értékelő táblázatokat (angolul
  rubric) alkalmaznak. Az értékelő táblázatban szerepelnek az értékelés
  szempontjai és szempontonkénti szintek, rövid leírásokkal. A
  táblázatok formája minden visszajelzés esetén (értsd
  foglalkozásonként, trimeszterenként, célonként) változtatható. Az
  értékelőtáblázat tekinthető struktúrált szöveges értékelésnek.
\item
  A gyerekek folyamatosan „gyűjtik'' a tanulási eredményeket, ami
  folyamatosan visszajelzést ad nekik arról, hogy az évfolyamhoz tartozó
  tantárgyi követelmények hány százalékánál tartanak.
\end{enumerate}

Ez alapján tanév közben nemcsak a gyerek, hanem tanárai és családja is
mindig nyomon követheti, hogy hol tart a gyerek és mit kell tenni azért,
hogy a tanév végére elérje a az évfolyamhoz tartozó követelményeket.

\hypertarget{osztalyozas}{%
\paragraph{Osztályozás}\label{osztalyozas}}

A gyerekek portfóliója alapján egyértelműen megállapítható a
teljesítésre került tanulási eredmények. Egy tanulási eredményt vagy
teljesített a gyerek, vagy nem, részben teljesíteni nem lehetséges.

Osztályzatot félévkor és év végén kapnak a gyerekek valamennyi
tantárgyból, ahogy az Nkt. előírja. Az osztályozás egyedi módja a
következő:

\begin{itemize}
\tightlist
\item
  az elért tanulási eredmények száma: a gyerek által adott
  félévben/évben teljesített tanulási eredményeinek számossága
\item
  a félévre/évre számított összes tanulási eredmény: a Nat. adott
  tantárgyhoz rögzített valamennyi tanulási eredményének száma, osztva
  az adott időszak számosságával (tehát pl. etika tantárgyban a NAT 1-4.
  osztályra 50 tanulási eredményt ír elő, az egy félévre számított
  összes tanulási eredmény 7)
\item
  az elért tanulási eredmények számát elosztjuk a félévre/évre számított
  összes tanulási eredmény számával, és megszorozzuk 100-zal
\item
  a kapott százalékos mutatót az alábbi táblázat alapján osztályzattá
  alakítjuk:
\end{itemize}

\begin{longtable}[]{@{}ll@{}}
\toprule
Teljesítmény & Osztályzat\tabularnewline
\midrule
\endhead
0 - 40\% & 1 (elégtelen)\tabularnewline
41 - 50\% & 2 (elégséges)\tabularnewline
51 - 60\% & 3 (közepes)\tabularnewline
61 - 70\% & 4 (jó)\tabularnewline
71 - 100\% & 5 (példás)\tabularnewline
\bottomrule
\end{longtable}

\hypertarget{magatartas-es-szorgalom-szoveges-ertekelese}{%
\paragraph{Magatartás és szorgalom szöveges
értékelése}\label{magatartas-es-szorgalom-szoveges-ertekelese}}

Az Ntk. 54. § (1) előírja a tanuló magatartásának és szorgalmának
értékelését és minősítését. Ezt a gyerek mentora trimeszterenként
szövegesen vagy értékelő táblázatokban végzi el. Tehát nem az
osztályfőnök végzi és nem érdemjegyekkel és osztályzatokkal.

\hypertarget{a-minosegpolitika-a-minoseggondozas-rendszere}{%
\subsection{A minőségpolitika, a minőséggondozás
rendszere}\label{a-minosegpolitika-a-minoseggondozas-rendszere}}

Az iskola és fenntartója a teljes szervezet minőségfejlesztését
PDCA-ciklusként (plan - tervezés, do -- cselekvés, check -- ellenőrzés,
act - beavatkozás) végzi. A gyerekek élményét leginkább befolyásoló
foglalkozások tervezése trimeszterenként történik, a többi esetben a
jogszabályok által megengedett legrövidebb csiklust alkalmazza az
iskola. Ellenőrzésként több metrikát vesz figyelembe a folyamat, de
mindenképp figyelembe kell venni a következőket:

\begin{enumerate}
\def\labelenumi{\arabic{enumi}.}
\tightlist
\item
  A szülő, a gyerek és a tanár közötti saját célokat megfogalmazó hármas
  megállapodások időben megszülettek, nincs olyan gyerek, akinek nincs
  elfogadott saját tanulási célja. Indikátor: elkészült szerződések
  száma.
\item
  A modulok végén a portfóliók bővülnek, és azok tartalma a
  tantárgyakhoz kapcsolódik. Indikátor: a portfólió elemeinek száma és
  kapcsolhatósága.
\item
  A tanulási eredményekre vonatkozó megkötések időben teljesülnek.
  Indikátor: elért tanulási eredmények száma gyerekekre bontva.
\item
  A szülők biztonságban érzik gyereküket, és eleget tudnak arról, hogy
  mit tanulnak. Indikátor: kérdőíves vizsgálat alapján.
\item
  A tanárok hatékonynak tartják a munkájukat. Indikátor: kérdőíves
  felmérés alapján.
\item
  A gyerekek úgy érzik, folyamatosan tanulnak, támogatva vannak, vannak
  kihívásaik. Indikátor: kérdőíves felmérés alapján.
\end{enumerate}

\hypertarget{a-tanulo-heti-kotelezo-oraszama}{%
\subsection{A tanuló heti kötelező
óraszáma}\label{a-tanulo-heti-kotelezo-oraszama}}

A NAT óraszámait tartjuk, a jogszabályokban előírtak szerint.

\hypertarget{a-pedagogus-vegzettseg-es--szakkepzettseg}{%
\subsection{A pedagógus végzettség és
-szakképzettség}\label{a-pedagogus-vegzettseg-es--szakkepzettseg}}

A BPS iskola egyedisége, hogy mentorok pedagógus munkakörben alkalmaz. A
program a mentorok esetén kötelezően előírja, hogy csak olyan felnőtt
alkalmazható mentorként, aki a köznevelési jogszabályok által pedagógus
munkakörben alkalmazható és elvégezte a BPS saját 30 órás
mentorképzését.

\hypertarget{a-mukodeshez-szukseges-feltetelek}{%
\subsection{A működéshez szükséges
feltételek}\label{a-mukodeshez-szukseges-feltetelek}}

\DIFdelbegin %DIFDELCMD < \hypertarget{tobb-telephely-egy-intezmeny-nincs-tagintezmeny}{%
%DIFDELCMD < \subsubsection{Több telephely, egy intézmény, nincs
%DIFDELCMD < tagintézmény}%DIFDELCMD < \label{tobb-telephely-egy-intezmeny-nincs-tagintezmeny}%%%
}
%DIFDELCMD < %%%
\DIFdelend \DIFaddbegin \hypertarget{helyisegek}{%
\subsection{Helyiségek}\label{helyisegek}}
\DIFaddend 

A \DIFdelbegin \DIFdel{Budapest School több telephellyel rendelkezik. Minden telephely egy
intézményhez tartozik, az egységes iskola működéséhez nincs szükség
tagintézmények létrehozására (azaz a BPS nem hoz létre tagintézményeket,
kizárólag telephelyeket), tekintettel arra, hogy az iskolák szervezeti
irányítása a BPS modell és azt leképező szoftverrendszer segítségével
megoldható anélkül, hogy az egyes helyszíneket telephelyek helyett
tagintézményként lenne szükséges működtetni.
}%DIFDELCMD < 

%DIFDELCMD < %%%
\DIFdel{A telephelyek kialakíthatók nem nevelési, oktatási ingatlanokban.
}%DIFDELCMD < 

%DIFDELCMD < \hypertarget{helyisegek}{%
%DIFDELCMD < \subsubsection{Helyiségek}%DIFDELCMD < \label{helyisegek}%%%
}
%DIFDELCMD < 

%DIFDELCMD < %%%
\DIFdel{A }\DIFdelend telephelyeken található helyiségeket az iskola a 20/2012.(VIII.31.)
EMMI rendelet 2.számú mellékletében meghatározott felhatalmazás alapján
alakítja ki: \emph{Az eltérő pedagógiai elveket tartalmazó nevelési
program az eszköz- és felszerelési jegyzéktől eltérően határozhatja meg
a nevelőmunka eszköz és felszerelési feltételeit.}

BPS modellben részletezett strukturális, szervezeti és tanulásszervezési
módok miatt az iskola az iskolában az alábbi helyiségek megléte
szükséges és elégséges:

\begin{longtable}[]{@{}lll@{}}
\toprule
\begin{minipage}[b]{0.13\columnwidth}\raggedright
\textbf{helyiség megnevezése}\strut
\end{minipage} & \begin{minipage}[b]{0.26\columnwidth}\raggedright
\textbf{mennyiségi mutató}\strut
\end{minipage} & \begin{minipage}[b]{0.51\columnwidth}\raggedright
\textbf{megjegyzés}\strut
\end{minipage}\tabularnewline
\midrule
\endhead
\begin{minipage}[t]{0.13\columnwidth}\raggedright
tanterem (1)\strut
\end{minipage} & \begin{minipage}[t]{0.26\columnwidth}\raggedright
16 gyerekenként egy\strut
\end{minipage} & \begin{minipage}[t]{0.51\columnwidth}\raggedright
1,5 nm / gyerek előírás figyelembevétel\strut
\end{minipage}\tabularnewline
\begin{minipage}[t]{0.13\columnwidth}\raggedright
laboratóriumok és szertárak (2)\strut
\end{minipage} & \begin{minipage}[t]{0.26\columnwidth}\raggedright
iskolánként egy\strut
\end{minipage} & \begin{minipage}[t]{0.51\columnwidth}\raggedright
laboratóriumi tevékenység végzésére alkalmas létesítmény üzemeltetőjével
kötött megállapodással is teljesíthető\strut
\end{minipage}\tabularnewline
\begin{minipage}[t]{0.13\columnwidth}\raggedright
tornaterem\strut
\end{minipage} & \begin{minipage}[t]{0.26\columnwidth}\raggedright
iskolánként egy\strut
\end{minipage} & \begin{minipage}[t]{0.51\columnwidth}\raggedright
kiváltható szerződéssel\strut
\end{minipage}\tabularnewline
\begin{minipage}[t]{0.13\columnwidth}\raggedright
tornaszoba\strut
\end{minipage} & \begin{minipage}[t]{0.26\columnwidth}\raggedright
székhelyen, telephelyeken egy\strut
\end{minipage} & \begin{minipage}[t]{0.51\columnwidth}\raggedright
csak ha ha a székhelynek v.telephelynek nincs saját tornaterme és akkor
is kiváltható szerződéssel\strut
\end{minipage}\tabularnewline
\begin{minipage}[t]{0.13\columnwidth}\raggedright
sportudvar\strut
\end{minipage} & \begin{minipage}[t]{0.26\columnwidth}\raggedright
székhelyen, telephelyeken egy\strut
\end{minipage} & \begin{minipage}[t]{0.51\columnwidth}\raggedright
kiváltható szerződéssel, vagy helyettesíthető alkalmas szabad
területtel\strut
\end{minipage}\tabularnewline
\begin{minipage}[t]{0.13\columnwidth}\raggedright
sportszertár (4)\strut
\end{minipage} & \begin{minipage}[t]{0.26\columnwidth}\raggedright
székhelyen, telephelyeken egy\strut
\end{minipage} & \begin{minipage}[t]{0.51\columnwidth}\raggedright
tornateremhez kapcsolódóan (kiváltható szerződéssel)\strut
\end{minipage}\tabularnewline
\begin{minipage}[t]{0.13\columnwidth}\raggedright
általános szertár (4)\strut
\end{minipage} & \begin{minipage}[t]{0.26\columnwidth}\raggedright
székhelyen, telephelyeken egy\strut
\end{minipage} & \begin{minipage}[t]{0.51\columnwidth}\raggedright
tornateremhez kapcsolódóan (kiváltható szerződéssel)\strut
\end{minipage}\tabularnewline
\begin{minipage}[t]{0.13\columnwidth}\raggedright
iskolatitkári iroda (5)\strut
\end{minipage} & \begin{minipage}[t]{0.26\columnwidth}\raggedright
iskolánként egy\strut
\end{minipage} & \begin{minipage}[t]{0.51\columnwidth}\raggedright
teljesen külön épületben, irodában is kialakítható\strut
\end{minipage}\tabularnewline
\begin{minipage}[t]{0.13\columnwidth}\raggedright
nevelőtestületi szoba\strut
\end{minipage} & \begin{minipage}[t]{0.26\columnwidth}\raggedright
iskolánként (telephelyenként) egy\strut
\end{minipage} & \begin{minipage}[t]{0.51\columnwidth}\raggedright
\strut
\end{minipage}\tabularnewline
\begin{minipage}[t]{0.13\columnwidth}\raggedright
könyvtár\strut
\end{minipage} & \begin{minipage}[t]{0.26\columnwidth}\raggedright
iskolánként egy\strut
\end{minipage} & \begin{minipage}[t]{0.51\columnwidth}\raggedright
nyilvános könyvtár elláthatja a funkcót, megállapodás alapján\strut
\end{minipage}\tabularnewline
\begin{minipage}[t]{0.13\columnwidth}\raggedright
orvosi szoba\strut
\end{minipage} & \begin{minipage}[t]{0.26\columnwidth}\raggedright
iskolánként egy\strut
\end{minipage} & \begin{minipage}[t]{0.51\columnwidth}\raggedright
amennyiben külső egészségügyi intézményben a gyerekek ellátása
megoldható, nem kötelező; lehet ideiglenesen kialakított\strut
\end{minipage}\tabularnewline
\begin{minipage}[t]{0.13\columnwidth}\raggedright
ebédlő (6)\strut
\end{minipage} & \begin{minipage}[t]{0.26\columnwidth}\raggedright
székhelyen, telephelyeken egy\strut
\end{minipage} & \begin{minipage}[t]{0.51\columnwidth}\raggedright
gyerek- és felnőttétkező közös helyiségben; tanteremmel közös helységben
is kialakítható, de egyidőben csak egy funkció\strut
\end{minipage}\tabularnewline
\begin{minipage}[t]{0.13\columnwidth}\raggedright
főzőkonyha\strut
\end{minipage} & \begin{minipage}[t]{0.26\columnwidth}\raggedright
székhelyen, telephelyeken egy\strut
\end{minipage} & \begin{minipage}[t]{0.51\columnwidth}\raggedright
ha helyben főznek\strut
\end{minipage}\tabularnewline
\begin{minipage}[t]{0.13\columnwidth}\raggedright
melegítőkonyha (7)\strut
\end{minipage} & \begin{minipage}[t]{0.26\columnwidth}\raggedright
székhelyen, telephelyeken egy\strut
\end{minipage} & \begin{minipage}[t]{0.51\columnwidth}\raggedright
ha nem helyben főznek, de helyben étkeznek\strut
\end{minipage}\tabularnewline
\begin{minipage}[t]{0.13\columnwidth}\raggedright
tálaló-mosogató\strut
\end{minipage} & \begin{minipage}[t]{0.26\columnwidth}\raggedright
székhelyen, telephelyeken egy\strut
\end{minipage} & \begin{minipage}[t]{0.51\columnwidth}\raggedright
ha nem helyben főznek, de helyben étkeznek; melegítőkonyhával közös
helyiségben is kialakítható\strut
\end{minipage}\tabularnewline
\begin{minipage}[t]{0.13\columnwidth}\raggedright
szárazáru raktár\strut
\end{minipage} & \begin{minipage}[t]{0.26\columnwidth}\raggedright
székhelyen, telephelyeken egy\strut
\end{minipage} & \begin{minipage}[t]{0.51\columnwidth}\raggedright
ha helyben főznek\strut
\end{minipage}\tabularnewline
\begin{minipage}[t]{0.13\columnwidth}\raggedright
földesáru raktár\strut
\end{minipage} & \begin{minipage}[t]{0.26\columnwidth}\raggedright
székhelyen, telephelyeken egy\strut
\end{minipage} & \begin{minipage}[t]{0.51\columnwidth}\raggedright
ha helyben főznek\strut
\end{minipage}\tabularnewline
\begin{minipage}[t]{0.13\columnwidth}\raggedright
éléskamra\strut
\end{minipage} & \begin{minipage}[t]{0.26\columnwidth}\raggedright
székhelyen, telephelyeken egy\strut
\end{minipage} & \begin{minipage}[t]{0.51\columnwidth}\raggedright
ha helyben főznek\strut
\end{minipage}\tabularnewline
\begin{minipage}[t]{0.13\columnwidth}\raggedright
élelmiszerhulladék-tároló (8)\strut
\end{minipage} & \begin{minipage}[t]{0.26\columnwidth}\raggedright
székhelyen, telephelyeken egy\strut
\end{minipage} & \begin{minipage}[t]{0.51\columnwidth}\raggedright
helyben étkeznek; nem kell külön helységben, ha biztonságosan
eltávolítható a hulladék\strut
\end{minipage}\tabularnewline
\begin{minipage}[t]{0.13\columnwidth}\raggedright
személyzeti WC\strut
\end{minipage} & \begin{minipage}[t]{0.26\columnwidth}\raggedright
székhelyen, telephelyeken egy\strut
\end{minipage} & \begin{minipage}[t]{0.51\columnwidth}\raggedright
ha fülkés, különálló WC kerül kialakításra, akkor a gyerek WC-t és
személyzeti WC-t nem kell megkülönbözteti\strut
\end{minipage}\tabularnewline
\begin{minipage}[t]{0.13\columnwidth}\raggedright
gyerek WC\strut
\end{minipage} & \begin{minipage}[t]{0.26\columnwidth}\raggedright
székhelyen, telephelyeken a gyereklétszám figyelembevételével\strut
\end{minipage} & \begin{minipage}[t]{0.51\columnwidth}\raggedright
ha fülkés, különálló WC kerül kialakításra, akkor nem kell elkülöníteni
a fiú és lány WC-t\strut
\end{minipage}\tabularnewline
\DIFaddbegin \begin{minipage}[t]{0.13\columnwidth}\raggedright
\DIFadd{mozgássérült WC}\strut
\end{minipage} & \begin{minipage}[t]{0.26\columnwidth}\raggedright
\DIFadd{a mozgássérült gyereklétszám figyelembevételével}\strut
\end{minipage} & \begin{minipage}[t]{0.51\columnwidth}\raggedright
\strut
\end{minipage}\tabularnewline
\DIFaddend \bottomrule
\end{longtable}

\begin{enumerate}
\def\labelenumi{\arabic{enumi}.}
\tightlist
\item
  Tanterem esetén az iskola azért nem osztályonként, hanem gyereklétszám
  arányában határozza meg a szükséges tantermek számát, mert
  osztálybontások esetén egy osztályra két tanterem jutna. Ezen felül a
  BPS iskolákban az osztályok gyerekszáma tipikus esetben 10 fő alatti,
  így a tantermek számosságának meghatározására a gyerekek száma a
  megfelelőbb mérőszám.
\item
  Amennyiben a 9-12. évfolyam tanulási eredményeinek eléréséhez
  szükséges laboratóriumi tevékenységek az iskolában található
  eszközökkel és termekben nem végezhetők el, az iskola és a fenntartó
  is kötelességet vállal arra, hogy a tanulási eredmény eléréséhez
  szükséges tárgyi feltételeket megállapodás, szerződés alapján
  biztosítsa.
\item
  A székhely és a telephelyek általánostól eltérően nagyobb száma és
  kisebb mérete más típusú intézmény vezetést tesz szükségessé, melynek
  része az egyes feladatellátási helyek rendszeres felkeresése. Az
  intézményvezető tanórát, foglalkozást nem, illetve minimális számban
  tart, így nem szükséges folyamatos jelenléte az adott székhelyen,
  telephelyen.
\item
  Az egyértelműség okán célszerű rögzíteni, hogy a sportszertár és
  általános szertár osztja a tornaterem sorsát.
\item
  A székhely és a telephelyek általánostól eltérően nagyobb száma és
  kisebb mérete miatt az adminisztráció működtetése célszerűen és
  hatékonyan megoldható a székhelyen kívül.
\item
  Technikailag megoldható hordozható hideg-meleg kézmosó alkalmazásával
  (népegészségügyi szerv által megerősített megoldás).
\item
  A melegítőkonyha funkciója az ebédlő részeként is kialakítható, ezen
  termek funkciója nem zárja ki az egy helyiségben történő kialakítást -
  különös tekintettel az Intézmény feladatellátási helyeinek
  általánostól eltérően nagyobb számára és kisebb méretére.
\item
  852/2004/EK rendelet alapján az élelmiszer-hulladékot, a nem ehető
  melléktermékeket és egyéb hulladékot a felgyülemlésük elkerülése
  érdekében a lehető leggyorsabban el kell távolítani azokból a
  helyiségekből, amelyekben élelmiszer található. A rendelet nem írja
  elő, hogy ezt külön helységben kell tárolni.
\end{enumerate}

A 20/2012. (VIII. 31.) EMMI rendelettől néhány terem esetén oly módon
kívánunk eltérni, hogy az adott terem ne legyen kötelező a
telephelyeken:

\begin{itemize}
\item
  \textbf{Aula}: Kevés gyereket befogadó épületben a legnagyobb tanterem
  általában elegendő a közösségi programokhoz. Nagyobb épületekben pedig
  a rendelet is kiválthatónak határozza meg.
\item
  \textbf{Csoportterem}: A tantermek csoportteremként is funkcionálnak.
  Tekintettel arra, hogy az iskolában az osztályok gyerekszáma tipikus
  esetben 10 fő alatti, és az egyes foglalkozásokra létrejött
  csoportokban sok esetben több osztály tanulói is részt vesznek, a
  tanulásra kialakított terek megkülönböztetése (tan-, csoport-,
  szakterem) nem szükséges.
\item
  \textbf{Egyéb raktár}: Amennyiben nem helyben étkeznek, a tanulói
  létszám feladatellátási helyenként alacsony mértéke okán nem szükséges
  egyéb helyiség kialakítása, közelebbről meg nem határozott tárgyak
  számára.
\item
  \textbf{Felnőtt étkező}: Gyermekvédelmi szempontból nem megoldható,
  hogy a gyerekek tanár jelenléte nélkül tartózkodjanak külön
  helyiségben az étkezés idejére. Ezért a felnőttek együtt étkeznek a
  gyerekkel.
\item
  \textbf{Intézményvezetői iroda}: A székhely és a telephelyek
  általánostól eltérően nagyobb száma és kisebb mérete más típusú
  intézmény vezetést tesz szükségessé, melynek része az egyes
  feladatellátási helyek rendszeres felkeresése. Az intézményvezető
  tanórát, foglalkozást nem, illetve minimális számban tart, így nem
  szükséges folyamatos jelenléte az adott székhelyen, telephelyen. Az
  iskolatitkári iroda olyan módon kerül kialakításra, hogy az az
  intézményvezető számára is használható.
\item
  \textbf{Iskolapszichológusi szoba}: Az iskolapszichológusi
  tevékenységek megoldhatók a tantermekben, különös tekintettel arra,
  hogy egész napos iskolaként működik az iskola.
\item
  \textbf{Logopédiai foglalkoztató, egyéni fejlesztő szoba}: A
  logopédiai foglalkozások megoldhatók a tantermekben, különös
  tekintettel arra, hogy egész napos iskolaként működik az iskola.
\item
  \textbf{Porta}: Funkciója, a gyerekek biztonságának biztosítása,
  szabályzatokkal, beléptető rendszerekkel megoldható.
\item
  \textbf{Szaktanterem a hozzá tartozó szertárral}: A szaktantermek
  funkciói a legtöbb esetben a tanteremben is megvalósíthatóak. Így
  számítástechnikai terem, társadalomtudományi szaktanterem, művészeti
  nevelés szaktanterem, technikai szaktanterem és gyakorló tanterem
  eszközigénye mobil eszközökkel is megoldható, úgy mint pl. laptop,
  rajztábla, satu, elektromos zongora. A természettudományi szaktanterem
  funkcióit pedig a szerződött laboratóriumokban elérhető a gyerekek
  számára. Amennyiben az adott tantárgyhoz, tantárgycsoporthoz tartozó
  tanulási eredmény az Intézményben található eszközökkel és termekben
  nem érhető el, az Intézmény és a fenntartó is kötelességet vállal
  arra, hogy a tanulási eredmény eléréséhez szükséges tárgyi
  feltételeket megállapodás, szerződés alapján biztosítsa.
\item
  \textbf{Személyzeti öltöző} és \textbf{személyzeti mosdó-zuhanyzó}: Az
  intézmény a tanárokon és óraadókon kívül állandó egyéb alkalmazottat
  egy feladatellátási helyszínen nem foglalkoztat. Az egyéb
  alkalmazottak (pl. karbantartó, takarító, kisegítők) ideiglenes, a
  feladatuk ellátásához szükséges ideig tartózkodnak az intézmény
  területén, ezért ilyen terem kialakítása szükségtelen.
\item
  \textbf{Teakonyha}: Tálaló-mosogató, ebédlő, és melegítőkonyha mellett
  külön teakonyha terem létesítése nem célszerű, az említett három
  ellátja a teakonyha funkcióját Intézményünkben, tekintettel az egy
  feladatellátási helyen egyszerre jelen lévő tanárok alacsony számára.
\item
  \textbf{Technikai alkalmazotti mosdó-zuhanyzó, WC helyiség}: Az
  intézmény a tanárokon és óraadókon kívül állandó egyéb alkalmazottat
  egy feladatellátási helyszínen nem foglalkoztat. Az egyéb
  alkalmazottak (pl. karbantartó, takarító, kisegítők) ideiglenes, a
  feladatuk ellátásához szükséges ideig tartózkodnak az intézmény
  területén, a tanári WC-t használata számukra elégséges.
\end{itemize}

\DIFdelbegin %DIFDELCMD < \hypertarget{helyisegek-butorzata-es-egyeb-berendezesi-targyai}{%
%DIFDELCMD < \subsubsection{Helyiségek bútorzata és egyéb berendezési
%DIFDELCMD < tárgyai}%DIFDELCMD < \label{helyisegek-butorzata-es-egyeb-berendezesi-targyai}%%%
}
%DIFDELCMD < %%%
\DIFdelend \DIFaddbegin \hypertarget{helyisegek-butorzata-es-egyeb-berendezesi-targyai}{%
\subsection{Helyiségek bútorzata és egyéb berendezési
tárgyai}\label{helyisegek-butorzata-es-egyeb-berendezesi-targyai}}
\DIFaddend 

A Budapest School telephelyein az alábbi feltételek megléte szükséges és
elégséges:

\begin{itemize}
\tightlist
\item
  fenntartónak stabil szélessávú internethozzáférést kell biztosítania
  minden telephelyen;
\item
  minden tanteremben minden órán elérhetőnek kell lennie legalább egy
  internethez kötött számítógépnek vagy tabletnek;
\item
  tanulói asztalok, székek a gyerekek számának megfelelően;
\item
  eszköztároló szekrény;
\item
  tábla vagy flipchart legalább három különböző színű tollal;
\item
  szeméttároló.
\end{itemize}

Amennyiben a tanároknak többletigénye merül fel, akkor a szervezeti és
működési szabályzatban lefektett módon tudják az iskola és a fenntartó
segítségét kérni.



\vspace{0.75cm}
\noindent
Tisztelettel,

\vspace{0.75cm}
\noindent
\begin{center}
      \begin{tabular}{p{8cm}}
            \begin{center}
                  \hrulefill \\
                  \textbf{MiSulink Nonprofit Kft.} \\
                  fenntartó\\
                  képviseletében: Halácsy Péter ügyvezető \\
            \end{center}
      \end{tabular}
\end{center}

\end{document}
