\hypertarget{operator-a-tanulokozossegek-alapitoja}{%
\section{Operátor, a tanulóközösségek
alapítója}\label{operator-a-tanulokozossegek-alapitoja}}

Vannak tanulóközösségek, amik szervezését, indítását egy \emph{alapító}
\emph{operátor} vagy \emph{operátorcsapat} kezdte el. Az operátorok a
létrehozók, az alkotók, az alapítók, akik minden nehézséget leküzdve a
semmiből létrehoznak egy önállóan működni képes tanulóközösséget.
Nélkülük biztos nem jött volna létre a tanulóközösség.

\hypertarget{az-operator-a-mindenes}{%
\subsection{Az operátor, a mindenes}\label{az-operator-a-mindenes}}

Kezdetben az operátor az egyszemélyes mindenes, hiszen egyedül kezdte a
tanulóközösség tanulás-szervezőjeként. Ő felel a pénzügyekért, a
tanárfelvételért, a csoportbontásokért, a tanulásszervezésért, a szülők
bevonásáért, a gyerekfelvételért és még a mosdókat is ki kell
tisztítani. Már, ha előtte elintéze, hogy legyenek mosdók. Az esetek
többségében ő találja, bérli ki, vagy vásárolja meg, újítja fel a tanuló
közösségnek otthont adó, később telephelyként működő épületet és dönt az
első tanár csatlakozásáról.

Az operátorok aktívak, tesznek, megoldanak, elintéznek, sokszor hirtelen
a Budapest School közösség képviselőjévé válnak.

\hypertarget{az-operator-a-tanulasszervezo-a-tanarcsapat-tagja}{%
\subsection{Az operátor a tanulásszervező, a tanárcsapat
tagja}\label{az-operator-a-tanulasszervezo-a-tanarcsapat-tagja}}

Amikor már megjelennek azok a tanárok, akik a tanulásszervezésért, a
tanulóközösség vezetésért felelnek, tehát a tanulásszervezők (mentorok,
mentortanárok és néhány kiemelt szaktanár), az operátor akkor is a
tanulásszervező csapat része marad. Nem átadja a tanulóközösséget a
tanulásszervezőknek, hanem elkezdik csapatban vezetni a közösséget.
Innentől kezdve a csapat minden tagja, így az operátor és minden
tanulásszervező be van vonva minden döntésbe: mindenki hozzájárulása
szükséges minden, a csapatot és a tanulóközösséget érintő kérdésben.

Ez azt is jelenti, hogy csak olyan döntést fogadhat el a csapat, ami
\emph{mindenkinek} elfogadható, amit mindenki \emph{„elég jónak és
biztonságosnak talál"}. Mindenki mondhatja azt, hogy egy javaslat nem
oké neki, szerinte inkább csinálja a csapat másképp. Ráadásul
javaslatokat mindenki hozhat. Ez azt jelenti, hogy mindenkinek lehet
hatása a működésre.

\begin{quote}
\emph{Példa: Anna, az egyik tanulásszervező javasolja, hogy
délutánonként kisebb csoportokba bontsák a gyerekeket, mert elfáradnak,
és figyelmüket könnyebb lenne úgy megtartani. Ehhez Anna azt javasolja,
hogy bízzanak meg egy külsős szaktanárt, hogy ő tudjon párhuzamosan
művészeti klubot tartani. Péter, a csapat másik tagja kifejezi
ellenvetését, mert épp így is alig van nullán a tanulóközösség, és
inkább tartalékot kéne képezni. Neki szüksége van az anyagi biztonságra.
A csapat megértve Anna és Péter igényeit is, kidolgoz egy olyan
megoldást, ahol extra külsős tanár nélkül is megoldódik az Anna által
felvetett probléma: ha a délutáni foglalkozás előtt időt töltenek az
udvaron, akkor jobban fogják tudni tartani a figyelmet, külön
csoportbontás nélkül is.}
\end{quote}

\hypertarget{az-operator-es-a-csapat}{%
\subsection{Az operátor és a csapat}\label{az-operator-es-a-csapat}}

A szociokrácia rendszerében mindenki vállalhat szerepeket, hatalommal
bíró feladatköröket, ha és amikor erre a csapat felruházza. Ezért az
operátor, aki ugye kezdetben az egyszemélyes mindenes volt,
feladatköröket ad át a többi tanulásszervezőnek. Nagy különbség van
azonban aközött, hogy megkérlek, hogy légy szíves a kirándulásra készíts
össze egészségügyi dobozt, vagy, hogy a csapattal megbeszéljük, hogy van
1. egy balesetvédelmi felelős szerep, amit javasolsz, hogy ezentúl 2. te
végezd el. Az első esetben szívességet kérünk egymástól, a másik esetben
feladatokat, felelősségeket, BPS terminológiában szerepeket adunk át,
delegálunk.

Fontos, hogy a szerepek átadása nem az operátor és az új felelős közti
megállapodás. Ez a csapat életét érintő változtatás, tehát mindenkit be
kell vonni a döntésbe: fontos, hogy mindenkinek elfogadható, hogy
kipróbáljuk, hogy ezentúl ezt a szerepet a javasolt ember viszi tovább?

\begin{quote}
Példa: Anna, egy tanulóközösség operátora sokszor elfelejti kiküldeni a
szülői előtt és után a szülőknek szánt kérdőíveket. Ebből több
konfliktus adódott, mert Péter egy mentor a csapatban, nagyon szeretné
tudni, hogy hogy vannak a szülők. Anna és Péter egy elég komoly vita
(veszekedés?) után vannak. Arra jutottak, hogy inkább Péter felvállalná
a szülői kérdőívek kiküldését és a szülők noszogatását. A következő
tanári meetingen elmondják a javaslatukat, hogy Péter legyen a szülői
visszajelzők kezelője, és kérik a csapatot, hogy Pétertől várják el,
hogy időben kimenjen a levél a visszajelzőkről, és legalább 2
emlékeztetőt küldjön Péter, és legalább egy nappal a szülői előtt
összesítse az eredményeket. Azaz adnak egy pontos szerepdefiníciót, és
javasolják a csapatnak, hogy ezt a szerepet ezentúl Péter vigye. A
csapat minden tagja nagyon örül, mert érezték, hogy ezt a feladatot
érdemes lenne jobban végezni. Kifejezik hálájukat Annának és Péternek,
hogy a nehézségek ellenére fejlesztik a rendszert.\_
\end{quote}

\hypertarget{mi-van-ha-az-operator-nem-akar-vagy-tud-mindenben-reszt-venni}{%
\subsubsection{Mi van, ha az operátor nem akar vagy tud mindenben részt
venni}\label{mi-van-ha-az-operator-nem-akar-vagy-tud-mindenben-reszt-venni}}

Vannak esetek, amikor nem javít a gyerekek tanulási élményén vagy a
tanárok jóllétén, ha az operátor minden döntésben részt vesz, vagy az
operátornak egyéb teendői miatt nincs kapacitása a közösség életében
nagy mértékben részt venni, feladatokat, szerepeket vállalni. Ilyenkor
mondhatja az\emph{t, hogy ``bízom bennetek, szóljatok, amikor úgy
érzitek, hogy kéne róla tudnom. De kérlek titeket, ami a pénzügyeket
érinti, arról szóljatok.''} Sokat kell ilyenkor dolgozni azon, hogy
mindenki tudja, értse és el is találja, hogy mikor kell bevonni az
operátort és mikor nem.

\hypertarget{mi-van-ha-a-tanarokat-nem-erdeklik-a-penzugyek}{%
\subsubsection{Mi van, ha a tanárokat nem érdeklik a
pénzügyek}\label{mi-van-ha-a-tanarokat-nem-erdeklik-a-penzugyek}}

Van, amikor azt érzed, hogy bizonyos dolgokról egyszerűen csak nem
akarsz tudni, csak működjön a dolog. Ilyenkor egyszerűen
felhatalmazhatjuk egymást, hogy egy adott kérdésben légy szíves hozd meg
a döntést, és akár még informálnod sem kell engem. Tanárként azt kérem,
hogy mondd meg, mennyi pénzt költhetek havonta filctollra és
programokra, és most egyelőre nem akarom tudni, hogy a bérleti díj és a
takarítóra költhető keret hogyan függ össze.

A probléma abból eredhet, hogy az operátor először egyedül csinált
mindent, nem volt kivel megbeszélnie a dolgokat. Aztán jöttél te, mint
tanár, és nem akartál mindent megbeszélni, mert sok volt az információ.
Aztán van az a kényelmetlen érzés, hogy szeretnél bevonva lenni, de nem
vagy. Mindenki azt hiszi, hogy ebben megállapodtatok, pedig nem. Csak
történetileg alakult így a helyzet. Ez az a pont, amikor nem az
SZMSZ-hez kell nyúlnod, hanem fel kell hívnod az operátort.

\hypertarget{az-operator-a-tanarcsapat-fonoke}{%
\subsubsection{Az operátor a tanárcsapat
főnöke}\label{az-operator-a-tanarcsapat-fonoke}}

Nem. Ha a főnök azt jelenti, hogy a főnök döntést hoz, a hierarchiában
alatta lévő munkavállalók pedig végrehajtják a döntést, akkor egyértelmű
NEM a válasz. A BPS szervezetek nem így működnek. Az operátor, mint más
is, javaslatokat tesz, és mindenki hozzájárulása esetén a csapat azt
végrehajtja. Ugyanakkor az valószínű, hogy az operátorok habitusuk,
sajátos helyzetük miatt több javaslatot fognak, legalábbis az első pár
évben tenni, mint a többiek.

Minden csapatnak van azonban egy
csapatvezetője (\ref{csapatok-vezetoi}.~fejezet \apageref{csapatok-vezetoi}.~oldalon),
aki azért felel, hogy „a csapat működjön: tiszták legyenek a szerepek és
megtörténjen az, amiben megállapodott a csapat, és működjenek és
fejlődjenek a folyamatok.'' Ahhoz, hogy ezt a célját elérje „facilitál,
moderál, szintetizál, kísér, kérdez.''

Kezdetben ezt a szerepet is az operátor látja el. A csapat dönthet úgy,
hogy ezt a szerepet az operátor átadja egy másik tanulásszervezőnek, aki
ezáltal a csapat koordinátora lesz. Azonban ehhez minden tag
hozzájárulása szükséges. Tehát a BPS-ben nem szavazunk ki senkit, hanem
megbeszéljük, hogy jobb lenne, ha bizonyos szerepet most más végezne.

\hypertarget{az-opertator-felelelosseget-vallal-mindenert}{%
\subsubsection{Az opertátor felelelősséget vállal
mindenért}\label{az-opertator-felelelosseget-vallal-mindenert}}

Nem. Nincs olyan ember a rendszerben, aki mindenért egy személyben
felelős lenne. Egy tornatanár nem felelős minden balesetért, ami
tornaórán történik. Történhet olyan baleset, ami ellen ő nem tudott
volna mit tenni. Azért viszont felelős, hogy ha egyszer már megtanultuk,
hogy balesetveszélyes mélyvízbe ugrania annak, aki nem tud úszni, hogy
megkérdezze mindenkitől, hogy tud-e úszni, mielőtt elkezdődik a
foglalkozás. Ugyanígy: egy operátor/ koordinátor nem felelős azért, hogy
mit csinál egy másik munkatárs, vagy egy gyerek az iskolában, vagy hogy
egy szülő hogyan viselkedik, de azért felelős, hogy ha nem tartunk be
szabályokat és megállapodásokat, akkor tegyen azért, hogy a helyzet
megváltozzon.

\hypertarget{az-operator-kulonleges-szerepe-o-a-penzugyi-stabilitas-biztositoja}{%
\subsection{Az operátor különleges szerepe: ő a pénzügyi stabilitás
biztosítója}\label{az-operator-kulonleges-szerepe-o-a-penzugyi-stabilitas-biztositoja}}

A pénz olyan, mint az oxigén. Szükségünk van rá a tanulóközösség
működtetéséhez. És mivel a Budapest School tanulóközösségek önálló
költségvetéssel bírnak, fontos, hogy egy tanulóközösségnek ne legyen
több a kiadása, mint a bevétele.

Az operátorok által életre hívott tanulóközösségek esetén az operátor
az, aki személyesen felelősséget vállal a pénzügyi stabilitásért. Értsd:
ha többet akar a tanulóközösség költeni, mint a bevétele, akkor neki
kell a zsebébe nyúlni. És ameddig az operátor ezt a felelősséget nem
osztja szét a többi tanulásszervező között (azaz nem szövetkezetként
működnek), addig ő mindig
\_\href{https://en.wikipedia.org/wiki/Primus_inter_pares}{első lesz
az}\_egyenlők között.

\textbf{Akkor ez azt is jelenti, hogy a pénzügyi felelősségvállalás
miatt ő maga meghozhat pénzügyeket érintő döntéseket?} Attól, hogy
operátorként az a dolgod, hogy 1. ne legyen hiány, 2. ha van, akkor
teremtsd elő a hiányt, attól még nem hozol meg minden döntést egyedül.
Mert a ne-legyen-hiány célt sokféleképpen el lehet érni. A csapatnak
kemény munka árán ki kell dolgoznia a saját megállapodásait arról, hogy
mi az a döntés, amire felhatalmazást kaptál.

\hypertarget{az-operator-kilepese}{%
\subsubsection{Az operátor kilépése}\label{az-operator-kilepese}}

Ha konfliktus alakul ki a csoportban, akkor bárki mondhatja azt, hogy
,,én kilépek''. Csak az operátor nem. Vagyis csak nagyon nehezen. Ha
eltűnik az operátor, akkor még az is lehet, hogy megszűnik a
tanulóközösség. Ha mégis olyan helyzet áll elő, hogy kilép, akkor új
operártort kell találni vagy át kell alakítani a struktúrát. Az operátor
ilyenkor is aktívan részt vesz a helyzet megoldásában (pl. az utódja
megtalálásában, az átadás-átvételben, stb).

\hypertarget{koordinator}{%
\subsection{Koordinátor}\label{koordinator}}

Amikor az operátor átadja a csapatvezető szerepet másnak, akkor ezt az
embert koordinátornak kezdjük el hívni, hogy mindenki értse: itt már a
pénzügyi felelősség és a csapatvezetőség szétvált ebben a csapatban. A
koordinátor szintén nem főnök, hanem facilitátor. Sokat gondolkodik
azon, hogy mindenki tudja-e, hogy ki mit és mikor csinál, ő a circle
leader.

Vannak azok a tanulóközösségek, ahol az operátor szerepet a Lab játsza.
Ilyenkor mindig van egy koordinátor a tanulásszervezők között.
